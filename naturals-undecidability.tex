\section{Невычислимое, недоказуемое}

Я как-то раз имел разговор с гражданином десяти лет от роду. Не помню зачем, но я пытался объяснить ему коммутативность умножения натуральных чисел. Он отлично знал, что <<$ab=ba$>>, потому что ему это уже рассказали в школе, но я его спросил: <<А слабо доказать?>>

Он начал пытаться мне объяснять:\\
--- Ну вот видишь, $2\times 5 = 5\times 2$, $3\times 4 = 4\times 3$, $7\times 9=9\times 7$>>. Значит, это свойство выполняется.\\
--- Не понятно с чего это свойство выполняется. Ты привёл всего три примера и устверждаешь, что это всегда так, а я вижу лишь три случая, когда оно работает, в других случаях может быть и не работает.\\
--- Можно ещё рассмотреть примеры. $5\times 7=7\times 5$...\\
--- Нет-нет, а что если мы возьмём какие-то большие числа?\\
--- $11\times12 = 12\times 11$, $13\times14=14\times 13$...\\
--- Я имел ввижу если взять числа типа $123456\times 976245173$, ты уверен, что в этом случае это свойство будет выполняться?

Я пытался подвести его к мысли, что перебрать много разных значений --- это не значит доказать что-то. В итоге я показал ему, что площадь прямоугольника можно посчитать в разном порядке, но из картинки видно, что результат будет один и тот же в обоих случаях, а это и доказывает свойство коммутативности.

Он сказал, что он меня понял, но я в этом не уверен, слишком уж много в его взгляде было сомнения. Скорее я добился того, что он признал мой авторитет, нежели то что его доказательство было не верно.

Вряд ли его непонимание вызвано тем, что он какой-то неумный или что в школе плохо учат. Скорее дело в том, что по природе человеческий мозг склонен к быстрым обобщениям, которые полезны на практике, но не заточен под научные доказательства. Человеку достаточно пару раз обжечься об открытое пламя (или даже один раз), чтобы понять, что огонь~--- это горячо и трогать его не надо. Биологическому организму не требуется строго доказывать, что любой огонь опасен. Северному человеку достаточно схватить пару раз солнечный удар где-нибудь на Кипре, чтобы в дальнейшем всегда перед выходом из дома использовать противозагарный крем и панаму. Ему не надо доказывать себе в уме, что солце всегда такое ужасное, и даже в общем-то будет неоправданным лазить каждый день на сайт с погодой, чтобы выяснять уровень ультрафиолета в своём городе именно сегодня.

Одна из задач школы --- отучить человека от примитивно-биологического мышления и привить ему навык требовать в каждом случае более веских доказательств, нежели простое обощение. Такое мышление необходимо, когда человек начинает рассуждать о политике, экономике, планировании семьи или просто если он занят в какой-то более-менее интеллектуальной сфере. Принцип математической индукции~--- это как раз важнейший шаг для ребёнка к формальному и критическому мышлению и уходе от обощений. Нам не достаточно доказать истинность утверждения для случаяв 1,2,3,4,5 и так далее, над надо доказать истинность лишь для 1, а затем доказать следствие $n+1$ из $n$.

Обобщения не то чтобы всегда не верны. С точки зрения статистики чем большее коричество раз мы убедимся в истинности какого-то утверждения, тем выше вероятность, что оно истинно в принципе всегда. Здесь речь идёт о вероятностях только, но вероятность может быть настолько большой, что часть такое утверждение можно принять за истинное.

Статистика как наука основывается на том факте, что что всегда изучаемых объектов всегда конечное число. Вспоминая ворон из параграфа~\S1.6.5: мы знаем, что множество ворон конечно, и если мы увидим достаточно большое количество ворон, и все они окажутся чёрными, то вероятность того, что и оставшиеся вороны чёрные, довольно высока. Однако, это работает только для физических экспериментов, но не сработает с математическими утверждениями о числах, поскольку сколько бы много мы не рассмотрели примеров типа $2\times 3=3\times 2$, всегда остаётся ещё бесконечное число непроверенных вариантов, которые не позволяют делать нам никаких выводов.

Так вот. Примерно 50 лет тому назад было доказано, что всё, что я написал сейчас выше касательно неправоты обобщений~--- не правда.

\begin{thm}
Для установления истинности почти любого утверждения $P(n)$ достаточно проверить лишь истинность конечного числа утверждений $P(n)$ для $n$ от 1 до некоторого $m$.
\end{thm}
\begin{proof}
Давайте представим себя на минуточку программистами, которые программируют на каком-то там языке программирования. Программа представляет собой некоторый текст конечной длинны, в которой написано по шагам что компьютер должен делать. Программы можно разделить на две категории: те, которые в какой-то момент сами собой прекратят свою работу, и программы, которые будут работать бесконечно долго. Это не очень правильно, но для удобства мы будем говорить, что такие программы <<зависают>>.

Поскольку алфавит, с помощую которого набирается текст программы, конечен, а так же конечна сама длинна программы, то мы можем утверждать, что множество программ длины $n$ так же конечно. Тем более конечно подмножество программ длины $n$, которые в итоге прекращают своё выполнение вместо зависания. Среди всех этих программ существует такая программа, которая не зависает, на работает дольше всех других программ по времени.

Таким образом мы можем определить максимальное время, требующееся для завершения работы программы длины $n$. Как именно его подсчитать не понятно, но сама по себе функция $B:\mathbb{N}\to\mathbb{N}$, ставящая в соответствие длине программы максимальное время выполнения, определена (эта функция по фольклорным причинам называется Busy Beaver Number, но это не важно).

Пусть у нас теперь есть конкретная программа длины n, мы её запускаем и начинаем ждать. Ждём довольно долго, и вдруг оказывается, что программа работает уже дольше, чем время $B(n)$. Поскольку $B(n)$~--- это максимальное время, за которое программа может остановиться, мы можем отсюда сделать вывод, что наша программа не остановится уже никогда.

Пусть у нас теперь есть некоторое математическое утверждение $P(n)$, которое мы можем для любого конкретного $n$ проверить на компьютере. Напишем программу, котроая будет перебирать все числа 1, 2, 3, 4, 5 и так далее подряд, проверяя для них утверждение $P$. Эта программа должна будет остановиться и вывести сообщение, когда она найдёт первое такое $k$, что $P(k)$ не верно. Сама программа, как и любая программа, будем иметь какую-то конечную длину $N$.

Запустим эту программу. Здесь возможно два итога: либо программа остановится раньше, чем за время $B(n)$, либо она не остановится вовсе, что значит, что она никогда не найдёт 
\end{proof}