\documentclass[a5paper,10pt,pagesize,DIV=classic]{scrbook}
\usepackage{cmap}
\usepackage[T2A]{fontenc}
\usepackage[pass]{geometry}

\usepackage{graphicx}
\usepackage{float}
\graphicspath{{images/}}

\usepackage[utf8]{inputenc}
\usepackage[english,russian]{babel}
\usepackage{tikz}
\usetikzlibrary{calc, shapes, arrows}
\usepackage{calc}

\usepackage[raggedright,small]{titlesec}
\usepackage[dotinlabels]{titletoc}

%\titlelabel{\thetitle.\quad\thispagestyle{fancy}}

\usepackage{ccaption}
\captiondelim{. }

\usepackage[bookmarks=false, a4paper, colorlinks, unicode, pdfstartview=FitH, pdftex]{hyperref}
\hypersetup{
plainpages=true,
linkcolor=blue,
citecolor=red,
menucolor=blue,
pdfnewwindow=true
}

\clubpenalty=10000
\widowpenalty=10000

\lccode`\-=`\-
\lccode`\+=`\+
\defaulthyphenchar=127
\hfuzz=1.5pt

\usepackage{amsmath}
\usepackage{amssymb}
\usepackage{amsthm}
\usepackage{amstext}
\usepackage{hyperref}


%% use this instead of slhape or textit for inline definitions %%
\newcommand{\term}[1]{\textit{#1}}

%% definitions of functions %%
\newcommand{\Mod}{\operatorname{Mod}}

%% srirling numbers
\DeclareRobustCommand{\fstirling}{\genfrac{[}{]}{0pt}{}}
\DeclareRobustCommand{\sstirling}{\genfrac{\{}{\}}{0pt}{}}

%% definition for theorems, excercies, etc. %%
\theoremstyle{plain}
\newtheorem{thm}{Теорема}[chapter]
\newtheorem{corollary}{Следствие}
\newtheorem{lemma}{Лемма}

\newtheorem*{Euclids}{Теорема Евклида}
\newtheorem*{EuclidsLemma}{Лемма Евклида}
\newtheorem*{Bezus}{Соотношение Безу}
\newtheorem*{FTA}{Основная теорема арифметики}
\newtheorem*{GodelsCompleteness}{Теорема Гёделя о полноте}
\newtheorem*{HandshakesLemma}{Лемма о рукопожатиях}

\theoremstyle{definition}
\newtheorem{exercise}{Упражнение}[chapter]
\newtheorem{problem}{Задача}[chapter]
\newtheorem{example}{Пример}[chapter]
\newtheorem{definition}{Определение}[chapter]

%% itemize use dashes instead of bullets%%
\def\labelitemi{---}


%% define lcm operator
\DeclareMathOperator{\lcm}{lcm}


\begin{document}

\title{Учебник по математике}
\author{Роман Добровенский\\ \\ \url{heller@heller.ru}\\ \url{http://heller.ru}}
\date{2012--2014, Москва}
\maketitle

\tableofcontents

\input{introduction}

\chapter{Логика}
Первая глава даёт неформальное описание основных законов логики. Главная цель~--- дать какую-то интуицию об излагаемых понятиях и о принципах логических выводов, прежде чем мы всё это формализуем и используем в конце второй главы. Если вам не особо интересны основания математики, то вы можете сразу переходить к середине третьей главы и читать более прикладные вещи, обращаясь сюда по мере необходимости как к справочнику.

Наиболее важными являются первый и четвёртый параграфы. Остальные параграфы имеют факультативный характер и не обязательны к прочтению.

\input{logic-basics}
\section{Представление функций}

В прошлом параграфе мы определили с помощью таблиц истинности пять логических операций. Нам никто не мешает определить и другие операции, задав их таблицу истинности, причём им не обязательно иметь один или два параметра — можно и больше, никто не запрещает. Для примера рассмотрим операцию (удобней, наверное, называть её функцией) $f$ с тремя параметрами, описанную в таблице 1.4. Записывать эту операцию мы будем как $f(a, b, c)$. По таблице легко находить её значения при конкретных параметрах, например: $f(0,0,1) = 0$ или $f(1,0,1) = 1$.

\begin{table}[h]
\centering
\begin{tabular}{ccc|c}
$a$&$b$&$c$&$f$\\
\hline
0&0&0&1\\
0&0&1&0\\
0&1&0&1\\
0&1&1&0\\
1&0&0&1\\
1&0&1&1\\
1&1&0&0\\
1&1&1&0
\end{tabular}
\caption{Представление булевой функции с тремя параметрами}
\end{table}

Какой физический смысл данной функции/операции? Пока никакого — мы делаем то, что мы делаем, просто потому, что мы можем это делать. В математике часто так поступают, а применение это всё находит позже.

Теперь давайте поймём, как определить эту же самую функцию, выразив её через уже известные нам логические операции.

Для начала рассмотрим более простую функцию $g_0$, которая имеет значение $1$ только при наборе параметров $g_0(1, 0, 1)$, а во всех остальных случаях её значением будет $0$. Легко догадаться, что такая функция в точности представляется как: $$g_0(a, b, c) = a\land \neg b \land c.$$ Рассмотрим также функцию $g_1(a, b, c)$, которая принимает значение $1$ только на наборе аргументов $g_1(0, 0, 0)$. Её, соответственно, можно представить как: $$g_1(a, b, c) = \neg a \land \neg b \land \neg c.$$

Теперь рассмотрим функцию $h$, которая принимает значение~$1$ на обоих наборах параметров $(1, 0, 1)$ и $(0, 0, 0)$. Это можно сформулировать так:   значение функции $h$ истинно, когда истинно значение хотя бы одной из функций $g_0$ и $g_1$. Сказанное дословно переносится на язык математической логики следующим образом: $$h(a, b, c) = g_0(a, b, c) \lor g_1(a, b, c).$$

Подставив сюда выражения для $g_0$ и $g_1$, получаем: $$h(a, b, c) = (a\land \neg b \land c) \lor (\neg a \land \neg b \land \neg c).$$

Продолжая рассуждать таким же образом, можно прийти и к выражению для функции $f$, заданной выше таблицей истинности: $$f(a, b, c) = (\neg a \land \neg b \land \neg c) \lor (\neg a \land b \land \neg c)\lor (a \land \neg b \land \neg c) \lor (a\land \neg b \land c). $$

Легко видеть, как получается такое представление: мы просто перечисляем все строки таблицы истинности, в которых функция принимает значение $1$, соединяя параметры функции операцией И (если значение параметра $0$, то перед ним добавляем отрицание), а сами наборы, при которых функция истинна, соединяя операцией ИЛИ. То есть если у нас есть таблица истинности, всегда возможно особо не думая записать, чему эта функция равна, используя только операции И, ИЛИ и НЕ.

Такая развёрнутая форма записи функций называется \term{дизъюнктивной нормальной формой} (ДНФ). Запоминать это не нужно — за пределами этого параграфа данный термин нам больше не понадобится. Суть такого представления заключается в том, что функция выражается как дизъюнкция (операция ИЛИ) некоторого количества конъюнктов (параметров функции, объединённых операцией И).

Можно получить и другое представление, если вначале выразить по нашей схеме не саму функцию, а её отрицание:
$$\neg f(a, b, c) = (\neg a \land \neg b \land c) \lor (\neg a \land b \land c) \lor (a\land b\land \neg c) \lor (a\land b \land c).$$

Теперь можно взять отрицание этого выражения и, используя закон де Моргана, получить такое выражение:
$$f(a, b, c) = (a \lor b \lor \neg c)\land(a\lor \neg b\lor \neg c)\land(\neg a\lor \neg b\lor c)\land(\neg a\lor\neg b\lor \neg c).$$

Мы здесь фактически перевернули знаки $\land$ и $\lor$, а также ко~всем высказываниям применили операцию отрицания — это, по сути, как раз и есть закон де Моргана. Проведите эти рассуждения подробнее самостоятельно.

Новую форму записи также легко получить по таблице истинности, особо не задумываясь над смыслом операции — мы делаем всё то же самое, что и в прошлый раз, но только на этот раз перечисляем не единицы таблицы, а нули, и везде подменяем И на ИЛИ и наоборот, а вхождения параметров в нашу формулу снабжаем отрицанием.

Такая форма записи называется \term{конъюнктивной нормальной формой} (КНФ), и представляет она из себя конъюнкцию (операция И) некоторого количества дизъюнктов (параметров функции, объединённых операцией ИЛИ).

В случае функции $f$ оба подхода оказались практически одинаковыми. В разных случаях, однако, какой-то из них может оказаться короче --- в зависимости от того, какие значения наша функция принимает чаще.

Рассмотрим импликацию:

\begin{table}[h]
\centering
\begin{tabular}{cc|c}
$a$&$b$&$a\rightarrow b$\\
\hline
0&0&1 \\
0&1&1 \\
1&0&0 \\
1&1&1
\end{tabular}
\caption{Таблица истинности для импликации}
\end{table}

Для этой операции описанные нами подходы представления функции дают следующие результаты:

\subparagraph{ДНФ:} $a \rightarrow b = (\neg a\land \neg b)\lor(\neg a \land b)\lor(a\land b)$

\subparagraph{КНФ:} $a\rightarrow b = \neg a \lor b$

\subparagraph{} КНФ здесь оказывается удобнее. Также этот подход можно рассматривать как альтернативу методикам доказательств, представленных в первом параграфе.

\begin{exercise} Запишите КНФ и ДНФ для операций «исключающее ИЛИ» и «эквиваленция».\end{exercise}

Вообще же следовать каждый раз именно такой схеме чаще всего оказывается расточительно. Если посмотреть вни\-ма\-тельнее на таблицу истинности для $f$, то можно заметить, что в случае истинности $a$ функция принимает истинное значение при ложном $b$, и значение $c$ там уже не важно. При ложности же $a$ нам важно лишь значение $c$ и не важно $b$. С учётом этих наблюдений функция $f$ может быть записана совсем просто: $$f(a, b, c) = (\neg a\land \neg c) \lor (a \land \neg b)$$

Получается гораздо красивее, чем было. Обычно в курсах логики значительная часть времени посвящается как раз упрощению формул. Мы могли бы рассмотреть много примеров и типичные приёмы, но вряд ли это окажется сильно полезно. Если вам когда-либо придётся заниматься подобной ерундой (что сомнительно), могу лишь посоветовать подходить к задаче творчески, а не искать универсальных подходов.

Нас же сейчас из всего сказанного больше интересует принципиальная возможность построения КНФ и ДНФ для произвольной формулы. Из наших рассуждений непосредственно следует такая простенькая теорема:

\begin{thm}Любая логическая функция может быть представлена с помощью операций <<И>>, <<ИЛИ>> и <<НЕ>>.\end{thm}

Данные операции можно называть базисом нашей логической системы (неформально говоря, базис~--- это некий набор компонентов, из которых можно собрать любой другой объект, оговорённый в нашей системе). Однако базис этот не единственный, можно придумать и другие наборы функций, через которые \mbox{также} можно будет выразить любую логическую операцию. Примером служит следующая теорема:

\begin{thm}Любая логическая функция может быть представлена с помощью операций <<И>>, <<исключающее ИЛИ>> и константы 1.\end{thm}

\begin{proof}Операции ИЛИ и НЕ можно представить через $\land$, $\oplus$ и $1$:

Представление ИЛИ: $a\lor b = a \oplus b \oplus (a\land b)$

Представление НЕ: $\neg a = a \oplus 1$

Ну а поскольку через функции ИЛИ, НЕ и И мы можем представить любую функцию, то отсюда понятно, что и через И, XOR и 1 мы также можем представить любую функцию.\end{proof}

Таким образом $\land$, $\oplus$ и $1$ также являются базисом нашей логической системы, который часто называется базисом Жегалкина. При работе в базисе Жегалкина оказываются удобными следующие соглашения о записи и названии логических операций:

\begin{enumerate}
\item $a\land b$ записывается просто как $ab$ и называется умножением,
\item $a \oplus b$ записывается как $a + b$ и называется сложением.
\end{enumerate}

С этими соглашениями основные свойства логических операций будут выглядеть так:

\begin{enumerate}
\item $a\lor b=a + b +ab$,
\item $\neg a = a + 1$,
\item $a(b+c) = ab + ac$,
\item $aa = a$,
\item $a + a = 0$.
\end{enumerate}

Эти формулы легко запоминаются и с ними оказывается очень просто работать (позже мы увидим, что такая запись и названия операций имеют и более глубинный смысл). Для примера выразим в базисе Жегалкина, следуя нашим соглашениям, импликацию: $$a \rightarrow b = \neg a \lor b = (1 + a) + b + b(1 + a) = 1 + a + b + b + ab = 1 + a + ab$$

Здесь мы сократили $b+b$.

\begin{exercise} Выведите формулу для эквиваленции: $$a \leftrightarrow b = 1 + a + b$$\end{exercise}

\begin{exercise} Выведите формулу для упомянутой ранее функции $f$.\end{exercise}

\begin{exercise} Докажите, что любая логическая функция может быть выражена с помощью лишь одной операции «штрих Шеффера», которая определяется как $a|b = \neg(a\land b) = 1 + ab$. (Использование значений $1$ или $0$ в записи тоже недопустимо.)\end{exercise}

Часто возникает и обратная задача --- зная какие-то свойства логической функции, определить её значения истинности в таблице. В следующем параграфе мы рассмотрим большую и сложную задачу, на первый взгляд с формальной логикой никак не связанную, которая решается как раз путём построения логической функции, удовлетворяющей заданным критериям. Пока же, как простой пример, мы построим таблицу истинности для импликации --- я уже упоминал, что она в значительной степени является следствием не каких-то логических соображений, а скорее формальной нужды. Сейчас мы уже готовы это продемонстрировать. Значения импликации, если рассматривать её как логическое следствие, для истинной посылки очевидны:

\begin{table}[h]
\centering
\begin{tabular}{cc|c}
$a$ & $b$ & $a \to b$ \\
\hline
0 & 0 & ? \\
0 & 1 & ? \\
1 & 0 & 0 \\
1 & 1 & 1
\end{tabular}
\caption{Неполная таблица истинности для импликации}
\end{table}

Чтобы определить значения в оставшихся ячейках, обозначенных символом <<?>>, необходимо рассмотреть, какими свойствами должно обладать логическое следствие, и проверить, какие ограничения эти свойства накладывают на таблицу истинности. Самое логичное требование к импликации, которое мы уже приво\-ди\-ли в первом параграфе --- это требование транзитивности: \mbox{<<Если} из $a$ следует $b$ и из $b$ следует $c$, то из $a$ следует $c$>>. Формально это свойство записывается так: $$((a\to b) \land (b \to c)) \to (a\to c).$$

Если подставить в это выражение вместо $a$ и $b$ истину ($1$), а вместо $c$ --- ложь ($0$), то, вычисляя (как в первом параграфе) значение истинности для логических связок, нам известных, мы сведём свойство транзитивности к следствию $0 \to 0$, и это выражение обязано быть истинным, если мы хотим, чтобы свойство транзитивности выполнялось для импликации. Если теперь, зная, что $(0 \to 0) = 1$, подставить вместо $a$ и $c$ истину, а вместо $b$ ложь, то мы получим, что также должно быть истинным и выражение $0 \to 1$. Это завершает построение таблицы истинности для импликации. Таким образом, её таблица истинности --- это в некотором смысле вынужденная таблица, в противном случае импликация не удовлетворяла бы тем свойствам, которые мы ожидаем от операции логического следствия.

\section{Самая сложная логическая задача}

В качестве примера построения логической функции, удовлетворяющей заданным критериям, мы рассмотрим довольно безумную задачку, которая даже имеет собственное название, и называется она, ни много ни мало, «Самой сложной логической задачей» («The Hardest Logic Puzzle Ever»). Её автор~--- известный логик и философ Джорд Булос.

\begin{problem}
Довелось нам повстречаться с тремя богами. Они всезнающие и могут отвечать на вопросы, которые предполагают ответ «да» или «нет», причём один из богов всегда отвечает правду, один всегда врёт, а третьему вообще наплевать — он отвечает на вопросы как чёрт на душу положит. Дополнительная проблема в том, что хоть русским они и владеют, но говорить на нём ниже их достоинства. Вместо «да» и «нет» они говорят «ня» и «ми», причём что из этого что означает — неизвестно. Такой вот божественный язык. У нас есть возможность задать этим богам три вопроса, подразумевающих ответ «да» или «нет». Задавать вопросы можно в произвольном порядке, можно задавать разные или одинаковые вопросы. Каждый вопрос адресуется только одному богу. В результате после всех трёх вопросов вы должны точно указать, где бог вранья, где бог правды и где бог случая. Как это сделать?
\end{problem}

В принципе, изложенного в курсе до сих пор материала уже достаточно для того, чтобы решить эту головоломку. Вы можете подумать над ней некоторое количество времени и лишь потом читать дальше. Либо можете сразу читать.

При общении с богами нам необходимо получать понятные для нас ответы на вопросы. При общении со случайным богом мы ничего достоверно узнать не можем, поэтому в первую очередь мы будем выяснять, кто из них случайный. Это, однако, нам ещё только предстоит в дальнейшем — для начала нам надо понять, каким образом сформулировать вопрос так, чтобы ответ на него и бога правды, и бога лжи одновременно давал нам какие-то полезные сведения.

Чтобы было проще, решим для начала упрощённую версию этой задачи. Упрощения будут следующими:

\begin{enumerate}
\item Боги всё же отвечают по-русски, и мы можем точно интерпретировать их ответ.
\item Богов всего два — бог правды и бог лжи. Случайного бога нет.
\item Нам надо выяснить истинность некоторого высказывания, задав только один вопрос.
\end{enumerate}

Такая упрощённая задача как-то попалась сыну моего начальника на окружной олимпиаде по математике — только там речь шла не о богах, а о двух братьях, и у них надо было выяснить, какая дорога из двух правильная.

Обозначим через $a$ тот факт, что мы общаемся с богом правды, а через $\gamma$ высказывание, истинность которого нам необходимо установить. Мы хотим каким-то образом реализовать функцию $f$, представленную в таблице 1.7.

\begin{table}[h]
\centering
\begin{tabular}{cc|c}
$a$ & $\gamma$ & $f$ \\
\hline
 0  &     0    &  0  \\
 0  &     1    &  1  \\
 1  &     0    &  0  \\
 1  &     1    &  1
\end{tabular}
\caption{Требуемая функция в упрощённой задаче о богах}
\end{table}

То есть мы должны получить ответ «да» лишь в том случае, когда $\gamma$ истинно. Мы могли бы задать одному из двух богов следующий вопрос: «Верно ли, что $f$ истинно?». Однако нам известно, что если мы говорим с богом лжи ($a = 0$), то он на все вопросы будет давать противоположный ответ, поэтому нам \mbox{надо} задавать вопрос про истинность функции, которая в случае разговора с богом лжи принимает значение, противоположное $f$. Обозначим её через $q$. Её значения приведены в таблице 1.8.

\begin{table}[h]
\centering
\begin{tabular}{cc|cc}
$a$ & $\gamma$ & $f$ & $q$ \\
\hline
 0  &     0    &  0  &  1  \\
 0  &     1    &  1  &  0  \\
 1  &     0    &  0  &  0  \\
 1  &     1    &  1  &  1
\end{tabular}
\caption{Функция $q$, учитывающая, что один из богов врёт.}
\end{table}

Теперь, если мы будем задавать произвольному богу вопрос «верно ли $q$?», то оба бога будут отвечать «да» в том и только в том случае, когда верно интересующее нас $\gamma$. На человеческом языке этот вопрос можно сформулировать так: «Верно ли, что и $\gamma$, и то, что ты бог правды, либо одновременно верно, либо одновременно неверно?». Сложно и косноязычно, но, тем не менее, это вполне себе ответ на нашу задачу.

Теперь понятно, как действовать и в первоначальном случае. Теперь в качестве параметров функции ответа надо добавить ещё одно высказывание $b$, истинное в случае, когда «ня» означает русское «да». В случае, когда $b = 0$, нам надо ещё один раз подменить значения $0$ и $1$, чтобы ответ «ня» звучал в точности тогда, когда истинно $\gamma$:

\begin{table}[h]
\centering
\begin{tabular}{ccc|ccc}
$a$ & $b$ & $\gamma$ & $f$ & $q$ & $p$ \\
\hline
 0  &  0  &     0    &  0  &  1  &  0  \\
 0  &  0  &     1    &  1  &  0  &  1  \\
 0  &  1  &     0    &  0  &  1  &  1  \\
 0  &  1  &     1    &  1  &  0  &  0  \\
 1  &  0  &     0    &  0  &  0  &  1  \\
 1  &  0  &     1    &  1  &  1  &  0  \\
 1  &  1  &     0    &  0  &  0  &  0  \\
 1  &  1  &     1    &  1  &  1  &  1
\end{tabular}
\caption{Требуемая функция для первоначальной задачи.}
\end{table}

Теперь на вопрос «Правда ли, что верно $p$?», как бог правды, так и бог вранья будут отвечать «ня» в том и только в том случае, когда верно высказывание $\gamma$.

В устной форме, конечно, вопрос этот будет звучать слишком длинно и нелепо. Я приведу лишь начало вопроса: «Верно ли одновременно, что ты бог лжи, „ня“ означает „нет“ и $\gamma$ неверно, либо что одновременно ты бог лжи, „ня“ означает „да“ и $\gamma$ неверно, либо что одновременно...» — ну и так далее, продолжаем словами надиктовывать ДНФ. Выглядит такой вопрос убого, но это тем не менее решение, и, согласитесь, если бы это был вопрос жизни и смерти, то вас даже такое решение устроило бы.

Приведённое выше высказывание можно упростить. Например, по таблице истинности можно заметить, что $p = (a \leftrightarrow b) \leftrightarrow \gamma$. Можно было бы задать наш вопрос, исходя из этого представления, и он был бы короче, но пришлось бы как-то явно оговаривать, в каком смысле мы имеем в виду эквивалентность (смотрите замечание в первом параграфе об ассоциативности импликации и о том, что $(a\leftrightarrow b)\leftrightarrow c$ и $a \leftrightarrow b \leftrightarrow c$ традиционно интерпретируются по-разному).

Есть и более остроумный способ сформулировать вопрос. Опять же, проще понять этот способ, если сначала рассмотреть ситуацию, когда боги отвечают на русском языке. Вопрос тогда будет звучать так: «Что бы ты мне ответил, если бы я спросил тебя о верности $\gamma$?». С богом правды всё понятно — он бы ответил правду. Бог обмана же попадает с этим вопросом в ловушку — на наш вопрос он соврал бы, но именно в такой постановке вопроса он вынужден соврать ещё раз, и по правилу двойного отрицания он вынужден дать правдивый ответ.

Аналогичную идею можно использовать и когда боги отвечают «ня» или «ми». В этом случае вы можете задать такой вопрос: «Если бы я тебя спросил об истинности $\gamma$, ответил бы ты „ня“?» По таблице истинности вы можете убедиться в том, что ответ «ня» всегда однозначно будет указывать на истинность $\gamma$.

Из возможности поставить вопрос по-разному и трюка с двойным отрицанием можно вынести нравственный урок: не всегда хорошо известный способ, логично следующий из теории, приводит к лучшему ответу. Часто (чаще всего) то, что пишут в книгах или рассказывают на уроках, приводит к уродливым решениям, а чтобы найти какой-то более оптимальный подход, приходится проявлять фантазию.

Теперь, когда мы умеем вытаскивать информацию из бога правды и бога лжи, не зная, кто из них кто, и не зная их языка, мы можем приступать к окончательному решению задачи.

Когда мы задаём первый вопрос, всегда существует шанс, что мы задаём его богу случайности, и по его ответу мы не можем судить об истинности утверждения, которое нас интересует. Поэ\-то\-му в первую очередь нам надо узнать, какой бог не является случайным, и мы можем это сделать, задав богу вопрос о случайности какого-либо другого бога.

Вопрос может звучать так: «Правда ли, что бог, на которого я сейчас показываю пальцем, случаен?» И показываем при этом пальцем на какого-то другого бога, не на того, к которому мы обращаемся. Вопрос, понятно, мы задаём не напрямую, а изворотисто, в форме, найденной выше. Я просто уже не отвлекаюсь на мелочи, так как сейчас они и так понятны.

Если на наш вопрос о случайности бога мы получим ответ «да», мы можем однозначно сказать, что второй оставшийся бог~--- либо бог правды, либо бог лжи, но никак не бог случая. Дальше мы можем задавать вопросы гарантированно неслучайному богу, на этот раз рассчитывая на верный ответ в любом случае. Важно, что это работает даже в случае, когда мы изначально обращаемся к случайному богу — поскольку выбор мы делаем среди оставшихся двух богов, как бы бог случая ни ответил, мы всегда в точности можем определить одного неслучайного бога.

Когда один из неслучайных богов установлен, мы уже ему можем задать вопрос о случайности бога, которому мы задавали первый вопрос. Из ответа на этот вопрос мы будем точно знать, какой бог случайный, а какие два нет.

Последний, третий вопрос мы будем задавать не случайному богу о честности одного из богов. Ответ на этот вопрос окончательно расставит точки над тем, какой бог правдив, какой лжив, а какой случаен.

А это ровно то, что требовалось по условию задачи.

\input{logic-predicates}
\section{Теории: интуиция}

В этом пункте мы попытаемся на очень неформальном интуитивном уровне понять, в чем состоят базовые вопросы логики и откуда они возникают при рассмотрении математических теорий, которые, на первый взгляд, не имеют непосредственного отношения к логике. Это будет очень неформальный материал, и я заранее должен оговориться, что подходы и определения, которые я буду здесь приводить, утрированы и упрощены. Абсолютная строгость нам здесь и не нужна, важно пока понять лишь принципы. Формальные определения будут даны в \S~1.8.

Давайте для начала снова рассмотрим теорему~1.4 и попробуем доказать её в общем виде, но несколько более абстрактно.

Итак, нам дано высказывание $\forall x (P(x)\to Q)$. Поскольку нам сказано <<для любого $x$>>, то давайте возьмём для начала какой-нибудь конкретный элемент $a$. Что это за элемент мы не знаем, но он вполне себе конкретен. Имеем $P(a)\to Q$. Предположим далее, что нам совершенно точно известно, что $P(a)$ истинно. Тогда, в этом предположении, используя импликацию, нам так же будет достоверно известно (по таблице истинности), что истинно $Q$. Обозначим тот факт, что из $P(a)$ мы вывели $Q$ как $P(a)\vdash Q$. Теперь трюк: мы изначально не знали что такое $a$, мы выбрали его на самом деле произвольным образом, мы могли бы вместо него выбрать и любой другой элемент. По сути нам тут важно лишь то, что  в принципе есть некоторое $a$, для которого оказывается верно $P(a)$. Из этих соображений мы можем заменить $P(a)$ на $\exists x, P(x)$. Теперь $\exists x, P(x) \vdash Q$ и отсюда мы можем вернуться к импликации (опять же, в соответствии с таблицей истинности): $(\exists x, P(X))\to Q$. Внимательный читатель заметит, что мы наполовину доказали теорему~1.4.

\begin{exercise}
Докажите теорему 1.4 в обратную сторону тем же способом.
\end{exercise}

У читателя могли справедливо возникнуть некоторые сомнения в этом доказательстве, однако они вызваны не каким-то недостатком в доказательстве, а скорее слишком высоким уровнем абстракции. Если перейти не к общим словам о высказываниях, а к конкретным примерам, то подобные рассуждения, как окажется, будут встречаться в математике повсеместно. Например, позже мы докажем, что любое комплексное уравнение имеет решение. Мы можем не знать это решение, мы лишь знаем, что оно существует, но из одного этого факта мы можем делать далее какие-то общие выводы. На уровне логики мы будем проделывать те же самые трюки, но они не вызовут у нас никаких сомнений.

Тем не менее, допустить ошибку в таких рассуждениях так же довольно легко. Рассмотрим элементарное высказывание $$\exists x \exists y, x\not= y$$
Поскольку мы утверждаем существование некоторых $x$ и $y$, удовлетворяющих неравенству, то давайте возьмём конкретные значения $a$ и $b$, такие что $a\not= b$. А теперь, поскольку $b$ мы никак не выводили предварительно, а просто взяли первое попавшееся число, не равное $a$, то давайте заменим $b$ на $\forall b$: $\forall b, a\not= b$. Очевидно, что это приводит к противоречию, так как если теперь взять вместо $b$ значение $a$, то получится $a\not= a$, что явно не правда.

В данном примере мы допустили очевидную ошибку, закрыв глаза на то, что мы выбирали изначально $b$ не произвольным элементом, а исходя из того, чтобы оно не было равно $a$. На практике, в более сложных и длинных рассуждениях, подобные ошибки, связанные с потерей каких-то условий, могут так быстро и не обнаружиться. Можете ли вы с достоверностью утверждать, что в нашем доказательстве теоремы~1.4 мы не допустили ошибки? Ошибки там действительно нет, но у честного человека останутся сомнения.

Все эти рассуждения приводят нас к тому, что хорошо бы иметь более чёткую систему математических доказательств. Хочется формально описать все ходы, которые мы можем делать в наших рассуждениях и которые гарантированно не приведут нас к ошибке. Этим мы и займёмся.

Возьмём за базовый строительный элемент наших рассуждений понятие \term{предложения}, под которым мы будем подразумевать некоторое высказывание, составленное по каким-то простым правилам, возможно, из других высказываний. Формально мы определим это понятие в \S~1.8, пока же просто будем полагаться на интуицию. Например, высказываниями являются выражения $\forall x (P(x)\to Q)$ и <<любые две прямые либо пересекаются ровно в одной точке, либо не пересекаются вообще>>. Последнее можно формализовать: $\forall l \forall k, P(l, k) \oplus C(l, k)$, где $P$ и $C$~--- это соответствующие предикаты, которые в свою очередь так же могут быть более подробно расписаны на языке логики. Несмотря на то, что на практике формализацией на логическом языке математических теорем занимаются довольно редко, для наших текущих целей будет важным помнить, что это по крайней мере всегда возможно.

Если обозначить высказывания греческими буквами, то любое доказательство теперь можно определить как последовательность высказываний $\alpha, \beta, \gamma, \ldots, \omega$. Эта последовательность должна удовлетворять следующему критерию: каждое предложение доказательства должно быть \term{логическим следствием} некоторых предыдущих предложений. Осталось определить лишь логические следствия, но это легко: каждое логическое следствие является набором предложений-посылок и связанными с ними символом $\vdash$ предложением-результатом, причём в качестве составных частей этих предложений выступают уже не конкретные высказывания, а некоторые шаблонные символы, в которые мы можем подставлять произвольные предложения.

Приведём пример. Простейшим правилом вывода является правило, называемое \term{дедукцией}: $\alpha, \alpha\to\beta \vdash \beta$ (если $\alpha\to\beta$ и истинно $\alpha$, то отсюда следует $\beta$). Рассмотрим так же правило $\alpha\lor\beta, \neg\alpha \vdash \beta$ (если верно $\alpha\lor\beta$ и известно, что $\alpha$ ложно, то тогда отсюда следует, что истинно $\beta$). То что эти правила корректны, легко убедиться по таблице истинности: если все выражения слева от знака $\vdash$ истинны, то и выражения справа так же будут истинны. Проверка этого совершенно не сложна, и я оставляю её читателю в качестве упражнения.

\begin{exercise}
В таблице 1.10 приведены типичные правила вывода. Докажите, что они сочетаются с введёнными нами до сих пор определениями логических операций.
\end{exercise}

\begin{table}[h]
\centering
\begin{tabular}{c c | c}
Посылка & Следствие & Наименование \\
\hline
$\neg\neg\phi$ & $\phi$ & сокращение двойного отрицания \\
$\phi$&$\neg\neg\phi$ & введение двойного отрицания \\
$\phi, \chi$ & $\phi \land \chi$ & введение конъюнкции \\
$\phi\land\chi$ & $\phi$ & сокращение конъюнкции \\
$\phi$ &  $\phi\lor\chi$ & введение дизъюнкции \\
$\phi\lor\chi, \neg\phi$ & $\chi$ & дизъюнктивный силлогизм \\
$\phi\leftrightarrow\chi, \phi$  & $\chi$ & сокращение эквиваленции (1) \\
$\phi\leftrightarrow\chi, \neg\phi$  & $\neg\chi$ & сокращение эквиваленции (2) \\
$\phi\leftrightarrow\chi, \phi\lor\chi$  & $\phi\land\chi$ & сокращение эквиваленции (3) \\
$\phi\leftrightarrow\chi, \neg\phi\lor\neg\chi$  & $\neg\phi\land\neg\chi$ & сокращение эквиваленции (4) \\
$T, \phi \vdash \chi$  & $T\vdash \phi\to\chi$ & теорема дедукции (1) \\
$T \vdash \phi\to\chi$  & $T, \phi\vdash\chi$ & теорема дедукции (2) \\
$\phi, \phi\to\chi$ & $\chi$ & modus ponens \\
$\neg \chi, \phi\to\chi$ & $\neg\phi$ & modus tollens \\
$\phi\lor\chi, \phi\to\theta, \chi\to\theta$ & $\theta$ & анализ частных \\
$\phi\to\chi, \chi\to\phi$ & $\phi\leftrightarrow\chi$ & введение эквиваленции
\end{tabular}
\caption{Типичные правила вывода}\label{table:kleene-or}
\end{table}

Пусть теперь нам известно, что истинными являются высказывания $\neg a, a\lor b$ и $b\to c$. Такой набор высказываний, которые мы принимаем за основные и никак их не доказываем, называется \term{аксиомами}. Все вместе эти аксиомы и возможные их следствия обозначим как $T$. Например, используя данные аксиомы мы можем построить следующее доказательство: $T\vdash b$ из первых двух аксиом и второго правила вывода. Затем $T\vdash c$ из правила дедукции и третьей аксиомы. Предложение, которое всегда истинно при предположении истинных аксиом, называется \term{теоремой}. В нашем примере мы получили две теоремы: $b$ и $c$. Множество всех теорем, выводимых из заданных аксиом (включая сами аксиомы), называется \term{теорией}.

\begin{example}
Рассмотрим конкретный пример как можно применять правила вывода. Возьмём за аксиому предложение $\neg (p\land q)$. Тут сразу напрашивается применение закона де Моргана, но у нас нет такого правила вывода, поэтому нам надо его доказать. Для удобства доказательства перенумеруем все шаги.
\begin{enumerate}
\item $\neg (p\land q)$~--- это нам дано;
\item (*) $\neg (\neg p \lor \neg q)$ --- пойдём от противного и \term{предположим}, что верно так же это; факт наличия дополнительного предположения мы отметили звёздочкой
\item (**) $\neg p$ --- введём еще второе предположение;
\item (**) $\neg p \lor \neg q$ --- введение  дизъюнкции;
\item (*) $\neg \neg p$ --- поскольку строчки 2 и 4 противоречат друг другу, мы получаем, что было неверным предположение строки~3(**). Это косвенное применение правила modus tollens~--- из выведенного ложного утверждения мы получили, что ложна посылка;
\item (*) $p$ --- сокращение двойного отрицания;
\item (***) $\neg q$ --- сделаем теперь такое предположение (три звёздочки);
\item (***) $\neg p \lor \neg q$ --- введение дизъюнкции;
\item (*) $\neg \neg q$ --- из противоречия 8 и 2 заключаем, что было ложным предположение~7(***);
\item (*) $q$ --- сокращение двойного отрицания;
\item (*) $p\land q$ --- введение конъюнкции для 6 и 10;
\item $\neg\neg(\neg p\lor \neg q)$ --- из противоречия 11 и 2 заключаем, что предположение~2(*) ложно;
\item $\neg p\lor \neg q$ --- сокращение двойного отрицания; никаких дополнительных предположений не осталось.
\end{enumerate}
Фух, мы доказали один из законов де Моргана в одну сторону. Остальное доказывать я не буду, если читатель мне не верит, то может самостоятельно доказать все остальные утверждения теоремы~1.1.
\end{example}

Очевидно, что подобные доказательства с использованием только лишь правил вывода куда сложнее, чем подходы, которые мы рассматривали до сих пор. Однако, у такого сложного подхода есть и преимущества. Во-первых, такой подход максимально строг и формален, что в некотором смысле защищает нас от ошибок. В последующих параграфах мы обсудим почему первоначальный наш подход строгим считаться не может. Во-вторых, процедура применения правил вывода хоть и сложна, но может быть автоматизирована~--- на правилах логического вывода строятся некоторые системы искусственного интеллекта и автоматического доказательства теорем. На заметку: несколько сложных теорем в математике были доказаны компьютером именно подобным способом; человек провести такие доказательства оказался неспособен. Так же использование правил вывода может использоваться компьютером для проверки того, что уже найденное человеком доказательство не содержит ошибок.

Несмотря на то, что рассматриваемый нами математический аппарат появился сравнительно недавно (точную дату дать сложно, так как всё это создавалось в течение столетий), сами подходы к аксиомам и теоремам оформились именно в таком идейном виде ещё во времена Евклида, чья книга <<Начала>> стала первой известной исторической попыткой построить математическую теорию строго в соответствии с законами логики. После нескольких неформальных определений геометрии на плоскости, Евклид приводил 15 аксиом, которые предлагалось принять без доказательства, а далее, уже исходя из этих аксиом, выводилось шесть томов теорем.

Определения были им даны довольно невразумительные. Так, первыми двумя определениями он определял точку как <<то, что не имеет никакой части>> и линию как <<длину без ширины>>. Это, вероятно, как-то отражает интуитивные представления, но не даёт ни точного определения, ни свойств. Например, Евклид, ссылаясь в этих определениях на <<части>>, <<длину>> и <<ширину>> сами эти понятия нигде не определяет.

Это на самом деле общая ситуация с любой теорией. Если мы даём определение, то мы обязаны пользоваться некими понятиями, определёнными ранее. При этом для этих более ранних определений также должны быть определения. Процесс может продолжаться бесконечно, и мы никогда не сможем прийти к определению, которое не пользуется никакими другими определениями.

Таким образом, какую бы теорию мы не строили, мы неминуемо приходим к тому, что в самом начале нам необходимо ввести некое понятие, которые мы никак не определяем, просто констатируем как факт, что есть некий объект (неопределённый), которому мы придумываем название и говорим, какие мы с ним можем делать операции и какими свойствами он обладает --- это и есть аксиомы.

Мы не будем приводить определения Евклида, так как они лишены смысла, но и не будем вводить пока собственных определений: процесс этот хоть и не слишком сложный, но долгий и кропотливый, к тому же позже мы найдём более удобный способ определить евклидову геометрию. Пока будем полагаться на интуицию читателя о понятиях линии, окружности, пересечения, параллельности и подобном, и перейдём сразу к аксиомам.

Хотя сам Евклид приводил 15 аксиом, с современной точки зрения их можно резюмировать всего пятью следующими простыми утверждениями (остальные аксиомы оказались избыточны либо были переформулированы):

\begin{enumerate}
\item Пусть $a$ и $b$ --- две различные точки. Тогда через них можно провести единственную прямую $L$.
\item Пусть $a$ --- точка, не лежащая на прямой $L$. Тогда из $a$ на $L$ можно опустить единственный перпендикуляр.
\item Пусть $a$ --- точка, лежащая на прямой $L$, а $M$ --- некоторый отрезок. Тогда на прямой $L$ можно построить различные точки $b$ и $c$, лежащие от $a$ на расстоянии, равном $M$.
\item Пусть $a$ --- точка и $M$ --- отрезок. Можно построить единственную окружность с центром $a$ и радиусом, равным $M$.
\item Пусть $L$ --- прямая, и $a$ --- точка, на ней не лежащая. Тогда можно построить единственную прямую $M$, которая будет проходить через $a$ и будет параллельна $L$.
\end{enumerate}

Далее в <<Началах>> следует первая теорема: для любого отрезка с концами $ab$ возможно построить равносторонний треугольник $abc$.

Доказательство строится таким образом (см. рисунок 1.1 для наглядной интерпретации): проведём окружность с центром $a$ и радиусом $ab$, затем окружность того же радиуса, но с центром $b$. Пусть $c$ --- их точка пересечения. Поскольку обе окружности имеют один радиус $ab$, то точка $c$ отстоит от обоих точек $a$ и $b$ на одно и то же расстояние, равное отрезку $ab$. Стало быть точки $a$, $b$ и $c$ образуют равносторонний треугольник.

\begin{figure}[h]
\centering
\begin{tikzpicture}
    \draw (-1,0) circle (2) node [anchor=east] {a};
	\draw (1,0) circle (2);
	\draw (-1,0) -- (1,0) node [anchor=west] {b};
	\draw (-1,0) -- (0,1.7321) node [anchor=south] {c};
	\draw (0,1.7321) -- (1,0);
\end{tikzpicture}
\caption{Построение равностороннего треугольника}
\end{figure}

Вроде бы вполне себе наглядное и очевидное доказательство, какие к нему могут быть вопросы? На самом деле это доказательство некорректно. Если вместо рассуждений на пальцах начать рассуждать более строго (например, на языке логики или около того), то окажется, что у нас возникнет проблема: мы не сможем никак доказать, что окружности с центрами в точках $a$ и $b$ вообще пересекутся. Это кажется очевидно нашей интуиции, но из сформулированных аксиом доказать этого не получится. Когда мы сталкиваемся с ситуацией, когда мы не можем доказать ни утверждение, ни его отрицание, вариантов тут может быть три:

\begin{enumerate}
\item мы просто не смогли найти пока доказательство, так как это может быть сложно;
\item что-то не так с нашей логикой, возможно нам не хватает правил вывода;
\item что-то не так с нашими аксиомами, возможно их слишком мало и их надо дополнить.
\end{enumerate}

Я сразу могу сказать, забегая вперёд, что в случае евклидовой геометрии проблема именно в аксиомах: их недостаточно. Пока мы не сможем этого доказать, но мы к этой теме ещё вернёмся в последующих главах.

Уточним теперь понятие <<что-то не так>>. Для примера введём такую систему аксиом:

\begin{enumerate}
\item Любые две прямые пересекаются ровно в одной точке;
\item Через любые две точки проходит ровно одна прямая;
\item Существуют четыре точки такие, что они не лежат на одной прямой.
\end{enumerate}

Это довольно расплывчатые и странные аксиомы. Например, первая аксиома утверждает, что параллельных прямых не существует. Тем не менее эти аксиомы не лишены смысла. Их можно интерпретировать, например, как аксиомы геометрии, которую мы наблюдаем на плоских изображениях (фото, картины) пространства. На фото возникает это странное свойство: любые две прямые пересекаются~(см.~рис.~1.2). Если рассмотреть, например, фото железной дороги, уходящей за горизонт, то на картинке две колеи сольются на горизонте в одну точку. В физическом мире они конечно не сливаются, но фотография создаст иллюзию этого. Такая геометрия (называемая в науке проективной) вполне имеет право на существование, и очень широко применяется, например, в компьютерной трёхмерной графике, не говоря уже о чисто научных применениях.

\begin{figure}[H]
\centering
\includegraphics[width=6cm]{images/parallels.png}
\caption{В бесконечности прямые пересекаются на фото}
\end{figure}

\begin{figure}[H]
\centering
\begin{tikzpicture}
	\def\point{node [circle, draw, fill, inner sep = 0, minimum size = .2cm] }
	\draw (0, 0) \point (p1) {};
	\draw (0, -1cm) \point (p2) {};
	\draw (-.866cm, .5cm) \point (p3) {};
	\draw (.866cm, .5cm) \point (p4) {};
	\draw (-1.732cm, -1cm) \point (p5) {};
	\draw (1.732cm, -1cm) \point (p6) {};
	\draw (0, 2cm) \point (p7) {};

	\draw (0, 0) circle [radius = 1cm] {};
	\draw (p5) -- (p7);
	\draw (p6) -- (p7);
	\draw (p5) -- (p6);
	\draw (p5) -- (p4);
	\draw (p6) -- (p3);
	\draw (p2) -- (p7);
\end{tikzpicture}
\caption{Плоскость Фано}
\end{figure}

С другой стороны, мы можем отойти от привычного интуитивного определения линий и пространства, и рассмотреть конструкцию, изображённую на~рис.1.3, которая называется плоскостью Фано. Эта <<плоскость>> состоит всего из семи точек и семи линий. Каждая линия состоит из трёх точек. Очевидно, что такое построение не имеет ничего общего с первой нашей интерпретацией в виде фотографий (и даже с нашей интуицией о пространстве и линиях в нём), но простым перебором всех возможных вариантов легко убедиться, что эта <<плоскость Фано>> удовлетворяет всем трём нашим аксиомам.

Рассматривая теперь плоскость Фано и фотографию, мы можем обнаружить, что хотя эти две картинки совершенно не похожи друг на друга, у них есть общие теоремы. Один такой простейший пример демонстрирует следующее упражнение:

\begin{exercise}
Докажите, что из сформулированных аксиом можно вывести, что существуют четыре линии, не пересекающиеся в одной точке.
\end{exercise}

Есть, однако, и утверждения, которые в различных интерпретациях будут отличаться. Самое очевидное касается количества точек: на фотографии их бесконечно много, а вот на плоскости Фано их всего семь. А раз какое-то утверждение может быть либо верным, либо неверным в зависимости от интерпретации, то мы можем понять, что это утверждение из заданных аксиом вывести невозможно.

Схожим образом мы докажем в дальнейшем и невозможность доказательства первой теоремы Евклида из его аксиом: мы предъявим две интерпретации его аксиом, и в одной интерпретации близколежащие окружности будут пересекаться, а в другой нет. Это очень важный приём, так как он даёт нам однозначный ответ на вопрос о том, что это именно наши аксиомы неполны, а не что-то ещё. Мы готовы сформулировать теперь такие определения:

\begin{definition}
Система аксиом называется \term{неполной}, если существует предложение $\phi$ такое, что из аксиом невозможно вывести ни $\phi$ ни $\neg\phi$.
\end{definition}

\begin{definition}
Система аксиом называется \term{полной}, если для любого предложения $\phi$ можно вывести либо $\phi$, либо $\neg\phi$.
\end{definition}

Опять же, забегая вперёд, могу порадовать читателя: практически все полезные аксиоматические системы являются неполными. Об этом будет сказано в третьей главе, и мы даже докажем, что простейшая школьная арифметика не полна и полной быть никак не может (мы даже предъявим целый набор теорем, которые невозможно ни доказать ни опровергнуть).

В каждой конкретной интерпретации любое предложение либо истинно, либо ложно и это не зависит от возможности это доказать. Может быть такое, что у аксиом существуют разные фактически интерпретации, но при этом в них наборы истинных и ложных утверждений совпадают. Это даёт нам основание для того, чтобы в некоторых случаях рассматривать лишь такие наборы утверждений в отрыве от самой интерпретации:

\begin{definition}
\term{Структурой} называется такое множество предложений $X$, что для любого предложения $p$ либо $p\in X$, либо $\neg p \in X$.
\end{definition}

Структуры могут быть интересны иногда сами по себе и связаны с рядом задач (например, по заданной структуре найти аксиомы, которые её породили, желательно автоматически с помощью компьютера), но нами будут использоваться лишь в неотрывной связи с конкретными теориями. Желание увязать структуру с теорией приводит к следующему определению:

\begin{definition}
\term{Моделью} теории $T$ называется такая структура $M$, что для любого $p\in T$ верно, что $p\in M$. Множество всех моделей теории $T$ обозначается как $\Mod T$.
\end{definition}

\begin{definition}
Теория называется \term{удовлетворимой}, если она обладает моделью. В противном случае она называется \term{неудовлетворимой}.
\end{definition}

Разные интерпретации могут обладать одной моделью, если в них верны одинаковые предложения, а вот модель полностью определяется набором своих истинных предложений. Для неполной теории могут существовать различные модели, в которых различные предложения истинны.

Однако, даже если теория полна, это не значит, что мы можем любое истинное утверждение в ней доказать. Подходы к определению истинности высказывания, основанные на возможности доказать теорему и на возможных моделях теории, приводят к таким различным определениям:

\begin{definition}
Предложение $p$ называется \term{синтаксически истинным} в теории $T$ (обозначение $T\vdash p$), если его возможно вывести в данной теории пользуясь правилами вывода.
\end{definition}

\begin{definition}
Предложение называется \term{семантически истинным} в теории $T$ (обозначение $T\models p$), если оно истинно в любой модели данной теории.
\end{definition}

Если предложение семантически истинно, то вовсе не обязательно мы сможем это доказать правилами вывода, это зависит от того, какие собственно говоря, правила вывода мы используем. Правила вывода мы можем взять различные, и они будут обладать различными свойствами, из которых всегда хочется доказать следующее:

\begin{definition}
Набор правил вывода называется \term{полным}, если из семантической истинности предложения следует его синтаксическая истинность.
\end{definition}

Обычно в качестве правил вывода используются правила, приведённые в таблице~1.10, а также четыре правила для кванторов: обобщение и переход к частному для кванторов $\forall$ и $\exists$. Последние правила относительно сложны и мы сформулируем их строго в~\S~1.7, пока что можно думать о них интуитивно как в начале этого параграфа.

\begin{GodelsCompleteness}
Указанный набор правил вывода полон.
\end{GodelsCompleteness}

Обратите внимание на два различных термина: полный набор правил вывода и полный набор аксиом. Первый говорит о выводимости всех истинных семантических предложений  и характеризует нашу логику, которой мы пользуемся, а второй говорит не о логике, а об аксиоматической системе. Это часто вносит путаницу. Например, существуют две теоремы с похожими названиями: сформулированная нами теорема Гёделя о полноте и теорема Гёделя о \term{неполноте}. Первая говорит о том, что логика, используемая нами, в некотором смысле действительно хороша для доказательства истинных утверждений, а вторая говорит уже о неполноте аксиом арифметики. Несмотря на похожее название, эти теоремы относятся по сути совершенно к разным областям.

Доказывать теорему о полноте мы не будем, это доказательство довольно нетривиально (хоть принципиально и не сложно) и выходит за рамки данной книги. Вы можете найти его в любом подробном учебнике по логике либо в статьях в Интернете. Для практических нужд достаточно лишь знать ту логику, о которой мы до сих пор говорили, и о возможной неполноте отдельных формальных систем не задумываться, хотя чистые математики рассматривают и другие виды логики.

Многие операции, которые мы до сих пор рассматривали с точки зрения таблиц истинности и синтаксиса, можно так же рассмотреть и с точки зрения семантики.  В качестве примера возьмём импликацию:

\begin{definition}
Операцией импликации $p\to q$ называется такая операция, истинность которой эквивалентна выражению $\Mod(T, q)\subset\Mod(T, p)$.
\end{definition}

Здесь под $\Mod(T, p)$ понимается множество всех моделей теории $T$, дополненной предложением $p$.

\begin{exercise}
Докажите, что приведённое утверждение порождает ровно те же значения истинности, которые мы приводили в~\S1.1.
\end{exercise}

Прежде чем двинуться дальше, введем последнее простое, но важное, определение.

\begin{definition}
Теория называется противоречивой, если в ней выводимы одновременно некоторое высказывание $p$ и $\neg p$. В противном случае теория называется \term{непротеворечивой}.
\end{definition}

\begin{thm}
В противоречивой теории любое предложение синтаксически истинно.
\end{thm}
\begin{proof}
Пусть мы вывели $p$ и $\neg p$ и хотим вывести $q$. Из истинности $p$ вытекает истинность $p\lor q$ (для истинности <<или>> достаточно, чтобы истинным было одно из двух высказываний, а первое высказывание, как мы знаем, истинно). Однако теперь из того, что истинно $\neg p$ и $p\lor q$ мы получаем, что истинно так же и $q$.
\end{proof}

Это довольно странно выглядит, но на самом деле это нормально: противоречивая теория на то и противоречивая, что она не соответствует здравому смыслу. Очевидно так же, что противоречивая теория не может иметь модели, а следовательно и интерпретации. При построении аксиоматических систем мы должны всеми силами стараться доказать, что аксиоматика, с которой мы работаем, не протеворечива. Опять же, я могу заранее обрадовать читателя: вторая теорема Гёделя о неполноте утверждает, что для большинства полезных аксиоматик непротиворечивость доказать невозможно. Об этом мы кратко поговорим в третьей главе.

То что противоречивая теория не может иметь модели~--- это довольно очевидно. А верно ли обратное? То есть верно ли, что если нам известно, что теория модели не имеет, то она противоречива? Можно так же поставить схожий вопрос: если известно, что теория обладает единственной моделью, то верно ли, что она полна? (То что полная теория имеет единственную модель, опять же, вполне очевидно). В общем случае ответ на эти вопросы отрицательный. С другой стороны логика, которую мы рассматривали до сих пор (и которая, кстати, вполне достаточна для нужд других разделов математики~--- за её рамки нам ни разу выходить в дальнейшем не придётся и за пределами чистой логики что-либо кроме неё вообще редко встречается), всё же обладает этими свойствами. Подробности мы, однако, пока оставим в стороне, поскольку они не относятся напрямую к нашему повествованию.

\section{Парадоксы}

Рассмотрим теперь несколько известных парадоксов классической логики.

\subsection{Парадокс импликации}

Обычно этот парадокс излагается так. Таблица истинности для импликации утверждает, что из ложного высказывания следует любое другое произвольное высказывание: $0 \rightarrow a$ всегда истинно, каким бы ни было $a$. В соответствии с этим определением по идее оказываются справедливы следующие высказывания:

\begin{itemize}
\item Из того, что Луна квадратная, следует, что деревья умеют летать.
\item Из того, что Жанна д’Арк~--- первый космонавт, следует, что Путин~--- краб.
\item Из того, что небо твёрдое, следует, что апельсины оранжевые.
\end{itemize}

Это всё полнейший бред, но он тем не менее идеально соотносится с таблицей истинности для импликации.

Можно привести и другую интерпретацию импликации: легко убедиться, что тавтологией также является высказывание $a \rightarrow (b \rightarrow a)$. Здесь по сути утверждается, что любое истинное высказывание следует из любого другого произвольного высказывания (неважно, верного или нет). Так, например, верно следующее:

\begin{itemize}
\item Вращение Земли вокруг Солнца следует из непогрешимости патриарха.
\item Смертность всех людей следует из того, что в 98-м году был дефолт.
\end{itemize}

Одним словом, бред. А если ввести в поиск по Интернету фразу «парадоксы импликации», то можно и не такое найти.

Собственно, никакого особого парадокса тут, конечно, нет. Все высказывания, которые мы привели выше, взяты с потолка. Мы не определили никак интерпретацию нашей логики, с которой мы работаем, а первичный смысл импликации всё же семантическое следование. Если мы рассматриваем импликацию просто по таблице истинности, вырвав из контекста интерпретаций, то она уже представляет собой не более чем арифметическую операцию с ноликом и единичкой, и искать в ней какой-то особый смысл~--- довольно глупое занятие.

С другой стороны, представление о теориях и моделях на самом деле появилось позже, чем понятие импликации. Изначально была придумана импликация с её таблицей истинности, и только потом люди стали рассуждать о парадоксах, следующих из неё, а представление о моделях появилось ещё позже.

Эта историческая последовательность и привела к  <<парадоксу импликации>>. Как мы показали в \S~1.2, таблица истинности для импликации продиктована необходимостью соблюсти свойство транзитивности. Однако до того момента, как математики развили концепцию теорий и моделей, попытки увязать логически формулы с высказываниями на человеческом языке приводили к таким парадоксам.

С современной точки зрения никакого парадокса уже не осталось, поскольку прежде чем наделять смыслом логическую формулу, мы должны представить её интерпретацию в смысле прошлого параграфа. Если заходить с этой  стороны, то никакого парадокса конечно же уже не произойдёт.

Чтобы увидеть, что парадокса не произойдёт, будет уместно вспомнить про понятие выводимости: из противоречивой теории (когда одновременно предполагались верными и $\psi$ и $\neg\psi$, откуда в общем-то по введению конъюнкции следует высказывание $\psi\wedge\neg\psi=0$) мы показали, что возможно вывести любое произвольное высказывание. Другое дело, что как стало понятно в прошлом параграфе, для такой теории не будет никакой модели — подобная теория является противоречивой.

Таким образом импликация из ложной посылки в любом случае никогда не может быть применена и парадокс не возникает. Её таблица истинности — не более чем необходимость, следующая из того, что логические связки должны быть определены для всех значений истинности.

\subsection{Парадокс брадобрея}

В некотором царстве, в некотором государстве, живёт брадобрей~--- такой мужик, который бреет бороды только тем, кто не бреется сам, а тем кто сам бреется, он бороды не бреет. Вопрос: а кто бреет самого брадобрея?

Допустим, брадобрей сам бреется. Но, будучи брадобреем, он не должен брить тех, кто бреется сам, то есть брить себя не должен. Если же допустить, наоборот, что он не бреется сам, то он должен себя брить, как человека, который сам не бреется. Парадокс.

Ситуация становится понятнее, если переформулировать условия задачи на языке логики. Пусть предикат $B(x)$ выражает тот факт, что $x$ является брадобреем, а предикат $S(y, z)$ говорит о том, что $y$ бреет $z$. Тогда условие задачи можно сформулировать следующим образом:

$$\exists x, B(x), \forall y (\neg S(y, y) \leftrightarrow S(x, y))$$

Однако поскольку $y$ снабжён квантором $\forall$, мы в качестве $y$ можем взять любое значение, в том числе и $x$. Тогда в скобках получится такое выражение:

$$\neg S(x, x) \leftrightarrow S(x, x)$$

Это условие всегда ложно, и значит изначальное высказывание тоже всегда ложно. Таким образом само условие задачи оказывается противоречивым и нереализуемым (не имеющим модели) — такого города и такого брадобрея просто не может существовать в принципе, даже в теории.

А значит, что и любые последующие вопросы, какими бы они ни были, в том числе вопрос «Кто бреет брадобрея?», физически не имеют смысла. Мы это строго доказали.

\subsection{Парадокс пьяницы}

{\bfseries Теорема.} В любом баре найдётся такой посетитель, что если он пьёт, то пьют и все остальные посетители.

Теорема кажется абсурдной — пить теоретически может любое количество людей, без каких либо условий. Однако, эту теорему легко доказать.

{\bfseries Доказательство.} Введём предикат $D(x)$, означающий, что $x$ пьёт. Тогда условия теоремы можно сформулировать следующим образом: $\exists x, (D(x) \rightarrow \forall y, D(y))$. Данное выражение оказывается всегда истинным, что легко увидеть, если рассмотреть две ситуации: когда в баре пьют все, и когда кто-то все же не пьёт. Если пьют все, то $\forall y, D(y)$ и доказывать нечего. Если же в баре кто-то не пьёт, то найдётся такой $x$, что выражение $D(x)$ окажется ложным, а из ложного утверждения следует любое утверждение. Следовательно, приведённая нами формула оказывается всегда истинна. \qed

В чем подвох? На первый взгляд может показаться, что подвох в импликации и это только разновидность первого нашего парадокса. Однако это не так — импликация здесь используется вполне законно, придраться к импликации тут негде — рассуждения первого парадокса здесь не пройдут.

Реальная причина парадокса в том, что формулировка теоремы как бы подразумевает, что есть некий один человек, который является причиной всеобщего пьянства, причём это происходит каждый раз. В формулировке же, используемой при доказательстве теоремы, говорится лишь об одном моменте времени и не говорится ни о какой систематичности. Если бы люди собрались выпить в баре в следующий раз, то теорема осталась бы верной, но выбор человека, который «если пьёт, то пьют все», был бы уже другим. Наша теория, доказанная выше, ровным счётом ничего не утверждает про завтрашний или вчерашний день.

Приведённый парадокс является примером частой ошибки, которую допускают люди при трактовке импликации: импликация ни в коем случае не символизирует собой  какое-то физическое следствие, она лишь является логической связкой, которая показывает невозможность истинности одного высказывания без другого в какой-то статичной модели.

\subsection{Парадокс воронов}

Рассмотрим следующее высказывание: «Все вор\'{о}ны чёрные».\footnote{Я не знаю, так это или не так с биологической точки зрения, но для целей математики предположим, что это так.} Легко увидеть, что данное высказывание эквивалентно высказыванию «Если объект не чёрный, то он не может быть вороной». На языке логики это символизируется тавтологией $$a \rightarrow b = \neg b \rightarrow \neg a$$ и в общем-то довольно очевидно.

Представим, однако, себя натуралистами-исследователями, которые путешествуют по свету и пытаются ответить на вопрос: «Являются ли все вороны чёрными, или все же есть цветные?» Увидеть сразу всех ворон у нас не получится, мы можем видеть только какую-то незначительную долю всех живущих ворон. Однако логично, что чем больше чёрных ворон в разных уголках планеты мы видим, тем больше мы убеждаемся в том, что и все вороны вообще — чёрные.

Возьмём теперь наше эквивалентное черноте всех ворон высказывание «Если объект не чёрный, то он не может быть вороной». Оно равнозначно прошлому высказыванию, и из него как бы следует, что чем больше мы видим не чёрных, а цветных объектов, которые не являются воронами, тем больше должна расти в нас уверенность в том, что все вороны чёрные. То есть даже если мы увидели миллионы апельсинов и яблок, то мы должны делать какие-то выводы о цвете ворон.

Выглядит парадоксально, и, действительно, наши рассуждения не совсем верны. Когда мы говорим что-то на языке формальной логики, мы работаем с абсолютной истиной и абсолютной ложью. Когда же мы рассуждаем по индукции и видим лишь часть мира, мы уже говорим о вероятностях, и здесь законы формальной логики не особо работают.

В данной ситуации следовало бы применять теорию вероятностей. Интересно, что даже используя теорию вероятностей, мы можем прийти к тому же самому результату: наблюдение большого числа цветных объектов должно убеждать нас в черноте всех ворон. Другое дело, что степень нашей убеждённости в этом случае будет расти не с той же скоростью. При наблюдении чёрной вороны наша убеждённость будет расти довольно значительно, а вот при наблюдении зелёного яблока убеждённость в черноте ворон должна расти настолько несущественно, что ей можно вообще пренебречь.

Таким образом мы можем сделать наш первый вывод о том, что при рассмотрении каких-либо утверждений мы должны адекватно подбирать математический аппарат для решения задачи. Иначе легко прийти к ложным выводам.

Второй нюанс, тесно связанный с первым, заключается в том, что мы также всегда должны более подробно формулировать постановку задачи и описывать, как именно мы проводим эксперимент. Наши высказывания и сформулированная тавтология слишком поверхностны для того, чтобы применять их в контексте какого-либо эксперимента.

Предположим себе такой сюжет для научно-фантастической книжки: существует два мира, населённых птичками. В одном мире вороны встречаются крайне редко. Например, их всего нес\-коль\-ко штук на всю планету, и все они чёрные. А во втором мире ворон летает миллионы, но среди них есть всё же одна белая.

Предположим теперь, что наш юный натуралист знает о существовании этих двух миров и знает ситуацию с воронами, однако он не знает, в каком именно из этих миров он находится. И вдруг он встречает чёрную ворону. В этом случае подобная встреча с чёрной вороной должна, напротив, убедить его в том, что в этом мире существуют также и белые вороны — если бы он был в том мире, где вороны только чёрные, то ему было бы вообще какую-либо ворону найти крайне сложно.

Таким образом, при некотором уточнении постановки задачи мы вообще смогли вывернуть наизнанку все наши выводы.

\subsection{Парадокс лжеца}

Парадокс заключается в формулировке следующего высказывания: «Данное высказывание ложно». Является ли оно ложным или истинным? Если оно истинно, то оно же само утверждает, что оно ложно, и должно, следовательно, быть ложным. И нао\-бо\-рот.

На первый взгляд сильно похоже на парадокс брадобрея, но только на первый.  Как сформулировать это высказывание на языке математической логики — совершенно непонятно. Можно было бы сослаться на определение высказывания из первого параграфа, где мы чётко указали на то, что мы рассматриваем лишь те высказывания, про которые можно чётко сказать, истинны они нет, и таким образом просто отмазаться от парадокса лжеца.

Проблема тут в том, что если формулировка, как я её привёл выше, кажется слишком надуманной и от неё вроде как можно было бы отмахнуться, то многие распространённые переформулировки задачи нам этого уже не позволят сделать.

Например, популярен парадокс Пиноккио (такой мужик из сказки, у которого вырастал нос, когда он врёт). Заключается он в том, что Пиноккио как-то заявил: «Ой, у меня нос растёт». Тут применимы всё те же рассуждения.

Сам же парадокс происходит ещё из древнего мира. Самая распространённая формулировка называется парадоксом Платона и Сократа:

\begin{quote}
---~Следующее высказывание Сократа будет ложным,~--- сказал Платон.\\
---~То, что только что сказал Платон, истинно,~--- ответил Сократ.
\end{quote}

Придумали его правда не они, а древнегреческий жрец и провидец, «не предсказывающий будущего, но разъясняющий тёмное прошлое», Эпименид, известный двумя фактами биографии: тем, что заснул в зачарованной пещере, и проснулся лишь спустя 57 лет, а также тем, что утверждал, будто все критяне постоянно врут. Сам он при этом тоже был критянином. Учёные, впрочем, сомневаются не только в том, что он проспал 57 лет, но и в том, что он вообще существовал. Если, однако, он существовал, то было это где-то порядка 600 лет до нашей эры.

Этот парадокс был потом многократно обыгран в разных вариациях, в том числе в художественной литературе. На того же Эпименида ссылается апостол Павел в Новом завете, называя его просто «одним из них же самих». Но встречается он не только в древних текстах. Вот, например, фраза из «Автостопом по галактике» Дугласа Адамса: «Старик постоянно говорил, что всё вокруг — неправда. Правда, потом оказалось, что он лгал» (этот парадокс однако при дополнительных допущениях можно легко разрешить, чем я предлагаю заняться читателю).

Над этим парадоксом долгое время думали философы всех мастей. Легенда утверждает, что поэт, грамматик и древний грек Филит Косский даже помер от бессонницы, пытаясь разрешить этот парадокс. В результате возникло много трактовок этого парадокса, а также много новых формулировок.

Распространённый подход — ввести в рассмотрение высказывания о высказываниях. То есть рассматривать отдельно высказывание $s_0$, а также высказывание $s_1$, утверждающее, что высказывание $s_0$ истинно либо ложно, и они совершенно не обязаны быть как-то согласованны. Например, $s_1$ может быть частным суждением некоторого человека. В этом случае парадокс уже как бы пропадает.

Есть направление, называемое нечёткой логикой. В ней высказывания не являются истинными либо ложными достоверно, а лишь только с некоторой степенью вероятности. Высказывания в формулировке парадокса оказываются верными либо ложными с одинаковой вероятностью. Некое подобное решение предлагает направление интуиционистской логики, в котором вероятностей как таковых нет, но есть высказывания, истинность которых не установлена и не может быть установлена.

Многие философы и математики стояли на позиции, что парадокс лжеца возникает из-за того, что сама его формулировка опирается на собственную формулировку. И подобная рекурсия недопустима и не имеет смысла вообще. Однако тогда была предложена переформулировка парадокса в виде такого бесконечного «рассказа»:

«Все последующие предложения данного рассказа являются ложными. Все последующие предложения данного рассказа являются ложными. Все последующие предложения данного рассказа являются ложными. Все последующие предложения данного рассказа являются ложными. Все последующие предложения данного рассказа являются ложными. Все последующие...»

И вот думай теперь какие из этих предложений истинные, а какие ложные. Тут интересно, что ссылки предложений самих на себя (что философы и математики видели причиной парадокса) уже не используются, но парадокс при этом остаётся парадоксом. Здесь, правда, уже используется бесконечность, что также может смутить.

{\bfseries Упражнение.} Объясните, почему бесконечность «рассказа» в данном случае принципиальна.

Мы рассмотрим этот парадокс с точки зрения той логики, которую я излагал. Для того, чтобы сформулировать условия высказывания, сделаем такую придумку: введём помимо самого понятия высказывания, которое может быть истинным или ложным, понятие «название высказывания» (можно интерпретировать это как разграничение между собственно высказыванием и названием переменной для него, которое мы можем подсовывать в предикаты).

Само высказывание «это высказывание ложно» обозначим как $L$, а его «название» как $\bar{L}$. Введём также предикат $Tr$, определённый для «названий» высказываний, который имеет следующий смысл: «Высказывание с данным „названием“ истинно». Данный предикат можно описать следующим образом: $s = Tr(\bar{s})$. Теперь само наше высказывание может быть описано как $L = \neg Tr(\bar{L})$.

В рассуждении ниже мы используем два правила вывода формул, которые мы не использовали ранее, но в справедливости которых довольно легко убедиться:

\begin{enumerate}
\item \term{введение конъюнкции}: $p, q\vdash p\land q$
\item \term{анализ частных}: $(p\to q), (\neg p\to q)\vdash q$
\end{enumerate}

Посмотрим теперь что из этого можно вывести:

\begin{enumerate}
\item  $T\vdash Tr(\bar{L})\vee\neg Tr(\bar{L})$ — закон исключённого третьего;
\item  $T, Tr(\bar{L})\vdash L$ — по определению нашего предиката;
\item  $T, Tr(\bar{L})\vdash \neg Tr(\bar{L})$ — по определению высказывания $L$;
\item  $T, Tr(\bar{L})\vdash Tr(\bar{L})\wedge\neg Tr(\bar{L})$ — введение конъюнкции с уже имеющейся теоремой;
\item  $T\vdash Tr(\bar{L})\rightarrow Tr(\bar{L})\wedge\neg Tr(\bar{L})$ — дедукция;
\item  $T, \neg Tr(\bar{L})\vdash L$ — по определению $L$;
\item  $T, \neg Tr(\bar{L}) \vdash Tr(\bar{L})$ — по определению предиката $Tr$;
\item  $T, \neg Tr(\bar{L})\vdash \neg Tr(\bar{L}) \wedge Tr(\bar{L})$ — введение конъюнкции;
\item  $T\vdash \neg Tr(\bar{L})\rightarrow Tr(\bar{L})\wedge\neg Tr(\bar{L})$ — дедукция;
\item  $T\vdash Tr(\bar{L})\wedge\neg Tr(\bar{L})$ — анализ частных для 5) и 9).
\end{enumerate}

Последняя полученная формула никогда не может быть истинна, следовательно теория противоречива и не может иметь моделей. В нашей логической модели мы пришли к противоречию, и соответственно сама постановка задачи была некорректна.

Здесь у читателя могут возникнуть сомнения: почему в парадоксе брадобрея мы так запросто приняли противоречивость постановки задачи, а в случае с лжецом, философы и математики тысячелетиями придумывали различные логические системы и трактовки? Не считая того, что сам парадокс брадобрея исторически появился намного позже, дело тут также и в том, что парадокс брадобрея довольно элементарно формулируется в терминах классической логики, и из формулировки легко следует, что постановка задачи некорректна. В случае с парадоксом лжеца задача так просто сформулирована быть уже не может, и более того: сформулирована она может быть различными способами.

В этой задаче также интересно заметить, что причина противоречивости формулировки в классической логике кроется вовсе не в ссылке высказывания самого на себя, как можно было бы предположить, а именно в содержании утверждения касательно истинности высказывания. Здесь можно рассмотреть такую нелепую вроде бы теорему:

\begin{thm}Данная теорема не может быть доказана.\end{thm}

Оказывается, что как только мы заменили слово «истинное» на «доказуемое» (более точно, конечно — «выводимое»), то парадокс лжеца перестаёт вести к противоречиям в классической логике, хотя эта теорема действительно невыводима. Но об этом речь у нас пойдёт уже значительно позже и несколько в другом виде, когда у нас появятся инструменты, чтобы подробнее формализовать приведённое утверждение. Более того, мы увидим, что теоремы, подобные этой, могут иметь вполне себе глубокий теоретический и иногда даже прикладной смысл.

%\section{Другие логики}

Та логика, которую мы рассматривали до сих пор, сама по себе на самом деле не имеет никакой монополии на то, чтобы быть единственно верной. Она удобна и правдоподобна почти во всех ветвях математики, однако помимо неё существует множество других разновидностей логики. В этом параграфе мы ознакомимся с некоторыми из них очень кратко и главным образом неформально, исключительно для того, чтобы у читателя сложилось какое-то впечатление. В дальнейшем эта логика нам в курсе не понадобится (кроме единичных необязательных задач), так что даже без этого параграфа дальнейший материал будет понятен. Однако краткий неформальный экскурс в различные ветви формальной логики может быть полезен для кругозора и приятен для ума.

Всё что мы рассматривали до сих пор называется \term{классической логикой}, которая характеризуется правилами вывода, приведёнными в таблице~1.10 (а так же правила обобщения и сведения к частному для кванторов, что мы пока договорились отложить на время). Внутри самой классической логики так же есть градация: если не рассматривать кванторы, то такая логика будет называться \term{логикой высказываний}, а вместе с кванторами она называется \term{логикой первого порядка}. Эта логика допускает выражения вида $\forall x, P(x)$, но не допускает выражений $\forall P, P(x)$. Если допустить последнее (то есть разрешить не только выражения типа <<для любого объекта $x$>>, но и выражения <<для любого предиката $P$>>), то такая логика будет называться \term{логикой второго порядка}. Но всё это разновидности классической логики. В математике практически всегда дело ограничивается классической логикой первого порядка.

В этой главе у нас пойдёт речь о неклассической логике. Простейший пример, когда возникает нужда в такой логике~--- это компьютерные базы данных. Базу данных можно представить себе как набор таблиц с какой-то информацией. Для определённости будем считать, что мы имеем таблицу участников накопительной программы в косметическом салоне. Среди прочих данных в таблице участников имеется графа <<возраст>>, который участники программы могут сообщать, а могут и нет. То есть эта графа может быть пустой. Это вполне реальная ситуация и любая база данных обычно имеет специально выделенное значение \texttt{NULL}, которым забиваются те данные, которыми мы не располагаем или которые вообще не определены.

Пусть предикат $Y(a, b)$ означает, что участник акции $a$ моложе участника акции $b$. Этот предикат не вызывает вопросов до тех пор, пока мы сравниваем участников, которые сообщили возраст. А что должна вывести программа, если мы задали ей вычислить этот предикат для участников, которые свой возраст не сообщили? Значение этого предиката не определено и мы приходим к необходимости помимо истины (1) и лжи (0) ввести так же понятие неопределённости ($U$) в нашу логику.

Когда мы ввели новое логическое значение, мы должны определить как с этим значением будут работать логические операции. Сделать это возможно многими способами, самый простой и естественный из которых называется \term{логикой Клини} и именно она чаще всего реализована в базах данных. Чаще всего в учебниках для программистов на неё ссылаются просто как на \term{тернарную (или третичную) логику}, но это не совсем корректно: тернарной логикой называется любая логика, в которой есть три значения истинности. Значения истинности приведены в таблицах 1.10, 1.11, 1.12 и 1.13.

\begin{table}[h]
\centering
\begin{tabular}{c | c}
$a$ & $\neg b$ \\
\hline
0 & 1 \\
U & U\\
1 & 0
\end{tabular}
\caption{Связка <<НЕ>> в логике Клини}\label{table:kleene-not}
\end{table}

\begin{table}[h]
\centering
\begin{tabular}{c | c c c}
$\land$ & 0 &U &1 \\
\hline
0 & 0 & 0 & 0 \\
U & 0 & U & U\\
1 & 0 & U & 1
\end{tabular}
\caption{Связка <<И>> в логике Клини}\label{table:kleene-and}
\end{table}

\begin{table}[h]
\centering
\begin{tabular}{c | c c c}
$\lor$ & 0 &U &1 \\
\hline
0 & 0 & U & 1 \\
U & U & U & 1\\
1 & 1 & 1 & 1
\end{tabular}
\caption{Связка <<ИЛИ>> в логике Клини}\label{table:kleene-or}
\end{table}

\begin{table}[h]
\centering
\begin{tabular}{c | c c c}
$\to$ & 0& U& 1 \\
\hline
0 & 1 & 1 & 1 \\
U & U & U & 1\\
1 & 0 & U & 1
\end{tabular}
\caption{Импликация в логике Клини}\label{table:kleene-or}
\end{table}

Проработайте эти таблицы и попытайтесь понять почему они именно такие, а не какие-то другие.

Однако надо иметь ввиду, что это не единственный вариант тернарной логики. Самый распространённый альтернативный вариант~--- это \term{логика Лукаcевича}, которая отличается от логики Клини лишь тождеством $U\to U = 1$. Проблема логики Клини в том, что никакое предложение в нём не может быть всегда истинным. Например, в классической логике мы имели полезнейший закон де Моргана
$$\neg(a \land b) \leftrightarrow \neg a \lor \neg b$$
а в логике Клини он уже не работает, если вспомнить, что эквивалентность задаётся как
$$(a \leftrightarrow b) = (a\to b)\land (b\to a)$$
Более того: в логике Клини нет вообще ни одной тавтологии. Логика же Лукасевича хоть и не сохраняет все законы классической логики (это было бы и невозможно), она по крайней мере сохраняет часть тавтологий.

\begin{exercise}
Докажите в логике Лукасевича, что
$$(a\lor b) \leftrightarrow (a \to b) \to b$$
\end{exercise}

\begin{exercise}
Докажите в логике Лукасечива закон де Моргана.
\end{exercise}

\begin{exercise}
Докажите в логике Лукасевича закон двойного отрицания
$$\neg\neg a = a$$
\end{exercise}

\begin{exercise}
Докажите, что в логике Лукасевича не работает закон исключённого третьего
$$a\lor \neg a = 1$$
\end{exercise}

\begin{exercise}
Докажите, что в логике Лукасевича не работает закон противоречия
$$a\land \neg a = 0$$
\end{exercise}

\begin{exercise}
Докажите, что в логике Клини нет ни одной тавтологии, использующей только переменные и приведённые логические операции (если мы будем вводить новые операции, то мы очевидно можем подогнать тавтологии под бесполезные операции~--- это не интересно совершенно).
\end{exercise}

\begin{exercise}
Не смотря на то, что логика Клини не имеет тавтологий, она допускает естественные правила вывода.Покажите, например, что сохраняется правило дедукции
$$p, p\to q\vdash q$$
\end{exercise}

В классической логике и логиках Клини и Лукасевича мы задали сами логические значения и правила, которыми они связаны. На самом деле мы могли бы задать совершенно произвольные логические значения и функции, лишь бы они были нам как-то полезны. Такой подход называется \term{семантическим}, поскольку мы изначально отталкиваемся от конкретного содержания логики, и лишь затем строим правила логического вывода.

Тем не менее, этот подход не лишён недостатков, одним из которых является то, что часто конкретные логические значения, которые может принимать наша логика, нам не ясны, либо они слишком сложны. В этом случае мы можем воспользоваться \term{синтаксическим} подходом к определению логики, который предполагает, что мы задаём лишь правила вывода теорем, но никак не говорим об истинности значений. Последнее упражнение в частности демонстрирует, что хоть мы и не имеем в логике Клини никаких тавтологий, это не мешает нам выводить теоремы пользуясь правилами вывода.

\begin{exercise}
Докажите, что не существует никакой тернарной логики, в которой работали бы все утверждения теоремы~1.1.
\end{exercise}

\begin{exercise}
Приведите пример четверичной логики (то есть логики, в которой помимо 1 и 0 существуют ещё некие неравные логические значения $\alpha$ и $\beta$), удовлетворяющей всем утверждениям теоремы~1.1.
\end{exercise}

Последнее упражнение демонстрирует, что одним и тем же синтаксическим правилам может соответствовать на самом деле множество семантик. Для классической логики высказываний мы в следующей главе покажем, как можно ввести бесконечное количество семантических интерпретаций, все из которых будут удовлетворять всем утверждениям теоремы~1.1. Таким образом получается, что синтаксический подход к логике оказывается в некотором смысле более богатым: даже не зная того заранее, мы всегда описываем потенциально гораздо более широкий класс возможных логик. Впрочем, есть и обратная сторона медали: в конкретной модели работать гораздо проще, чем применять только синтаксические правила преобразования формул, как мы убедились в прошлом параграфе.

Здесь нет строгого закона, но обычно логику на основе логических значений строят люди, преследующие прикладную цель, поскольку с ней проще работать и они хорошо понимают предметную область, которую собираются исследовать. Логику же на основе правил вывода строят люди, которым важна чистота и строгость выкладок безотносительно какой-либо физической интерпретации, то есть главным образом философы и математики, занимающиеся основаниями науки.

В качестве наиболее простого для восприятия конкретного примера синтаксического построения логики рассмотрим \term{модальную логику}. С точки зрения интуиции в этой логике существует три разновидности истинности: \term{необходимая истина} ($\Box p$), \term{возможная истина} ($\diamondsuit p$) и \term{фактическая истина} ($p$). Неформально это можно интерпретировать как возможность вообразить себе альтернативы. Вот, например, сегодня я очень сильно замёрз, пусть это высказывание $p$. Это фактическая истина~--- я это прочувствовал на себе, это правда. Но теоретически это могло бы быть и по-другому: например, я мог бы не торчать в Москве, а полететь на далёкие острова в эмиграцию. Поэтому нельзя сказать, что то что я сегодня мёрз весь день является какой-то необходимой истиной: могло бы быть и по-другому. В то же время если рассмотреть высказывание $q$ <<в открытом космосе нельзя дышать без скафандра>>, то это истина необходимая, поэтому мы это обозначим как $\Box q$.

Возможная истина~--- это истина, которая теоретически возможна или была бы возможна при каких-то обстоятельствах. Например, рассмотрим высказывание <<американцы первыми запустили человека в космос>>, которое обозначим как $r$. Фактически, это не правда, поэтому мы пишем $\neg r$. В то же время мы легко можем представить себе ситуацию, при которой американцы обогнали бы СССР в космической гонке: это не есть что-то предопределённое природой, это могло случиться. Поэтому мы можем так же написать $\diamondsuit r$. 

Сформулируем правила вывода модальной логики. Во-первых, они вбирают в себя все правила вывода классической логики из таблицы~1.10. Во-вторых, мы дополним их правилами для модальных операторов:

\begin{enumerate}
\item $\Box p \vdash \neg \diamondsuit \neg p$
\item $\diamondsuit p \vdash \neg \Box \neg p$
\item N-правило: если $A$~--- набор аксиом (необходимых истин) и $A\vdash p$, то $A\vdash \Box p$
\item K-правило: $\Box (p\to q) \vdash (\Box p) \to (\Box q)$
\item T-правило: $\Box p \vdash p$
\item 5-правило: $\diamondsuit p \vdash \Box \diamondsuit p$
\end{enumerate}

Рекомендую вам попытаться понять на интуитивном уровне что означает каждая из этих аксиом.

Аксиоматик модальной логики существует много разных, между ними есть маленькие и большие философские разногласия. Система аксиом, которую я привёл, называется $S5$-аксиоматикой и она наиболее часто встречается. Название это главным образом историческое, так же как и названия правил N, K, T и 5.

Возникает вопрос: а почему мы ввели именно правила вывода и сказали, что эта логика определяется синтаксически, вместо того, чтобы просто задать какие-то дополнительные логические значения и работать с ними так же как мы работали с классической логикой в самом начале этой книги? На самом деле задать какой-то набор логических значений для модальной логики было бы невозможно и мы можем это продемонстрировать.

Во-первых, двух значений было бы не достаточно в любом случае. Возьмём любое фактически истинное высказывание $p = 1$. Каким логическим значением должно обладать выражение $\Box p$? В терминах только истинности и ложности на этот вопрос явно нельзя ответить.

Аналогично можно увидеть, что нам не хватит и трёх логических значений. Если к истине и лжи добавить неопределённое значение U, и мы знаем, что истинны одновременно высказывания $p$ и $\diamondsuit \neg p$, то какое должно быть значение истинности для $\diamondsuit p$? Если считать это высказывание истинным, то мы придём к тому, что любое высказывание, начинающееся с символа $\diamond$ будет истинным. Задать его ложным было бы вроде как вообще не правильным. Если определить его за $U$, то тогда любое выражение с операторами $\diamondsuit$ или $\Box$ будет иметь это значение, что лишает модальную логику смысла.

\begin{table}[h]
\centering
\begin{tabular}{c | c c c}
$a$ & $\neg a$ & $\Box a$ & $\diamondsuit a = \neg\Box\neg a$ \\
\hline
0 & 3 & 0 & 0 \\
1 & 2 & 0 & 3 \\
2 & 1 & 0 & 3 \\
3 & 0 & 3 & 3
\end{tabular}
\caption{Попытка модальной конечнозначной логики}\label{table:kleene-or}
\end{table}

Предположим, что всё же мы можем свести модальную логику к логическим значениям, если рассмотреть четвертичную логику: 0 для необходимой ложности, 1 для возможной ложности, 2 для возможной истинности и 3 для необходимой истинности. Здесь легко поставить таблицу для таблицы истинности операторов $\Box$  и $\diamondsuit$ (таблица~1.15).

Вроде бы пока все значения кажутся логичными и сходятся. Попробуем определить таблицу истинности для $\land$ (таблица~1.16). Основная масса значений в таблице очевидна, но что делать со связкой $2\land 3$?  Их никак нельзя определить так, чтобы это соответствовало нашим интуитивным представлениям о модальности. Если $p=2$, то очевидно $p\land\neg p = 0$ по закону противоречия. В то же время если взять два независимых высказывания $p=1$ и $q=2$, то $p\land q$ хоть и непонятно чему должно быть равно, но это явно никак не 0, поскольку если оба высказывания возможно истинны и не зависимы друг от друга, то вероятно они могут быть фактически истинными и одновременно.

\begin{table}[h]
\centering
\begin{tabular}{c | c c c c}
$\land$ & 0 &1 &2 & 3 \\
\hline
0 & 0 & 0 & 0 & 0\\
1 & 0 & 1 & ? & 1\\
2 & 0 & ? & 2 & 2\\
3 & 0 & 1 & 2 & 3
\end{tabular}
\caption{Связка <<И>> в модальной логике}\label{table:kleene-or}
\end{table}

Такие рассуждения приводят нас к заключению, что для модальную логику не удастся задать кратко через таблицы истинности с конечным числом значений и нам остаются только правила вывода.

Если так же большой соблазн определить необходимо истинные предложения как утверждения, которые могут быть доказаны. Такой подход так же не срабатывает 


Самым известным примером логики, которая не описывается подобным тривиальным способом семантически, является \term{интуиционистская логика}, появившаяся в начале прошлого столетия как попытка уйти от скользких моментов классической логики. Основная идея этой логики заключается в том, чтобы пользоваться лишь теми правилами вывода, которые никак не предполагают, что каждое утверждение может быть лишь в двух состояниях: либо истинным либо ложным, а вместо этого опираться лишь на уже доказанные утверждения.

Как один из примеров давайте рассмотрим закон исключённого третьего. Классическая логика говорит, что для любого предложения $p$ либо оно само, либо $\neg p$ истинно. Предположим первое, и в этом предположении докажем некое утверждение $q$. Теперь предположим $\neg p$ и докажем отсюда некоторое предложение $r$ (оба доказательства могут быть длинными и сложными и само утверждение может быть нетривиальным). Закон исключённого третьего из классической логики утверждает, что у нас в любом случае будет истинно либо $r$ либо $q$, но это может вызвать сомнения. Что, если ни $p$ ни $\neg p$ в принципе недоказуемы в нашей системе аксиом? Тогда мы не можем доказать $p \lor \neg \neg p$, но в классической логике мы принимаем это как аксиому и доказываем отсюда теоремы. Кто-то скажет, что это нормально, а кто-то усомнится.

Подобная ситуация возникает и с двойным отрицанием. Если мы каким-то образом доказали формулу $\neg\neg p$, то в классической логике это автоматически означает истинность $\neg p$. Интуиционистская логика же предполагает, что даже при доказанной $\neg\neg p$ сама истинность $p$ может быть по-прежнему неустановленной. Максимум, что говорит доказанное $\neg\neg p$ о самом $p$, так это то что $p\neg$

%\section{О боге}

\begin{thm}
Бог существует.
\end{thm}
\begin{proof}
Это доказательство принадлежит Гёделю, которого мы уже неоднократно поминали в нашем курсе. Некоторые исследователи считают\footnote{<<Reflections on Gödel’s Ontological Argument>>, Christopher G. Small}, что попытка доказать существование Бога было одним из главных движущих стимулов для Гёделя заниматься логикой. Достоверно известно, что Гёдель был ревностным католиком, но, видимо, отлично понимал, что вещи, высказываемые в церкви, совершенно смехотворны с точки зрения логической обоснованности и никакого умного человека не убедят. Поэтому он пытался привести строгое доказательство. Рассуждения на тему доказательства существования Бога у него появились в ранних черновиках и занимался он этим доказательством на протяжении всей жизни. Само доказательство он так и не опубликовал, вероятно, считая его неполноценным и неубедительным, и та форма, в которой доказательство сейчас приводится в многочисленных источниках, восстановлена по его черновикам после смерти и комментариям его студентов, с некоторыми из которых он обсуждал доказательство.

Я думаю, что привести здесь это доказательство будет довольно интересно и познавательно <<для общего развития>>. В конце концов есть подозрения, что современная логика развивалась во многом с целью развития именно этого доказательства, так же оно является неплохим примером применения модальной логики. Сегодня вариации этого доказательства появляются с завидной регулярностью, и когда в новостях иногда проскакивает что-нибудь вроде <<математики доказали существование Бога>>~--- это вовсе не журналистская утка, это появление одного из вариантов подобного доказательства.

Прежде чем мы перейдём к формальному доказательству, я разъясню базовые идеи Гёделя.

Во-первых, Гёдель опирается на идею о том, что любой объект обладает некоторыми \term{свойствами}, которые его целиком описывают. Это довольно простая идея: те же физики, когда получают новую частицу, тщательно описывают её в каждом мельчайшем аспекте. По этому описанию мы всегда можем понять о какой частице идёт речь. Договоримся, что запись $Fx$ будет означать, что объект $x$ обладает свойством $F$. Таким образом $F$ является предикатом.

Определим объединение ($F\lor H$) и пересечение ($F \land H$) свойств следующим образом:
$$\Box \forall x, (F\land H) x \leftrightarrow \Box \forall x, Fx\land Hx$$
$$\Box \forall x, (F\lor H) x \leftrightarrow \Box \forall x, Fx\lor Hx$$


Набор свойств (предикатов) $\textbf{F}$ мы будем обозначать жирным шрифтом. Свойство, являющееся объединением всех свойств набора $\textbf{F}$ будем обозначать как $\bigvee\textbf{F}$, а свойство-пересечение как $\bigwedge\textbf{F}$. Набор всех свойств объекта $x$ обозначим как $\textbf{X}$, а \term{сущностью} этого объекта назовём свойство $X = \bigvee \textbf{X}$.


\end{proof}

\section{Формализм}

До сих пор все наши рассуждения были главным образом интуитивными, мы апеллировали к каким-то физическим образам и вводили нестрогие вспомогательные понятия вроде множеств (само понятие множества определяется в математике строго, но мы это сделаем лишь в следующей главе). Такой подход нельзя назвать безупречным с математической точки зрения, поэтому в этой главе мы формально введём уже рассматриваемые нами ранее понятия.  Вернее, не совсем формально, но я покажу, как это в целом делается. Этот параграф необязателен для дальнейшего понимания книги, но тем, кому важно доскональное понимание основ, он сможет немного помочь.

Состоять наше изложение логики будет из трёх частей: языка (то, как мы записываем предложения), синтаксиса (правила вывода одного предложения из другого) и семантики (наделение предложений предполагаемым смыслом). Главным образом причём мы будем говорить именно о языке, поскольку синтаксис и семантику мы в принципе уже рассмотрели и формальное изложение отличается не сильно.

\subsection{Язык}

Как мы уже отмечали, когда мы даём какое-то определение, мы всегда вынуждены пользоваться другими определениями. В конечном итоге мы обязаны ввести какое-то понятие, которое мы никак не определяем. Это называется <<принципом Мюнгхаузена>>: если бы мы не начинали построение математики от какого-то неопределяемого понятия, то получилось бы, что наши определения как-то зависимы друг от друга: условно говоря определение А базировалось бы на определении Б, а определение Б на определении А, и это в самом явном случае (зависимости могли бы быть самыми сложными теоретическими, но такие определения всегда были бы ошибочны). Хотя в определениях в общем-то допускаются перекрёстные ссылки друг на друга или даже самих на себя, где-то в любом случае любое определение должно ссылаться на понятие, которое мы ещё не определили.

В качестве понятия, которое мы никак не будем определять, у нас будет выступать \term{<<символ>>}. С точки зрения интуиции, символ~--- это некоторая закорючка на бумаге. С точки же зрения логики это понятие, которое мы принимаем без попыток понять что это.

\begin{definition}
\term{Алфавитом} назовём набор символов.
\end{definition}

\begin{example}
Примером алфавитов может служить русский или английский алфавит.
\end{example}

\begin{example}
Для нужд логики мы определим алфавит, состоящий из символов $\land$, $\lor$, $\to$, $\neg$, $\oplus$, $\leftrightarrow$, $\forall$, $\exists$, $=$, (, ) а так же всех символов английского языка, как строчных, так и заглавных. Сам этот алфавит будем обозначать как $\Sigma$.
\end{example}

Да, опять же нам надо как минимум определить что значит слово <<набор>>, используемое в определении алфавита. Это возможно сделать, но в такую степень формализма мы уже не будем углубляться, и если вы в дальнейшем найдёте где-то подобные дырки в определениях, отнеситесь к этому с пониманием. Мы могли бы привести все определения, но рассказ просто очень сильно тогда затянулся бы. Я же излагаю сейчас лишь то, что реально имеет какое-то отношение к практике и показываю как в целом строится фундамент логики.

\begin{definition}
\term{Строкой} (или так же \term{словом}) называется конечная упорядоченная последовательность символов (возможно, пустая) некоторого алфавита. Пустая строка для удобства обозначается как $\epsilon$.
\end{definition}

Строками алвавита $\Sigma$, определённого выше, являются например такие выражения как <<$\land\land P\oplus\leftrightarrow$>>. Строками русского алфавита будут такие последовательности как <<аплотфдц>>, а английского такие как <<shehs>>. Это совершенно бессмысленный набор символов, и отсюда ясно, что нам необходимо как-то из всех возможных строк выделить допустимые.

\begin{definition}
Набор слов (возможно, бесконечный) называется \term{языком}.
\end{definition}

Простейший, но одновременно с тем и почти бесполезный способ задания языка~--- это простое перечисление всех строк алфавита. В каких-то частных случаях это было бы возможно, но в целом это неинтересные либо непрактичные примеры. Очевидно, что перечислить все возможные предложения логики (или любого другого языка) без использования каких-то специальных механизмов явно невозможно.

\begin{definition}
\term{Грамматикой} называется некоторое формальное описание структуры допустимых слов языка.
\end{definition}

Это довольно нечёткое определение и как именно грамматику задавать может решать каждый сам для себя. Например, мы могли бы сконструировать язык, описывающий все возможные положения игры крестики-нолики. В качестве алфавита мы выберем набор $\{x, o, ?,1, 2\}$, а в качестве языка условимся называть все строки длины 10 этого алфавита, в которых первым символом идёт либо 1 либо 2 (что обозначает игрока, которому принадлежит ход), а оставшиеся символы будут обозначать подряд все клетки поля, где помимо крестиков и ноликов мы могли бы ставить символ $?$ для незанятых клеток. Примером такого слова может служить строка <<20?0?x???xx>>~--- она описывает ситуацию, изображённую в таблице~1.11.

\begin{table}[h]
\centering
\begin{tabular}{c | c | c}
o & & o\\
\hline
  & x & \\
\hline
 & x & x
\end{tabular}
\caption{Ход второго игрока}
\end{table}

Мы могли бы перечислить все слова языка крестиков-ноликов, но это было бы сложно, так как возможных слов, которые нам подходят, слишком много. Вместо этого мы явно указали, что вначале указывается чей ход, а затем позиция в девяти клетках. Такое описание можно считать грамматикой языка крестиков-ноликов.

В общем случае для задания грамматик и работы с ними существует целый ряд стандартных механизмов, которыми занимается раздел математики под названием <<Теория формальных языков>> и этими механизмами удобно пользоваться.

Самый простой способ задания грамматики языка~--- это разбить предложения языка на составные единицы и указать правила, по которым они составляются. Такие грамматики называются \term{контекстно-свободными}. Тут можно вспомнить уроки русского языка в школах: все изучали, что предложение состоит из составных частей вроде подлежащего, сказуемого, дополнения, вводного предложения, деепричастного оборота. Определения частей языка могут быть и рекурсивными, то есть ссылающимися на самих себя. Так, предложением является так же и набор из нескольких предложений, соединённых союзами. Эти элементы языка так же описываются с помощью других языковых элементов: подлежащее может быть представлено местоимением, существительным, числительным и т.п.

Таким же путём мы сейчас опишем грамматику предложений логики. Для удобства мы будем записывать правила в форме, подобной следующей:
$$a \to bad | ef | g$$
Здесь $a$~--- это объект языка, вертикальными чертами разделены различные варианты чем $a$ может являться, а между вертикальных черт записывается конкретная форма. Запись может разбиваться на несколько строк.

\begin{example}
Правило, приведённое выше говорит, что, например, запись  $bbefdd$ является элементом $a$. Действительно, $ef$ является элементом $a$. То есть мы можем сказать, что $bbefdd = bbadd$. Теперь, $bad$ в середине строки так же является элементом $a$: $bbadd = bad$. Собственно мы получили, что изначальная строка является элементом $a$.
\end{example}

Я не буду выписывать грамматику целиком, определив лишь базовые конструкции для понятия \term{формулы}:\\
\\
Формула $\to$ атом | $\neg$ формула\\
\hspace*{2cm}| формула $\land$ формула | формула $\lor$ формула\\
\hspace*{2cm}| формула $\to$ формула | формула $\leftrightarrow$ формула\\
\hspace*{2cm}| $\forall$ \term{переменная} формула | $\exists$ \term{переменная} формула\\
Атом $\to$ терм = терм | \term{предикат} (списоктермов)\\
Терм $\to$ \term{константа} | \term{операция} (списоктермов)\\
Списоктермов $\to$ терм списоктермов | $\epsilon$\\
\\
Слова, которые я выделил курсивом (предикат, переменная, операция, константа)~--- это некоторые символы, которые мы каким-то произвольным образом разбили в группы. В каждой конкретной ситуации вы разбиваем эти символы по-разному. Например, так:\\
\\
Предикат $\to$ P | Q | R| $\ldots$\\
Переменная $\to$ x | y | z| $\ldots$\\
Операция $\to$ f | g | h | $\ldots$\\
Константа $\to$ a | b | c | $\ldots$\\
\\
Это уже целиком зависит от того, как мы собираемся использовать эти символы.

\begin{example}
Давайте разберём формулу $\forall x P(x) \to Q(f(y))$. Эта формула явно имеет вид <<$\forall$ переменная формула>>, в роли переменной выступает $x$, а в роли формулы $P(x) \to Q(f(y))$. Последняя так же состоит из двух формул $P(x)$ и $Q(f(y))$, соединённых символом $\to$. Обе эти формулы являются атомами вида <<предикат (списоктермов)>>. В случае $P(x)$ список термов состоит из единственного терма $x$ (более точно~--- из терма $x$ и спискатермов $\epsilon$), а в случае $Q(f(y))$ из единственного терма $f(y)$. $x$ является переменной, а терм $f(y)$ имеет вид <<операция список термов>>. В качестве операции тут выступает $f$, а в качестве терма $y$, который является переменной.
\end{example}

Как видно, простая на вид запись на самом деле, если определять её формально, имеет довольно сложную структуру. Тем не менее для математической строгости мы обязаны это всё определять именно таким образом.

\begin{definition}
Переменная $v$ в составе формулы называется \term{свободной}, если перед этой формулой не написано $\exists v$ или $\forall v$.
\end{definition}

Здесь требуется сделать несколько дополнительных ремарок уже по самому смыслу того, что мы определили.

Во-первых, читателя могло смутить, что мы никак не обмолвились о множествах и пишем теперь не $\forall x\in A$, а просто $\forall x$. Это связано с тем, что теория множеств сама по себе требует определения в терминах логики и мы введём её формально лишь в следующей главе. Теория множеств хороша для интуитивного понимания предикатов, но пользоваться ей прежде, чем она определена, мы не имеем права.

Во-вторых, вас могла смутить запись $P(f(x))$ и термин <<операция>>. Формально это просто правило для записи предложений, интуитивно об этом можно думать как о некотором правиле, которое некоторым термам ставит в соответствие другой терм. Например, опять же воспользовавшись нестрогой интуицией, мы могли бы считать, что операция $f$ сообщает достаток человека. Если предикат $P$ означает достаток свыше миллиона, а $x$~--- это некоторый человек, то формула, выражающая высокий доход человека $x$ будет записываться как $P(f(x))$. Такие конструкции входят традиционно в язык логики, но с точки зрения интуиции и практики о них гораздо лучше рассуждать опираясь на теорию множеств, и поэтому я это оставил для следующей главы.

Ну и в заключение заметим, что введённый нами язык допускает запись $\forall x\ P(x)$, но не допускает записи $\forall P\ P(x)$. Это не просто так, поскольку если бы мы допустили выражения в духе <<для любого предиката>>, то наша логика оказалась бы несколько сложнее. Такая логика используется, но для наших нужд она не пригодится. Для общей информации я лишь замечу, что логика, рассмотренная нами называется \term{логикой первого порядка}, если добавить к ней выражения <<для любого предиката>>, то это будет \term{логика второго порядка}, а если же, наоборот, кванторы убрать вовсе, то получится \term{логика нулевого порядка}. Если из логики нулевого порядка убрать операции и переменные, то получится \term{логика высказываний}. Это всё нам не понадобится в дальнейшем, но упомянуть <<для общего развития>> будет не лишним.

\subsection{Синтаксис}

Язык логики определён, перейдём к синтаксису. Читатель теперь может понять формальное значение правил вроде
$$\alpha, \alpha\to\beta \vdash \beta$$
Здесь в качестве греческих букв может выступать любое предложение, а каждая из записей в левой части определяет структуру предложения, которая соответствует нашей контекстно-свободной грамматике. Символ $\vdash$ означает <<выводимость>>. Обычно для правил вывода  используют какую-то другую нотацию, чтобы не путать правило вывода и выводимость предложения в теории, но мы не будем различать эти понятия специально.

Как мы уже упоминали, мы начинаем с некоторого набора аксиом, которые являются набором предложений языка логики. Применяя правила вывода, мы получаем из них теоремы. Набор всех выводимых теорем называется теорией. Символом $\Gamma$ мы будем обозначать некоторый произвольный набор предложений (для интуиции это удобно рассматривать как некоторый ограниченный набор теорем, выводимых из наших аксиом). В таблице~1.10 мы привели основные правила вывода, не касающиеся кванторов (правила вывода логики высказываний), теперь же мы готовы сформулировать и их.

\term{Universal instantiation (UI)}: Если $\Gamma \vdash \forall x\ P(x)$, то $\Gamma \vdash P(c)$.

Интуитивно это правило очень понятное: если мы доказали, что что-то выполняется для любого элемента, то для конкретного $c$ оно так же будет выполняться, где $c$ может быть как свободной переменной, так и константой.

\term{Existential generalization (EG)}: Если $\Gamma \vdash P(c)$, то $\Gamma\vdash \exists x P(x)$.

Опять же тут довольно всё просто: если мы вывели предложение $P$ для какого-то конкретного $c$, то среди всех возможных элементов найдётся такой, для которого $P$ истинно.

\term{Existential instantiation (EI)}: Если $\Gamma \vdash \exists x\ P(x)$, то $\Gamma \vdash P(c)$.

Это правило работает в том случае, если $c$ не используется ни в каком из предложений $\Gamma$ и не встречается в $P(x)$. Интуиция здесь так же простая: если мы доказали существование некоторого элемента, для которого выполняется $P$, то мы можем ввести для этого элемента специальную константу $c$, при условии, что мы уже не используем это обозначение для чего-то другого.


\begin{example}
Давайте докажем импликацию $(\forall x\ A(x)) \to (\exists x\ A(x))$. Для этого первоначальный набор аксиом $\Gamma$ будем содержать лишь одно предложение $\forall x\ A(x)$.
\begin{enumerate}
\item $\Gamma\vdash \forall x\ A(x)$~--- дано;
\item $\Gamma \vdash A(c)$~--- universal instantiation;
\item $\Gamma, A(c) \vdash \exists A(x)$~--- existential generalization.
\end{enumerate}
\end{example}

И последнее правило:

\term{Unversal generalization (UG)}: Если $\Gamma \vdash P(x)$, то $\Gamma\vdash \forall x\ P(x)$.

Здесь $x$~--- это переменная, не зависящая от предыдущих шагов вывода, $P$~--- формула со свободной переменной. Правило работает только в том случае, если $\Gamma$ не содержит ни одного предложения, содержащего $c$, P не содержит переменных с именем $x$.

Интуитивно это можно интерпретировать так: если нам удалось вывести предложение $P(c)$, где $c$~--- произвольная константа, то мы точно так же могли бы вывести это предложение для любого элемента той теории, которую мы рассматриваем. Кто изучал геометрию в школе, мог заметить такой довольно общий шаблон доказательства теорем: <<Возьмём произвольный треугольник ABC... бла-бла-бла доказательство бла-бла-бла... значит, любой треугольник обладает свойством бла-бла-бла>>. Эта структура и отражает идею правила universal generalization: если нам удалось доказать что-то для произвольного объекта, пусть как-то специально и поименованного, мы можем доказать это для любого объекта и соответственно можем использовать квантор $\forall$.

Независимость переменной $x$ довольно сложно определить и она требует внимательного к себе отношения. Мы можем рассмотреть пример 1.8 в обратном порядке: применяя к $\exists x \ A(x)$ правило EI получим $A(c)$, а к нем можно применить UG, получая $\forall x\ A(x)$. В итоге имеем следствие
$$(\exists x\ A(x)) \to (\forall x\ A(x))$$
которое очевидно ошибочно. Интуитивно ошибка ясна: $c$ во втором шаге зависела от правила EI предыдущего шага, поэтому нам требуется формальный запрет применять правило UG к константам, которые были введены в рассмотрение правилом EI. В нашем случае мы можем использовать формальный приём, связанный с тем, что после шага EI мы получаем формулу, в которой $c$ является не переменной, а константой, а правило UG работает только для переменных. Этот подход используется далеко не всегда, и есть множество других подходов к определению логики первого порядка, которые при полном формальном построении логики оказываются часто лучше, но для наших нужд такой формальный приём будет достаточен.

\begin{example}
Докажем, что можно ввести правило вывода $$(\forall x\ A(x)\to B(x)), (\forall x\ A(x)) \vdash (\forall x\ B(x))$$
Разобьём доказательство на шаги.
\begin{enumerate}
\item $\forall x\ A(x) \to B(x)$ --- дано;
\item $\forall x\ A(x)$ --- дано;
\item $A(c)$ --- UI;
\item $A(c) \to B(c)$ --- UI;
\item $B(c)$ --- из 3 и 4 по modus ponens;
\item $\forall x\ B(x)$ --- UG;
\end{enumerate}
\end{example}

\begin{exercise}
Докажите, что существует тавтология $$\vdash \forall x\ (x\lor\neg x)$$
\end{exercise}

Приведённые пример и упражнение демонстрируют типичное применение правил UG и UI. Как бы формально ни определялась логика в отношении кванторов, целью этих определений всегда является возможность подобных выводов с одновременным отграничением их от правил EG и EI.

\subsection{Семантика}

Семантика~--- это тот смысл, который мы в логику вкладываем. До сих пор всё что мы говорили являлось лишь операциями с символами на бумаге и не более того. Мы могли бы взять язык крестиков-ноликов и ввести правила вывода типа $xx\vdash00$ и $??\vdash01$ и начать доказывать какие-то теоремы. Однако, эти теоремы хоть и были бы формально корректны с точки зрения правил вывода, смысла бы они никакого не несли~--- это были бы просто кружочки и крестики, которые как-то меняются друг на друга. 

Если мы правила вывода в логике придумываем не просто так, а с каким-то умыслом, то говорят, что мы наделяем их семантикой. Это не строгое определение (строго его и не определить, видимо), но более-менее надеюсь, понятное. Так, мы могли бы разрешить лишь те правила вывода для крестиков-ноликов, которые соответствуют возможным ходам игроков, и если бы мы держали в уме, что это не просто символы на бумаге, а закодированное игровое поле, то это было бы семантическим значением.

Логику высказываний мы уже наделили семантикой, когда сказали в первом параграфе, что каждое высказывание является либо истинным либо ложным и задали таблицы истинности для логических операций. Семантика для логики первого порядка~--- это возможные интерпретации системы аксиом. Обычно на практике они задаются с помощью теории множеств, но тут есть проблема, что сама теория множеств задаётся на языке логики. Все эти препятствия в принципе можно попытаться обойти, но это путь скучный и муторный~--- довольно сложно определить всё формально корректно.

Как альтернативу чуть менее распространённую, но более простую, можно задать семантику через модели, как мы их определили в \S~1.5: модель $M$ теории $T$ это просто набор предложений, такой что если $T\vdash\phi$, то $M\models \phi$, а так же для любого $\psi$ либо $M\models \psi$, $M\models \neg\psi$. Этого упрощённого определения мы и будем придерживаться. Здесь у нас получается, что сама семантика определяется в терминах синтаксиса, что не понравится многим философам, но так оно просто проще и мы будем придерживаться этого подхода.

Мы уже упоминали такое важное свойство полноты некоторых логических систем: если для любой модели $M$ теории $T$ предложение истинно, то оно может быть выведено в теории $T$. Есть так же второе не менее важное свойство: если любой конечный набор предложений теории удовлетворим (то есть обладает моделью), то удовлетворима и вся теория. Это свойство называется компактностью и оно крайне важно для многих теорем логики. Из него в частности следует, что любая непротиворечивая теория обладает моделью. Я не буду вдаваться здесь в детали, лишь скажу, что возможно доказать, что логика первого порядка~--- это самая общая из возможных логик, которая удовлетворяет обоими этими свойствами. Если бы мы использовали логику второго порядка или какую-то другую более выразительную логику, то ни теорема о полноте, ни теорема о компактности уже не работали бы (последнюю, впрочем, можно сделать и полной и компактной, если немного ограничить рассматриваемые семантики, что накладывает дальнейшие ограничения).

По большому счёту все доказательства в математике происходят на уровне синтаксиса и мы не используем ничего больше: мы просто выводим одно из другого. Это соответствует обычной аргументации, которую мы используем в спорах на любую тему. Мы произносим набор фраз, из которых делаем какие-то следствия, прибавляя их к тому, что мы принимаем за истину. Семантика как таковая нами не используется ни в математике ни в повседневной речи, хотя конечно мы какой-то смысл всегда и подразумеваем.

Как только мы начинаем обсуждать не само доказательство, а поднимаемся на тот уровень абстракции, когда пытается что-то утверждать о самой используемой логике~--- здесь уже обычно возникают вопросы, связанные с семантикой. Мы не можем чисто синтаксически доказать, что предложение $\phi$ недоказуемо, и чтобы выяснить это, нам надо строить различные интерпретации. Мы будем редко обращаться к таким рассуждениям, ограничившись в дальнейшем буквально несколькими примерами.

\section{Зачем это всё?}

В популярных книгах мотивационные вступления обычно принято писать в самом начале, чтобы человек понимал, что он и зачем изучает. В книгах по чистой математике это не принято, поскольку никаких практических применений материал почти никогда не находит, и читают математики книги просто для извлечения их них знаний. Само содержание книги~--- уже мотивация.

Тем не менее, какие-то общие слова о том, зачем всё изложенное нужно, я всё же могу сказать.

Я как автор учебника рассматривал для себя аж сразу пять причин, по которым я включил эту главу в курс:

\begin{enumerate}
\item Классическая логика является формальной основой для 99\% современной математики. На практике правила вывода, модели и логические операции математики почти никогда не применяются, но тем не менее, чтобы сформулировать аксиомы теории множеств, логика необходима. Мне хотелось показать читателю как все же строится современная математика, и чтобы иметь эту возможность, я сделал небольшой экскурс в логику.
\item Есть буквально пара мест даже в классических областях институтской математики, где формальная логика встречается (в основном это касается кванторов и законов де Моргана). В институте правда все обычно ограничивается словами «этот значок означает „для любого“». Знать более подробно что же это такое все же полезно.
\item Несмотря на то, что логика вроде бы как-то стоит особняком от остальной математики, есть все же несколько теорем, которые интересны широким слоям населения. Доказательство их в курсе также надеюсь будет приятным занятием.
\item Математическая логика в современном мире является довольно базовым знанием, которое в скором времени скорее всего будут преподавать во всех школах. Простой пример: любая база данных является множеством, а пользователь, задавая вопрос базе данных, на самом деле на специализированном языке формулирует предикат. Эту операцию выполняет часто и продвинутый бухгалтер и юрист, и финансист, только они чаще всего не знают слов типа «подмножество» и «предикат». Понимание логики и теории множеств может здорово упростить им работу, если они захотят более подробно изучать работу с базами данных (что не сложно). Можно привести более сложные примеры с языками программирования, но это уже в общем-то дебри относительно нашего курса, поэтому не будем углубляться. Просто скажем, что все же эти знания пусть и не в чистом виде, но могут оказаться полезны.
\item Из всего, что я могу написать в этой книге, с одной стороны логика в значительной степени абстрактна, с другой же стороны она сравнительно проста и позволит расширять в дальнейшем набор примеров, которые мы будем использовать. Данная глава на мой взгляд является довольно не плохой тренировкой для мозгов.
\end{enumerate}

Если уйти от контекста книги с нашими узкими целями и говорить о более академических направлениях, то основное практическое применение логики (за вычетом философии и оснований математики, о чем я упоминал) --- это попытки разработки систем автоматического доказательства теорем. Ситуация здесь двоякая --- с одной стороны пока не существует никакой адекватной автоматизированной системы автоматических доказательств и работа в этом направлении вроде бы идёт не особо успешно. С другой стороны в качестве частных случаев были примеры компьютерного доказательств сложных теорем именно методами логики, которые человек до этого доказать оказывался не способен. Правда, эти доказательства оказываются небольшой гигантской последовательностью символов, которые формально с точки зрения логики верны, но интуитивно совершенно непонятны. Поэтому такого рода доказательства значительная часть математиков не признает.

Также применения изложенный материал имеет в электронных схемах. Происходящее там уже правда мало походит на логику --- $1$ означает наличие заряда, а $0$ его отсутствие, а логические операции в этом случае превращаются уже в чистую арифметику. Об этом в этом курсе будет параграф в третьей главе.

Какие-то ещё применения логики встречаются в разработках искусственного интеллекта, где окружающий мир описывается набором высказываний, и компьютер знает какие действия как на эти высказывания влияют. При наличии информации об окружающем мире и конечной цели (или недостающей информации), компьютер может самостоятельно построить последовательность действий, необходимую для достижения результата. В ограниченных количествах подобные технологии вроде бы даже применяются где-то в робототехнике, но это скорее единичные случаи, нежели постоянная практика.



\chapter{Множества}
В этой главе очень формально и кратко будут изложены основные понятия множеств. Запомнить и понять сходу всё, что здесь написано, очень сложно, поэтому лучшей стратегией будет поверхностно ознакомиться с содержимым главы, а затем возвращаться к ней по мере необходимости как к справочному материалу. Важными параграфами являются 1, 2 и 4. Остальные можно пропустить, если будут возникать трудности при прочтении.

\section{Основные понятия}

В прошлой главе мы уже упоминали вскользь множества, сейчас же займёмся ими более плотно. Вспомним во-первых то что уже определяли ранее:

\begin{definition}
\term{Множеством} называется неупорядоченный набор различимых объектов, называемых \term{элементами} множества.
\end{definition}

Это определение не является строгим математическим, а скорее аппелирует к интуиции. Строгую аксиоматизацию мы дадим~в~\S2.5, а пока же будем опираться на неформальные рассуждения.

Задать множества можно многими способами, самый простой из которых заключается в перечислении элементов множества. Например, множество всех гласных английского алфавита можно записать как $V = \{a, e, i, o, u\}$. Можно было бы впрочем описать его и просто словами: «$V$ — это множество гласных английского алфавита». Суть от этого не поменялась бы.

Обратим внимание на слова «неупорядоченный» и «различимые», которые фигурируют в определении. Неупорядоченность означает, что $\{a, e, i, o, u\} = \{u, o, a, e, i\}$, то есть нам не важно в каком порядке множество задано, важен лишь его состав. «Различимость» говорит о том, что набор элементов $\{a, a, e, e, i\}$ множеством в математическом смысле уже не является — множество не может по определению содержать совершенно одинаковые объекты, которые мы не можем различить.

В первой главе мы уже сталкивались со многими понятиями, которые являются множествами (могут рассматриваться как множества), и которые могут быть хорошими примерами: множество логических значений $\{0, 1\}$, множество всех формул, множество моделей заданной теории $\mathrm{Mod}(T)$, множество логических функций и так далее. Из более приземлённых примеров можно рассматривать множество всех звёзд на небе, множество учащихся младших классов, множество монет у меня в кармане.

Отдельного упоминания заслуживает следующее тривиальное, но очень важное множество:

\begin{definition}
Множество $\varnothing=\{\}$, не содержащее ни одного элемента, называется \term{пустым множеством}.
\end{definition}

Если $x$ является элементом множества $S$, то это обозначается как $x \in S$. Если не является, то это обозначается как $x \not\in S$.

Пустое множество таким образом можно охарактеризовать следующим логическим утверждением: $\forall x\ x \not \in \varnothing$.

Второе определение, которое было давно в первой главе, это определение подмножества:

\begin{definition}
Множество $A$ называется \term{подмножеством} множества $B$ (обозначение $A\subset B$), если любой элемент из $A$ содержится так же и во множестве $B$.
\end{definition}

На языке логики это определение можно записать так:
$$\forall x\ (x\in A\rightarrow x\in B)$$

Очевидно следующее простое свойство: $\forall X\ \varnothing\subset X$, то есть пустое множество является подмножеством любого другого множества. Это может быть непонятно исходя из интуитивного словесного определения, но идеально вписывается в приведённую логическую формулу для подмножеств: высказывание $x\in\emptyset$ всегда ложно, следовательно, импликация $x\in\varnothing \rightarrow X$, всегда истинна, чем бы не являлось $X$. Это демонстрирует преимущество строгих логических формулировок над интуитивными определениями: определение на языке логики всегда позволяет однозначно ответить на ряд вопросов, которые в противном случае могут даже не иметь смысла. По этой причине (а так же в качестве упражнения для мозга) последующие определения мы будем давать сразу и на языке логики, и на обычном человеческом языке.

Так же полезно вспомнить, что каждое подмножество может быть задано некоторым предикатом, и по каждому подмножеству можно построить предикат, хотя и не однозначно. Множества, заданные предикатами, записываются так: $\{x \in A|P(x)\}$ — так описывается подмножество множества $A$, для элементов которых оказывается истинным предикат $P$.

\begin{definition}
Множества $A$ и $B$ называются \term{равными}, если они состоят из одних и тех же элементов:
$$A=B \leftrightarrow \forall x\ (x\in A \leftrightarrow x\in B)$$
\end{definition}

Кто читал первую главу, знает, что
$$(x\in A \leftrightarrow x\in B) = (x \in A \rightarrow x \in B) \wedge (x \in B \rightarrow x \in A)$$
поэтому можно сформулировать следующее свойство:

\begin{thm}
$A=B$ тогда и только тогда, когда $A\subset B$ и $B \subset A$.
\end{thm}

\begin{definition}
\term{Объединением} множеств $A$ и $B$ ($A\cup B$) называется множество, которое содержит элементы обоих множеств:
$$\forall x\ (x\in A \vee x \in B \leftrightarrow x\in A\cup B)$$
\end{definition}

Понятно, что если какие-то элементы входят одновременно и во множество $A$ и во множество $B$, то в объединение множеств они войдут лишь в единственном экземпляре:

\begin{example}
Пусть $A = \{a, b, c, d\}$ и $B = \{b, c, d, e, f\}$. Тогда
$$A\cup B = \{a, b, c, d, e, f\}$$
\end{example}

\begin{definition}
\term{Пересечением} множеств $A$ и $B$ ($A\cap B$) называется множество, которое содержит лишь те элементы, которые принадлежат сразу обоим множествам:
$$\forall x, (x\in A \wedge x \in B \leftrightarrow x\in A\cap B)$$
\end{definition}

\begin{example}
Пусть $A = \{a, b, c, d\}$ и $B = \{b, c, d, e, f\}$. Тогда $A\cap B = \{b, c, d\}$
\end{example}

\begin{figure}[h]
\centering
\def\seta{(-1,0) circle (2)}
\def\setb{(1,0) circle (2)}
\begin{tikzpicture}
     \tikzstyle{element}=[minimum size=1mm, font=\large]
    \begin{scope}[even odd rule]
        \clip \seta;
        \fill[fill=red] \setb;
    \end{scope}
    \draw \seta;
    \draw \setb;
    \foreach \t/\x/\y in { z/-3/-2, x/3/-2, y/3/2, a/-2/0, b/-.1/1, c/.3/0, d/-.2/-1,
    e/2/.5, f/1.5/-.5 }
        \node[element] (\t) at (\x,\y) {$\t$};
\end{tikzpicture}
\caption{Пересечение двух множеств}
\end{figure}

Операции над множествами можно наглядно изобразить с помощью так называемых \term{кругов Эйлера}. Рисунок~2.1 изображает круги Эйлера для примера, который я только что привёл. Здесь $x$, $y$ и $z$ — это просто некоторые элементы, которые не вошли ни в одно из множеств. Всегда при работе со множествами (да и вообще всегда) важно рассматривать не только объекты, с которыми мы непосредственно работаем, но и внешние факторы. Красным цветом обозначено пересечение двух множеств, представляемых кругами.

\begin{exercise}
Нарисуйте круги Эйлера для объединения множеств (какую часть рисунка надо закрасить цветом?)
\end{exercise}

\begin{definition}
Множества называются \term{непересекающимися}, если $A\cap B = \varnothing$ (то есть множества не имеют общих элементов). В противном случае множества называются пересекающимися.
\end{definition}

\begin{example}
Множества $\{a, b, c\}$ и $\{d, e, f\}$ не пересекаются. Множества $\{a, b\}$ и $\{b, c\}$ пересекаются, и их пересечением является множество $\{b\}$.
\end{example}

\begin{definition}
\term{Разностью} множеств $A\setminus B$ называется множество, содержащее все те элементы $A$, которые не содержатся в $B$:
$$\forall x\ (x \in A\cap B \leftrightarrow x\in A \wedge x \not \in B)$$
\end{definition}

\begin{example}
Пусть $A = \{a, b, c, d\}$ и $B = \{b, c, d, e, f\}$. Тогда $A\setminus B = \{a\}$
\end{example}

\begin{exercise}
Нарисуйте круги Эйлера для разности множеств.
\end{exercise}

\begin{definition}
\term{Симметрической разностью} $A\bigtriangleup B$ называется множество $(A\setminus B)\cup (B\setminus A)$, то есть множество элементов, принадлежащих либо $A$, либо $B$, но не их пересечению:
$$\forall x\ (x\in A\bigtriangleup B \leftrightarrow x\in A \oplus x\in B)$$
\end{definition}

\begin{example}Пусть $A = \{a, b, c, d\}$ и $B = \{b, c, d, e, f\}$. Тогда
$$A\bigtriangleup B = \{a, e, f\}$$
\end{example}

\begin{exercise}
Нарисуйте круги Эйлера для симметрической разности множеств.
\end{exercise}

Часто при работе со множествами, мы держим в уме, что все элементы нашего множества являются элементами некоторого другого более крупного множества, содержащего все возможные объекты рассматриваемой нами задачи. Например, если мы говорим о множестве звёзд на небе, то мы можем держать в уме так же множество всех звёзд вообще, а не только видимых нам. Если мы говорим о множестве учеников десятого класса школы N469, то как более общее множество мы можем подразумевать множество вообще всех учеников этой школы, либо же множество всех школьников страны, либо же множество всех людей. Смотря что за задачу мы решаем. Поэтому оказывается полезным ввести следующее определение:

\begin{definition}
\term{Универсальным множеством}, или \term{универсумом}, называется множество всех возможных элементов, имеющих смысл в решаемой задаче.
\end{definition}

\begin{example}
Если посмотреть на круги Эйлера, приведённые выше для иллюстрации объединения множеств и считать, что на картинке представлены все интересные нам элементы, то там универсумом в этом случае является множество
$$U = \{a, b, c, d, e, f, x, y, z\}$$
\end{example}

\begin{definition}
\term{Дополнением} множества $A$ (обозначается как $A^C$) называется множество элементов универсума, не принадлежащих множеству $A$:
$$\forall x\ x\in A^C \leftrightarrow x\in U \wedge x \not \in A$$
где $U$ — универсум. Это же можно записать и без упоминания универсума, если предположить, что мы держим его «в уме»:
$$\forall x\ x\in A^C \leftrightarrow x\not \in A$$
\end{definition}

Понятно, что для операции дополнения необходимо строгое задание универсума, иначе она теряет смысл.

\begin{example}
Пусть $U = \{a, b, c, d, e, f, x, y, z\}$ и $A = \{a, b, c, d\}$. Тогда $A^C = \{e, f, x, y, z\}$
\end{example}

\begin{exercise}
Нарисуйте круги Эйлера для дополнения.
\end{exercise}

Основные понятия мы определили, теперь надо разобраться с их свойствами. Однако прежде чем мы сформулируем нашу первую теорему о множествах, сделаем такое существенное наблюдение: практически все логические операции и операции над множествами находятся в соответствии друг с другом и операции над множествами определяются просто через логические операции. Так, логическое И задаёт пересечение множеств. Логическое ИЛИ — объединение. Исключающее ИЛИ — симметрическую разность. Отрицание высказываний — дополнение множеств. Эквиваленция — равенство множеств. Импликация — подмножества. В некотором смысле можно так же провести аналогию между универсумом и истинным высказыванием, а так же пустым множеством и ложным высказыванием.

Для продвинутых читателей, которые целиком осилили и поняли первую главу, отмечу, что сильно извратившись определить можно не только операции над множествами через операции над высказываниями, но и наоборот. Пусть, например, у нас есть теория $T_0$ и формулы $\phi$ и $\psi$. Пусть у нас так же есть теория $T_1$, для которой известно, что $\mathrm{Mod}(T_1) = \mathrm{Mod}(T_0, \phi) \cup \mathrm{Mod}(T_0, \psi)$. Тогда можно показать (сделайте это самостоятельно), что теория $T_1 = T_0 \cup \{\phi\vee \psi\}$ будет иметь как раз требуемое множество моделей (хотя такая теория может быть конечно не единственна), и именно через свойства моделей при добавлении формул можно определить логическое ИЛИ. Нечто аналогичное мы делали, когда определяли понятие импликации. Нечто аналогичное можно сделать и для всех других логических операций.

\begin{thm}Для операций над множествами справедливы следующие законы:
\subparagraph{Ассоциативность:}
\begin{enumerate}
\item $(A \cap B) \cap C = A \cap (B \cap C)$
\item $(A \cup B) \cup C = A \cup (B \cup C)$
\item $(A \bigtriangleup B) \bigtriangleup C = A \bigtriangleup (B \bigtriangleup C)$
\end{enumerate}
\subparagraph{Коммутативность:}
\begin{enumerate}
\item $A \cap B = B \cap A$
\item $A \cup B = B \cup A$
\item $A \bigtriangleup B = B \bigtriangleup A$
\item $A = B$ равносильно $B = A$.
\end{enumerate}
\subparagraph{Дистрибутивность:}
\begin{enumerate}
\item $A \cap (B \cup C) = (A \cap B) \cup (A \cap C)$
\item $A \cup (B \cap C) = (A \cup B) \cap (A \cup C)$
\item $A \cap (B \bigtriangleup C) = (A \cap B) \bigtriangleup (A \cap C)$
\end{enumerate}
\subparagraph{Двойное дополнение:}
\begin{enumerate}
\item $(A^C)^C = A$
\end{enumerate}
\subparagraph{Законы де Моргана:}
\begin{enumerate}
\item $(A \cap B)^C = A^C \cup B^C$
\item $(A \cup B)^C =A^C \cap B^C$
\end{enumerate}
\subparagraph{Еще по мелочам (здесь $U$ — универсальное множество):}
\begin{enumerate}
\item $A \cap U = A$
\item $A \cap \varnothing = \varnothing$
\item $A \cup U = U$
\item $A \cup \varnothing = A$
\item $A \bigtriangleup \varnothing = A$
\item $A \bigtriangleup U = A^C$
\item $A \cap A^C = \varnothing$
\item $A \cup A^C = U$
\item $A \bigtriangleup A^C = U$
\item $A\cap A = A$
\item $A\cup A = A$
\item $A \bigtriangleup A = \varnothing$
\item $A \cap (A^C \cup B) = A \cap B$
\item $A \cup (A^C \cap B) = A \cup B$
\end{enumerate}
\subparagraph{Новенькое для отрицания:}
\begin{enumerate}
\item $A \setminus B = A \cap B^C$
\end{enumerate}
\subparagraph{Свойства подмножеств:}
\begin{enumerate}
\item Если $A \not \subset B$, то $A$ и $B^C$ пересекаются.
\item $A \subset A$
\item Транзитивность: $A \subset B \wedge B \subset C \rightarrow A \subset C$ (думаю на всякий случай это свойство полезно проговорить словами: если $A\subset B$ и $B\subset C$, то $A\subset C$)
\item $A \subset B\cap C \rightarrow A \subset B$
\item Если $A \subset B$, то $B^C \subset A^C$ и наоборот.
\end{enumerate}
\end{thm}
\begin{proof}Во-первых, как можно заметить, все эти свойства дублируют соответствующие свойства для логических операций. Такой вот поворот событий. Некоторые свойства пришлось немного переформулировать (в основном в части с импликацией), одно свойство добавилось для разности множеств, несколько свойств потеряли смысл в теории множеств либо стали неинтересны. Но в целом мы имеем то же самое один в один.

Доказательства всех этих свойств оказываются совершенно элементарны и сводятся, как можно догадаться, к простой переформулировке на языке логики. Давайте докажем, например, свойство~10 (мы здесь активно используем задание подмножеств предикатами, которые в нашем случае являются простой логической формулой, а так же дистрибутивность из теоремы~1.1):
\begin{align*}
A \cap (B \bigtriangleup C) &= \{x|x\in A \wedge x \in B\bigtriangleup C\} \\
& = \{x|x\in A \wedge (x \in B \oplus x \in C)\}\\
& = \{x|(x\in A \wedge x \in B) \oplus (x \in A \wedge x \in C)\}\\
& = \{x|(x\in A \cap B) \oplus (x \in A \cap C)\} \\
& = \{x|x\in (A \cap B) \bigtriangleup (A \cap C)\}\\
& = (A \cap B) \bigtriangleup (A \cap C)
\end{align*}

Что и требовалось. Как видно из этих рассуждений, операции над множествами и операции над высказываниями — это действительно очень близкие понятия, которые во многом отражают одно и то же, только под разным углом.

Остальные свойства докажите самостоятельно в качестве упражнения, а так же нарисуйте круги Эйлера для этих свойств — они должны дать довольно не плохую интуицию относительно свойств множеств (и заодно логики).
\end{proof}

В завершение параграфа определим ещё одну операцию, которая уже не имеет никакого прообраза в логике.

\begin{definition}
\term{Булеаном} множества $A$ (обозначается $2^A$) называется множество всех его подмножеств.
\end{definition}

\begin{example}
Пусть $A = \{a, b, c\}$. Тогда
$$2^A = \{\varnothing, \{a\}, \{b\}, \{c\}, \{a, b\}, \{a, c\}, \{b, c\}, A\}$$
\end{example}

Обратите внимание, что всегда $\varnothing \in 2^A$ и $A\in 2^A$. Так же обратите внимание на то, что булеан, сам являясь множеством, содержит в качестве своих элементов другие множества. Это ничему не противоречит — элементами множеств могут быть и другие множества, почему бы и нет?

Важно отметить такой нюанс: если $a \in A$, то $\{a\} \in 2^A$, но $a \not \in 2^A$. Это довольно очевидно: $a\not = \{a\}$, ведь множество состоящее из одного элемента и сам этот элемент логически разные сущности. Это примерно как яблоко и яблоко в пакете являются разными объектами.

Понятие булеана будет активно использоваться нами в дальнейшем, а пока мы рассмотрим опять же аналогию булеана с логикой (это только для дотошных читателей). Пусть $A$ — некоторое одноэлементное множество. Тогда $2^A = \{\varnothing, A\}$.Теперь, если рассматривать наши операции над множествами, ограничившись лишь этим булеаном, то если назвать $\emptyset$ ложью, а $A$ истиной, то наша аналогия между логическими операциями и операциями над множествами станет уже не примерной, а совершенно однозначной.

Таким образом можно считать, что логика — это в некотором смысле частный случай теории множеств, которая в свою очередь является обобщением логики. Если рассматривать $A$, который состоит из многих элементов, то можно говорить, что $2^A$ — это модель нечёткой логики, где $A$ — истинное высказывание, $\varnothing$ — ложное высказывание, а остальные множества являются истинными высказываниями лишь с некоторой степенью вероятности. Это далеко не самый удобный подход для определения нечёткой логики, и на практике математиками не используется наверное никогда, кроме очень узких областей, однако мы будем иногда обращаться к этому примеру в качестве иллюстраций и более интуитивного понимания отдельных понятий.

Так же у дотошного читателя может возникнуть определённый дискомфорт от той последовательности изложения, которую он до сих пор наблюдает: говоря о логике и предикатах мы ввели понятие множеств, говоря о множествах мы во всю опирались на логику. Это как в России: чтобы получить работу по специальности, надо иметь опыт работы по этой специальности, а чтобы получить опыт, надо проработать по этой специальности. Такие ситуации допустимы в быту, но не в науке, поэтому порочные круги необходимо разрывать.

Пока я стараюсь дать просто интуицию о множествах, поскольку строгое формальное изложение без порочных кругов и без начальной интуиции вряд ли окажется сильно полезно и понятно читателю. Поэтому пока мы оставим все как есть, а формальное изложение проведём в конце главы, где уже окончательно расставим все на свои места и избавимся от всех неточностей и нечёткости в определениях.

\section{Отношения}

\begin{definition}
\term{Упорядоченным набором}, или же \term{кортежем}, или же \term{отношением}, называется набор объектов, в котором в котором порядок следования элементов имеет значение.
\end{definition}

Для обозначения упорядоченного набора мы будем использовать круглые скобки: $(a, b, \ldots, x)$. Первое отличие упорядоченного набора от множества заключается в упорядоченности элементов, которая отсутствует для множеств. Так в случае множеств $\{a, b\} = \{b, a\}$, но для упорядоченных наборов $(a, b)\not=(b, a)$. Так же в упорядоченных наборах могут встречаться повторения, а в множествах нет. $(a, a)$ является упорядоченным набором, а $\{a, a\}$ множеством не является. Пока отнеситесь к этому просто как к данности, позже будет понятно какой смысл это имеет в приложениях.

Как более-менее практический пример можно рассмотреть базы данных. Чтобы не говорить слишком абстрактно, а рассмотреть более конкретную ситуацию, можно рассмотреть простую записную книжку мобильного телефона, которая тоже по сути является компьютерной базой.

Сама база данных представляет из себя множество записей, каждая из которых состоит из нескольких полей. Пример базы данных представлен в таблице 2.1.

\begin{table}[h]
\centering
\begin{tabular}{lllll}
Фамилия & Имя & Телефон & ДР & Комментарий\\
Авраам & Линкольн & +1(270)... & 12.02.1802 & Президент\\
Бенито & Муссолини & +3(39)... & 29.07.1883 & Фашик\\
Владимир & Ленин & +7(499)... & 22.04.1870 & Свой мужик\\
Григорий & Распутин & +7(3459)... & 21.01.1869 & Есть чему завидовать\\
Джорджина & Байер & +6(64)... & 11.1957 & Транссексуал\\
... & ... & ... & ... & ...
\end{tabular}
\caption{Пример записей в базе данных}
\end{table}

Если предположить, что $F$ — множество фамилий, $N$ — множество имён, $P$ — множество телефонов, $D$ — множество дат календаря и $S$ — множество произвольных текстовых строк, то можно сказать, что данная таблица иллюстрирует множество упорядоченных наборов следующего вида:

$\{(f, n, p, b, c)|f\in F, n \in N, p \in P, b \in D, c \in S\}$

(Строго говоря с точки зрения логики правильнее было бы писать не символ запятой справа в предикате, а символ $\wedge$, но чаще в случаях, подобных нашему, употребляется именно запятая в силу большей наглядности и удобства — означает она при этом логическое И).

Практически вся теория баз данных построена на самом деле на теории множеств в рамках того, что я уже рассказал, и декартова произведения, о котором речь пойдёт ниже. База данных — это множество, элементами которого являются упорядоченные наборы. Команды работы с базой данных — это операции, задающие предикаты и операции над множествами. Я не могу позволить себе углубляться сейчас в алгебру реляционных баз данных и синтаксис языка SQL, но вообще рекомендую всем читателям с этими вещами ознакомиться — будет небесполезно, а заодно увидите кучу примеров тому, о чем я сейчас говорю.

Здесь стоит ещё сделать такое наблюдение по терминологии. Чаще всего термин <<кортеж>> применяется именно в теории баз данных, термин <<отношение>> применяется в случае упорядоченных пар, в которых оба элемента принадлежат одному и тому же множеству $\{(x, y)|x\in A, y\in A\}$, а термин <<упорядоченный набор>> используется в остальных случаях. Здесь нет строгого формального разграничения, это просто сложившаяся практика, которая однако может часто нарушаться и в этом нет ничего преступного.

\begin{definition}
\term{Декартовым}, или \term{прямым}, \term{произведением} множеств $A$ и $B$ называется множество $A\times B = \{(a, b)|a\in A, b\in B\}$.
\end{definition}

Говоря человеческим языком, $A\times B$ — это множество всех возможных упорядоченных пар $(a, b)$, таких что $a\in A$ и $b \in B$.

\begin{example}
Пусть $A = \{0, 1\}$, а $B = \{a, b, c\}$. Тогда
$$A\times B = \{(0, a), (0, b), (0, c), (1, a), (1, b), (1, c)\}$$
\end{example}

Кое-какую интуицию относительно декартова произведения вероятно поможет развить иллюстрация в виде таблицы. Столбцы в ней отводятся для элементов одного множества, строки — для другого множества, а на пересечении стоит их декартово произведение. Таблица 2.2 соответствует примеру 2.9.

\begin{table}[h]
\centering
\begin{tabular}{c|ccc}
$\times$ & a&b&c\\
\hline
0 & (0,a) & (0, b) & (0, c) \\
1 & (1, a)& (1, b) &(1, c)
\end{tabular}
\caption{Декартово произведение для примера 2.9}
\end{table}

\begin{example}
Часто с помощью декартова произведения можно описывать какие-то физические понятия. Пусть $A$ — множество мастей (червы, бубны, крести, трефы), а $B$ — множество достоинств (6, 7, 8, ..., король, туз). Тогда $A \times B$ — множество игральных карт в колоде.
\end{example}

\begin{exercise}
Покажите, что в общем случае $A\times B \not = B \times A$. В каких частных случаях все же $A\times B = B \times A$?
\end{exercise}

Легко заметить, что $(A\times B)\times C \not= A\times (B \times C)$, поскольку в случае множества до знака равенства будут получаться пары вида $((a, b), c)$, а для множества справа от знака равенства будут пары $(a, (b, c))$. Однако как и раньше удобно считать, что произведение $A\times B\times C$ без скобок даёт нам упорядоченные тройки $(a, b, c)$ — также без каких-либо специальных группировок. Для удобства часто, впрочем, считают, что и при наличии скобок декартово произведение даёт обычные упорядоченные наборы: 
$$(A\times B)\times C = \{(a, b, c)|a\in A, b\in B, c\in C\}$$

Каждое отношение таким образом является подмножеством декартова произведения $\rho \subset X\times X$ — в этом параграфе нас будут интересовать как раз такие отношения. Часто для удобства применяется следующая запись: если $(x, y) \in \rho$, то пишут $x\rho y$.

\begin{example}
Пусть $S$ — множество высказываний. Тогда любая логическая операция задаёт отношение на этом множестве. В простейшем случае, если рассматривать только одно истинное и ложное высказывание $S = \{0, 1\}$, то $\vee = \{(1, 0), (0, 1), (1, 1)\}$, $\leftrightarrow = \{(0, 0), (1, 1)\}$ и так далее. По аналогии любому предикату на множествах можно сопоставить в соответствие некоторое отношение (возможно, не из двух элементов).
\end{example}

\begin{definition}
Отношение называется \term{рефлексивным}, если для любого $x$ выполняется $x\rho x$.
\end{definition}

\begin{definition}
Отношение называется \term{антирефлексивным}, если для любого $x$ отношение $x\rho x$ не имеет места быть.
\end{definition}

\begin{definition}
Отношение называется \term{транзитивным}, если для любых $x$, $y$ и $z$ из того, что $x\rho y$ и $y \rho z$ следует, что $x\rho z$.
\end{definition}

\begin{definition}
Отношение называется \term{симметричным}, если из того, что $x\rho y$ следует $y\rho x$.
\end{definition}

\begin{definition}
Отношение называется \term{антисимметричным}, если из того, что $x \rho y$ и $y \rho x$ следует, что $x = y$.
\end{definition}

Можно привести и больше возможных общих свойств отношений, но для наших нужд достаточно и того что я уже привёл.

\begin{exercise}
Запишите эти свойства на языке логики.
\end{exercise}

\begin{exercise}
Приведите пример нетранзитивного отношения на множестве $\{0, 1\}$.
\end{exercise}

\begin{definition}
Отношение называется \term{отношением эквивалентности}, если оно рефлексивно, транзитивно и симметрично.
\end{definition}

Отношение эквивалентности часто обозначается символами $=$, $\approx$, $\sim$ и подобными (хотя есть и другие обозначения, которые мы будем по ситуации использовать). Если на множестве задано несколько отношений эквивалентности, то символьное обозначение отношения часто указывается справа внизу от знака эквивалентности: $\approx_R$.

\begin{exercise}
Пусть $S$ — множество учеников средней школы №469. Отношение $x\approx_Y y$ означает, что ученики $x$ и $y$ заканчивают школу в одном году, отношение $x \approx_C y$ означает, что они учатся в одном классе, $x \sim y$ что имеют одинаковую успеваемость. Проверьте, что все заданные отношения — это отношения эквивалентности (для этого необходимо проверить, что они рефлексивны, транзитивны и симметричны).
\end{exercise}

Следующие три упражнения уже посложнее, и их начинающие могут пропустить.

\begin{exercise}
Пусть $S$ — некоторое множество, а $P$ — множество предикатов, определённых на этом множестве. Пусть нам задано такое отношение, что $p \approx q$ тогда и только тогда, когда $p$ и $q$ определяют одинаковые подмножества $S$. Докажите, что это отношение эквивалентности.
\end{exercise}

\begin{exercise}
Пусть нам задано множество логических формул, все с одинаковым числом параметров. Отношение $f=g$ определено для формул, которые на одинаковых значениях параметров принимают одинаковые значения истинности. Докажите, что это отношение эквивалентности.
\end{exercise}

\begin{exercise}
Пусть нам задана некоторая теория. Докажите, что семантическая эквивалентность формул является отношением эквивалентности. (Напомню, что формулы $\phi$ и $\psi$ семантически эквивалентны, если в любой модели теории $\psi\leftrightarrow\phi$).
\end{exercise}

Примеры, приведённые в упражнении, демонстрируют основную идею отношения эквивалентности: это набор пар, которые имеют какие-то общие характеристики, которые и отражают отношение. Причём мыслить об отношении как о парах элементов чаще всего неудобно с интуитивной точки зрения — удобнее мыслить об отношении эквивалентности именно как о наличии какого-то общего свойства.

\begin{definition}
Пусть дано множество $A$ и на нем задано отношение эквивалентности $\sim$. Множество всех элементов $x$, для которых $a\sim x$ называется \term{классом эквивалентности} элемента $a$ и обозначается как $[a]$.
\end{definition}

\begin{thm}
Для любых элементов $a, b \in A$ классы эквивалентности $[a]$ и $[b]$ либо совпадают, либо не пересекаются.
\end{thm}
\begin{proof}
Проведём доказательство от противного и предположим, что теорема неверна. Пусть классы эквивалентности $[a]$ и $[b]$ пересекаются, но не совпадают. Тогда найдётся элемент $x$, принадлежащий сразу обоим классам эквивалентности, и элемент $b' \in [b]$, такой что $b' \not\in [a]$. Для него однако верно, что $b' \sim x$, но поскольку $x \in [a]$ и $x\sim a$, то в силу транзитивности $b' \sim a$ и соответственно $b' \in [a]$. Полученное противоречие говорит, что наше предположение «от противного» было не верно, и значит теорема верна.
\end{proof}

Это доказательство на первый взгляд может показаться странным и непонятным — вспомните тогда свойства импликации и как она связана с выводимостью (см.~\S\S1.5,~1.7):
$$(a\rightarrow b) \leftrightarrow (\neg b \rightarrow \neg a)$$

Смыслом теоремы является тот факт, что каждое множество с заданным на нем отношением эквивалентности можно разбить на непересекающиеся классы эквивалентности.

\begin{example}
Пусть опять $S$ — множество учеников школы №469. Всех учеников можно сгруппировать по классам, по успеваемости или по году окончания школы. Пусть $A$ — множество двоечников, $B$ — троечников, $C$ — хорошистов и $D$ — отличников. Очевидно, что эти множества не пересекаются и $S = A\cup B\cup C\cup D$, причём любые два ученика из одного и того же множества находятся в отношении $\sim$ («имеют одинаковую успеваемость») друг с другом, а ученики из разных множеств в этом отношении не находятся.
\end{example}

\begin{definition}
Процесс разбиения множества на классы эквивалентности называется \term{факторизацией}, а само множество классов эквивалентности называется \term{фактор-множеством}. Если $\rho$ — отношение эквивалентности на $A$, то фактормножество $A$ по $\rho$ обозначается как $A/\rho$.
\end{definition}

\begin{example}
Продолжая пример с успеваемостью, можно записать, что $S/\sim = \{A, B, C, D\}$.
\end{example}

Обратите внимание в какую сторону рисуется черта фактормножества и не путайте её с разностью множеств. $A\setminus B$ — разность, а $A/\rho$ — факторизация.

Интуитивно таким образом можно рассматривать факторизацию как разбиение множества на подмножества в соответствии с некоторым отношением («фактором»). Обратное кстати тоже верно — если множество разбито на непересекающиеся подмножества, то по ним можно задать отношение эквивалентности.

\begin{exercise}
Пусть $A = \{a, b, c, d\}$ разбито на подмножества $\{a, b\}$ и $\{c, d\}$. Запишите по этому разбиению отношение эквивалентности $\rho$ как множество упорядоченных пар (начинается эта запись как $\rho = \{(a, a), (a, b), \ldots\}$).
\end{exercise}

Перейдём теперь ко второму не менее важному типу отношений, которые мы часто будем использовать.

\begin{definition}
Отношение называется \term{отношением частичного порядка} (или \term{отношением нестрогого частичного порядка}), если оно рефлексивно, транзитивно и антисимметрично.
\end{definition}

\begin{definition}
Отношение называется \term{отношением строгого частичного порядка}, если оно антирефлексивно, транзитивно и антисимметрично.
\end{definition}

Отношение частичного порядка удобно обозначать как $\le$. Отношение строгого частичного порядка — как $<$. Если $a \le b$ и $a \not= b$, то $a<b$. Если $b \le a$, то можно писать $a \ge b$ или $a > b$, если $a\not= b$. Легко видеть, что отношения частичного и строгого частичного порядка — это фактически одного и то же. Различаются они лишь тем, находится ли каждый элемент в отношении с самим собой (то есть можем ли мы записать отношение $x\le x$). В теории в основном используется отношение частичного порядка, в конкретных задачах часто может оказаться удобным и то и другое. Хотя погоды это разграничение не делает: если у нас есть строгий порядок, то его всегда можно сделать нестрогим, дополнив парами $x\le x$, и наоборот нестрогий порядок можно сделать строгим, выкинув такие пары.

\begin{exercise}
Докажите, что если элементами множества являются множества (например, это булеан), то отношение $\subset$ является отношением частичного порядка.
\end{exercise}

\begin{exercise}
Докажите, что отношение «является предком» на множестве людей является строгим частичным порядком.
\end{exercise}

\begin{exercise}
Докажите, что отношение «не ниже чем» на множестве всех людей образует нестрогий частичный порядок, а отношение «выше чем» строгий частичный порядок.
\end{exercise}

\begin{exercise}
Покажите, что отношение «является родителем» не является отношением частичного порядка.
\end{exercise}

\begin{exercise}
Покажите, что на множестве картонных коробок отношение «влезает в» является отношением строгого частичного порядка.
\end{exercise}

\begin{exercise}
Рассмотрим множество всех логических функций с одинаковым числом параметров. Определим на них отношение следующим образом: $f\le g$ тогда и только тогда, когда на одинаковых наборах параметров либо эти функции принимают одинаковое значение, либо $f=0$, а $g=1$. Докажите, что это отношение частичного порядка. Как оно соотносится с порядком на булеане множества?
\end{exercise}

\begin{example}
В микроэкономике рассматриваются отношения предпочтений, задаваемые на множествах альтернатив (читай товаров). Считается, что потребители могут сравнивать альтернативы по степени предпочтительности и данное отношение является как раз отношением частичного порядка. (Психологи однако доказывают, что предпочтения людей не являются рациональными и таким образом не образуют частичный порядок на множестве товаров).
\end{example}

Отношение частичного порядка удобно представлять в виде диаграмм. Например, так будет выглядеть диаграмма для булеана множества $\{x, y, z\}$:

\includegraphics{hasse1.png}

(Картинка из Википедии, свободной энциклопедии, потому что сам не умею пока такие рисовать; LibreOffice Draw не справляется с задачей).

Здесь стрелка обозначает отношение «является подмножеством». Элементы, располагающиеся выше, оказываются «больше» элементов, располагающихся внизу, если между ними есть путь из стрелок. Если на диаграмме есть стрелка $a\to b$ и стрелка $b\to c$, то по транзитивности должна быть и стрелка $a\to c$, которая не указывается, чтобы не загромождать картинку.

Отношение частичного порядка таким образом упорядочивает элементы множества. Существенно, однако, что если задан частичный порядок, то совершенно не факт, что удастся сравнить произвольные два объекта. Как видно из диаграммы выше, элементы $\{x, y\}$ и $\{z\}$ несравнимы между собой. Точно так же для любых двух людей не обязательно кто-то один из них должен быть предком другого, и в этом смысле несравнимыми являются брат и сестра.

{\bfseries Определение.} {\slshape Линейным порядком} называется отношение частичного порядка, относительно которого сравнимы любые два элемента множества.

{\bfseries Пример.} Отношение «не ниже чем» является линейным порядком на множестве людей.

{\bfseries Определение.} {\slshape Максимальным элементом} называется такой элемент, что не существует элемента, большего него.

{\bfseries Определение.} {\slshape Наибольшим элементом} называется такой элемент, что он больше любого другого элемента.

{\bfseries Пример.} На множестве коробок максимальным элементом будет коробка, которая не может влезть ни в какую другую коробку. Наибольшим элементом будет коробка, в которую может поместиться любая другая коробка.

В полной аналогии можно ввести понятия {\slshape минимального} и {\slshape наименьшего элемента}, но мы не будем на это отвлекаться, так там вся теория совершенно аналогична.

Очевидно, что если множество обладает наибольшим элементом, то он же будет и единственным максимальным элементом. Однако же максимальных элементов может быть несколько, и наибольшего элемента в этом случае не будет, как в случае следующей диаграммы:

\begin{picture}(120,60)
\put(0,0){a}
\put(40,0){b}
\put(20,50){e}
\put(22,45){\line(-1,-2){18}}
\put(22,45){\line(1,-2){18}}
\put(80,0){c}
\put(120,0){d}
\put(100,50){f}
\put(102,45){\line(-1,-2){18}}
\put(102,45){\line(1,-2){18}}
\end{picture}

На приведённой диаграмме два максимальных элемента: $e$ и $f$. Наибольшего элемента нет. На следующей же диаграмме уже имеется один наибольший элемент $g$, он же является и максимальным:

\begin{picture}(120,100)
\put(0,0){a}
\put(40,0){b}
\put(20,50){e}
\put(22,45){\line(-1,-2){18}}
\put(22,45){\line(1,-2){18}}
\put(80,0){c}
\put(120,0){d}
\put(100,50){f}
\put(102,45){\line(-1,-2){18}}
\put(102,45){\line(1,-2){18}}
\put(60,90){g}
\put(62,85){\line(-1,-1){40}}
\put(62,85){\line(1,-1){40}}
\end{picture}

В обоих этих примерах порядки не являлись линейными. В случае же линейного порядка, очевидно, что понятия максимального и наибольшего элемента совпадают.

{\bfseries Определение.} {\slshape Верхней гранью} подмножества $S\subset U$ называется такой элемент $x\in U$, что для любого элемента $s\in S$ выполняется $s \le x$.

{\bfseries Определение.} {\slshape Нижней гранью} подмножества $S\subset U$ называется такой элемент $x\in U$, что для любого элемента $s\in S$ выполняется $s \ge x$.

Если $x$ — верхняя грань подмножества $S$, то любой элемент больший $x$ так же будет являться верхней гранью $S$. Таким образом, верхние грани сами по себе образуют множество (аналогично с нижними) и есть смысл ввести следующие определения:

{\bfseries Определение.} {\slshape Наименьшей}, или {\slshape точной}, {\slshape верхней гранью}, называется наименьшая из верхних граней. Обозначается она как $\sup S$.

{\bfseries Определение.} {\slshape Наибольшей}, или {\slshape точной}, {\slshape нижней гранью}, называется наибольшая из нижних граней. Обозначается она как $\inf S$.

(С этого места начинаются абстракции, которые начинающему читателю вероятно и не являются необходимыми — если будет непонятно о чем речь, можно смело пропускать. Если однако разобраться с материалом, то это будет хорошим упражнением для ума.)

Если подмножество $S$ состоит лишь из двух элементов, то операции «точных граней» можно рассматривать как арифметические операции над этими элементами. В этом случае используется запись $\sup\{a, b\} = a\vee b$ и $\inf\{a, b\} = a\wedge b$.

{\bfseries Упражнение.} Докажите, что на булеане относительно порядка, образованного включением множеств, $A\wedge B = A\cap B$ и $A\vee B = A \cup B$ и они обладают соответственно всеми свойствами логического И и логического ИЛИ.

Сформулированное в упражнении утверждение в общем случае неверно, если рассматривать произвольный частичный порядок. Во-первых, могут не выполняться сами свойства логических операций, а во-вторых самих точных граней может и не быть. Поэтому вводятся следующие определения:

{\bfseries Определение.} {\slshape Решёткой} называется частично упорядоченное множество, у которого для любого двухэлементного подмножества существуют точные верхние и нижние грани.

{\bfseries Определение.} {\slshape Дистрибутивной решёткой} называется решётка, для которой оказываются верны свойства дистрибутивности операций $\wedge$ и $\vee$ (см §1.1, теорема 1).

{\bfseries Упражнение.} Докажите, что решётка, изображённая на диаграмме ниже, не является дистрибутивной:

\includegraphics{lattice.png}

Дистрибутивные решётки уже практически дублируют законы логики, которым была посвящена первая глава. Недостаёт только $1$ и $0$. Следующие определения исправляют ситуацию:

{\bfseries Определение.} Решётка называется ограниченной, если в ней существуют наибольший и наименьший элементы, обозначаемые соответственно как $0$ и $1$.

{\bfseries Упражнение.} Если решётка имеет минимальный или максимальный элементы, то они будут единственны и будут являться наименьшим и наибольшим элементами. Докажите это.

{\bfseries Определение.} {\slshape Дополнением} элемента $x$ ограниченной решётки такой элемент $y$, что $x\wedge y = 0$ и $x \vee y = 1$. Дополнение $x$ обозначается как $\neg x$.

{\bfseries Определение.} Если в решётке каждый элемент обладает дополнением, то такая решётка называется {\slshape решёткой с дополнением}.

{\bfseries Определение.} {\slshape Булевой алгеброй} называется дистрибутивная решётка, обладающая наименьшим и наибольшим элементами (обозначаемыми соответственно $0$ и $1$).

{\bfseries Пример.} Если рассмотреть двухэлементное множество $\{0, 1\}$ и определить на нем порядок $0 < 1$, то мы получим в точности классическую логику, которую рассматривали в первой главе, которая как теперь видно является разновидностью булевой алгебры.

{\bfseries Пример.} Булеан любого множества $A$ задаёт булеву алгебру относительно включения множеств, где наибольшим элементом является само множество $A$, а наименьшим пустое множество.

{\bfseries Упражнение.} Для булевых алгебр справедливы все законы логики, которые мы приводили в теореме 1.1. Докажите это.

Если у нас есть два частично упорядоченных множества $A$ и $B$, то на их декартовом произведении можно задать частичный порядок несколькими способами:

{\bfseries Определение.} {\slshape Лексикографическим порядком} на $A\times B$ называется порядок, при котором $(a, b) \le (a', b')$ либо когда $a < a'$, либо когда $a=a'$ и $b\le b'$.

{\bfseries Определение.} {\slshape Естественным порядком} на $A\times B$ называется порядок, при котором $(a, b) \le (a', b')$ тогда и только тогда, когда одновременно $a \le a'$ и $b \le b'$.

Оба этих порядка естественно распространяются на декартово произведение любого количества множеств.

{\bfseries Упражнение.} Пусть $B = \{0, 1\}$ и на нем введён порядок $0 < 1$. Будем рассматривать множество $B\times B\times \ldots \times B$ (фактически упорядоченные наборы единиц и нулей) с естественным частичным порядком. Докажите, что такой частичный порядок задаёт булеву алгебру (кстати, как раз ту самую, которая повсеместно используется в программировании и называется там побитовыми логическими операциями).

{\bfseries Упражнение.} Докажите, что линейно-упорядоченное множество является дистрибутивной решёткой, и соответственно при наличии наименьшего и наибольшего элемента является булевой алгеброй.

{\bfseries Упражнение.} Является ли булевой алгеброй множество $B\times B \times \ldots \times B$, введённое выше, относительно лексикографического порядка?

{\bfseries Упражнение.} Пусть теперь $B$ — некоторая произвольная булева алгебра. Является ли булевой алгеброй $B\times B \times \ldots \times B$ относительно лексикографического порядка? А относительно естественного порядка?

\section{Графы}

Этот параграф не содержит каких-то очень важных для нашего курса сведений — он призван служить демонстрацией отдельных понятий, которые мы вводили. Всё, о чем мы говорили до сих пор, может показаться слишком абстрактным и непрактичным на первый взгляд, и в этом параграфе я покажу, что на самом деле даже то, что я изложил до сих пор, может оказаться вполне полезным, если грамотно это применить. Материал, связанный со второй задачей и многомерной геометрией на самом деле очень жёсткий (и одновременно с тем неформальный), так что при первом знакомстве его можно пропустить.

Если говорить неформально, то \term{граф} — это набор точек, называемых \term{вершинами}, некоторые из которых соединены линиями, называемыми \term{рёбрами}. Графы бывают многих разновидностей, но нас пока будут интересовать \term{простые графы} и \term{ориентированные графы} (кратко \term{орграфы}). В орграфах каждое ребро изображается в виде стрелки, которая имеет начало и конец (такое ребро называется \term{дугой}), в простых же графах начало и конец ребра не различимы.

Мы уже встречали графы в прошлом параграфе: диаграммы частично упорядоченных множеств как раз представляют собой орграфы. Графы и орграфы часто используются для описания многих понятий нашей жизни: карта дорог между городами является графом, где дороги — это ребра графа, а города — вершины. Графом является схема компьютерной сети: каждый компьютер в таком графе является вершиной, а кабель, соединяющий их в сеть, является ребром. Ориентированные графы применяются всякими там менеджерами для планирования всяких там задач: у них вершинами графов являются задачи, которые необходимо исполнить, а дугами показывается зависимость между задачами (если из задачи $A$ в задачу $B$ идёт стрелочка, то выполнение задачи $B$ не может быть начато до окончания выполнения задачи $A$). Инженеры используют разновидности графов для представления электрических цепей. Некоторые из этих задач мы будем рассматривать в дальнейшем (в основном в виде простых самостоятельных упражнений).

\begin{figure}[h]
\centering
\tikzstyle{blackdot}=[circle,fill, inner sep=1.5pt]
\begin{tikzpicture}

\node [blackdot, label=3] (a3) at (-3,1) {};
\node [blackdot, label=below:2] (a2) at (-3,-1) {};
\node [blackdot, label=below:1] (a1) at (-5,-1) {};
\node [blackdot, label=4] (a4) at (-5,1) {};

\draw  (a4) edge (a1);
\draw  (a3) edge (a2);
\draw  (a4) edge (a3);
\draw  (a1) edge (a2);

\node [blackdot, label=3] (b3) at (-1,1) {};
\node [blackdot, label=below:2] (b2) at (1,-1) {};
\node [blackdot, label=4] (b4) at (1,1) {};
\node [blackdot, label=below:1] (b1) at (-1,-1) {};

\draw  (b3) edge (b4);
\draw  (b3) edge (b2);
\draw  (b1) edge (b4);
\draw  (b1) edge (b2);

\node [blackdot, label=left:1] (c1) at (3,1) {};
\node [blackdot, label=right:2] (c2) at (5,2) {};
\node [blackdot, label=right:4] (c4) at (6,0) {};
\node [blackdot, label=right:3] (c3) at (5,-2) {};
\draw  (c1) edge (c2);
\draw  plot[smooth, tension=.7] coordinates {(c1)};
\draw (3,1);
\draw (3,1);
\draw (3,1);
\draw (3,1) .. controls (3,0) and (6,0) .. (6,0);
\draw (6,0) .. controls (4,-1) and (4,-2) .. (5,-2);
\draw (5,2) .. controls (3,-3) and (5,-2) .. (5,-2);
\end{tikzpicture}
\caption{Изоморфные графы.}
\end{figure}

Важно ответить, что граф — это именно набор вершин и рёбер, но не их представление на бумаге. Если изменить форму рёбер или положение вершин, граф от этого не изменится, хотя визуально может выглядеть и сильно по-другому. Три графа, изображённые на рисунке~2.6 одинаковы, хотя и выглядят по-разному. На языке математики говорят, что эти графы \term{изоморфны} друг другу.

Что такое граф интуитивно, думаю, понятно. Давайте теперь все формализуем и определим графы через теорию множеств.

\begin{definition}
\term{Графом} называется упорядоченная пара $(V, E)$, где $V$ — множество, элементы которого называются вершинами графа, а $E$ — симметричное антирефлексивное отношение на $V$, элементы которого называются рёбрами графа.
\end{definition}

\begin{definition}
Если ослабить условия прошлого определения и разрешить, чтобы отношение $E$ не было симметричным, то такая конструкция будет называться \term{ориентированным графом}. Сами элементы множества $E$ называются в этом случае \term{дугами}.
\end{definition}

Мы будем говорить преимущественно о неориентируемых графах. В общем случае антирефлексивность не требуется, но мы для простоты будем исключать из рассмотрения ребра вида $(a, a)$, чтобы не отвлекаться на различные частные случаи.

\begin{definition}
Eсли $(v, w) \in E$, то вершины $v$ и $w$ называются \term{инцидентными}. Так же говорят, что $v$ и $w$ инцидентны ребру $(v, w)$, а ребро соответственно инцидентно его вершинам.
\end{definition}

\begin{exercise}
	Запишите отношения, соответствующие графам, изображённым выше, и убедитесь в том, что они действительно изоморфны.
\end{exercise}

Перейдём сразу к задачам. Первая из них простая и вполне себе классическая.

\begin{problem}
Есть три дома и три колодца. Надо от каждого колодца к каждому дому провести тропинку. Возможно ли сделать это так, чтобы эти тропинки не пересекались?
\end{problem}

Вот пример запоротого решения:

\includegraphics{bells.jpg}

Попробуйте вначале решить задачу самостоятельно, а потом читайте дальше.

{\bfseries Определение.} Граф, который возможно изобразить на плоскости без пересечения рёбер, называется {\slshape планарным}.

Очевидно, что это ровно то, что нам требуется: надо проверить, является ли граф для домиков и колодцев планарным. Этот граф имеет специальное обозначение $K_{3, 3}$ и в простейшем виде, если не заморачиваться о пересечениях, может быть изображён так:

\includegraphics{k331.png}

Можно ли его нарисовать без пересечения линий? Для начала запишем его ребра в виде множества: $E = \{a1, a2, a3, b1, b2, b3, c1, c2, c3, \ldots\}$ Для удобства мы не стали перечислять симметричные отношения (понятно, что если $a1\in E$, то и $1a \in E$ и лишний раз упоминать это скучно) и ребра $(v, w)$ кратко записывали как $vw$.

{\bfseries Определение.} Граф $G' = (V', E')$ называется {\slshape подграфом} графа $G = (V, E)$, если $V' \subset V$ и $E' \subset E$.

Рассмотрим такое подмножество рёбер: $C = \{a1, 1b, b2, 2c, c3, 3a\}$. Эти ребра образуют такой подграф:

\includegraphics{cycle.png}

Подобные последовательности имеют специальные названия. Просто упомянем их для порядку:

{\bfseries Определение.} Последовательность чередующихся вершин и рёбер, в которой каждое ребро стоит между различными вершинами, которым оно инцидентно, называется {\slshape путём}.

В нашем случае путь записывается как последовательность $a, a1, 1, 1b, b, b2, 2, 2c, c, c3, 3, 3a, a$.

{\bfseries Определение.} Если в пути нет повторяющихся рёбер, то он называется {\slshape цепью}.

{\bfseries Определение.} Если первая и последняя вершины цепи совпадают, то такой путь называется {\slshape замкнутой цепью}, либо {\slshape циклом}. Если не совпадают, то цепь {\slshape незамкнута}.

Приведённый нами подграф $C$ очевидно является циклом и иллюстрация даёт понять откуда берётся такое называние. И хочешь не хочешь, но как этот цикл не рисуй он всегда либо будет иметь самопересечения, либо будет разбивать плоскость на две части: внутреннюю и внешнюю, и в этом смысле наш цикл имеет простейший вид какой только возможно (По хорошему это тоже надо строго доказывать да и вообще определять что значит «разбивает плоскость», но мы это сделаем намного позже, когда будем говорить о геометрии, а пока что просто доверимся интуиции).

Множество рёбер нашего первоначального графа $E = C \cup \{a2, b3, c1\}$ и нам необходимо дорисовать к подграфу $C$ три ребра, чтобы получить $E$. Здесь можно просто рассмотреть все возможные варианты. Если нарисовать $a2$ внутри цикла, то $b3$ и $c 1$ должны быть снаружи цикла, иначе они будут пересекать ребро $a2$. Но оба ребра $b3$ и $c 1$ тоже не могут быть одновременно снаружи, так как в этом случае они так же будут пересекаться. Значит, $a2$ не может быть нарисован внутри цикла так, чтобы граф $K_{3, 3}$ не имел пересечений. Однако если нарисовать $a2$ снаружи цикла, то так же можно увидеть, что $b3$ и $c 1$ неминуемо пересекутся внутри цикла. Таким образом мы перебрали все возможные варианты и убедились:

{\bfseries Лемма.} Граф $K_{3, 3}$ не является планарным.

Это ровно то что требовалось выяснить в условиях задачи.

Термин «лемма» обычно (но не всегда) используется для обозначения промежуточных результатов, которые будут задействованы далее в каких-то более сильных утверждениях. Самостоятельно предлагаю доказать вам ещё следующую лемму:

{\bfseries Лемма.} Граф $K_5$ имеющий вершины $\{a, b, c, d, e\}$, каждая пара которых инцидентна, не планарен.

В простом виде этот граф выглядит так:

\includegraphics{k5.png}

Его не планарность доказывается так же, как и в случае с графом $K_{3, 3}$.

Из двух лемм, сформулированных выше, можно получить такую теорему:

{\bfseries Теорема.} Граф, который содержит в качестве подграфов $K_{3, 3}$ или $K_5$ не планарен.

Оказывается, что наличие подграфов, которые структурно схожи с $K_{3, 3}$ или $K_5$ является и необходимым условием. «Структурно схожи» можно выразить строго следующими терминами (в следующих определениях я опять не упоминаю симметричных отношений для краткости):

{\bfseries Определение.} Пусть дан граф $G=(V, E)$ и $(a, b) \in E$. Тогда {\slshape разбиением} графа $G$ по дуге $(a, b)$ называется граф $G' = (V\cup\{x\}, E\cup\{(a, x), (x, b)\}\setminus\{(a, b)\})$.

{\bfseries Определение.} Графы $A$ и $B$ называются {\slshape гомеоморфными}, если существует граф, такой что оба графа $A$ и $B$ получаются из него некоторыми разбиениями дуг.

{\bfseries Теорема Куратовского.} Граф является планарным тогда и только тогда, когда он не содержит подграф, гомеоморфный $K_{3, 3}$ или $K_5$.

Разбиение графа — это вставка вершины посреди какого-то ребра. Эта операция, очевидно, не влияет на возможность нарисовать граф планарно: подразбиение это просто выделение точки на линии. Например, мы на каждой тропинке от колодца к домику могли бы поставить опорный пункт полиции. Понятно, что граф уже не был бы равен $K_{3, 3}$, но в целом выглядел бы он точно так же.

Доказывать теорему Куратовского мы не будем, потому что я не знаю как её доказывать, да и не особо она нужна по большому счету (в рамках этого курса точно не пригодится). Я когда сам в своё время начинал читать доказательство, мне оно показалось скучным и длинным, и я бросил это занятие. Если хотите, можете найти доказательство в интернете. Однако зато теперь, если вам кто-то предложит решить головоломку с колодцами и домиками, вы можете с умным видом заявить: «Эта задача не имеет решения по теореме Куратовского», — и все сразу подумают, что вы умный. Я подобным образом пытался даже когда-то соблазнять девчонок (я правда рассказывал им о теореме Гёделя о неполноте), но это ни к чему так и не привело. Как был одинокий хрен так и остался.

Шутки шутками, но я бы хотел обратить внимание на то зачем я вообще завёл речь об этой задаче. По ходу изложения мы активно использовали терминологию теории множеств. Могли бы мы обойтись без неё? Вообще говоря да. В теории множеств у нас не было никаких особых теорем, которыми бы мы тут воспользовались, и мы вполне могли провести все те же рассуждения и не обращаясь к множествам и определению графа. Какие-то понятия нам все равно пришлось бы вводить, но это не сложно. Тем не менее теория множеств дала нам язык, на котором мы смогли это все строго сформулировать, а заодно дала нам какие-то общие идеи, хоть и очень простые, как вообще можно поступать с этими графами.

Следующая задача тоже может быть решена без теории множеств и теории графов, но однако если к задаче с колодцами ещё можно было как-то подступиться без множеств, то тут это уже вряд ли получится, хотя, опять же, напрямую свойства множеств никак не используются (замечу, что теоретический материал тут будет довольно сложный, так что при первом прочтении можно его и пропустить).

{\bfseries Задача.} В некоторой компании работает некоторое количество менеджеров. Директор вынудил их играть в странную игру. Каждому из менеджеров он надевает либо чёрную, либо белую шапку на голову. Менеджеры не знают цвета своей шапки, но видят цвет шапок других менеджеров. В какой-то момент в комнате звенит звонок и менеджеры должны одновременно по звонку назвать цвет собственной шапки (то есть менеджер называет цвет шапки, которую он не видит). Называть цвет должны не обязательно все — кто-то может и промолчать, но те кто называют цвет, должны делать это одновременно. Если хоть один менеджер ошибётся, то увольняют всех.  Если все промолчат, то тоже всех увольняют. Если же все, кто все же осмелится назвать цвет своей шапки, назовут его правильно, то все получают повышение. У них всего одна попытка и никак переговариваться/перемигиваться они не могут, называть цвета последовательно они тоже не могут — лишь одновременно и с первой попытки. Вопрос: как быть менеджерам?

Здесь надо отметить, что конечно же гарантированного способа избежать увольнения у менеджеров нет. Каждый менеджер и вправду не может знать своего цвета шапки, поэтому всегда есть шанс, что менеджер, называющий свой цвет, ошибётся. Однако поступая мудро, шанс ошибки можно минимизировать. Попробуйте вначале решить задачу самостоятельно, а затем читайте дальше. Для произвольного количества менеджеров задача действительно сложна, но для трёх менеджеров, довольно легко придумать простое и интуитивно-понятное решение. (Его я сформулирую в самом конце). Пока же обратимся опять к теории множеств и графов и посмотрим что можно сделать с этой задачей.

{\bfseries Определение.} Пусть даны графы $G_0 = (V_0, E_0)$ и $G_1 = (V_1, E_1)$. Их {\slshape декартовым произведением} называется граф $G_0 \square G_1 = (V, E)$, такой что $V = V_0 \times V_1$ и $((a, b), (v, w)) \in E$ тогда и только тогда, когда либо $a=v$ и $(b, w) \in E_1$, либо $b=w$ и $(a, v) \in E_0$.

В новом графе вершины записываются как упорядоченные пары из $V_0 \times V_1$ и  соответственно ребра этого графа являются упорядоченными парами упорядоченных пар, что формально можно записать так:

$E \subset (V_0\times V_1)\times(V_0\times V_1) \\ E = \{((a, b), (v, w))|(a=v \wedge (b, w)\in E_1)\vee(b=w \wedge (a, v)\in E_0)\}$

Выглядит чудовищно, но если вдуматься, то это вполне логичное и естественное определение для рёбер, если задавать его на произведении $V_0 \times V_1$. Тут, вероятно, помогут примеры.

Обозначим как $K_2$ простенький граф, состоящий из двух вершин, соединённых ребром ($V = \{0, 1\}$, $E = \{(0, 1), (1, 0)\}$).

Тогда можно увидеть, что в соответствии с определением $K_2$:

\begin{picture}(20,75)
\put(10,0){0}
\put(12,10){\line(0,1){50}}
\put(10,64){1}
\end{picture}

Его декартово произведение самого с собой $K_2 \square K_2$:

\begin{picture}(20,75)
\put(5,0){(0,0)}
\put(5,66){(1,0)}
\put(55,0){(0,1)}
\put(55,66){(1,1)}
\put(12,11){\line(0,1){50}}
\put(12,11){\line(1,0){50}}
\put(62,61){\line(0,-1){50}}
\put(62,61){\line(-1,0){50}}
\end{picture}

Умножим это ещё раз на него же ($K_2 \square K_2 \square K_2$, так как вершин много, я на этот раз указываю лишь некоторые из них):

\includegraphics{k23.png}

И ещё раз ($K_2 \square K_2 \square K_2 \square K_2$):

\includegraphics{k24.png}

Тут так и напрашиваются геометрические ассоциации, и это в общем-то неспроста. Действительно, на рисунках по сути изображены отрезок, квадрат, куб и, как мы позже выясним, четырёхмерный куб (в свою очередь так же выяснится, что квадрат — это двумерный куб, а отрезок — одномерный куб). По сути каждый раз умножая наш граф на $K_2$, мы вводим ещё одно измерение, в котором мы копируем наш куб, и эти два куба склеиваем по вершинам. Четырёхмерный куб (то есть его граф), приведённый выше, можно нарисовать в четырёхмерном пространстве без пересечения рёбер графа. Вообразить себе такое пространство и такой куб оказывается довольно сложно, но представление в виде графов даёт об этих кубах (и вообще многомерных объектах) довольно много информации. Например, из рисунка выше мы видим сколько всего у четырёхмерного куба будет вершин и рёбер, какие вершины какими рёбрами соединены, сколько из каждой вершины исходит рёбер и так далее. Это согласитесь уже не мало.

Развивая идею можно заметить, что декартово произведение графов ассоциативно и с точностью до переименования вершин коммутативно (и вообще тут можно построить целую арифметику графов при желании, так же как мы поступали с логическими операциями и операциями на множествах). Отсюда в силу ассоциативности сразу следует, что декартово произведение многомерных кубов — тоже является многомерным кубом, и можно сразу понять размерность такого куба.

Топологические приложения изложенного материала идут ещё дальше. Есть довольно общий приём триангуляции — раскрашивание поверхностей прилегающими друг к другу треугольниками. Каждая триангуляция является графом. Поверхности называются гомеофорфными, если, представляя, будто они сделаны из резины, можно одну поверхность как-то деформировать (растянуть-перекрутить без разрывов и склеиваний) в другую поверхность. Это довольно близко к понятию гомеоморфности графов. Позже мы докажем в нашем курсе, что если две поверхности возможно триангулировать гомеоморфными графами, то эти две поверхности сами гомеофорфны, а конструкции, подобные декартову произведению, позволяют выяснить это сравнительно простым путём. Более того, гомеоморфность — это отношение эквиваленции, и таким образом все поверхности распадаются на классы топологически эквивалентных поверхностей. Дальше на эти классы эквивалентности можно завязать всяческую алгебру и углубляться дальше в абстрактные конструкции сколько вздумается.

Эти рассуждения конечно очень абстрактны и не строги, и пока не очень понятно как именно все это описывается и как с этим работать. Я это рассказал только в обзорных целях для того, чтобы показать куда это все вообще может развиваться, поскольку читатели жалуются, что курс слишком теоретический и непрактичен.

{\bfseries Упражнение.} Пусть $P$ — незамкнутая цепь. Постройте (и нарисуйте) граф $P\square K_2$ (такой граф называется лестницей).

{\bfseries Упражнение.} Пусть опять $P$ — незамкнутая цепь. Постройте (и нарисуйте) граф $P\square P$ (такой граф называется сетью).

Следующие упражнения можно решать только при наличии каких-то геометрических образов в голове. Я их никак не объяснял, поэтому материала курса пока явно не хватит для решения, однако вы можете их хотя бы попробовать. Если вы ничего не понимаете, то это не страшно (и даже справедливо), смело пропускайте их — далее в курсе это всё будет рассмотрено более подробно и формально, и тогда к ним можно будет вернуться.

{\bfseries Упражнение.} Докажите, что куб, пирамида и сфера гомеоморфны (их можно триангулировать одним и тем же графом; вершины графа не обязательно могут совпадать с геометрическими вершинами трёхмерного объекта — важна лишь возможность его нарисовать).

{\bfseries Упражнение.} Докажите, что если $C$ — это цикл, то $C\square P$ — это цилиндр, и что цилиндр гомеоморфен сфере.

{\bfseries Упражнение.} Докажите, что $C\square C$ — это тор (выглядит как бублик) и он не гомеоморфен сфере.

В общем случае, если нам даны графы $A$ и $B$, то их произведение $A\square B$ строится, как теперь должно быть понятно, следующим образом: каждая отдельная вершина графа $B$ заменяется на точную копию графа $A$, а ребра графа $B$ соответственно заменяются на ребра, соединяющие все соответствующие вершины копий графа $A$.

Но вернёмся к нашим менеджерам. Какое они имеют отношение ко всей этой многомерной геометрии? А вот какое.

Каждого менеджера можно условно считать графом $K_2$. Одна вершина этого графа символизирует белую шапку на нём, другая чёрную, а ребро, их соединяющее — тот факт, что для него эти цвета неразличимы (он не видит какая шапка на нем надета). Тогда несколько менеджеров можно интерпретировать как многомерный куб $K_2 \square K_2 \square \ldots \square K_2$, у которого каждая вершина соответствует некоторому набору шапок на менеджерах. В момент, когда на менеджеров надевают шапки, фактически совершается выбор одной вершины куба. Но никакой отдельный менеджер не знает что это за вершина, так как он не знает цвета собственной шапки — он знает лишь то, что он «находится» в одной вершине куба из двух, то есть его знание может быть охарактеризовано ребром куба.

Пусть, например менеджеров трое, и всем надели белые шапки, что мы будем записывать как $(0, 0, 0)$. Первый менеджер не знает какая на нем надета шапка, и для него выбор стоит между $(0, 0, 0)$ и $(1, 0, 0)$. Для второго между $(0, 0, 0)$ и $(0, 1, 0)$. Для третьего $(0, 0, 0)$ и $(0, 0, 1)$. Это и есть ребра, которые исходят из вершины, характеризующей расстановку шапок.

После того, как менеджеры увидели шапки друг друга, им надо принять решение какой цвет назвать и называть ли его вообще. Поведение менеджеров таким образом можно описать ориентированным графом с теми же вершинами, но где каждое ребро либо удалено, либо заменено на дугу (ребро, направленное в одну сторону). Продолжая прошлый пример, если первый менеджер выберет сказать, что у него белая шапка, это будет представлено дугой $(1,0,0)\to (0, 0, 0)$, если предпочтёт сказать, что шапка чёрная, то это будет дуга $(0, 0, 0)\to (1, 0,0)$. Если предпочтёт промолчать, то вершины $(0, 0, 0)$ и $(1, 0, 0)$ вообще не будут инцидентны.

Попытаемся понять, как нам надо выбрать этот орграф таким образом, чтобы менеджеры в большинстве случаев справлялись с задачей. Если у нас в некоторую вершину идёт хоть одна дуга, то это значит, что если расстановка шапок будет соответствовать этой вершине, то кто-то из менеджеров назовёт цвет своей шапки. Если из какой-то вершины исходит какая-то дуга в другую вершину, то это значит, что при расстановке шапок для этой вершины менеджер, соответствующей этой дуге, ошибётся, и все проиграют. Если какая-то вершина не будет инцидентна никакой другой вершине, то это значит, что при соответствующем расположении шапок, все промолчат, и менеджеры снова проиграют.

Наша задача сводится к такому заданию ориентированного графа, чтобы количество проигрышных вершин (таких, из которых исходят дуги), было как можно меньше, и как можно больше вершин, в которые идёт хотя бы по одной стрелке, но для которых нет исходящих стрелок. Такую оптимальную стратегию для трёх менеджеров можно выразить следующим рисунком:

\includegraphics{sol3.png}

Здесь фактически говорится, что если некий менеджер видит, что шапки двух других менеджеров имеют одинаковый цвет, то ему надо назвать цвет противоположный. Это действительно очень простая стратегия: вероятность того, что всем оденут шапку одного цвета меньше, чем вероятность того, что цвета будут все же разными. Поэтому можно предполагать, что существуют все же разноцветные шапки, и поэтому если ты видишь, что у двух менеджеров одинаковые шапки, то на тебе самом с большой степенью вероятности шапка имеет противоположный цвет. До этого решения, конечно, можно было бы догадаться и без многомерных кубов.

А вот решение для четырёх менеджеров:

\includegraphics{sol4.png}

Эту стратегию уже так просто не сформулируешь. Если все четверо видят только белые шапки, то надо называть чёрный цвет. Во всех остальных ситуациях четвёртый менеджер вообще молчит, говорят оставшиеся. Если на всех кроме четвёртого чёрные шапки, то все кроме четвёртого называют белый цвет. Если на четвёртом менеджере чёрная шапка, то первый менеджер должен молчать. Если второй видит на третьем и четвёртом чёрные шапки, то он должен называть цвет, который видит на первом менеджере. Третий же менеджер должен называть цвет второго менеджера, если он не совпадает с цветом первого. В остальных случаях менеджеры молчат. (Вероятно я где-то ошибся, потому что глядя на этот график можно глаза сломать и я не уверен, что корректно его прочитал; но идея думаю понятна). Решение с трудом можно назвать изящным, но как видно из нашего графа, в большинстве случаев менеджеры все же будут выигрывать. Поэтому это пусть и не красивое (и даже уродливое), но все равно решение. Согласитесь, что если бы это был вопрос жизни и смерти, то оно бы вас устроило.

Если менеджеров больше, то задача решается аналогично, хотя графы будут ещё больше по размеру и перебирать придётся ещё больше вариантов. С этим может справиться компьютер, хотя задача все равно сложная. Какого-то универсального и простого решения к этой задаче видимо не существует, но тем не менее с помощью графов мы смогли получить хоть что-то. Без подобных абстракций и строгой формализации, задача не решалась бы вообще никак.

Ещё хуже была бы ситуация, если бы цветов шапок было бы не два, а больше. Тогда помогло бы следующее понятие:

{\bfseries Определение.} {\slshape Гиперграфом} называется пара $(V, E)$, где $E\subset 2^V$.

Фактически это граф, в котором ребра соединяют сразу множество вершин.

Любой гиперграф может быть представлен с помощью обычного графа. Для этого надо рассмотреть граф, множество вершин которого разбито на два подмножества — одно подмножество вершин соответствует вершинам гиперграфа ($V_0$), а второе рёбрам гиперграфа ($V_1$). Множество рёбер, инцидентных одной и той же вершине из $V_1$ соответствует грани гиперграфа.

Используя гиперграфы можно решать задачу менеджеров и с произвольным количеством цветов шапок. Задача однако в этом случае становится менее интересной, так как вероятность выигрыша менеджеров резко падает.

Гиперграф — не единственное возможное обобщения понятия граф. Гораздо чаще применяется следующее определение:

{\bfseries Определение.} {\slshape Абстрактным симплициальным комплексом} на множестве $S$ называется множество $\Delta\subset 2^S$, такое, что если $x\in \Delta$, то для любого $y\subset x$ так же $y\in\Delta$. Элементы множества $\Delta$ называются {\slshape абстрактными симплексами}.

Абстрактные симплициальные комплексы удобно строить последовательно: вначале выбирается множество вершин $S$ (так называемые нульмерные симплексы). Затем выбираются обычные ребра графа, соединяющие вершины из $S$ (одномерные симплексы). Затем из различных объединений этих рёбер и вершин выбираются трёхмерные симплексы, которые можно рассматривать как сплошные плёнки, которыми заклеиваются ребра графа. Объединяя эти плёнки можно выбрать трёхмерные симплексы, затем четырёхмерные и так далее. Здесь правда надо следить за тем, чтобы «наклеивая плёнки» мы не нарушали условий определения комплекса. Например, если есть грани $\{a, b\}$ и $\{b, c\}$, но нет грани $\{a, c\}$, то наклеить плёнку $\{a, b, c\}$ мы не имеем права, поскольку по определению для её наклеивания требуется так же наличие грани $\{a, c\}$, что неплохо соответствует геометрической интуиции.

Такие конструкции используются опять же в многомерной топологии и с помощью них можно выяснять подробности устройства многомерных объектов. В этом курсе все это будет значительно позже, пока что я привёл эти определения лишь для того, чтобы показать каким образом самые базовые понятия теории множеств могут быть полезны простому человеку и как она позволяет простым способом обобщать привычные в общем-то понятия. Написанное в этом параграфе можно смело забыть до поры, это все у нас будет значительно позже.

\section{Функции}

{\bfseries Определение.} {\slshape Функцией}, или {\slshape отображением}, из множества $A$ во множество $B$ (обозначение $f:A\to B$) называется множество упорядоченных пар $f\subset A\times B$, таких что для любого $a\in A$ найдётся единственный $b\in B$, такой что $(a, b) \in f$.

Если $(a, b) \in f$, то это часто записывается как $f(a) = b$ или как $f:a \mapsto b$. Множество всех функций $\{f: A\to B\}$ обозначается как $B^A$.

{\bfseries Определение.} Если $f:A \to B$, то множество $A$ называется {\slshape областью определения} функции $f$ и обозначается как $\mathrm{Dom} f$.

Область определения может представлять собой декартово произведение нескольких множеств. В этом случае говорят, что функция является {\slshape функцией нескольких переменных}, где каждое множество соответствует отдельной переменной.

{\bfseries Определение.} Если $f:A \to B$, то множество $B$ называется {\slshape областью значений} функции $f$ и обозначается как $\mathrm{Codom} f$.

{\bfseries Пример.} Логические операции И, ИЛИ, Исключающее ИЛИ, импликация и эквиваленция рассматриваемые нами ранее, являются функциями $f:B\times B \to B$ (это всё функции нескольких переменных), где $B = \{0, 1\}$. Функция НЕ является функцией типа $f: B \to B$.

{\bfseries Пример.} Объединение и пересечение множеств являются функциями типа $f: S \times S \to S$, где $S$ — некоторое множество, состоящее из других множеств.

{\bfseries Пример.} Любой предикат $F$, заданный на некотором множестве $S$ является функцией типа $F: S \to B$, где $B = \{0, 1\}$.

Если говорить о чистой интуиции, то понятие функции имеет две основных трактовки. Первая — это некоторый объект, который по заданному элементу множества $A$ каким-то образом выдаёт какой-то элемент множества $B$. Иногда он берет его из таблицы, иногда есть какая-то формула, по которой можно этот самый $b$ вычислить, иногда какая-то написанная на компьютере программа, которая по $a$ даёт $b$. Если у нас есть некоторое физическое устройство (например, чёрный ящик) с клавиатурой, свалившееся из космоса, которое при наборе какого-то числа даёт в ответ другое число, и мы не знаем как именно оно это делает — это все равно тоже функция.

Мы можем привести множество примеров функций в быту. Если $A$ — множество женатых мужчин, а $B$ — замужних женщин, то функция, сопоставляющая каждому мужчине в соответствие его жену, является функцией вида $f: A\to B$. Если у нас например есть база данных, в которую мы можем вводить имена мужчин, а она отдаёт нам в ответ имена их жён, то эта база данных как раз и будет реализовывать данную функцию.

Можно рассмотреть множество рабочих дней, прошедших от начала торгов на Нью-Йоркской фондовой бирже, и каждому дню сопоставить лидеров роста и падения. Если дни обозначить за $D$, а компании за $C$, то такое сопоставление будет функцией вида $D\to C\times C$.

В дальнейшем мы будем временами приводить примеры из кодирования и криптографии. Шифрование и кодирование — это тоже функции. Если $T$ — множество всех возможных текстов, а $B^\infty$ — множество последовательностей нулей и единиц (как это принято в компьютере на низком уровне), кодирование — это функция $f:T\to B^\infty$. Задачей теории кодирования является построение такой функции $f$, чтобы она обладала какими-то полезными свойствами, например чтобы запись была максимально краткой, либо чтобы она была устойчива к ошибкам, и в случае каких-то сбоев можно было восстановить весь текст и по фрагменту кодировки.

Если $K$ — множество ключей, а $C$ — множество шифровок, то шифрование сводится к реализации функции шифрования $E: T\times K \to C$ и дешифрования $D: C\times K \to T$, которые должны быть выбраны таким образом, чтобы не зная ключа $k\in K$ нельзя было восстановить исходный текст $t \in T$ по шифру $c \in C$.

Если $A$ — множество карточных мастей, а $B$ — множество достоинств карт, то функция, которая ставит каждой карте в соответствие её достоинство, имеет вид $f: A\times B \to B$ и может быть записана формулой как $f: (a, b) \mapsto b$.

Второй интуитивный смысл, который часто имеют функции — это установление соответствия между различными объектами, которое говорит нам что-либо о свойствах этих объектов. Пусть, например, множество $A$ состоит из упорядоченных элементов $a<b<c$, а множество $B$ из элементов $a<b<c<d<e$. Тогда функция $f:A \to B$, которая сопоставляет любому элементу тот же самый элемент другого множества ($f: x\mapsto x$) может показать нам, что множество $A$ в некотором смысле является начальным отрезком множества $B$ — любой элемент последнего множества, не нашедшего себе пары во множестве $A$, будет больше любого другого элемента. Это примитивный пример, но вероятно он как-то продемонстрирует общую идею (если нет, то позже вероятно вы это поймёте на практике).

Рассмотрим более содержательный пример. В первом параграфе мы говорили, что любому предикату на множестве $S$ соответствует подмножество $S$ и наоборот. Это соответствие — тоже функция. Каждый такой предикат — это функция вида $f:S\to B$, и стало быть $f\in B^S$. Тогда функция $p$, которая по предикату даёт подмножество, ему соответствующее, имеет тип $p:B^S \to 2^S$. Забегая вперёд можно сказать, что $2=\{0, 1\}$ (это будет объяснено в третьей главе), и именно отсюда происходит обозначение для булеана.

Отметим так же, что функции используются часто не только для отображения отдельных элементов, но и для отображения подмножеств элементов.

{\bfseries Пример.} Пусть опять $A$ — множество женатых мужчин, $B$ — замужних женщин, а $f: A\to B$ ставит каждому мужчине в соответствие его жену. Пусть теперь $M\subset A$ — подмножество мужчин, работающих в Макдональдсе. Тогда $f(M)$ — это множество женщин, чьи мужья работают в Макдональдсе.

Это можно формализовать при желании и назвать отдельными словами:

{\bfseries Определение.} Если $f(a) = b$, то $b$ называется {\slshape образом} элемента $a$ по отображению $f$.

{\bfseries Определение.} Множество $f(S) = \{y\in \mathrm{Codom}f|\exists x \in S, f(x) = y \}$ называется {\slshape образом} множества $S$ по отображению $f$.

{\bfseries Определение.} Множество $f^{-1}(y) = \{x | f(x) = y \}$ называется {\slshape прообразом} элемента $y$.

{\bfseries Определение.} Множество $f^{-1}(S) = \{x | f(x) \in S \}$ называется {\slshape прообразом} множества $S$.

Обратите внимание на то, что образом любого элемента является только один элемент, а прообразом является целое множество элементов (вполне возможно, что пустое).

{\bfseries Пример.} Пусть $A = \{a, b, c\}$, $f = \{(a, a), (b, c), (c, c)\}$. Тогда $f^{-1}(a) = \{a\}$, $f^{-1}(b) = \emptyset$, $f^{-1}(c) = \{b, c\}$.

{\bfseries Определение.} Множество $\mathrm{Im} f = f(\mathrm{Dom} f)$ называется {\slshape образом} функции $f$.

Обратите внимание на то, что в общем случае $\mathrm{Im} f \not= \mathrm{Codom} f$. Так, в последнем примере $\mathrm{Im} f = \{a, c\}$, но $\mathrm{Codom}f = \{a, b, c\}$.

{\bfseries Определение.} Единичной функцией на множестве $A$ называется функция $1_A: A\to A$, ставящая любому элементу в соответствие его же самого: $1_A: a\mapsto a$.

{\bfseries Определение.} {\slshape Композицией} функций $f:B\to C$ и $g:A\to B$ называется функция $f\circ g: A\to C$, такая что если $f(b) = c$ и $g(a) = b$, то $(f\circ g)(a) = c$.

{\bfseries Теорема.} Для любой функции $f: A\to B$, $1_B \circ f = f \circ 1_A = f$.

Доказательство в качестве простого упражнения.

{\bfseries Теорема.} Композиция функций ассоциативна: $f\circ (g \circ h) = (f\circ g)\circ h$.

{\bfseries Доказательство.} Достаточно выписать напрямую два значения функции для произвольного элемента $x$, чтобы увидеть это:

Слева: $(f\circ (g \circ h)) (x) = f((g\circ h)(x)) = f(g(h(x)))$

Справа: $((f\circ g) \circ h) (x) = (f\circ g)(h(x)) = f(g(h(x)))$

Как видно, в обоих случаях получается одно и то же значение. \qed

{\bfseries Определение.} Функция называется {\slshape инъективной}, или {\slshape инъекцией}, если $f(a)\not= f(b)$ для любых $a\not= b$.

{\bfseries Определение.} Пусть $f:A\to B$. Функция $f^{-1}_l$, такая что $f^{-1}_l\circ f = 1_A$ называется {\slshape левой обратной}.

{\bfseries Теорема.} Функция имеет левую обратную функцию тогда и только тогда, когда она инъективна.

Докажите эту теорему в качестве упражнения.

{\bfseries Пример.} Любая функция кодирования обязана быть инъективной, поскольку в противном случае была бы возможна ситуация $f(a) = f(b) = c$, и было бы непонятно как мы должны раскодировать $c$ обратно.

{\bfseries Определение.} Если $\mathrm{Im}f = \mathrm{Codom}f$, то функция называется {\slshape сюръективной}, или {\slshape сюръекцией.}

{\bfseries Определение.} Пусть $f: A\to B$. Функция $f^{-1}_r$. такая что $f\circ f^{-1}_r = 1_B$ называется {\slshape правой обратной}.

{\bfseries Теорема.} Функция имеет правую обратную функцию тогда и только тогда, тогда она сюръективна.

Доказательство опять же не сложно и я оставляю его читателю в качестве упражнения.

{\bfseries Упражнение.} Пусть $f:A \times B \to A$ и $f: (a, b)\mapsto a$. Докажите, что эта функция сюръективна.

{\bfseries Определение.} Если функция одновременно и сюръективна и инъективна, то она называется {\slshape биективной}, либо {\slshape биекцией}.

{\bfseries Теорема.} Если $f: A\to B$ — биекция, то левая обратная функция будет совпадать с правой обратной функцией.

Доказательство снова в качестве не сложного упражнения. Понятно, что в случае с биекциями разница между левой обратной и правой обратной функцией пропадает (в случае же сюръекции или инъекции существует лишь одна из них), и поэтому такая функция называется просто {\slshape обратной}. Если функция плюс ко всему является обратной самой себе (то есть $f(f(x)) = x$, очевидно, что это возможно лишь для отображений вида $A\to A$), то она называется \term{инволюцией}.

Так же полезно заметить, что произвольную инъективную функцию возможно сделать биективной, если заменить её область значений лишь её образом, то есть если $C = \mathrm{Im} f$, то вместо функции $f: A\to B$ рассмотреть функцию $f: A \to C$. Легко проверить, что в этом случае функция действительно станет биекцией.

Для сюръекции подобное утверждение тоже верно, но только при отдельных оговорках.

{\bfseries Определение.} Ограничением функции $f: A\to B$ на $S\subset A$ называется функция $f|_S: S\to B$, такая что для любого $x\in S$ верно, что $f|_S(x) = f(x)$.

Несколько более формально и точно, но менее понятно можно написать, что $f|_S = f \cap S \times B$.

Для произвольной сюръективной функции можно было бы попробовать искать такое ограничение этой функции, чтобы она стала биекцией. Предположение это на первый взгляд довольно очевидно, однако оказывается, что оно эквивалентно так называемой аксиоме выбора, которую во-первых нельзя взять и доказать, а во-вторых из которой следует множество парадоксов. Подробнее мы будем обсуждать эту тему далее в этом курсе (и совсем немного в следующем параграфе), пока что можно просто принять к сведению (хотя это и не принципиальной важности факт), что доказать существование такого ограничение невозможно.

{\bfseries Определение.} Множества $A$ и $B$ называются {\slshape равномощными} (обозначение $|A| = |B|$), если существует биекция $f: A\to B$.

Равномощность говорит о том, что элементы множеств $A$ и $B$ можно поставить во взаимооднозначное соответствие. Часто это интерпретируется как то, что они содержат одинаковое число элементов. Это правда довольно опасная интерпретация, что станет ясно, когда мы начнём говорить о бесконечных множествах. Пока же в принципе довольно удобно воспринимать равномощность именно так. имея  правда ввиду, что это сгодится лишь только для довольно маленьких множеств.

{\bfseries Пример.} Пусть $A = \{1, 2, 3\}$, $B = \{a, b, c\}$. Тогда эти множества равномощны, поскольку существует биекция $f=\{(1, a), (2, b) , (3, c)\}$.

{\bfseries Пример.} Пусть $A$ — множество женатых мужчин, а $B$ — множество замужних женщин. Эти множества равномощны.

{\bfseries Упражнение.} Приведите пример неравномощных множеств и двух отображений на них: сюръективного и инъективного.

\section{Формализм}

Как обычно в завершение главы немного пожестим и рассмотрим довольно сложные темы, которые не обязательны для понимания дальнейшего материала, и которые при возникающих сложностях можно пропустить. Если вы молодая, одинокая, красивая девушка из Москвы или около, и вы вдруг поймёте что здесь написано, то вам надо обязательно написать мне письмо на почту. Я на вас женюсь.

Обычно рассмотрение формализма теории множеств начинают с обсуждения парадокса Рассела, на примере которого объясняются ряд тонкостей аксиом теории множеств и мотивируются определения. Мы не будем исключением.

{\bfseries Парадокс Рассела.} Пусть $U$ — множество всех множеств. $A = \{x\in U|x\not\in x\}$, то есть множество таких множеств, которые не содержат себя самого в качестве элемента. Вопрос: принадлежит ли $A$ самому себе, то есть верно ли, что $A\in A$?

Если предположить, что $A\in A$, то по определению $A$, он не должен быть элементом $A$. Если $A\not\in A$, то по определению $A$, $A \in A$. Это сильно напоминает парадокс брадобрея, рассмотренный в первой главе, и утверждение парадокса Рассела на самом деле легко сводится к тому же самому утверждению: $\exists A\forall x\in A (x\in A \leftrightarrow x\not\in x)$. Однако в качестве $x$ мы можем взять $A$ и получить запись $A\in A \leftrightarrow A\not\in A$. Это высказывание всегда ложно.

Когда мы рассматривали парадокс брадобрея, то придя к такому выражению мы сделали вывод о том, что изначальная постановка задачи была некорректна — такого брадобрея просто не могло существовать. Так и здесь очевидно, что множества $A$, описанного в условии парадокса, очевидно не может существовать. Стало быть где-то мы использовали запись, которую мы использовать не имеем права. Первое предположение, которое обычно делают люди, глядя на этот парадокс, что некорректна запись $x\in x$, и соответственно первое желание — запретить множеству быть элементом самого себя.

Аксиоматика Цермело-Френкеля (сокращённо ZF), которую мы будем рассматривать в этом параграфе, как раз запрещает выражения вида $x\in x$. Для такого запрета специально вводится аксиома Foundation Axiom: «Для любого непустого множества $x$ найдётся такое $y\in x$, что $x$ и $y$ не пересекаются». Такая сложная формулировка на самом деле оправдана. Если бы вместо неё мы написали что-то простое вроде $\neg \exists x, x\in x$, то в таком виде эта аксиома все равно продолжала бы приводить к парадоксам вроде парадокса Рассела. Например, оказалась бы корректной запись $x=\{y\}$, $y=\{x\}$. В этом случае, действительно, $x\not\in x$, но однако $x\in y\in x$, а это ничуть не лучше того, что было изначально.

При этом важно заметить, что даже при этой аксиоме, у нас возможна ситуация, когда некоторый элемент множества является одновременно и его подмножеством: $x\in y \wedge x\subset y$. Например, вот: $a = \{b, c\}$, где $c=\{ b\}$. Очевидно, что $c\in a \wedge c\subset a$.

Может показаться, что эта ситуация тоже нежелательна. Однако, это не так. На самом деле вся арифметика построена именно на таких множествах. Если быть точным, то в современной математике натуральные числа определены следующим образом:

0) $0 = \emptyset$

1) $1 = 0 \cup \{0\} = \{0\}$

2) $2 = 1 \cup \{1\} = \{0, 1\}$

3) $3 = 2\cup \{2\} = \{0, 1, 2\}$

4) $4 = 3 \cup \{3\} = \{0, 1, 2, 3\}$

И так далее. В общем случае определена операция инкремента $S: n\mapsto n\cup \{n\}$, через которую и определяется множество натуральных чисел, обозначаемое символом $\mathbb{N}$. Сама функция инкремента часто обозначается как $+1$, то есть $S(n) = n+1$. $+1$ в данном случае не какая-то операция над числом и единицей, а просто другое обозначение для функции $S$. Мы пока примем сформулированное определение натуральных чисел как факт, а подробнее будем рассматривать его в следующей главе.

Как видно, накладывая ограничения на структуру множеств, Foundation Axiom тем не менее оставляет значительный простор для конструирования множеств, и это ещё одна причина, по которой она формулируется именно в таком виде.  Иные виды формулировок, которые могли бы придти на ум, могли бы не оставить нам возможности определить подобным образом натуральные числа. Мы конечно могли бы определить их и как-нибудь по-другому, однако приведённое определение пока является наиболее простым и удобным из всех известных науке. Об этом опять же будет в следующей главе.

Тем не менее с этой аксиомой оказывается все не совсем гладко, о чем свидетельствует следующая теорема.

{\bfseries Теорема.} В предположении Foundation Axiom, не существует бесконечной последовательности $x_1, x_2, x_3, \ldots$ такой, что $x_2 \in x_1$, $x_3 \in x_2$, $x_3 \in x_4, \ldots$, то есть последовательности, в которой каждый элемент является множеством, причём каждый элемент является так же элементом предыдущего элемента.

Прежде чем перейти к доказательству, сделаем некоторые замечания по терминологии.

{\bfseries Определение.} {\slshape Последовательностью} элементов множества $A$ называется функция $x:\mathbb{N}\to A$. Элементы $x(n)\in A$ называются {\slshape элементами последовательности} и обозначаются как $x_n$.

При перечислении элементов последовательности, обычно предполагают, что первым элементом является $x_1$, затем $x_2$ и так далее. Формально это не совсем соответствует определению, которое мы только что привели: элементы последовательности мы начинаем нумеровать с единицы, однако множество натуральных чисел «начинается» с нуля. Чтобы добиться точности, мы могли бы либо нумеровать элементы последовательности начиная нулём, либо определять последовательность как отображение $\mathbb\{N\}\setminus \{0\} \to X$. На практике же эта неточность ни на что не влияет и поэтому можно оставить все как есть.

Если в нашем условии теоремы рассмотреть некоторое множество множеств $U$ и вспомнить о функции инкремента $+1$, то теорему можно переформулировать таким образом:

{\bfseries Теорема.} В предположении Foundation Axiom, не существует последовательности $\{x_n\}$ такой, что $x_{n+1} \in x_n$.

{\bfseries Доказательство.} Предположим противное. Пусть $f: \mathbb{N}\to U$ — такая функция, что $f(n+1) \in f(n)$. Foundation Axiom требует, чтобы существовало $A\in \mathrm{Im} f$, такое, что $A \cap \mathrm{Im}f = \emptyset$. Однако для любого $A\in \mathrm{Im} f$ найдётся такое $n$, что $A = f(n)$ в силу определения $f$. Отсюда $f(n+1) \in A$ и $f(n+1)\in \mathrm{Im}f$. Это значит, что $f(n+1)\in A \cap \mathrm{Im}f$, то есть $A$ и $\mathrm{Im}f$ все же пересекаются вопреки нашему предположению. Полученное противоречие доказывает теорему. \qed

Рассмотрим множество людей не Земле. Каждый из людей состоит из множества органов. Каждый орган из множества тканей. Ткани — из клеток. Клетки из органоидов, органоиды из молекул, молекулы из атомов и так далее. Это не очень точная и не очень формальная последовательность с научной точки зрения, но для наших целей этого вполне достаточно. Атом можно продолжать разбивать на субатомные частицы, их можно разбивать далее, получив кварки, которые, пока только предположительно, могут состоять из преонов (это открытый вопрос науки). Если будет доказано существование преонов, вполне вероятно, что встанет вопрос о том, из чего состоят преоны. В общем виде можно задаться таким вопросом, сформулированную ещё в древние времена: можно ли различные объекты физического мира разбивать на составные части сколь угодно долго, либо же мы неминуемо придём к некоторым неделимым составным частям?

Foundation Axiom внезапно отвечает на этот вопрос: да, мы обязательно должны придти к неделимым составным частям. Foundation Axiom конечно работает с формальными множествами, а приведённые физические рассуждения с объектами реального физического мира, поэтому считать, что Foundation Axiom действительно говорит что-то о реальном мире, мы не можем. Однако же подобная интерпретация наводит на мысли о том, что данная аксиома вероятно и не особо-то хороша.

Как мы говорили, аксиоматика ZF, которую мы сейчас рассмотрим, все же содержит в себе Foundation Axiom, однако, ZF не единственная возможная формализация теории множеств. Существуют и другие системы аксиом, в которых, наоборот, используется Anti-Foundation Axiom (кратко AFA, которая тоже может формулироваться по-разному). Простейший вариант такой аксиоматики это так называемая New Foundation Set Theory, в которой на уровне аксиом определяются ориентированные графы, допускающие циклы (дуги вида $(a, a)$), и AFA явно постулирует, что каждая вершина графа является множеством.

«Как же тогда в New Foundation обстоит дело с парадоксом Рассела?» — спросит читатель, если он ещё тут. Ответ на этот вопрос неожиданный: на самом деле парадокс Рассела вообще почти никак с Foundation Axiom не связан. Сама постановка парадокса содержит множество других внутренних противоречий, которые и приводят к нему, но эти противоречия разрешаются и без Foundation Axiom. Проблема с записью $x\in x$, которую мы пытались разрешить выше, оказывается, является лишь «наведённым эффектом» от других проблем формулировки, а Foundation Axiom при всей своей очевидности, оказывается, что не особо-то и нужна.

Прежде чем перейти к формулировке аксиом ZF, мы должны внести ясность с понятиями логики из первой главы. В первом главе при определении понятий логики, мы активно пользовались теорией множеств. Даже само понятие теории мы формулировали как множество формул. После, определяя понятия теории множеств, мы пользовались логикой. Сейчас нам предстоит переформулировать все что было сказано заново, но только избавившись от порочных кругов.

Итак, нам надо с чего-то начать. Увы, чтобы определить какое-то понятие, нам неминуемо необходимо пользоваться другими понятиями, которые тоже в свою очередь надо как-то определять. Поскольку бесконечную цепочку определений мы физически не можем построить, нам придётся смириться с тем, что у нас будет одно базовое понятие, которое не будет иметь никакого определения. Таким понятием для нас станет понятие {\slshape символа}.

Говоря неформально, символ — это некоторая закорючка, которую мы можем нарисовать на листе бумаги. Мы будем считать, что различных символов очень много (сколько угодно сколько потребуется) и мы можем отличать один символ от другого. Определять более строго мы это понятие не будем и примем понятие символа за основу, на которой мы будем строить нашу теорию.

Следующие символы будем называть {\slshape логическими символами}:

— {\slshape логические связки} $\neg$, $\wedge$, $\vee$, $\rightarrow$ и $\leftrightarrow$;

— {\slshape квантификаторы} $\forall$, $\exists$;

— {\slshape равенство} $=$;

— {\slshape переменные}, просто некоторый набор до сих пор не задействованных символов.

Так же будем выделять {\slshape нелогические символы} (это так же просто классификация неиспользуемых ранее символов):

— символы {\slshape отношений}{\slshape };

— символы {\slshape операций};

— {\slshape константные} символы.

Это все пока просто классификация. Скажем, аксиоматика ZF требует лишь одного символа отношения $\in$. Можно было бы рассмотреть дополнительно символы вроде $\cup$, $\cap$, $\bigtriangleup$ и подобных в качестве операций, а символ $\emptyset$ в качестве константного символа, однако аксиомы ZF не используют их, как будет видно ниже. Я повторюсь, что на данный момент мы не наделяем символы никаким смыслом — для нас это пока не более чем закорючки на бумаге, которые мы пока просто классифицировали без какой-либо особой логики.

{\bfseries Определение.} {\slshape Термом} называется либо константный символ, либо переменная, либо запись $f(a, b, \ldots, z)$, где $a, b, \ldots, z$ — некоторые прочие термы, а $f$ — символ операции.

{\bfseries Определение.} {\slshape Атомом} называется либо запись $a = b$, где $a$ и $b$ — термы, либо запись $f(a, b, \ldots, z)$, где $a, b, \ldots, z$ — некоторые прочие термы, а $f$ — символ отношения.

{\bfseries Определение.} {\slshape Формулой} называется одна из следующих записей:

— отдельный атом является формулой;

— если $\phi$ и $\psi$ являются формулами, то записи вида $\neg \phi$, $\phi \wedge \psi$, $\phi \vee \psi$, $\phi \rightarrow \psi$, $\phi \leftrightarrow \psi$ так же являются формулами;

— если $\phi$ — формула, а $v$ — переменная, то $\forall v \phi$ и $\exists v \psi$ так же являются формулами

{\bfseries Определение.} Переменная $v$ в составе формулы называется {\slshape свободной}, если перед этой формулой не написано $\exists v$ или $\forall v$

{\bfseries Определение.} {\slshape Предложением} называется формула без свободных переменных.

Приведённые нами определения несколько упрощены, но в целом достаточны. Например, с точки зрения определений, данных выше, запись $a\cap b$ не является формулой, однако формулой является $\cap(a, b)$. Мы могли бы расширить наши определения, чтобы покрыть более широкий класс формул, но это был бы довольно скучный бессодержательный рассказ в угоду формализму. При желании формально это может проделать читатель самостоятельно, это не сложно. Мы же двинемся дальше, а для краткости будем полагать, что записи вроде $a\cap b$ являются сокращением записей в соответствии с определениями. Для нас важнее сейчас понять общую идею, а не достичь максимальной корректности с формальной точки зрения.

Теперь можно вспомнить правила вывода из первой главы. Они формулировались просто как правила, по которым можно преобразовывать формулы (хотя само понятие формулы мы строго не определяли). Например, правило modus ponens говорило, что две формулы $\phi$ и $\phi\rightarrow\psi$ можно преобразовать в формулу $\psi$. Эти правила можно применять к произвольным формулам, не наделённых никаким смыслом. Это как раз наш случай. Мы будем их применять просто к закорючкам на бумаге, совершенно не думая о том, что эти закорючки что-то вообще значат.

В общем виде теперь построение теорий будет выглядеть так: из некоторого начального набора предложений, называемых аксиомами, по правилам вывода, мы будем получать некие другие предложения, называемые теоремами. В самом общем виде смысла в этом — ноль. Однако если в качестве аксиом взять какие-то предложения, которые мы можем осознанно интерпретировать как какие-то имеющие физический смысл, то наши выводы теорем уже оказываются полезными.

То есть если подумать, то все человеческие рассуждения — это на самом деле некоторое преобразование высказывания языка. Любые природные или логические явления мы записываем на бумаге, и как бы мы не старались, они все равно в результате описываются словами, смысл которых ровно тот, которым мы их наделяем. Из каких-то предложений человеческого языка мы по некоторым правилам вывода можем строить некоторые рассуждения. Сейчас мы делаем то же самое, но только более формально: мы строго определили само понятие предложения и набор правил, по которым мы можем строить рассуждения. При кажущейся абстрактности и оторванности от физического мира, это на самом деле единственное что нам доступно и это именно то, как строятся все умозаключения и рассуждения.

Доказательства первой главы были конечно нестрогими и неформальными, в свете того, что я говорю сейчас. Увы, провести многие из них более формально и не получится вовсе. Весь материал первой главы был нацелен лишь на то, чтобы дать неформальную интуицию, касающуюся того, почему вообще мы можем наделять наши формулы интерпретацией, и почему отдельные правила вывода логично было бы принять как верные. Строго доказать этого конечно нельзя.

Впрочем, первый параграф был довольно большой, и даже в том виде как мы сформулировали понятие теории, набор теорем логики, которые мы предположительно можем использовать, тоже объёмный, и все это предлагается принять на веру, основываясь на каких-то неформальных рассуждениях. Чтобы объем принимаемого на веру оказался меньше, можно в принципе договориться о том, что многие логические символы являются лишь сокращениями для длинных записей. Так, можно считать, что $x\vee y$ — это не более чем сокращённая форма записи для $\neg (\neg x \wedge \neg y)$, а $\exists x P(x)$ — сокращённая форма записи для $\neg \forall x \neg P(x)$. Правила вывода так же многие можно в этом случае либо выкинуть, либо доказать используя только преобразования формул.

Прежде чем окончательно сформулировать аксиомы Цермело-Френкеля, сделаем ещё одну ремарку. До сих пор мы говорили, что множество — это набор элементов. Что такое элементы мы тем не менее не определяли. Говоря теперь о формальных символах, мы вообще больше никак не можем разграничить понятие множества и его элемента, у нас в соответствии с определением формулы есть лишь нелогические константы и логические переменные. Это может показаться довольно большим недостатком сформулированной нами логической системы, и долгое время при задании аксиом теории множеств математики вводили специальные предикаты, которые для каждой константы определяли является ли она множеством или нет. С учётом того, что любое множество может быть элементом кого-то другого множества, деление проводилось между элементами и урэлементами, которые множествами никак не являлись. Позже однако выяснилось, что деление на элементы и урэлементы излишне — если предположить, что в нашей теории не существует в принципе ничего кроме множеств, то мы ничего в действительности и не потеряем, а лишь наоборот сделаем нашу теорию проще.

Вот теперь мы можем формулировать аксиомы Цермело-Френкела (ZF):

{\bfseries 1. Extensionality Axiom}: $\forall x \forall y (\forall z (z \in x \leftrightarrow z \in y) \rightarrow x = y)$

Эта аксиома утверждает, что множество полностью определяется своими элементами, то есть если множества $x$ и $y$ состоит из одних и тех же элементов, то $x=y$. Довольно разумное требование, здесь сложно как-то ещё это прокомментировать.

{\bfseries 2. Foundation Axiom}: $\forall x(\exists y (y\in x)\rightarrow \exists y(y\in x\wedge \neg \exists z (z\in x \wedge z\in y)))$

Возможно после того, как я сформулировал эту аксиому выше словами обычного русского языка, у вас могло бы возникнуть желание записать эту аксиому как-то вроде $\forall x \not= \emptyset \exists y\in x, x\cap y = \emptyset$. Такая запись однако не может являться предложением нашей теории, так как у нас на данный момент пока не определены $\emptyset$ и операция $\cap$. Поэтому приходится расписывать это все подробно. Попробуйте разобраться каким именно образом предложение выше выражает желаемый смысл аксиомы.

Напомню, что реально большого смысла эта аксиома не несёт и используется лишь в узких областях, связанных непосредственно с самой теорией множеств — как мы увидим далее, все более-менее стандартные ветви математики никак не опираются на эту аксиому. Парадокс же Рассела разрешается на самом деле с помощью следующей аксиомы:

{\bfseries 3. Comprehension Scheme Axiom}: Для каждой формулы $\phi$ со свободными переменными $v_1, \ldots, v_n, x$ верно, что $\forall z \exists y \forall v_1 \ldots \forall v_n (x\in y \leftrightarrow x \in z \wedge \phi(x, v_1, \ldots, v_n))$.

Опять же выглядит страшно, но это станет понятнее, когда мы изложим её смысл. Изначальный посыл этой аксиомы дать возможность формулировать множества по заданной формуле. Мы говорили уже о о том, что каждый предикат даёт возможность формировать подмножество, и смысл этой аксиомы в том, что нам нужна возможность формировать множества типа $\{x|\phi(x)\}$.

При заданном фиксированном $\phi$ можно было бы сформулировать эту аксиому как-то так: $\exists y \forall x (x\in y \leftrightarrow \phi(x))$. Однако, эта формулировка привела бы все к тому же парадоксу Рассела, если определить $\phi(x) = x \not \in x$. Действительно, в этом случае $\exists y \forall x (x\in y \leftrightarrow x\not \in x)$, и если положить $x=y$ (мы имеем на это право из-за квантора общности), то получим противоречие $y \in y \leftrightarrow y\not \in y$.

Чтобы избежать парадокса Рассела в данном случае, мы должны потребовать, чтобы можно было формировать не произвольные множества, а лишь подмножества некоторых уже существующих множеств: $\forall z \exists y \forall x (x\in y \leftrightarrow x \in z \wedge \phi(x))$. Данная формальная уловка действительно «лечит» парадокс Рассела: теперь для формулы $x \in y \leftrightarrow x \in z \wedge x\not \in x$ всегда можно взять в качестве $y$ пустое множество, и оба атома $x \in y$ и $x \in z$ окажутся в этом случае ложны, что сделает истинной эквиваленцию (вспоминаем первую главу).

Здесь есть два тонких места. Первое заключается в том, что мы пока не определили понятие пустого множества. Однако с помощью Comprehension Axiom это делается довольно легко, если сформировать подмножество некоторого множества $z$ по предикату $x \not= x$: $\emptyset = \{x \in z |x \not= x\}$. Понятно, что такое множество не содержит ни одного элемента, это же можно считать и как определение пустого множества (если бы нашёлся такой $y$, который принадлежал бы определённому нами множеству, то было бы $y\not= y$, что всегда ложно). В силу Extensionality Axiom такое множество единственно, и мы имеем право ввести следующее определение:

{\bfseries Определение.} $\emptyset$ это уникальное множество, такое что $\forall x (x\not\in \emptyset)$.

Второе тонкое место заключается в потенциальной возможности существования множества всех множеств. Если предположить, что существует такое $z$, что $\forall x (x \in z)$, то в формуле $y \in y \leftrightarrow y\in z \wedge y \not \in y$, рассматриваемой выше, даже если $y = \emptyset$ использованный нами для разрешения парадокса, будет верным $y \in z$, и тогда выражение $y \in y \leftrightarrow y\in z \wedge y \not \in y$ окажется ложным. Отсюда следует следующая теорема:

{\bfseries Теорема.} $\neg \exists x \forall y (y \in x)$.

Это и есть теорема о том, что не существует множества всех множеств. Предположение же парадокса Рассела «Пусть $U$ — множество всех множеств» уже само по себе заключает в себе противоречие.

Как теперь видно, парадокс Рассела кроется совершенно не в Foundation Axiom, а в Comprehension Axiom, хотя на первый взгляд казалось очевидным совершенно другое. Реальная причина парадокса не в возможности написать формально $x\in x$ (от такой возможности мы в действительно не можем уйти, если пользоваться определениями формул и теорий как мы их определили), а в слишком вольных терминах при определении подмножеств и самом предположении о существовании множества всех множеств.

Но вернёмся к аксиоме. Общий её смысл я думаю теперь понятен. Переменные $v_i$ необходимы лишь для обобщения аксиомы на случаи формул с несколькими свободными переменными, не более того (рассмотренные нами выше примеры имели лишь одну свободную переменную). Может возникнуть ещё вопрос почему мы саму формулу $\psi$ не снабдили квантором $\forall$, ведь в этом случае мы могли бы записать аксиому без дополнительных слов вроде «для каждой формулы $\psi$...».

Дело в том, что квантор мы имеем право навешивать лишь на переменные, а формула переменной понятное дело не является. Поэтому несмотря на интуитивное желание написать $\forall\psi$, делать мы этого не имеем права. При этом как уже говорилось, любое предложение это некоторая вполне конкретная запись. Предложение не может содержать в себе «некоторую произвольную формулу», потому что в терминах «некоторой произвольной формулы» с формальной точки зрения было бы непонятно как совершать логические выводы и что вообще внутри себя эта формула может иметь.

Поэтому фразу «для каждой формулы $\phi$» надо трактовать как то, что на самом деле для каждой отдельной формулы $\psi$ мы записываем отдельную аксиому Comprehension Scheme с этой вот конкретной формулой. Всего формул у нас бесконечное число, и соответственно аксиом у нас получается бесконечно много. Это, однако, ничему не противоречит, просто это надо иметь ввиду. По этой причине говорят, что $ZF$ — бесконечная система аксиом.

Возможность формировать подмножества с помощью Comprehension Scheme ведёт так же в возможности определить операцию пересечения множеств:

{\bfseries Определение. }$\cap z = \{x \in a| \forall y \in z, x\in y\}$ для произвольного $a \in z$.

Здесь $z$ можно интерпретировать как множество множеств, которые пересекаются операцией $\cap$. Пересечение лишь двух множеств (то что мы определяли ранее) определяется отсюда как $A\cap B = \cap\{A, B\}$.

Возникает желание определить аналогичным образом и объединение множеств. Для объединения, однако, оказывается, что сформулированных до сих пор аксиом недостаточно. Comprehension гарантирует лишь существование подмножеств, но однако для произвольных $a$ и $b$ совершенно не факт, что будет существовать множество, содержащее их обоих. Разным аспектам этого посвящается следующие три аксиомы.

{\bfseries 4. Pairing Axiom}: $\forall a \forall b \exists z (a\in z \wedge b \in z)$.

Напрямую однако эта аксиома не говорит о том, что для любых $a$ и $b$ существует множество $\{a, b\}$. Оно действительно существует, но для того, чтобы его получить, необходимо применить Comprehension Scheme. С помощью Pairing Axiom мы можем гарантировать существование некоторого $z$ который будет содержать $a$ и $b$ в том числе возможно наряду и с некоторыми другими элементами. Однако из этого мы все же можем выразить подмножество $\{a, b\} = \{x \in z| x = a \vee x = b\}$.

Если в рассуждениях выше задать $a = b$, то можно утверждать, что для любого $a$ существует так же множество $\{a\}$, что тоже ранее было не очевидно.

Используя возможность строить множества $\{a, b\}$ мы теперь можем определить и упорядоченные пары как $(a, b) = \{a, \{a, b\}\}$. Выглядит возможно странно, но действительно для множества $\{a, \{a, b\}\}$ всегда можно определить его элементы, и более того их порядок. Это именно то что требовалось от упорядоченных пар, и определить их по-другому конечно было бы можно, но мы все равно как-то были бы обязаны действовать исходя из определения множеств.

{\bfseries 5. Union Axiom:} $\forall f \exists z \forall y \forall x (x \in y \wedge y \in f \rightarrow x \in z)$.

Здесь можно интерпретировать $f$ как некоторое множество множеств, обозначаемых в данном случае как $y$, элементы которых в свою очередь обозначаются как $x$. Множество $z$ — это некоторое множество, содержащее в качестве подмножества все $y \in f$, то есть их объединение. По Comprehension Scheme опять же можно получить и точное объединение множеств:

{\bfseries Определение.} $\cup z = \{x \in a|x\in y, y\in z\}$, где $\forall y (y \subset a)$.

Здесь я для удобства ввёл символ $\subset$. Его можно трактовать как сокращённую форму записи для $\forall x (x\in A \rightarrow x \in B)$, что впрочем мы и определяли в первом параграфе (я просто отмечаю на всякий случай, что в данном случае нет нарушения определённого нами формализма и всё легко определяется).

Аналогично с пересечением можно определить и объединение лишь двух множеств $A\cup B = \cup \{A, B\}$.

Рассмотренные до сих пор аксиомы уже гарантируют нам существование натуральных чисел: для любого $n$ можно определить $x\cup\{x\}$, а так же мы уже имеем $\emptyset$. Не хватает только пока определения функций и отношений, определение которых требует наличия декартова произведения. Его возможно ввести с помощью следующей аксиомы.

{\bfseries 6. Replacement Scheme Axiom:} для каждой формулы $\psi$ со свободными переменными $x, y, v_1, \ldots, v_n$ верно, что $\forall X \forall v_1 \ldots \forall v_n (\forall x\in X \exists ! y \phi(x, y, v_1, \ldots, v_n) \rightarrow \exists Y \forall x \in X \exists y \in Y \phi(x, y, v_1, \ldots, v_n))$.

В этой аксиоме мы использовали сокращение $\exists!y$ для фразы «существует единственный $y$». Запись $\exists!x\phi(x)$ можно рассматривать как сокращённую форму для $\exists x \phi(x) \wedge \neg \exists y (y\not=x \wedge \phi(y))$.

По аналогии с Comprehension Scheme данная аксиома является на самом деле сразу целым набором аксиом, по одной для каждой возможной формулы. Так же по аналогии с Comprehension Scheme переменные $v_1, \ldots, v_n$ служат лишь для обобщения аксиомы и не несут особо глубокого смысла.

Мотивация за этой аксиомой такая: пусть $\phi(x, y)$ — описание некоторой функции $X\to Y$. На самом деле понятия функции у нас пока нет, поэтому правильнее говорить, что это не функция, а формула, такая что $\forall x\in X \exists!y \phi(x, y)$. Логично, что образ этой функции $\{y|\exists x \in X \phi(x, y)\}$ должен являться множеством. Из аксиом 1-5 этого доказать не получится, поэтому это выражает отдельная аксиома Replacement Scheme. Так же как и Pairing Axiom она требует лишь существования некоторого надмножества, содержащего множество $Y$, но далее точное множество $Y$ как обычно можно получить применив Comprehension Scheme.

С помощью Replacement Scheme можно определить декартово произведение. Во-первых вспомним, что для любых $a\in A$ и $b\in B$ мы уже успешно определили упорядоченную пару $(a, b)$. Тогда, если зафиксировать некоторое $b$, можно рассмотреть отображение $a \mapsto (a, b)$. На языке формул это можно записать как $\forall a\in A \exists!z (z = (a, b))$. Образом этого отображения должно быть множество $A\times\{b\} = \{(a, b)| a\in A\}$, которое действительно будет являться множеством по Replacement Scheme.

Определив $A\times\{b\}$ мы можем рассмотреть отображение $b\mapsto A \times \{b\}$ (или в виде формулы $\forall b\in B \exists! z (z = A\times\{b\})$) и по аналогии образом этого отображения будет множество $X = \{A\times \{ b \} | b \in B\}$. Декартово произведение теперь можно определить как $A\times B = \cup X$.

Отсюда можно уже определять отношения, функции, да и вообще все что угодно. Таким образом, всё, рассматриваемое до сих пор, мы смогли определить и доказать используя только первые 6 (даже 5 за вычетом Foundation) аксиом. Следующие аксиомы мы сформулируем очень кратко, так как для наших целей они пока не нужны и потребуются значительно позже, и соответствующие комментарии по их поводу у нас будут позже.

{\bfseries 7. Infinity Axiom:} $\exists x (\emptyset \in x \wedge \forall y \in x ((y\cup\{y\}\in x)))$

Эта аксиома гарантирует, что все натуральные числа все же образует множество $\mathbb{N}$. Без неё мы могли бы работать с натуральными числами, но не могли бы работать с множеством натуральных чисел собственно как со множеством. Например, мы не могли бы определить множество $\mathbb{N}\times\mathbb{N}$, без которых затруднительно определить арифметику отрицательных и рациональных чисел (будет позже в этом курсе).

{\bfseries 8. Power Set Axiom.} $\forall x \exists y \forall z (z\subset x \rightarrow z\in y)$

Гарантирует существование множества $2^X$ для любого $X$. Необходима для определения вещественных чисел (опять же будет позже).

Эти восемь аксиом собственно и образуют аксиоматику ZF. К перечисленным аксиомам почти всегда добавляется ещё следующая «аксиома выбора», вместе с которой аксиоматика носит название ZFC:

{\bfseries 9. Choice Axiom.} Формулировку разобьём на несколько строк:

$\forall A \forall B \forall f\subset A\times B \\(\forall x \in A \exists! y\in B, (x, y)\in f\rightarrow\\ (\exists S\forall y\in B(\exists x\in A, (x, y)\in f \rightarrow \\ \exists!x\in X, (x, y)\in f\cap S\times B)))$

Выглядит, опять же, страшно, но на деле ничего особого. Все, что здесь сказано — это то что из произвольного множества мы можем выбрать некоторый элемент, и на деле формальная запись этой аксиомы никогда не применяется — обычно просто говорят «возьмём некоторый элемент $x$ из множества $X$», и это как раз подразумевает использование аксиомы выбора. Ниже я кратко опишу как выбор элемента из множества связан с формальной записью выше, но это в принципе уже программа максимум — на практике подобные рассуждения да и само приведённое формальное определение совершенно излишни и никем никогда не рассматриваются.

Итак, пусть нам дана функция $f: A\to B$. Каждый $b \in B$ имеет прообразом некоторое множество. Причём для разных $b\in B$ эти прообразы не пересекаются. Мы можем сказать, что на $A$ задано отношение эквивалентности следующим образом: $x\sim y$, если $f(x) = f(y)$. Тогда выбор некоторого элемента из множества — это все равно что выбор элемента из класса эквивалентности. Пусть мы выбрали из каждого такого класса эквивалентности по одному представителю, и сформировали из них множество $S$. Ограничение $f|_S$ будет инъективным. Возможность такого выбора (то есть утверждение аксиомы выбора) эквивалентно возможности такого ограничения произвольной функции $f$, чтобы она стала инъективной.

Осталось формализовать эти понятия. Если $f: A\to B$, то формально это можно записать как $f\subset A\times B \wedge \forall x \in A \exists!y\in B, (x, y)\in f$

Если дополнительно к этому функция $f$ инъективна, то мы можем сказать, что $\forall y\in B (\exists x\in A, (x, y)\in f \rightarrow \exists!x, (x, y)\in f)$. Здесь необходима проверка $\exists x\in A, (x, y)\in f$, чтобы убедиться в том, что конкретный $y\in B$ вообще имеет прообраз.

Теперь, вспомнив, что $f|_S = f\cap S\times B$, мы после некоторых раздумий получаем формулировку аксиомы выбора в том виде, как я её привёл.

Из аксиомы выбора следует множество парадоксов, самый пожалуй душещипательный их которых следующий:

{\bfseries Парадокс Банаха-Тарского.} Любой шар можно разрезать на конечное число кусков, а потом из этих кусков собрать два таких же точно шара.

Этого парадокса не стоит пугаться — в нем нет никаких противоречий и нет ничего непонятного. Мы это докажем далее в нашем курсе, когда вы будете к этому готовы. Пока что же мы отложим подробное рассмотрение аксиомы выбора до времён, когда мы уже освоимся на практике со следствиями остальных аксиом и поупражняемся в работе с бесконечными множествами.

%\section{Теоремы Гёделя}

В прошлом параграфе мы строго определили аксиомы теории множеств. Сейчас мы копнём немного глубже и попытаемся понять насколько приведённый нами формализм полноценен.

Когда мы в первой главе говорили о теориях в самом общем виде, то мы упоминали два возможных свойства теорий: полноту и непротиворечивость. Напомню эти определения (теперь в несколько более формальном виде):

{\bfseries Определение.} Теория $T$ называется {\slshape полной}, если для любого предложения $\phi$ либо $T\vdash\phi$, либо $T\vdash\neg\phi$.

{\bfseries Определение.} Теория $T$ называется {\slshape непротиворечивой}, если ни для какой формулы не может быть одновременно $T\vdash \phi$ и $T\vdash\neg\phi$.

В первой главе мы так же доказывали, что из противоречивых теорий можно вывести любую формулу. Этот факт будет нами использоваться ниже.

Является ли ZFC полной и непротиворечивой? Ответ на этот вопрос дают следующие две теоремы (доказательства мы проведём ниже, а пока только сформулируем и обсудим эти утверждения):

{\bfseries Первая теорема Гёделя о неполноте.} Никакая аксиоматическая система, в которой возможно описать арифметику натуральных чисел, не является полной.

{\bfseries Вторая теорема Гёделя о неполноте.} Ни для какой аксиоматической системы, в которой возможно описать арифметику натуральных чисел, невозможно доказать её непротиворечивость.

То есть результаты прямо скажем неутешительные. Если говорить простыми словами, то здесь говорится о том, что существуют некоторые теоремы, которые мы никогда не сможем доказать, причём доказуемость этих теорем совершенно никак не зависит от системы аксиом, которую мы придумаем. Что бы мы там ни напридумывали вместо ZFC, все равно полученная аксиоматика будет неполной.

Это, впрочем, не особенно большая проблема: недоказуемую теорему можно принять как новую аксиому и спокойно себе жить дальше (можно так же принять и её отрицание, и тоже спокойно себе жить дальше). Вторая теорема Гёделя звучит куда страшнее: если наша теория непротиворечива, то мы никогда не сможем этого доказать. То есть если противоречие есть, мы доказать это можем: достаточно найти формулу $\phi$, чтобы одновременно с ней мы могли вывести и её отрицание. Если же такой формулы в действительности нет и теория наша непротиворечива, то мы этого доказать никогда не сможем. Впрочем, за примерно сотню лет истории ZFC (смотря с какого момента начинать отсчитывать её историю), противоречий найти так и не удалось, поэтому есть надежда на то, что ZFC всё же непротиворечива.

Может возникнуть вопрос, что же случится в случае, если в ZFC все же обнаружат противоречия? На самом деле и тут страшного ничего не приключится: большинство теорий, базой которых является теория множеств, могут быть переформулированы и без неё, хотя и менее удобно. Например, мы могли бы сформулировать теорию ориентированных графов с петлями не прибегая к множествам следующим образом: константные символы мы бы разбили на вершины и дуги, и ввели бы операции $\mathrm{dom}$ и $\mathrm{codom}$, которые каждой дуге ставили бы в соответствие её начальные и конечные вершины. Формально дугу от вершины можно было бы отличить формулой типа $\exists v, v=\mathrm{dom}(a)$, которая истинна тогда и только тогда, когда $a$ является дугой. Продолжая подобные рассуждения можно было бы сформулировать целиком теорию графов, а так же сформулировав дополнительные аксиомы избавиться от ориентированности дуг и от петель. Другое дело, что определение теорий в таком виде оказывается куда сложнее, чем на языке теории множеств, однако при этом мы уже не зависим от ZFC и не страшимся её противоречий. Аналоги для теорий, переформулированные без теории множеств, обычно называются метатеориями. Так, сформулированное мной только что можно было бы назвать метаграфом. Никто всерьёз, правда, таким не занимается, да и если найдутся все же противоречия в ZFC, они скорее всего будут исправлены переформулировкой аксиом, а в целом фундамент теории множеств останется таким же.

Здесь полезно вспомнить так же и о моделях. В первой главе мы вводили их довольно неформально, но теперь мы можем определить понятие модели более-менее строго:

{\bfseries Определение.} {\slshape Сигнатурой} называется набор нелогических символов с правилами их использования, то есть с указанием, являются ли они символами операции, отношения либо константными, и указанием количества параметров, используемых вместе с этими символами.

Мы можем говорить о сигнатуре некоторой теории или отдельной формулы. Например, сигнатуру ZFC можно описать единственным символом $\in$, для которого указано, что это символ отношения, и что он имеет ровно два параметра. Любая формула ZFC имеет ту же сигнатуру. То есть сигнатура это в некотором смысле язык, на котором мы описываем нашу теорию.

{\bfseries Определение.} {\slshape Структурой} заданной сигнатуры называется такое множество формул $S$, что для любой формулы заданной сигнатуры либо $\phi\in S$, либо $\neg\phi \in S$ (если $\phi\in S$, то это обозначается как $S\models\phi$).

{\bfseries Определение.} Структура $M$ теории $T$ называется {\slshape моделью} этой теории, если сигнатуры структуры $M$ и теории $T$ совпадают, и если для любого предложения $T\vdash\phi$ так же верно $M\models\phi$.

Фактически структура — это набор утверждений об истинности конкретных формул. Модель — структура, в которой истинны все формулы заданной теории.

В терминах теории моделей можно переформулировать понятия полноты и непротиворечивости следующим образом:

{\bfseries Определение.} Теория называется {\slshape полной}, если для неё существует единственная модель.

{\bfseries Определение.} Теория называется {\slshape непротиворечивой}, если для неё в принципе существуют модели.

Эквивалентность двух определений непротиворечивости довольно очевидна. Если теория полна, то очевидно так же, что она имеет единственную модель. Если моделей теория имеет несколько, то это означает, что существуют предложения, которые могут быть ложны либо истинны в зависимости от модели, и они стало быть не могут быть выведены.

Однако с моделями имеется определённая напасть: чтобы сформулировать определение модели, мы опять использовали понятие множеств, и стало быть мы закладывались на непротиворечивость ZFC. Может показаться, что в таком свете теория моделей становится бесполезной для изучения теории множеств, однако это не так (хотя многие математики избегают при определении моделей пользоваться терминологией теории множеств, и говорят, что структура $S$ это не множество, а просто набор записей вида $S\models\phi$ либо $S\models \neg\phi$ для каждого предложения $\phi$, что в общем-то конечно избегает использования ZFC, но не приводит к каким-то сильно ощутимым бонусам). Вполне справедливы и полезны рассуждения, когда из самой гипотезы непротиворечивости мы определяем понятие модели и предполагаем существование хотя бы одной из них, а отсюда мы уже доказываем существование каких-то других моделей. Так, например, если предположить непротиворечивость ZF, то из существования хотя бы какой-то модели ZF (неизвестно даже точно какой именно), мы можем используя конструкции теории множеств показать, что будут неминуемо существовать ещё по крайней мере две модели: та, в которой верна аксиома выбора, и та, в которой она неверна.

Мы пока не будем углубляться в подобные доказательства (хотя в последующих главах я приведу наброски подобных рассуждений), а отметим просто такой факт: если в неполной теории мы нашли невыводимую формулу $\phi$, то мы можем прибавить её к нашему набору аксиом и вновь получим непротиворечивую аксиоматику. А можем прибавить и отрицание этой формулы, и полученная аксиоматика так же будет непротиворечива. Нечто подобное произошло с геометриями Евклида и Лобачевского: в евклидовой геометрии утверждается, что через точку, не лежащую на заданной прямой, возможно провести лишь одну прямую, параллельную заданной, а в геометрии Лобачевского утверждается, что таких параллельных прямых окажется бесконечно много. И причём если геометрия Евклида непротиворечива, то именно из геометрии Евклида можно вывести и непротиворечивость геометрии Лобачевского. Несколько подробнее, хотя тоже не формально, а в виде исторического очерка, мы рассмотрим это в следующем параграфе.

Перейдём теперь непосредственно к доказательству теорем Гёделя.

{\bfseries Доказательство первой теоремы.} Для начала мы сделаем набросок доказательства, опускающий ряд технических деталей, чтобы была просто понятная идея. Эти технические детали мы ниже сформулируем отдельной леммой с доказательством.

Предположим, что каждую формулу и каждое доказательство можно каким-то образом пронумеровать. Номер формулы  будем обозначать символом $g$, а номер доказательства символом $G$. Эти номера мы будем называть {\slshape гёделевскими номерами}. Так же предположим, что мы можем каким-то образом записать формулу $pf(x, y)$, которая обращается в истину тогда и только тогда, когда $x$ является номером доказательства для формулы под номером $y$.

Идея доказательства  сводится к формулированию такого предложения теории, которое каким-то образом ссылалось бы само на себя. В общих чертах это делается таким образом: если рассмотреть формулу $F(x)$, имеющую одну свободную переменную, то для неё существует некоторый номер $n=g(F)$. Тогда формула $F(n)$ будет косвенно ссылаться на саму себя. Наша задача сейчас сводится к тому, чтобы сформулировать $F(x)$ таким образом, чтобы она заключала в себе утверждение о доказуемости формулы с номером $x$ (вспомните теорему из первой главы «эту теорему невозможно доказать» — мы сейчас как раз занимаемся построением такой теоремы).

Чтобы получить в теореме ссылку на саму себя, определим формулу $q(x, g(F)) = \neg pf(x, g(F(g(F))))$. То есть если словами, то $q(x, g(F))$, означает, что $x$ не является номером доказательства для $F(g(F))$. Последнее как раз и содержит утверждение, ссылающееся само на себя. Сама форма $q(x, y)$ вначале преобразовывает $y$ (некоторый номер) в номер формулы со ссылкой на себя, и обращается в истину только тогда, когда $x$ не является доказательством для полученной формулы. Возможность формирования такой формулы мы опять же докажем ниже как лемму.

Теперь мы можем рассмотреть такую формулу: $P(x) = \forall y q(y, x)$. Она, во-первых, имеет некоторый номер $g(P)$, во-вторых имеет свободную переменную $x$, и в третьих содержит утверждение о недоказуемости $x$, преобразованного в формулу, ссылающуюся на саму себя. Тогда если мы подставим вместо $x$ номер $g(P)$, то получим $P(g(P))$, а это то что нам и требовалось.

Действительно, предположим, что $P(g(P))$ доказуемо. По определению это означает, что $\forall y q(y, g(P))$ и то есть $P(g(P))$ не может быть доказано. Противоречие.

Предположим обратное. Пусть доказуемо отрицание $\neg P(g(P))$. Это значит, что $\exists y \neg q(y, g(P))$, то есть существует доказательство $P(g(P))$, и последнее должно быть истинно. Опять же противоречие.

Мы получили, что ни предложение $P(g(P))$ ни $\neg P(g(P))$ действительно не могут быть доказаны, а это и есть утверждение о неполноте. \qed

Обещанную лемму мы теперь сформулируем так:

{\bfseries Лемма 1.} Существует такая нумерация формул и доказательств, что в этой нумерации возможно определить формулы $pf(x, y)$ и $\pi^x_n(g(F))$, где $\pi^x_n$ по номеру формулы $F(x)$ даёт номер формулы, в которой свободная переменная $x$ заменена натуральным числом $n$.

Формула $\pi^x_n(g(F))$ необходима для того, чтобы иметь возможность сформулировать $q(n, F(x))$ — без неё мы не могли бы подставить вместо $x$ какое-либо значение.

Доказательство и уточнение формулировки мы проведём чуть ниже, а пока докажем сразу и вторую теорему:

{\bfseries Доказательство второй теоремы.} Опять же мы приведём лишь набросок доказательства, а повысим планку строгости в следующем параграфе. Мы пока допустим здесь некоторую вольность, и не будем различать сами формулы и их номера, поскольку именно в данном доказательстве это и не принципиально. Пусть $p$ — недоказуемая ({\slshape гёделевская}) формула, введённая при доказательстве первой теоремы. Тот факт, что она недоказуема, было нами строго доказано, то есть если ввести формулу $P(f) = \exists x, pf(x, f)$, то мы можем утверждать, что $ZF\vdash \neg P(p)$. Однако теперь можно вспомнить смысл формулы $p$, который формулируется так: «Данное предложение не может быть доказано». Но если мы доказали $\neg P(p)$, то мы значит доказали её недоказуемость, и, соответственно доказали $p$! Это похоже на порочный круг, приводящий к противоречиям, и нам надо разобраться откуда он появился. Проблема собственно в последнем следствии, которое мы сделали: из $\neg P(x)$ мы предположили, что не может быть одновременно с тем $P(x)$, то есть мы сделали предположение о непротиворечивости теории. Значит, это предположение было ложным, и такое предположение мы делать не имели права. Важно, что речь тут идёт не о реальной противоречивости теории, а лишь о предположении непротиворечивости.

Несколько поднимем планку строгости и рассмотрим приведённое рассуждение более формально. Пусть $Con(ZF)$ — формула, соответствующая непротиворечивости ZF (её можно определить, например, как $\neg P(\exists x\not=x)$, поскольку $\exists x \not= x$ выводима тогда и только тогда, когда ZF противоречива). В предположении непротиворечивости ZF, и, следовательно, невыводимости $p$, мы можем утверждать существование моделей, в которых $p$ истинно, и моделей, в которых $p$ ложно. В одном случае будет выполняться импликация $Con(ZF)\rightarrow p$, в другом случае импликация $Con(ZF)\rightarrow \neg p$. Мы предполагаем, что мы можем доказать непротиворечивость ZF, то есть $ZF\vdash P(Con(ZF))$. Довольно легко видеть, однако, что если $T\vdash P(A)$ и $T\vdash P(A\rightarrow B)$, то так же $T\vdash P(B)$. Получается, что в некоторых моделях при предположении $P(Con(ZF))$ будет $P(p)$, а в некоторых $P(\neg p)$, однако обе эти формулы противоречат определению $p$ и первой теореме Гёделя о неполноте. Следовательно, моделей, в которых $P(Con(ZF))$, не существует (что однако не говорит ничего об истинности самого $Con(ZF)$). \qed

Отмечу, что я приводил доказательство для ZF лишь для того, чтобы не отвлекаться на лишние слова. То же самое доказательство можно провести в рамках любой теории, которая допускает определение натуральных чисел.

Докажем теперь используемые нами предположения из этих двух доказательств. Начнём с упомянутого во втором доказательстве утверждения, поскольку оно проще и короче.

{\bfseries Лемма 2.} Если $T\vdash P(A)$ и $T\vdash P(A\rightarrow B)$, то и $T\vdash P(B)$.

{\bfseries Доказательство.} Обратим вначале внимание на то, что $T\vdash X$, тогда и только тогда, когда $T\vdash P(X)$ по определению $P$. Тогда:

1) $T\vdash P(A)$  по условию;

2) $T\vdash A$ по определению $P$;

3) $T\vdash P(A\rightarrow B)$ по условию;

4) $T\vdash A \rightarrow B$ по определению $P$;

5) $T\vdash B$ по правилу modus ponens из 2 и 4;

6) $T\vdash P(B)$ по определению $P$. \qed

Доказательство леммы 1 требует однако некоторых знаний из базовой арифметики и теории чисел. Мы пока этим не занимались, и поэтому провести полноценное доказательство не сумеем. Тем не менее, сформулировать доказательство нам удастся почти целиком, так как свойства чисел используются лишь в одном небольшом фрагменте, который останется пока недоказанным, и закроем эту дырку мы в следующей главе.

Сразу заметим, что перенумеровать все формулы и доказательства вообще говоря не составляет труда: мы можем расположить все формулы по порядку возрастания их длины, а формулы одинаковой длины мы можем расположить «по алфавиту» (задав некоторый порядок символов). Доказательства формул можно перенумеровать аналогично. Такой порядок доказывает саму возможность нумерации, но ниже мы увидим, что он не позволит целиком доказать лемму. Пока мы будем использовать его, а ниже станет понятно какой именно нумерация должна быть, чтобы удовлетворить нашим запросам, и уже с этим знанием сформулируем требуемую нумерацию в третьей главе.

Конструировать формулу $pf(x, y)$ мы будем в виде последовательности лемм, объединение которых и даст нам требуемое доказательство.

Во-первых, вспомним, что доказательство — это конечная последовательность формул $(x_1, \ldots, x_n) \in \mathbb{N}^n$. В формуле $pf(x, y)$ первым параметром является гёделевский номер такой последовательности, определяемый инъективным отображением $G: \mathbb{N}^n \to \mathbb{N}$. Нам надо доказать здесь две вещи: во-первых, что нашу нумерацию мы можем выразить в ZF, и во вторых, что мы всегда можем построить отображение, обратное к нему. Этому соответствуют следующие две даже более общие леммы:

{\bfseries Лемма 3.} Для любого $S\subset \mathbb{N}^n$ существует инъекция $S \to \mathbb{N}$.

{\bfseries Доказательство.} В качестве упражнения. В дальнейшем мы будем доказывать более общее утверждение. \qed

Обратите внимание, что данная лемма утверждает лишь возможность построить функцию (в отличие от формулы ZF) для нумерации доказательств, но не формул. То есть хоть мы и можем сопоставить каждой формуле гёделевский номер, мы не сможем построить функцию — областью определения в этом случае является набор формул, который мы не имеем права рассматривать как множество в соответствии с нашей формальной аксиоматикой (это та же причина, по которой Comprehension Scheme и Replacement Scheme мы формулировали как бесконечный набор аксиом, а не через кванторы).

{\bfseries Лемма 4.} Для любого отношения существует обратное.

{\bfseries Доказательство.} Элементарно. Пусть $r\subset A\times B$. Тогда обратное отношение определяется как $\{(b, a)\in B\times A|(a, b)\in r\}$. \qed

Итак, на данном этапе мы имеем конечную последовательность $(x_1, \ldots, x_n)$ и номер формулы $y$. Теперь, используя композицию с $G$, мы можем свести задачу построения функции $pf$ к построению аналогичной функции, но только работающей не с номером доказательства, а с последовательностью номеров формул. Когда у нас есть последовательность формул, мы должны во-первых показать $x_n = y$, и во-вторых показать, что сама последовательность является последовательностью корректных выводов.

{\bfseries Лемма 5.} Для любой конечной последовательности $s = (x_1, \ldots, x_n)$ и любого $i\le n$ возможно построить функцию проекции $p_i: (x_1, \dots, x_n)\mapsto x_i$.

{\bfseries Доказательство.} Здесь под знаком $i \le n$ подразумевается $i \subset n$ — именно так определяется отношение порядка на натуральных числах, как мы это увидим в следующей главе. Каждая конечная последовательность — это функция $f\in\mathbb{N}^n$. Тогда требуемая проекция — это значение $f(i)$. \qed

{\bfseries Лемма 6.} Из любой конечной последовательности $s = (x_1, \ldots, x_n)$ можно получить множество её элементов.

{\bfseries Доказательство.} Поскольку $s$ — функция, то множество элементов последовательности будет множеством значений этой функции $\mathrm{Im} s$. \qed

{\bfseries Лемма 7.} Из любой конечной последовательности $s = (x_1, \ldots, x_n)$ можно получить начальный отрезок этой последовательности $s_i = \{x_1, \ldots, x_i\}$

{\bfseries Доказательство.} Достаточно взять ограничение $s\cap i\times \mathbb{N}$. \qed

Пользуясь этими леммами мы легко можем получить последний элемент последовательности $x_n$ для сравнения его с $y$. Так же мы легко можем рассмотреть любой начальный фрагмент последовательности, чтобы убедиться в том, что каждый номер в нем действительно является корректным следствием предыдущих номеров.

Следствие — это значит некоторое правило вывода. Например, правило вывода modus ponens можно определить как $mp: g(a), g(a\rightarrow b)\mapsto g(b)$. Если определить множество всех правил вывода за $R$, то отношение, проверяющее, что $y$ следует из последовательности $s$ можно легко сформулировать с помощью формулы $\exists u\in \mathrm{Im} s \exists v \in \mathrm{Im} s \exists r\in R ((u, v), y) \in r$.

Здесь мы предположили, что все правила в качестве начальных параметров берут два номера формул. Это легко обобщается на случай многих входных параметров.

Единственное, что нам осталось для написания формулы $pf$ — это определить множество $R$, то есть сформулировать на языке формальных формул все правила вывода как арифметические операции над номерами. Здесь уже возникает проблема. Получить формулу по номеру мы на самом деле не можем — отображения из формул на натуральные числа вообще говоря не существует, поскольку формулы теории ZF вообще не образуют множество, они являются лишь предложениями языка, на котором мы описываем нашу теорию. Поэтому здесь и возникает потребность не просто в абы какой нумерации формул, а именно в нумерации, которая позволяла бы легко на языке ZF формулировать правила вывода. На данный момент мы такую нумерацию определить не способны, потому что здесь потребуются уже базовые свойства чисел, о чем мы будем говорить лишь в следующей главе.

Аналогичная ситуация с определением функции $\pi^x_n$. Здесь прежде всего надо уточнить что именно она делает. Фраза «заменяет вхождения $x$ на $n$» звучит просто и понятно, но на самом деле не совсем корректна, поскольку $x$ — свободная переменная, а $n$ — вполне конкретное натуральное число, являющееся вполне конкретным множеством в нашей терминологии. Поэтому в действительности замена, которая происходит, может неформально быть описана функцией $\pi^x_n: F(x)\mapsto \exists x F(x)\wedge x=n$. Осталось только описать это на языке арифметики, чтобы работа велась не с формулами (на это мы не имеем права с формальной точки зрения), а с натуральными числами. Здесь та же ситуация что и с определениями правил вывода: для того чтобы легко манипулировать формулами на языке их номеров, нам нужна соответствующая нумерация. Эту дырку в доказательстве мы закроем в следующей главе.

Теперь, если по клочкам собрать в единое целое все написанное выше, то мы как раз и получим формулу для $pf$. Выглядеть она будет совершенно чудовищно (и я даже не уверен записывал ли её реально хоть кто-нибудь), но однако мы доказали её существование за вычетом нумерации. Этим доказательство первой теоремы Гёделя можно считать завершённым.

В качества фаталити предлагаю попробовать доказать следующие две теоремы:

{\bfseries Упражнение.} Докажите, что для любой формулы $\phi$ с одной свободной переменной существует предложение $\psi$, такое, что $ZF\vdash \psi \leftrightarrow \phi(g(\psi))$.

{\bfseries Упражнение.} Применив утверждение прошлого упражнения, докажите {\slshape теорему Тарского о неопределяемости истины}: не существует формулы, которая по гёделевскому номеру могла бы сказать является ли предложение, соответствующим данному номеру, истинным или ложным.

\section{История}

Изложенный выше строгий аксиоматический подход применялся на самом деле далеко не всегда в математике. Первым его применил Евклид в своей книге «Начала» около 300 года до нашей эры, где сформулировал пять аксиом, из которых выводил все остальные геометрические теоремы. Формулировались эти аксиомы следующим образом (не точно, но нам в дальнейшем будет удобно ссылаться на них в этом виде; на деле аксиомы Евклида дошли до нас в различных формулировках):

1. Через любые две точки возможно провести прямую, причём только одну.

2. Из точки, не лежащей на прямой, возможно опустить на неё перпендикуляр.

3. Пусть задана прямая и отмеченная на ней точка, а так же некоторый отрезок; тогда от этой точки можно отмерить ещё две точки на прямой ровно на расстоянии заданного отрезка.

4. Пусть задана точка и отрезок; тогда можно задать окружность с центром в этой точке и с радиусом, равным данному отрезку.

5. Пусть есть прямая и точка на ней не лежащая; тогда через эту точку возможно провести прямую, параллельную заданной, и притом только одну.

Конечно Евклид не формулировал свои теоремы так строго как это делали выше мы, и не использовал никакого специального логического формализма. Тем не менее он был первым, кто вообще подошёл к математике строго. До него геометрия была набором разрозненных фактов слабо связанных между собой и не имела единого начала. Проводились строгие доказательства, но опирались эти доказательства часто на довольно интуитивные представления, а не на какие-то строго доказанные из аксиом утверждения.

Надо сказать, что сама аксиоматика Евклида по нынешним меркам была довольно отвратительна. Во-первых, конечно, она была не полна (и не могла быть полна, поскольку натуральные числа возможно определить геометрически). В первоначальной формулировке геометрии Евклида было невозможно доказать следующее простое утверждение: «Пусть есть две окружности, центры которых располагаются ближе друг к другу, чем их удвоенные радиусы. Пересекаются ли эти окружности?». Утверждение очевидное, если нарисовать чертёж, но аксиомы Евклида его доказать не позволяют.

Аксиомы Евклида позднее многократно расширялись, и наиболее полное расширение аксиом Евклида ныне принадлежит Гильберту и состоит из 20 аксиом. Впрочем, современная геометрия целиком стоит на фундаменте теории множеств и векторной алгебры, и то что раньше называлось аксиомами в новых реалиях является простенькими теоремами, которые можно доказать.

Из пяти аксиом Евклида последняя пятая вызывала большие сомнения. «Параллельность» означает отсутствие пересечения где-либо, в том числе «далеко-далеко». Если внимательно присмотреться, то пятая аксиома является единственной, которая неявно опирается на представление о бесконечности: как бы далеко мы не ушли от первоначальной точки, мы никогда не увидим пересечения прямых. Но поскольку прямая — это объект бесконечный, то нам надо убедиться, что пересечения нет нигде в бесконечности.

Сегодня понятие бесконечности довольно привычно всем в силу школьной программы и научной фантастики, но во времена Древней Греции понятие бесконечности было в новинку. Причём важно, что если сегодня мы принимаем формальный подход, где мы говорим о переменных и константах, а аксиомы — это просто формулы, сформированные по нашим правилам вывода, то в древние времена математики пытались описывать реальный мир. И если сегодня мы безоговорочно принимаем понятие прямой как объект бесконечной протяжённости, то древние греки пытались ответить на вопрос могут ли существовать такие объекты в реальности.

По этой причине многие люди, да и сам Евклид, очень недолюбливали пятую аксиому. Сам Евклид в своих «Началах» откладывал использование пятой аксиомы до последнего, и первые 28 теорем доказаны без использования её. Та часть геометрии, которую возможно рассматривать без привлечения пятой аксиомы, получила позже название «абсолютной геометрии».

Саму же пятую аксиому долгое время пытались доказать (и иногда опровергнуть) опираясь на первые четыре аксиомы. Успехов на этом поприще никто не добился.

В 1829 году Лобачевский в своей работе «О началах геометрии» первым предположил, что на самом деле пятая аксиома скорее всего не может быть доказана вообще никаким образом из первых четырёх, и что в этом предположении возможно вместо пятой аксиомы принять противоположное утверждение: «Пусть есть прямая и точка на ней не лежащая; тогда через эту точку возможно провести любое количество прямых, параллельных заданной». Развивая геометрию, опираясь на альтернативной пятой аксиоме, он развил полноценную геометрическую теорию, правда он так и не смог доказать её непротиворечивость или независимость пятой аксиомы от первых четырёх. Сам Лобачевский также считал именно Евклидову геометрию «правильной», а свою же теорию он сам называл «воображаемой геометрией», хотя и видел в ней перспективы для практического применения.

Первым доказал состоятельность геометрии Лобачевского итальянский математик Эудженио Бертрами в своей работе 1868 года. Модель, которая в России чаще всего носит название модели Клейна, выглядит как диск, в пределах которого и изображаются прямые:

\includegraphics{klein.png}

Эта модель предполагает, что наше пространство на самом деле как бы ограничено, и поэтому в силу именно ограниченности мы можем изобразить какое угодно количество прямых, не пересекающих заданную и проходящих через заданную точку. Единственный сложный вопрос в этой модели в измерении расстояния. При движении к краю окружности масштаб длин должен увеличиваться, чтобы модель удовлетворяла 3 и 4 аксиомам, и одно и то же расстояние в центре диска Клейна и на его краю будет выглядеть сильно по-разному.

Эта модель интересна тем, что она целиком формулируется в терминах геометрии Евклида. То есть модель геометрии Лобачевского получается из непротиворечивости геометрии Евклида: если непротиворечива евклидова геометрия, то непротиворечивой окажется и геометрия Лобачевского, хотя обе они содержат противоречащую друг другу аксиому. Позже выяснилось, что геометрия Лобачевского является не только интересным теоретическим построением но имеет физический смысл: скорости в теории относительности Эйнштейна описываются как раз законами геометрии Лобачевского.

Я к сожалению забыл называние и автора интересной книжки о вымышленном мире, который представлял собой шар в евклидовой геометрии, населённый людьми. Этот шар подчинялся следующему закону: чем ближе человек находился к границе шара, тем меньше он становился по размеру и тем большими для него оказывались расстояния. Эти вымышленные человечки проводили физические эксперименты, и таким образом выяснили, что их вселенная бесконечна и подчиняется законам геометрии Лобачевского. То что на самом деле они живут в евклидовом шаре с особыми свойствами они никаким образом понять не могли.

Как подобный шар и подобные люди могут быть описаны математически мы увидим позже, но сама подобная интерпретация довольно интересна тем, что показывает ограниченность возможности познания: как эти сферические люди ни будут пытаться, они никогда не смогут выяснить что находится за пределами их шара и ничего не узнают о геометрии Евклида, которой на самом деле подчиняется их мир.

Нам же этот пример с геометриями интересен прежде всего тем, что он является хорошей демонстрацией неполноты теории: в абсолютной геометрии утверждение пятой аксиомы является невыводимым, и его можно в результате принять как ещё одну аксиому, либо же можно принять как аксиому противоположное утверждение, и оно так же будет непротиворечиво.

Но все сказанное до сих пор относилось лишь к геометрии. Аксиоматизация арифметики натуральных чисел происходила гораздо позднее, чем геометрии, и вплоть до XIX века манипуляции с натуральными числами была неформализованы, и далеко не все видели в такой формализации вообще необходимость. Математик Леопольд Кронекер говорил: «Бог создал натуральные числа, а всё прочее — дело рук человеческих». На сегодняшний день самой известной формализацией (после изложенной мной в этой главе) является система аксиом Пеано, изложенная им в 1889 году в его книге «Принципы арифметики представленные новым методом».

Его аксиомы определяли множество натуральных чисел с помощью константы $0$ и операции инкремента $S$, однако как конкретно выглядят эти константы и операция на языке теории множеств не указывалось, а вместо этого формулировался ряд довольно косвенных свойств, которые были достаточны для определения арифметики:

1) Равенство чисел определялось как отношение эквивалентности.

2) Для любого $n\in\mathbb{N}$ формула $S(n) = 0$ определялась как ложная.

3) $S$ объявлялась инъективной функцией.

4) Оговаривался принцип индукции: если $\phi(0)$ истинно и $\forall n\in\mathbb{N}, \phi(n)\rightarrow \phi(S(n))$, то $\forall n\in\mathbb{N}, \phi(n)$.

Операции сложения и умножения чисел определялись следующим образом:

5) $a+0 = a$

6) $a+S(b) = S(a + b)$

7) $a0 = 0$

8) $aS(b) = ab + a$

Этих аксиом и определений оказывалось достаточным для доказательства всех базовых арифметических свойств.

{\bfseries Упражнение.} Докажите из аксиом Пеано коммутативность умножения $ab=ba$.

Долгое время производились попытки доказать непротиворечивость и полноту аксиом Пеано. Эти попытки были окончательно разбиты в 1931 году, когда Гёдель представил свои теоремы о неполноте. Здесь надо сказать, что Гёдель доказывал теоремы несколько не в том виде и несколько не так, как я представил их в прошлом параграфе. Он был нацелен именно на аксиомы Пеано и использовал в основном аппарат математической логики, а не теории множеств, как это сделал я. Впрочем, у нас не стояло задачи сформулировать максимально строго теорему Гёделя, а лишь понять её следствия, существенные для большинства математиков.

Для определения натуральных чисел аксиомы Пеано сегодня полностью вытеснены более удобными определениями ZFC. Определение сложения и умножения аксиом Пеано тем не менее осталось довольно полезным, поскольку оно обобщается, и с помощью них  можно удобно определять операции над числами более общего вида, так называемыми ординалами.

ZFC сейчас является базой для 99\% всей существующей математики, хотя детали аксиом и правил вывода чаще всего не используются и большинство математиков с ними не знакомы. Вообще основания математики в виде логики и теории множеств существуют довольно отдельно от остальной математики. Терминология теории множеств используется просто как удобный язык, а из логики математики вообще редко прибегают к чему-либо.

Изначально математическая логика развивалась больше как интересное философское направление, позже она постепенно прилаживалась для формализации аксиоматических систем, с развитием компьютерной техники были надежды, что она может стать основой для искусственного интеллекта, но на деле эти надежды оказались сильно преувеличенными. Сейчас математическая логика живёт в значительной степени отдельно от остальной математики и какие-то логические результаты лишь изредка вырываются в мир «большой математики». Пожалуй сколько-нибудь значимыми для широких кругов математиков являются лишь рассмотренные в прошлом параграфе теоремы Гёделя, а так же теорема Лёвенгейма-Скулема, которую мы сможем понять лишь позже. Впрочем, даже эти результаты мало кому известны и используются главным образом не сами они, а их следствия.



\chapter{Натуральные числа}
В этой части наконец-то вводится понятие натурального числа. Основные две темы ~--- само понятие натурального числа (включая минимум из теории чисел) и комбинаторика. Базовые комбинаторные формулы сами по себе часто оказываются полезны в самых разных областях математики, так же на них легко и приятно отрабатывать основные навыки работы с базовыми математическими объектами. В более абстрактные области мы уйдём со следующей главы, эта же глава будет во многом простая, расслабляющая и лёгкая для чтения (возможно, кроме первого параграфа, который в принципе многим читателям будет необязателен).

\section{Определение}

Натуральные числа~--- это 0, 1, 2, 3, и~т.д. Все это вроде знают. Но как определить понятие натурального числа строго? Чтобы оценить задачу, прежде чем читать дальше, попробуйте дать такое определение самостоятельно, заодно с определение арифметических операций и не опираясь на физическую интуицию, а лишь на логику. Замечу, что этот параграф носит малоприкладной характер~--- его цель лишь в определении натуральных чисел. Я же не могу написать: <<а с этого момента давайте использовать натуральные числа>>. Определить я их обязан как-то, но в то же время подробные определения, которые я здесь привожу, вряд ли будут кому-то действительно полезными. Поэтому данный параграф можно читать наискосок либо не читать вообще, если вы помните свойства натуральных чисел.

Дать строгое определение пытались многими простыми способами, но все более-менее интуитивные подходы неизменно заводят в тупик. На данный момент мейнстримом в определении натуральных чисел являются два подхода: определение непосредственно на основе теории множеств (определение Фреге-Рассела) и чуть более сложный, но также важный в силу полезных обобщений, подход на основании аксиом Пеано. Мы не будем рассматривать подробно и формально эти подходы со всеми выкладками, а лишь рассмотрим суть этих определений. Недостающие пробелы вы можете закрыть сами~--- это довольно большая работа, но без каких-либо принципиальных сложностей. Более подробно и обстоятельно мы вернёмся к аксиоматизации натуральных чисел позже в шестой главе, когда будем говорить о бесконечных множествах.

Определение Фреге-Рассела отражает идею о том, что натуральные числа определяют количество элементов во множествах. Это довольно понятно на пальцах, но надо дать строгое определение. Первая идея, которая обычно приходит на ум~--- это взять вообще все множества в принципе и разбить их на классы эквивалентности так, чтобы в одном классе оказались множества одинакового размера.

Ввести такую эквивалентность для множеств несложно: в \S~2.4 мы уже вводили понятие равномощности. В соответствии с тем определением, два множества $A$ и $B$ называются равномощными, если существует некоторая биективная (то есть взаимооднозначная) функция $f:A\to B$. Пусть например у нас есть множество трёх букв алфавита $A=\{a, b, c\}$ и странное множество $B=\{\heartsuit, \clubsuit, \spadesuit\}$. Эти два множества равномощны, так как существует биекция $$f=\{(a, \clubsuit), (b, \heartsuit), (c, \spadesuit)\}$$ --- то есть существует способ назначить каждому элементу множества $A$ некий элемент множества $B$ и обратно. Если мы рассмотрим теперь множество с одним дополнительным элементом $C=\{\heartsuit, \clubsuit, \spadesuit, \Diamond\}$, то увидим, что никаких способов задать тут биекцию не существует, и стало было множество $C$ не равномощно $A$ и $B$.

\begin{exercise}
Докажите, что на любом множестве, состоящим из множеств, отношение биекции~--- это отношение эквивалентности.
\end{exercise}

Теперь, когда у нас есть некое отношение эквивалентности, мы хотим задать классы эквивалентности~--- и эти классы эквивалентности мы могли бы объявить натуральными числами, тогда каждое натуральное число представляло собой класс всех множеств одинаковой мощности. К сожалению, поступить таким образом мы не можем, так как классы эквивалентности могут быть построены лишь на некотором множестве, и в нашем случае было бы необходимым рассматривать множество всех множеств. Последнего, однако, как мы показали в \S~2.5, не существует.

Тем не менее, эта идея всё же может быть доведена до конца, если вместо задания сразу всего класса эквивалентности, мы зададим лишь по одному представителю из этого класса, а затем, чтобы определить принадлежность некоторого множества к классу, мы будем сравнивать это множество с представителями классов. Это похоже на то, что делают в физике при измерениях: вначале определяют некий один эталонный объект (скажем, метр), а затем все остальные сравнения производятся уже относительно него.

Самый маленький класс множеств~--- это пустое множество. Представителем этого класса логично выбрать $\emptyset$. Обозначим его как $$0 = \emptyset$$

Следующим за ним класс должен состоять из только одного элемента. Этим элементом можно взять как раз число 0, и определить представителя нашего нового класса как $$1 = \{0\} = \{\emptyset\}$$

Теперь определим представителя для ещё более крупного множества, в котором на один элемент больше. Для этого естественно использовать уже имеющиеся у нас элементы 0 и 1: $$2 = \{0, 1\} = \{\emptyset, \{\emptyset\}\}$$

Ну и до кучи: $$3 = \{0, 1, 2\} = \{\emptyset, \{\emptyset\}, \{\emptyset, \{\emptyset\}\}\}$$

Процедура, в общем-то, проста. Если у нас есть число $n$, то мы можем определить следующий за ним элемент: $$S(n) = n \cup \{n\} = \{0, 1, \ldots, n\}$$

Применяя эту процедуру бесконечное число раз, мы можем получить все натуральные числа. Это на самом деле не столь очевидно, но Infinity Axiom гарантирует нам, что подобным образом действительно возможно определить некое множество, которое мы будем обозначать как $\mathbb{N}$ и называть его множеством натуральных чисел. Теперь мы готовы для наших первых определений.

\begin{definition}
Множеством \term{натуральных чисел} $\mathbb{N}$ называется множество чисел, полученных из $0=\emptyset$ применением функции $S$.
\end{definition}

\begin{definition}
\term{Последовательностью} элементов множества $X$ называется функция $x:\mathbb{N}\to X$. Обозначается последовательность как $\{x_n\}$.
\end{definition}

\begin{definition}
\term{Конечной последовательностью} элементов множества $X$ называется функция $n \to X$.
\end{definition}

\begin{definition}
Множество $X$ называется \term{конечным}, если существует равномощное ему множество $n\in\mathbb{N}$. В противном случае $X$ называется \term{бесконечным}.
\end{definition}

\begin{definition}
\term{Мощностью} конечного множества $X$ называется такое число $n\in \mathbb{N}$, что $X$ и $n$ равномощны. Обозначается это как $|X| = n$.
\end{definition}

Здесь, вероятно, требуются примеры. Возьмём опять наше множество $C=\{\heartsuit, \clubsuit, \spadesuit, \Diamond\}$. Если без формализма, а на уровне интуиции, то его мощность~--- это количество элементов в нём. Это очевидным образом связано с определением, данным нами выше, если вспомнить, что $4 = \{0, 1, 2, 3\}$. Тогда биекцией $4\to C$, устанавливающей равномощность, может быть функция $f = \{(0, \heartsuit), (1, \clubsuit), (2, \spadesuit), (3, \Diamond)\}$. Аналогичным образом получается связь и других определений с нашей интуицией~--- здесь может быть непривычным лишь то, что мы рассматриваем натуральные числа как множества, но именно это, если вдуматься, позволяет нам устанавливать между ними и множествами соответствия, используя привычный механизм функций.

\begin{definition}
Мы говорим, что число $m$ \term{меньше или равно} $n$ и обозначаем это как $m\le n$, если $m\subset n$.
\end{definition}

\begin{definition}
Мы говорим, что число $m$ \term{меньше} $n$ и обозначаем это как $m<n$, если $m\subset n$ и $m \not= n$.
\end{definition}

\begin{thm}
Отношение $\le$ задаёт линейный порядок на $\mathbb{N}$.
\end{thm}
\begin{proof}В качестве упражнения.\end{proof}

\begin{thm}
$S(n) > n$ для любого $n$.
\end{thm}
\begin{proof}
Элементарно.
\end{proof}

\begin{example}
Сравним числа 3 и 4. Мы знаем, что $3 = \{0, 1, 2\}$ и $4 = \{0, 1, 2, 3\}$. Очевидно, что $3\subset 4$, и, следовательно, $3\le4$. Поскольку $3\not= 4$, то $3<4$.
\end{example}

В полной аналогии с приведёнными определениями можно определить также сравнения \term{больше} ($>$) и \term{больше или равно} ($\ge$).

\begin{thm}
Множество $\mathbb{N}$~--- бесконечное.
\end{thm}
\begin{proof}
Пусть $\mathbb{N}$ конечно и $n = \max\{\mathbb{N}\}$. Тогда $m = |\mathbb{N}| = S(n) > n$, но поскольку $n$~--- максимальный элемент, $m\not\in\mathbb{N}$, что противоречит определению $\mathbb{N}$. Что и требовалось доказать.
\end{proof}

На самом деле для порядка натуральных чисел справедливо даже более сильное утверждение, к которому мы будем обращаться время от времени, а потом подробнее рассмотрим его в шестой главе:

\begin{definition}
Множество называется \term{фундированным}, если любое его подмножество имеет минимальный элемент.
\end{definition}

\begin{definition}
Множество называется \term{вполне упорядоченным}, если оно одновременно фундированное и линейно упорядочено.
\end{definition}

\begin{thm}
Множество $\mathbb{N}$ вполне упорядочено относительно порядка $\le$.
\end{thm}
\begin{proof}
В качестве упражнения.
\end{proof}

Теперь немного отвлечёмся от подхода Фреге-Рассела и посмотрим на аксиомы Пеано. Эти аксиомы не отвечают на вопрос что же вообще такое натуральные числа, но постулируют существование некоего множества $\mathbb{N}$ такого, что в нем существует выделенный элемент 0, и на котором задана некая инъективная функция $S:\mathbb{N}\to\mathbb{N}$ такая, что элемент 0 не имеет обратного. Это не совсем полная аксиоматика~--- мы будем её дополнять по мере надобности, но уже сейчас очевидно, что это определение практически дублирует подход Фреге-Рассела за исключением того, что мы не определяем конкретный вид элементов множества $\mathbb{N}$, но этого, на самом деле вполне достаточно~--- этим определением можно пользоваться, хотя жизнь при этом получается намного сложнее, как мы увидим ниже.

Перейдём теперь к определению арифметических операций. Вначале я буду давать интуитивное определение, затем доводить его до строгого вида в аксиоматике Фреге-Рассела, и затем определение для аксиом Пеано.

\begin{definition}
Пусть у нас есть непересекающиеся множества $M$ и $N$, такие что $|M|=m$ и $|N| = n$. Будем писать, что $|M\cup N| = m+n$, а число $m+n$ называть \term{суммой} $m$ и $n$.
\end{definition}

Это имеет очень простой комбинаторный смысл: если у нас есть некоторый набор, состоящий из $n$ объектов и мы его объединяем с набором, состоящим из $m$ объектов, то в результате мы получим набор, состоящий из $n+m$ объектов. Это то что объясняют в первом классе школы, но только более абстрактно.

Тем не менее, это не слишком хорошее определение. Во-первых, оно даёт нам понятие суммы не в терминах самих чисел, а в терминах неких множеств, причём непересекающихся. Это плохо, поскольку сами числа, будучи множествами, всегда пересекаются друг с другом (кроме числа 0). Поэтому если у нас есть два числа, не привязанных к конкретным множествам, это определение не даёт нам понять как определить их сумму.

Во-вторых, встаёт такой неприятный вопрос: пусть у нас есть непересекающиеся множества $A$ и $B$, такие что $|A|=|N|$ и $|B|=|N|$, можем ли мы в этом случае гарантировать, что $|A\cup B| = |N\cup M|$? Интуитивно это очевидно, но как это доказать~---вопрос нетривиальный.

И в третьих, всё ещё хуже: даже если $A$ и $B$ конечны, то где гарантия, что их объединение будет также конечным? Опять же, это очевидно, но поди докажи (а когда мы будем говорить о бесконечных множествах, выяснится, что подобные очевидные рассуждения часто банально неверны).

Первая проблема устраняется с помощью следующих двух упражнений:

\begin{exercise}
Докажите, что $|m\times 1| = m$ для $m\in\mathbb{N}$.
\end{exercise}

\begin{exercise}
Докажите, что $(m\times 1) \cap m = \emptyset$.
\end{exercise}

Пользуясь этими двумя утверждениями, можно ввести такое определение:

\begin{definition}
$m + n = |m\times1 \cup n|$.
\end{definition}

Это решает первую и вторую (после некоторых несложных раздумий) обозначенные нами проблемы, но не решает третью. Её можно решить либо с помощью метода матиндукции, который мы отложим на последующие параграфы, либо используя изначально более абстрактные конструкции и обобщая натуральные числа на случай бесконечных множеств, что мы оставим до шестой главы нашего учебника для начинающих.

А теперь то же самое определение, но уже в аксиоматике Пеано:

\begin{definition}
Сложение определяется следующим образом:\\*
$n + 0 = n$\\*
$n + S(m) = S(m + n)$
\end{definition}

\begin{example}
$$n + 3 = n + S(2) = S(n+2) = S(n+2)
=S(n+S(1)) = S(S(n+1)) = S(S(S(n)))$$
\end{example}

Как видно из примера, это определение фактически говорит, что выражение $m + n$ означает, что к числу $m$ применяется операция $S$ (фактически, увеличение на единицу) $n$ раз. Интуитивно это должно быть понятно, строгое же доказательство того, что такое определение правомочно, будет дано позже.

\begin{thm}
Справедливы следующие свойства сложения:
\begin{enumerate}
\item нейтральность нуля: $a + 0 = a$
\item коммутативность: $a + b = b + a$,
\item ассоциативность: $a + (b + c) = (a + b) + c = a + b + c$
\item если $a < b$, то $a + c < b + c$
\end{enumerate}
\end{thm}
\begin{proof}
Нейтральность нуля очевидна. Для коммутативности и ассоциативности используя определение Фреге-Рассела довольно легко построить биекцию между левыми и правыми частями равенства. Последнее равенство следует из того, что если $a < b$, то $S(a) < S(b)$ (доказывается элементарно), а прибавление любого натурального числа равносильно многократному применению операции $S$. Используя только аксиоматику Пеано это можно доказать, опять же, по индукции, о чем будет отдельный параграф.
\end{proof}

Пользуясь сложением легко определить линейный порядок на $\mathbb{N}$ в случае аксиом Пеано (для Фреге-Рассела это будет элементарная теорема):

\begin{definition}
$a < b$, если существует такое $n$, что $a + n = b$.
\end{definition}

\begin{definition}
Операция вычитания: мы пишем $a = b - c$, если $a + c = b$. В этом случае $a$ называется \term{разностью} $b$ и $c$.
\end{definition}

\begin{thm}
Операция $a - b$ определена только в том случае, если $a \ge b$.
\end{thm}
\begin{proof}В качестве упражнения\end{proof}

\begin{definition}
Операция умножения для Фреге-Рассела: $ab = |a\times b|$
\end{definition}

Здесь опять же надо внимательно отнестись к тем комментариям, которые я приводил для сложения, я на этом уже не буду подробно останавливаться.

Если смотреть на умножение комбинаторно, то получается простая интерпретация: для произвольных множеств $A$ и $B$ с мощностями $a$ и $b$ соответственно, имеем $|A\times B| = ab$. То есть произведение чисел~--- это количество элементов в декартовом произведении множеств соответствующих размеров. Это очень часто используется в самой базовой комбинаторике, например, так:

\begin{exercise}
Пусть у нас есть три бабы и два мужика. Сколько гетеросексуальных пар из них можно составить?
\end{exercise}

\begin{definition}
Операция умножения для аксиом Пеано:\newline
\item $m\cdot 0 = 0$\newline
\item $m\cdot S(n) = mn + m$
\end{definition}

\begin{thm}
Справедливы следующие свойства:
\begin{enumerate}
\item $0\cdot a = 0$
\item нейтральность единицы: $1\cdot a = a$
\item коммутативность: $ab = ba$
\item ассоциативность: $a(bc) = (ab)c = abc$
\item дистрибутивность: $a(b+c) = ab + ac$
\item если $a < b$ и $c \not= 0$, то $ac < bc$
\end{enumerate}
\end{thm}
\begin{proof}
Здесь всё аналогично доказательству подобных свойств для сложения~--- необходимо просто построить биекцию для левой и правой части (причём это не так просто, если ударяться прямо в формализм ZFC, хотя и возможно по индукции). Рекомендую самостоятельно попытаться строго проработать случай коммутативности, поскольку он несложен, но далеко не все понимают его.

Если отойти от формализма и посмотреть на вопрос геометрически, то $mn$ можно рассматривать как количество ячеек в таблице с $m$ строками и $n$ столбцами, а $nm$~--- количество ячеек в той же таблице, поставленной на бок~--- в этом случае строки и столбцы меняются местами, но количество ячеек при этом не меняется.

Из этой табличной интерпретации можно уже построить конкретную биекцию. Как увязываются таблицы и декартовы произведения рассматривалось в \S~2.2. Остальные приведённые здесь свойства могут быть интуитивно мотивированы подобным же образом.
\end{proof}

Используя обозначение $S(n) = n+1$ и свойство дистрибутивности можно легко понять смысл определения умножения в аксиомах Пеано: $mS(n) = m(n+1) = mn + m$.

\begin{thm}
Для любого $a$ и $b > 0$ найдутся такие числа $r<b$ и $q$, что $a = qb + r$.
\end{thm}
\begin{proof}
Будем строить конечную последовательность $\{r_n\}$ следующим образом: $r_n = a - nb$ для тех значений $n$, для которых вычитание будет определено. Множество элементов этой последовательности имеет минимальный элемент, который мы и обозначим как $r = a - qb$. Это и есть утверждение теоремы.
\end{proof}

\begin{exercise}
Где в доказательстве предыдущей теоремы неявно используется условие $b>0$?
\end{exercise}

\begin{definition}
Пусть $a = qb + r$. $q$ называется \term{частным от деления} $a$ на $b$, а $r$ \term{остатком от деления}.
\end{definition}

\begin{definition}
Если остаток от деления $a$ на $b$ равен нулю, то говорят, что $a$ \term{делится} на $b$, или что $b$ \term{делит} $a$. Обозначается это как $a\vdots b$ и  $b|a$ соответственно, а частное в этом случае обозначается как $a\over b$.
\end{definition}

\begin{exercise}
Докажите, что отношение делимости задаёт частичный порядок на $\mathbb{N}$.
\end{exercise}

\begin{definition}
Возведение в степень для Фреге-Рассела: пусть $m = |M|$ и $n = |N|$, тогда $m^n = |M^N|$
\end{definition}

Напомню, что $M^N = \{f|f:N\to M\}$~--- то есть это множество функций из $N$ в $M$. Это имеет простой комбинаторный смысл. Пусть, например, $M$~--- множество цветов рубашек и $N$~--- множество мужчин. Каждый мужчина выбирает себе цвет рубашки. Если все мужчины сделали свой выбор, то этот выбор представляется функцией $f: N \to M$. Сколько всего есть вариантов выбора цветов для всех мужчин сразу? Ровно столько, сколько есть таких функций, то есть ровно столько, какова мощность множества $M^N$. Остаётся только вопрос в том, как посчитать эту величину.

\begin{exercise}
Пусть для кодирования мы используем символы $\{a, b, c, d\}$. Сколько существует различных кодовых слов длины 5?
\end{exercise}

\begin{thm}
$m^n = \underbrace{m\cdot m \cdot \ldots \cdot m}_n$
\end{thm}
\begin{proof}
Пусть $|M| = m$ и $|N| = n$. Я приведу не самые строгие рассуждения, строго это опять же надо доказывать по индукции. Возьмём некоторый элемент $N$. Он может быть отображён в один из элементов $M$, которых $n$ штук. Возьмём другой элемент $N$, он также может быть отражён на один из $n$ элементов $N$. Итого для первых двух элементов $N$ существует $m\cdot m$ вариантов отображения. Если теперь рассмотреть ещё один элемент $N$, то он также может быть отображён в $m$ элементов, итого вариантов для отображения первых трёх элементов оказывается равно $m\cdot m\cdot m$. Продолжая рассуждения мы получим утверждение теоремы, поскольку в множестве $N$ всего $n$ элементов.
\end{proof}

Приведённое рассуждение подсказывает нам как можно определить степень для аксиоматики Пеано:

\begin{definition}
Степень в аксиоматике Пеано:\newline
$a^0 = 1$\newline
$a^{S(n)} = a^na$
\end{definition}

\begin{thm}
Для степеней справедливы следующие свойства:
\begin{enumerate}
\item $a^0 = 1$
\item $a^b a^c = a^{b+c}$
\item $(a^b)^c = a^{bc}$
\item если $a > b$, то для любого $c > 0$, $a^c > b^c$
\item если $a > b$, то для любого $c > 1$, $c^a > c^b$
\end{enumerate}
\end{thm}
\begin{proof}
Первое свойство дублирует определение Пеано, но его можно увидеть и из определения Фреге-Рассела: существует всего лишь одна функция из пустого множества в некоторое другое, и эта функция сама является пустым множеством. Функция вида $f:\emptyset\to X$ совершенно легальна и единственна, хоть она ничего и не отображает.

Второе свойство: $a^ba^c = \underbrace{\underbrace{a\cdot\ldots\cdot a}_b \cdot \underbrace{a \cdot\ldots \cdot a}_c}_{b+c} = a^{b+c}$

Третье свойство: $(a^b)^c = \underbrace{a^ba^b\ldots a^b}_c = a^{bc}$

Оставшиеся свойства предлагаю доказать самостоятельно в качестве упражнения.
\end{proof}

Приведённое определение Фрегге-Рассела может почти сразу дать нам ответ на вопрос о том сколько всего подмножеств имеет некоторое множество, а заодно объяснить обозначение $2^X$ для булеана. Прежде, однако, нам понадобится одно вспомогательное понятие.

\begin{definition}
Пусть $X\subset U$. Функция $\chi_X: U\to \{0,1\} = 2$ называется \term{индикаторной}, или \term{характеристической} функцией множества $X$, если $\chi_X(t) = 1$ при $t\in X$ и $\chi_X(t) = 0$ в противном случае.
\end{definition}

Каждому подмножеству $U$ соответствует своя характеристическая функция, ровно как и характеристической функции соответствует подмножество. Это соответствие позволяет нам легко получить желаемую теорему:

\begin{thm}
$|2^X| = 2^{|X|}$
\end{thm}
\begin{proof}
Для того, чтобы определить количество элементов в булеан, нам достаточно определить количество различных характеристических функций на множестве $X$. Поскольку характеристическая функция имеет вид $\chi:X\to 2$, множеством всех таких функций является $2^X$. По теореме 3.9 мощность этого множества $2^{|X|}$
\end{proof}

\section{Позиционные системы счисления}

После того как мы определили понятие натурального числа, встаёт вопрос о том, как натуральные числа записывать. Пока мы ввели только символы 0, 1, 2, 3 и 4 для нескольких чисел. Мы могли бы продолжить процесс и дальше (пока продолжим этот ряд последовательно символами 5, 6, 7, 8, 9, A), однако довольно быстро возникает проблема: множество $\mathbb{N}$ бесконечное, и соответственно символов нам потребуется бесконечно много, что, видимо, невозможно. Нам нужен способ, который позволит записывать любое натуральные число используя конечное число символов.

Пусть у нас есть некоторое число $n$, которое надо записать. Выберем некоторое произвольное $b > 0$, которое будем называть основанием нашей системы счисления и поделим одно на другое с остатком:
\begin{equation}\label{eq:n0}
n = q_0b + r_0
\end{equation}
Здесь $r_0 < b$ и $q_0 < n$. Поделим теперь на $b$ с остатком значение $q_0$: $$q_0 = q_1b + r_1$$ и подставим это выражение в \eqref{eq:n0}:
\begin{equation}\label{eq:n1}
n = (q_1b + r_1)b + r_0 = q_1b^2 + r_1b + r_0
\end{equation}

Аналогично можно представить $q_1 = q_2b + r_2$ подставив его в \eqref{eq:n1}, затем $q_2$, $q_3$ и так далее. Легко увидеть, что последовательность $\{q_i\}$ с каждым следующим элементом убывает, и, стало быть, в какой-то момент найдётся такое $k$, что $q_k = 0$. На этом процесс прекратится и мы получим такое выражение для $n$:
\begin{equation}\label{eq:nk}
n = r_kb^k + r_{k-1}b^{k-1} +\ldots + r_1b + r_0
\end{equation}

В этом выражении важно то, что каждое из значений $r_i$ оказывается меньше чем $b$, и при этом набора $\{r_i\}$ вполне достаточно для того, чтобы однозначно идентифицировать любое число. В этом и заключается основная идея позиционных систем счисления. Число $b$ определяет количество символов, необходимых для представления числа в системе с основанием $b$.

В компьютерах применяется так называемая двоичная система счисления, в которой $b=2$ и используются лишь два символа для записи чисел: 0 и 1. Это обусловлено тем, что на физическом уровне в вычислительных системах довольно просто отличить два принципиально различных состояния друг от друга: есть напряжение в проводе/нет напряжения, луч отражается от диска под большим углом/под маленьким, сектор на диске намагничен/не намагничен. И так далее. Возможно, конечно, и более детальное различение физических систем, например мы могли бы различать не просто наличие напряжения, но и его величину: слабое оно или сильное в дополнение к тому, если ли оно вообще. В этом случае $b$ было бы равно трём, и иногда это действительно используется, но технически это часто осуществляется сложнее, поэтому почти всегда используется $b=2$.

Рассмотрим пример. Как представить число $A$ в двоичной системе? (Напомню, что за $A$ мы обозначили число, следующее за числом 9). Проделывая процедуру с делением, описанную в начале параграфа, мы приходим к записи
$$A = 1\cdot 2^3 + 0\cdot 2^2 + 1\cdot 2 + 0$$
Здесь $r_3 = 1, r_2 = 0, r_1 = 1, r_0 = 0$. Можно кратко записать это как упорядоченный набор: $(1, 0, 1, 0)$, или же даже ещё короче, опустив скобки и запятые: 1010. Это и есть двоичное представления числа A. Чтобы не путать системы счисления, удобно также обозначать основание рядом с числом. В нашем случае получится $1010_2$. Впрочем, иногда нам будет удобно пользоваться и записью $(1, 0, 1, 0)_2$, так что следует иметь её ввиду, по крайней мере в течение ближайших нескольких параграфов. Количество символов, необходимых для представления числа, мы будем называть разрядностью, а выражение $r_ib^i$ $i$-ым разрядом. Иногда нам будет удобно считать, что число имеет больше разрядов чем необходимо, тогда старшие разряды будут иметь значение 0. Таким образом число $1010_2$ можно было бы эквивалентно записать как $00001010_2$. Потенциально мы можем считать, что слева в записи числа стоит бесконечное число нулей~--- это соображение часто упрощает рассуждения и мы будем пользоваться им ниже.

В повседневной жизни чаще всего применяется десятичная система счисления, в которой $b=A$ и помимо 0 и 1 используются также символы 2, 3, 4, 5, 6, 7, 8, 9. Рассмотрим, для примера, как представить число $10011001_2$ в десятичной системе счисления. Повторяя ещё раз процедуру деления с остатком, получаем:
$$10011001_2 = 1\cdot A^2 + 5\cdot A + 3 = 153_A$$

Рассматривая этот пример, у вас могут возникнуть сомнения по поводу того, как я это вычислил. Ответ тут очень простой: я использовал инженерный калькулятор, который умеет работать с разными системами счисления. Впрочем, даже без калькулятора можно было бы удостовериться в верности данного выражения. Самый простой способ поделить $a$ на $b$ с остатком заключается в многократном вычитании $b$ из $a$ до тех пор, пока результат не окажется меньше $b$. Этот способ легко понять, но он крайне неэффективен: для его реализации вам потребуется уже не калькулятор, а полноценный компьютер. Тем не менее вычислить это возможно. Пока мы остановимся на этом способе и на самом факте того, что это можно как-то вычислить, а в следующем параграфе я продемонстрирую более эффективный способ деления с остатком, который позволит провести все вычисления используя лишь ручку и клочок бумажки.

Везде далее, если не будет оговорено обратное, мы будем использовать десятичную систему счисления, при этом обозначать её мы не будем никак специально, то есть вместо $123_{10}$ мы будем ограничиваться записью $123$.

В качестве последнего примера рассмотрим шестнадцатеричную систему счисления ($b=16$), часто используемую программистами. В ней помимо символов десятичной системы применяются также символы $A, B, C, D, E, F$. Рассмотрим пример того, как можно понять десятичное значение числа в шестнадцатеричной записи:
$$ABF_{16} = A\cdot 16^2 + B\cdot 16 + F = 10\cdot 256 + 11 \cdot + 15 = 2751$$

Причина, по которой программисты любят шестнадцатеричную систему счисления, заключается в том, что она очень легко переводится в двоичную систему счисления и обратно. По сути для этого надо знать лишь представление в двоичной системе 16-ти цифр. Для примера выше мы уже видели, что $A_{16} = 1010_2$, также легко увидеть, что $B_{16} = 1011_2$ и $F_{16} = 1111_2$. Чтобы получить отсюда двоичную запись, достаточно объединить двоичные записи для отдельных шестнадцатеричных цифр: $$ABF_{16} = 1010\:1011\:1111_2$$

Возможность такого представления основывается на следующей несложной общей теореме (сложнее понять формулировку, чем доказать), доказательство которой мы оставим в качестве упражнения читателю (впрочем, я бы пока рекомендовал отложить это упражнение и вернуться к нему после прочтения следующего параграфа):
\begin{thm}
Записи в системах счисления с основаниями $b$ и $c = b^k$ связаны следующим образом: $$(d_{kn},\dots, d_0)_b = ((d_{kn}, \ldots, d_{(k-1)n + 1})_b, \dots, (d_{2k}, \dots, d_{k+1})_b , (d_k, \dots, d_0)_b)_c$$
\end{thm}

В компьютерной памяти чаще всего двоичные значения 0 и 1 (их называют битами) объединены в группы по восемь бит (число восемь берётся из соображений, близких к только что упомянутой теореме). Такая группа бит называется байтом. Во многих системах один байт представляет собой один печатный символ. Если же рассматривать байт как число, что его значения могут варьироваться от 0 до 255 (всего 256 различных значений), и таким образом каждому символу можно сопоставить некоторое числовое значение. Всего у нас может быть максимум 256 символов.

Если рассматривать не один, а сразу последовательность байт, то их можно считать числом, записанном в 256-ричной системе счисления. Это часто применяется в компьютерах для записи больших чисел. Если рассматривать два байта, то их максимальным значением может быть 65535. Если считать за символ не один байт, а два байта, то это значит, что наша система сможет поддерживать 65535 символов, что хватит даже китайцам с несколькими их диалектам, Египтянам, латинянам и евреям. Если нам и этого мало, то можно рассматривать четырёхбайтные значения. В этом случае мы сможем записать число 4294967295, то есть четыре байта позволяют записывать девятизначные числа и некоторые десятизначные. С точки зрения символов мы сможем уместить сюда не только все распространённые в мире языки, но и все вымершие языки, смайлики, музыкальные обозначения, математические знаки, несколько вариантов древней клинописи и так далее. Если нам и этого не хватит, то можно взять 8-байтные целые, которые позволят работать с 19-значными числами.

Если мы будет рассматривать текст как последовательность символов, то мы также эту последовательность можем интерпретировать как некоторое большое число. Например, для кодирования английского текста чаще всего применяется стандарт ASCII, устанавливающий какой букве соответствует какое число. Букве F в нём соответствует число 70, а букве o~--- 111 (ASCII использует только 1 байт для кодирования символов). Как число слово Foo в ASCII можно представить следующим образом:
$$Foo = 70 \cdot 256^2 + 111 \cdot 256 + 111 = 4587520 + 28416 + 111 = 4616047$$

Подобное отношение к тексту позволяет применять к нему математические функции. Например, многие криптосистемы представляют собой лишь некоторые арифметические действия над числами и даже не догадываются о том, что пользователь рассматривает данные как текст. Самым ярким примером является криптосистема RSA, на которой сейчас построена значительная доля всей криптографии, используемой на практике, и которую мы рассмотрим в четвёртой главе. Используя действия над числами можно также сжимать данные, чтобы они занимали меньше места. Этот подход называется арифметическим кодированием и мы также рассмотрим его в четвёртой главе. В четвёртом параграфе этой главы мы будем рассматривать математические формулы как обычный текст, который в свою очередь мы будем рассматривать как обычное число. Довольно неожиданным образом это позволит сделать нам важные фундаментальные выводы относительно всей математики в целом.


\section{Вычислительный аспект}

В первом параграфе мы определили арифметические операции над натуральными числами, но однако не сказали ни слова о том, как реально вычислять результат от их применения. Можно, конечно, использовать аксиомы напрямую. Так, для сложения $m+n$ в соответствии с аксиомами Пеано нам потребуется $n$ раз прибавить 1 к числу $m$. Скажем, вот так может начинаться процесс сложения $123+456$:
$$123+456 = 124 + 455 = 125 + 454 = 126 + 453 = \ldots$$
Очевидно, что этот способ никуда не годится. В этом параграфе мы рассмотрим каким образом можно проводить арифметические вычисления более-менее эффективно, но прежде введём новую для нас удобную нотацию.

Пусть $\{x_i\}$~--- некоторая последовательность и мы хотим сложить все элементы этой последовательности подряд начиная $x_a$ и заканчивая $x_b$. Кратко мы будем записывать эту сумму с помощью символа $\sum$:
$$\sum_{i=a}^b x_i = x_a + x_{a+1} + \ldots + x_b$$
Аналогичную краткую запись мы введём и для произведения:
$$\prod_{i=a}^b x_i = x_a \cdot x_{a+1} \cdot \ldots \cdot x_b$$
Для пронумерованного семейства множеств $\{S_i\}$ аналогично введём краткое обозначение для объединения и пересечения:
$$\bigcup_{i=a}^b A_i = A_a \cup A_{a+1}\cup\ldots\cup A_b$$
$$\bigcap_{i=a}^b A_i = A_a \cap A_{a+1}\cap\ldots\cap A_b$$
В полной аналогии можно обозначать конъюнкцию, дизъюнкцию и исключающее или  для логических высказываний да и вообще многие другие операции, но мы не будем на этом лишний раз останавливаться~--- эти обозначения и так очевидны и понятны.

Так же введём следующее сокращение для многократного применения функции:
$$f^n(x) = \underbrace{(f\circ f\circ\ldots\circ f)}_n (x)$$

Пусть мы теперь хотим просуммировать не все элементы $x_i$ в каком-то диапазоне, а в точности те значения $x_i$, где $i$ принадлежит некоторому наперёд заданному множеству $S$. Это можно кратко записать так:
$$\sum_{i\in S} x_i$$
Аналогично можно записывать и прочие операции.

Пользуясь введённой нотацией любое натуральное число $n$, для представления которого требуется $k$ разрядов, можно записать в системе счисления с основанием $b$ таким образом:
$$n = \sum_{i=0}^{k-1} r_i b^i = r_{k - 1} b^{k-1} + \ldots + r_1 b + r_0$$
Или даже, если посчитать, что при $i\ge k$ все $r_i = 0$, можно избавиться в этой сумме от величины $k$:
$$n = \sum_{i=0}^\infty r_i b^i = \sum_{i\in\mathbb{N}} r_i b^i$$
Символом $\infty$ мы абстрактно обозначили <<бесконечность>>, что означает, что мы будем суммировать по всем значениям $i$ вообще. В общем случае возможность сложения бесконечного количества чисел вызывает сразу ряд вопросов, однако в нашей ситуации мы знаем, что лишь конечное число слагаемых $r_ib^i$ будет отлично от нуля, так что реально здесь суммируется по сути конечное число значений и проблемы с этим не возникает.

Пусть мы хотим сложить числа $a$ и $b$. Их разряды в системе счисления с основанием $d$ мы обозначим как $a_i$ и $b_i$ соответственно. Тогда, если вспомнить, что сложение может осуществляться в любом порядке и мы можем как угодно переставлять в суммах скобки (см.~\S3.1), мы элементарно получаем следующее соотношение:
$$\left(\sum_{i=0}^\infty a_i d^i\right) + \left(\sum_{i=0}^\infty b_i d^i\right) = \sum_{i=0}^\infty (a_i + b_i) d^i$$
Буквально здесь говорится, что мы можем складывать числа поразрядно. То есть для нашего примера с 123 и 456 мы можем отдельно сложить 1+4, 2+5 и 3+6. Результатом будет 579 (если вам непонятны эти рассуждения, распишите эти числа как сумму в десятичной системе счисления и проведите вычисления аккуратно).

Если мы теперь попытаемся сложить 579 и 123, то у нас выйдет проблема: $3+9=12$, что больше, чем основание системы счисления. Это значит, что 12 надо представить как $1\cdot10 + 2$, и теперь единица переходит в старший разряд. В итоге вместо $7+2$ мы должны во второй разряд записать $7+2+1=10$, что опять не умещается в систему счисления. Снова единица перейдет уже в третий разряд: $5+1+1$. Результат: 702.

Мы не будем останавливаться на этом подробно, так как все эти вещи совершенно тривиальны. В школах используется именно этот способ сложения, только для удобства записи числа записываются в столбик, разряд под разрядом, и в школе редко объясняют почему такой способ сложения вообще работает. Мы это только что доказали.

Вычитание практически аналогично сложению и мы не будем его рассматривать отдельно.

Так же легко определить и формулу для умножения (мы могли бы получить много разных вариантов, но именно этот вариант простой перестановки скобок~--- стандартный школьный):
$$\left(\sum_{i=0}^\infty a_i d^i \right)\left( \sum_{j=0}^\infty b_j d^j \right) = \sum_{i=0}^\infty \left(\sum_{j=0}^\infty a_i b_j d^j \right)d^i$$
Это выражение в точности дублирует умножение в столбик~--- каждый разряд числа $a$ по отдельности умножается на число $b$ целиком (опять же поразрядно). Не будем вдаваться в скучные подробности умножения в столбик, а обратим внимание на следующий важный аспект:

\begin{exercise}
Пусть даны два $n$-разрядных числа $a$ и $b$. Для их умножения приведенным способом потребуется $n^2$ операций умножения отдельных разрядов.
\end{exercise}

Если взять числа 23 и 45 и сложить их, то нам потребуется сложить разряды $2+4$ и $3+5$. Для умножения же этих чисел, потребуется вычислить произведения $2\cdot 4$, $2\cdot 5$, $3\cdot 4$ и $3\cdot 5$. Если взять не двузначные числа, а трехзначные, то при сложении нам потребуется сложить три разряда, а при умножении перемножить девять разрядов. Для сложения 100-разрядных чисел потребуется 100 операций сложения разрядов, а для умножения их же~--- 10000 операций умножения.

Как видно, сложность умножения (если выражать сложность в количестве операций)  значительно выше, нежели сложность сложения, причем чем больше числа по разрядности, тем значительнее возрастает сложность умножения, в отличие от сложения. Компьютер, конечно, вычислит это всё равно за доли секунды, какими бы длинным ни были перемножаемые числа, однако часто этого оказывается недостаточно. Умножение~--- это одна из самых базовых операций и зачастую она выполняется компьютером сотни или даже тысячи раз в секунду, например, при обработке графики или сложных физических расчетах. В таких условиях, разработчикам компьютерных систем становится принципиальным, чтобы умножение чисел компьютер мог успевать совершать не несколько раз за секунду, а несколько миллионов раз за секунду (а лучше~---больше). Понятно, что приведенный нами алгоритм, даже если и сможет работать так быстро в силу совершенства оборудования, будет сильно падать в эффективности при увеличении перемножаемых чисел.

Один из общих подходов к созданию быстрых алгоритмов называется <<Разделяй и властвуй>>. Идея состоит в том, чтобы некоторую крупную по размерам задачу разбить на ряд маленьких подзадач, обработать каждую из этих подзадач в отдельности, а затем объединить результат. Это довольно общие слова и как именно надо действовать зависит от конкретной задачи. Мы рассмотрим этот подход на примере умножения чисел.

Пусть числа $a$ и $b$ имеют $2d$ десятичных разрядов (возможно, меньше, тогда дополним их нулями). Запишем их в виде $a = a_H10^d + a_L$ и $b = b_H10^d + b_L$, где $a_H, a_L, b_H, b_L$ имеют по $d$ разрядов. Тогда их произведение может быть расписано простым раскрытием скобок:
\begin{align*}
ab &= (a_H10^d + a_L)(b_H10^d + b_L) \\
         &= a_Hb_H10^{2d} + a_Hb_L10^d + a_Lb_H10^d + a_Lb_L\\
         &= a_Hb_H10^{2d} + (a_Hb_L + a_Lb_H)10^d + a_Lb_L
\end{align*}
Здесь мы свели одно умножение $2d$-разрядных чисел к умножению четырех чисел, но уже $d$-разрядных и суммированию результатов. К сожалению, это нам пока ничего не даёт. Например, для двузначных чисел мы что раньше выполняли четыре умножения однозначных чисел, что теперь. Однако, этот метод возможно ускорить, применив так называемый трюк Карацубы (для многих неожиданно, что Карацуба был советским математиком родом из Чечни, закончившим грозненскую школу, и звали его Анатолий), который стал исторически первым эффективным алгоритмом типа <<Разделяй и властвуй>> и первым алгоритмом умножения, работающим быстрее чем за время $n^2$. Трюк заключается в вычислении произведения
$$c = (a_H + a_L)(b_H + b_L) = a_Hb_H + a_Hb_L + a_Lb_H + a_Lb_L$$
Если вычислить предварительно произведения $a_Hb_H$ и $a_Lb_L$, то получаем
$$a_Hb_L + a_Lb_H = c - a_Hb_H - a_Lb_L$$
В итоге теперь нам требуется вычислить не четыре произведения, а лишь три: $c$, $a_Hb_H$ и $a_Lb_L$, каждое из которых оперирует числами разрядности $d$, а затем вычислить требуемое значение $a_Hb_L + a_Lb_H$, используя $c$ и операцию вичитания, что намного быстрее. Каждое из этих произведений опять же может быть вычислено по алгоритму Карацубы, для чего придется опять перемножить три числа, на этот раз разрядности $d \over 2$. Процесс следует продолжать, пока мы не дойдём до одноразрядных чисел.

Подробный анализ быстродействия алгоритма Карацубы требует довольно отвлеченных от нашего нынешнего рассказа тем математики (их мы тоже рассмотрим, но позже), поэтому мы пока ограничимся лишь бездоказательным утверждением о том, что алгоритм Карацубы за счет сокращения четырех умножений до трёх, действительно даёт вычислительное преимущество при умножении чисел, в сравнении с умножением в столбик. Для умножения с использованием ручки и бумажки он не удобен, однако на компьютере он может быть запрограммирован элементарно в несколько строчек кода. Часто именно таким образом реализуется компьютерное умножение, хотя сейчас есть и более совершенные методы, один из которых мы будем рассматривать в этом курсе позже.

Возведение в степень легко осуществить по определению Пеано:
$$a^b = \prod_{i=1}^b a$$
но можно и быстрее, что опять же важно для компьютерных вычислений.

Пусть для начала нам надо возвести число $a$ в степень $2b$. По свойствам степени $a^{2b} = (a^b)^2$ и мы можем вначале возвести $a$ в степень $b$, а затем результат возвести в квадрат~--- одно это уже намного проще, чем производить умножение $2b - 1$ раз в соответствии с аксиомами Пеано.

Если же нам надо возвести число $a$ в степень $2b +1$, то мы можем поступить похожим образом: $a^{2b+1} = (a^b)^2a$. Мы вначале вычисляем $a^b$, затем возводим результат в квадрат, и в итоге умножаем результат опять на $a$. Подробный анализ быстродействия этого способа опять же выходит за рамки этого параграфа, но довольно очевидно, что преимущество, получаемое нами в данном случае, будет огромно.
\begin{exercise}
Покажите, что для возведения числа $x$ в степень 64 достаточно произвести всего 6 операций умножения в соответствии с приведенным подходом, против 63 умножений в соответствии с аксиомами Пеано.
\end{exercise}

Осталось рассмотреть лишь одну операцию~--- деление с остатком. Совсем простой подход я уже озвучивал в прошлом параграфе: чтобы поделить $a$ на $b$ с остатком, надо вычитать $b$ из $a$ до тех пор, пока результат не окажется меньше $b$. Например, поделим 123 с остатком на 40. После первого вычитания получаем 83 ($123 = 40 + 83$). После второго вычитания~--- 43 ($123 = 2\cdot 40 + 43$). После третьего~--- 3 ($123 = 3\cdot 40 + 3$). Итого и частное и остаток оказались равны трём.

Этот способ очень прост, но и очень неэффективен: как, например, таким способом поделить с остатком 12347 на 5?

Простая идея для ускорения вычислений заключается в том, чтобы вычитать делитель не по одному за итерацию, а сразу по нескольку. В случае с 12347 и 5 вместо того, чтобы вычитать многократно число 5, мы можем вычесть сразу некоторое число $5n$, где $n$ надо подобрать таким образом, чтобы $5n$ было не больше 12345. Удобнее всего здесь брать степени десяти (или основания той системы счисления, в которой мы работаем), помноженные на какое-то небольшое число. Для нашего примера удобно взять $n=2000$. Отсюда после первого вычитания получаем: $12347 = 5\cdot 2000 + 2347$. Для второго вычитания возьмём $n=400$:
$$12347 = 5\cdot 2000 + 5\cdot 400 + 347 = 5\cdot 2400 + 347$$
Теперь возьмём $n=60$: $12347 = 5\cdot 2460 + 47$. Взяв в последний раз $n=9$, получаем $12347 = 5\cdot 2469 + 2$. Это и есть желаемый ответ.

Приведенные рассуждения~--- это в точности школьное деление в столбик, но только расписанное подробно. Примерно таким образом работает и деление чисел в компьютере. На практике при компьютерной реализации этого алгоритма возникает целый ряд дополнительных вопросов, например каким именно образом наиболее быстро можно выбрать значение $n$, плюс к этому можно рассмотреть целый ряд усовершенствований, основанных на использовании двоичной системы счисления.

В заключение этого раздела мы спустимся в вычислительных системах еще на уровень ниже и кратко рассмотрим, каким образом реализуется арифметика на уровне компьютерного железа. Мы ограничимся рассмотрением только сложения, остальные операции читатель сможет сам домыслить по аналогии.

Очень абстрактно мы будем представлять себе один бит как контакт на компьютерной плате, по которому может либо идти напряжение (значение 1), либо не идти (значение 0). Мы будем так же рассматривать логические блоки, которые имеют несколько входных контактов, и несколько выходных и которые реализуют некоторые логические операции. Одним из простейших логических блоков является блок, реализующий логическую операцию <<штрих Шеффера>>. В \S~1.2 мы упоминали, что абсолютно любую логическую операцию можно выразить с помощью него. Как пример, мы можем рассмотреть операцию <<И>>. Её выражение на языке логики с помощью штриха Шеффера такое: $a\land b = (a|b)|(a|b)$ (проверьте). Абстрактное представление этого выражения в том виде, как оно будет реализовано на логической схеме, представлено на рисунке~3.1.
\begin{figure}[h]
\centering
\begin{tikzpicture}
	\def\lblock{node [minimum height=1cm,minimum width=1.5cm,draw] }
	\def\spoint{node [circle, draw, fill, inner sep = 0, minimum size = 0.05cm] }
	\draw (-2cm, .5cm) \spoint (x) {};
	\draw (-2cm, -.5cm) \spoint (y) {};
	\node [above] at (x.north) {a};
	\node [below] at (y.south) {b};
	\draw (0, .7cm) \lblock (s1) {|};
	\draw (0, -.7cm) \lblock (s2) {|};
	\draw (3cm, 0) \lblock (s3) {|};
	\draw (4.5cm, 0) \spoint (xy) {};
	\node [above] at (xy.north) {$a\land b$};
	\draw [->] (s1.east) -- ($(s3.south west)!.75!(s3.north west)$);
	\draw [->] (s2.east) -- ($(s3.south west)!.25!(s3.north west)$);
	\draw [->] (x.east) -- ($(s1.south west)!.75!(s1.north west)$);
	\draw [->] (x.east) -- ($(s2.south west)!.75!(s2.north west)$);
	\draw [->] (y.east) -- ($(s1.south west)!.25!(s1.north west)$);
	\draw [->] (y.east) -- ($(s2.south west)!.25!(s2.north west)$);
	\draw [->] (s3.east) -- (xy.west);
\end{tikzpicture}
\caption{Схема операции <<Логическое И>>}
\end{figure}

Теперь перейдём к построению схемы сложения чисел. За $n$ обозначим количество разрядов в складываемых числах. Результатом будет $(n+1)$-битное число из-за возможности переполнения последнего разряда. Складываемые числа будем обозначать как $x$ и $y$, а их разряды как $x_i$ и $y_i$. Результат сложения и его разряды будем обозначать $z$ и $z_i$ соответственно. Введем так же переменные $c_i$, которые будут равны 1, если в разряде $i$ произошло переполнение. Для того, чтобы выразить значения $z_i$ и $c_i$, нам доступны лишь логические (двоичные) операции и только они. На деле нам обычно доступно лишь какое-то подмножество логических операций, но во многих случаях мы можем из них выразить все остальные операции, как это описано в~\S~1.2.

Очевидно, что $z_0 = x_0 \oplus y_0$ и $c_1 = x_0\land y_0$. В общем же случае нам требуется, чтобы $z_i$ равнялось 1 когда либо одно из чисел $x_i, y_i, c_{i-1}$ равно 1, либо все они одновременно, а $c_i$ равно 1, когда по крайней мере два из них равно 1. Используя таблицу истинности, легко увидеть, что этим критериям соответствуют логические формулы
$$z_i = x_i\oplus y_i \oplus c_{i-1}$$
$$c_i = (\neg x_i\land y_i \land c_{i-1}) \lor (x_i \land (y_i \lor c_{i-1}))$$

Диаграмма для этих формул показана на рисунке~2.

\begin{figure}[h]
\centering
\begin{tikzpicture}
	\def\lblock{node [minimum height=1cm,minimum width=1.5cm,draw] }
	\def\spoint{node [circle, draw, fill, inner sep = 0, minimum size = 0.05cm] }
	\draw (-2cm, 1cm) \spoint (x) {};
	\draw (-2cm, 0) \spoint (y) {};
	\draw (-2cm, -1cm) \spoint (c) {};
	\draw (-2cm, -2.95) \spoint (c1) {};
	\node [left] at (x) {$x_i$};
	\node [left] at (y) {$y_i$};
	\node [left] at (c) {$c_{i-1}$};
	\draw (-.5cm, -1.7cm) \lblock (p1) {$\oplus$};
	\draw (2.2cm, -2.7cm) \lblock (p2) {$\oplus$};
	\draw (-.5cm, 2cm) \lblock (n) {$\neg$};
	\draw (1.5cm, 1.8cm) \lblock (and1) {$\land$};
	\draw (3.5cm, 1.6cm) \lblock (and2) {$\land$};
	\draw (3.5cm, 0) \lblock (and3) {$\land$};
	\draw (1.5cm, -.8cm) \lblock (or1) {$\lor$};
	\draw (5.5cm, .8cm) \lblock (or2) {$\lor$};
	\draw (7cm, -1cm) \spoint (z) {};
	\draw (7cm, 1cm) \spoint (cn) {};
	\node [right] at (z) {$z_i$};
	\node [right] at (cn) {$c_i$};
	\draw [->] (x.east) -- ($(p1.south west)!.75!(p1.north west)$);
	\draw [->] (y.east) -- ($(p1.south west)!.25!(p1.north west)$);
	\draw [->] (p1.east) -- ($(p2.south west)!.75!(p2.north west)$);
	\draw [->] (c.east) -- (c1) -- ($(p2.south west)!.25!(p2.north west)$);
	\draw [->] (p2.east) -- (z.west);
	\draw [->] (x.east) -- (n.west);
	\draw [->] (n.east) -- ($(and1.south west)!.75!(and1.north west)$);
	\draw [->] (y.east) -- ($(and1.south west)!.25!(and1.north west)$);
	\draw [->] (and1.east) -- ($(and2.south west)!.75!(and2.north west)$);
	\draw [->] (c.east) -- ($(and2.south west)!.25!(and2.north west)$);
	\draw [->] (y.east) --  ($(or1.south west)!.75!(or1.north west)$);
	\draw [->] (c.east) --  ($(or1.south west)!.25!(or1.north west)$);
	\draw [->] (x.east) --  ($(and3.south west)!.75!(and3.north west)$);
	\draw [->] (or1.east) --  ($(and3.south west)!.25!(and3.north west)$);
	\draw [->] (and2.east) --  ($(or2.south west)!.75!(or2.north west)$);
	\draw [->] (and3.east) --  ($(or2.south west)!.25!(or2.north west)$);
	\draw [->] (or2.east) -- (cn.west);
\end{tikzpicture}
\caption{Схема сложения одного разряда}
\end{figure}

Подобная схема должна быть реализована на каждый бит данных для сложения чисел.

\begin{exercise}
Сколько всего логических элементов и соединений потребуется для реализации операции сложения для двубайтовых чисел? Можете ли вы предложить способ, которым можно сократить это число?
\end{exercise}

Приведенная диаграмма является очень сильным упрощением того, то в действительности происходит и отражает лишь общую идею, на которой всё базируется. В физическом устройстве было бы важно реальное геометрическое расположение элементов и связей между ними, их размеры, материалы, напряжение, стоимость, общее количество элементов и много прочих сложных инженерных деталей, которые математикам уже мало интересны.

\section{Делимость}

В этом параграфе мы одним махом рассмотрим все базовые свойства делимости и связанные с этим понятия. Напомню, что мы ввели обозначение $a|b$ для обозначения делимости $b$ на $a$ (равносильно $a$ делит $b$).. Формально это значит, что $b=ka$ для некоторого $k$.

\begin{thm}
Если в сумме $\sum_{i=0}^na_i$ все слагаемые делятся на $b$, то и вся сумма делится на $b$.
\end{thm}
\begin{proof}
По условию теоремы $a_i = k_i b$. Тогда
$$\sum_{i=0}^na_i = \sum_{i=0}^n bk_i = b\sum_{i=0}^n k_i$$
Последнее выражение как раз и означает, что первоначальная сумма делится на $b$, поскольку мы смогли из всей суммы вынести этот множитель за скобки.
\end{proof}

\begin{thm}
Если в сумме $\sum_{i=0}^na_i$ все слагаемые кроме одного делятся на $b$, то сумма не делится на $b$.
\end{thm}
\begin{proof}
Поскольку мы можем переставлять слагаемые как нам вздумается, будем считать, что на $b$ не делится слагаемое $a_0$, а остальные слагаемые делятся. Это значит, что $a_0 = k_0 b + r, r>0$ и для всех $i>0$ имеем $a_i=k_i b$. Тогда мы можем записать сумму в следующем виде:
$$\sum_{i=0}^na_i = k_0 b + r + \sum_{i=1}^n b k_i = b\sum_{i=0}^n k_i + r$$
Последнее есть ни что иное как деление с остатком суммы на $b$, где остаток $r>0$. Это ровно то, что требовалось доказать.
\end{proof}

\begin{exercise}
В теореме 3.14 принципиально то, что не делится на $b$ лишь одно из слагаемых, но не больше. Покажите, что если в сумме не делится на $b$ ровно два слагаемых, то сама сумма на $b$ всё же может делиться.
\end{exercise}

\begin{exercise}
Докажите, что для того, чтобы произведение $\prod_{i=0}^n a_i$ делилось на $b$, достаточно, чтобы хотя бы один из множителей $a_i$ делился на $b$. Это условие не является необходимым: покажите также, что возможна ситуация, когда ни один из множителей не делится на $b$, но само произведение на $b$ делится.
\end{exercise}

\begin{definition}
Число называется \term{чётным}, если оно делится на 2, и \term{нечётным} в противном случае.
\end{definition}

Любое чётное число может быть представлено в виде $2k$, а нечётное в виде $2k+1$.

\begin{exercise}
Докажите, что сумма двух чётных чисел чётна, сумма двух нечётных также чётна, а сумма чётного и нечётного нечётна. Докажите также, что произведение нечётных чисел всегда нечётное, и что если хотя бы один из множителей чётный, то и всё произведение чётно.
\end{exercise}

Понятие чётности/нечётности вроде на первый взгляд совершенно тривиально, однако оно постоянно возникает в математике. Например, в прошлом параграфе оно нам уже встречалось, хотя мы тогда не обратили на него внимания. С новой терминологией мы можем переформулировать алгоритм возведения в степень так: если $b$~--- чётное, то $a^b = (a^{b/2})^2$, а если нечётное, то $a^b = (a^{(b-1)/2})^2 a$. Мелочь, но подобная терминология в математике и её приложениях встречается сплошь и рядом, это довольно удобно.

Используя теоремы выше мы можем легко вывести школьные <<признаки делимости>>.

\begin{example}
Для того, чтобы число делилось на 4, необходимо и достаточно, чтобы число, составленное из двух младших разрядов делилось на 4. Действительно:
$$\sum_{i=0}^\infty a_i 10^i = 100 \sum_{i = 2}^\infty a_i 10^{i-2} + (10a_1 + a_0)$$
В последнем выражении сумма слева делится на 4, поэтому для делимости всего числа на~4 по теоремам~3.13-14 необходимо и достаточно, чтобы делилось на 4 выражение $10a_1 + a_0$, а это и есть две последние цифры. Так, число 4133 не делится на 4, так как 33 не делится на 4, а число 12344 делится на четыре, так как 44 делится на 4.
\end{example}

\begin{exercise}
Докажите, что для того, чтобы число было чётным, необходимо и достаточно, чтобы младший разряд числа был чётным.
\end{exercise}

\begin{exercise}
Докажите, что для того, чтобы число делилось на 5, необходимо и достаточно, чтобы младший разряд был равен 0 либо 5.
\end{exercise}

\begin{example}
Для того, чтобы число делилось на 3, необходимо и достаточно, чтобы сумма его цифр делилась на 3. Действительно:
\begin{align*}
\sum_{i=0}^\infty a_i 10^i &= a_0 + (9+1)a_1 + (99+1)a_2 + \ldots\\
    &= \sum_{i=0}^\infty a_i + 9a_1 + 99a_2 + 999a_3 +\ldots \\
    &= \sum_{i=0}^\infty a_i + 3(3a_1 + 33a_2 +\ldots)
\end{align*}
Второе слагаемое всегда делится на три, поэтому для делимости всей суммы необходимо и достаточно, чтобы делилась на три сумма $\sum_{i\in\mathbb{N}} a_i$.
\end{example}

\begin{exercise}
Докажите, что для того, чтобы число делилось на 9, необходимо и достаточно, чтобы сумма его цифр делилась на 9.
\end{exercise}

\begin{example}
Можно придумать ещё множество критериев подобным образом, а также сочетая существующие критерии. Например, для того, чтобы число делилось на 6, необходимо и достаточно, чтобы оно одновременно делилось на 2 и на 3. Придумайте ещё подобные критерии.
\end{example}

Как мы уже отмечали в первом параграфе, отношение делимости задаёт частичный порядок на $\mathbb{N}$. Как и прочие частичные порядки (см.\S2.2) этот порядок можно наглядно изобразить в виде диаграммы. На рисунке~3.3 изображено несколько натуральных чисел, связанных отношением делимости. Полную диаграмму делимости для чисел $\mathbb{N}$ вряд ли возможно себе вообразить, но понимание того, что числа возможно частично упорядочить таким образом (причем концептуально это ни чуть не хуже привычного упорядочивания <<больше-меньше>>, это просто ещё один способ), будет необходимым для нашего следующего определения.

\begin{figure}[h]
\centering
\begin{tikzpicture}
\def\point{node [circle, draw, fill, inner sep = 0, minimum size = .1cm] }
\draw (0, 0) \point (p1) {};
\draw (-2cm, .8cm) \point (p2) {};
\draw (-1cm, .8cm) \point (p3) {};
\draw (-2cm, 1.6cm) \point (p4) {};
\draw (1cm, .8cm) \point (p5) {};
\draw (-1.5cm, 1.6cm) \point (p6) {};
\draw (2cm, .8cm) \point (p7) {};
\draw (-2cm, 2.4cm) \point (p8) {};
\draw (0, 1.2cm) \point (p9) {};
\draw (-.5cm, 2.4cm) \point (p10) {};
\draw (-1.5cm, 2.4cm) \point (p12) {};
\draw (1.2cm, 2.4cm) \point (p14) {};
\draw (.35cm, 2.4cm) \point (p15) {};

\draw (p1) -- (p2);
\draw (p1) -- (p3);
\draw (p1) -- (p5);
\draw (p1) -- (p7);
\draw (p2) -- (p4);
\draw (p2) -- (p6);
\draw (p3) -- (p6);
\draw (p4) -- (p8);
\draw (p3) -- (p9);
\draw (p2) -- (p10);
\draw (p5) -- (p10);
\draw (p2) -- (p12);
\draw (p6) -- (p12);
\draw (p2) -- (p14);
\draw (p7) -- (p14);
\draw (p3) -- (p15);
\draw (p5) -- (p15);

\node [below] at (p1) {1};
\node [left] at (p2) {2};
\node [left] at (p3) {3};
\node [left] at (p4) {4};
\node [right] at (p5) {5};
\node [above right] at (p6) {6};
\node [right] at (p7) {7};
\node [left] at (p8) {8};
\node [right] at (p9) {9};
\node [right] at (p10) {10};
\node [right] at (p12) {12};
\node [right] at (p14) {14};
\node [right] at (p15) {15};

\end{tikzpicture}
\caption{Частичный порядок, заданный отношением делимости}
\end{figure}

Напомню, что в~\S~2.2 мы ввели понятие точных нижних и точных верхних граней для частично упорядоченных множеств и обозначили это как $\inf$ и $\sup$ соответственно. Эти понятия можно применить и к отношению делимости:

\begin{definition}
\term{Наибольшим общим делителем (НОД)} чисел множества $S$ называется $\inf S$, а \term{наименьшим общим кратным (НОК)} $\sup S$ относительно порядка делимости.
\end{definition}

Это определение требует, видимо, расшифровки. Для простоты будем считать, что множество $S$ состоит из двух чисел $x$ и $y$, хотя всё сказанное легко обобщается на случай произвольного конечного количества чисел.

Величина $d$ называется общим делителем чисел $x$ и $y$, если одновременно $d$ делит и $x$ и $y$. Множество всех нижних граней множества $\{x, y\}$ есть на самом деле множество общих делителей этих чисел~--- это хорошо видно из рисунка. Например, нижними гранями множества $\{10, 15\}$ являются числа 1 и 5. Если взять, например, число 2, то оно будет являться нижней гранью для $\{10\}$, но поскольку оно не сравнимо с 15, нижней гранью $\{10, 15\}$ оно не является.

Можно привести такой наглядный образ: пусть из точки диаграммы $x$ вниз по линиям стекает жидкость $A$, а из точки диаграммы $y$ стекает жидкость $B$. Если опять же рассматривать пример чисел $x=10$ и $y=15$, то жидкость $A$ течет через точки 10, 2, 5 и 1, а жидкость $B$ через точки 15, 3, 5 и 1. Точки, в которых жидкости смешиваются~--- это и есть общие грани множества $\{x, y\}$, а по совместительству ещё и множество общих делителей. Самая большая из таких точек есть наибольший общий делитель.

\begin{exercise}
Докажите, что наибольший общий делитель всегда определен однозначно.
\end{exercise}

Аналогично число $d$ называется общим кратным чисел $x$ и $y$, если оно делится одновременно и на $x$ и на $y$. Наименьшее из таких чисел~--- это наименьшее общее кратное, а по совместительству точная нижняя грань из всех общих кратных.

Часто в учебниках принято обозначать НОД как $\gcd$ (greatest common divisor), а НОК как $\lcm$ (least common multiplier). Нам НОК будет требоваться довольно редко, а вот для НОД мы примем другое обозначение: вместо $\gcd(x, y)$ будем писать просто $(x, y)$. На данный момент это может показаться странным и бессмысленным, но на это есть глубокая причина, которую читатель увидит в следующей главе.

\begin{definition}
Число называется \term{простым}, если имеет в точности два делителя.
\end{definition}

\begin{definition}
Число называется \term{составным}, если оно не простое.
\end{definition}

Два делителя~--- это само число и единица. Можно было бы сказать <<не имеет делителей, кроме себя и единицы>>, это было бы понятнее, но тогда такое определение распространялось бы и на саму единицу, которая в соответствии с данным определением простым числом не является. Может возникнуть вопрос: а почему не считать единицу за простое число? Быстрый ответ таков, что это просто удобнее в большинстве случаев. Более глубокий ответ будет дан в следующей главе.

\begin{definition}
Числа $x$ и $y$ называются \term{взаимопростыми}, если $(x, y) = 1$.
\end{definition}

Любое положительное число, если оно не простое, можно представить в виде произведения некоторых других чисел. Эти числа, если они не простые, опять же можно представить в виде какого-то произведения. Например,
$$120 = 10 \cdot 12 = 2\cdot 5 \cdot 2 \cdot 6 = 2\cdot 5 \cdot 2 \cdot 2 \cdot 3$$
Таким образом, любое положительное натуральное число мы можем представить в виде произведения простых чисел и простые числа оказываются в некотором смысле строительными блоками для всех остальных натуральных чисел. Возникает естественный вопрос: а сколько простых чисел всего? Или хотя бы их конечное число, или бесконечное? Ответ даёт следующая теорема.

\begin{Euclids}
Простых чисел бесконечно много.
\end{Euclids}
\begin{proof}
Пойдём от противного и предположим, что это не так. Пусть простых чисел всего $N$ штук. Обозначим их как ${p_i}$. Тогда число $1 + \prod_{i=1}^N p_i$ не делится ни на одно из $p_i$, и соответственно само является простым. Это противоречит тому, что простых чисел всего $N$ штук, откуда следует что простых чисел бесконечно много.
\end{proof}

Касательно разложения чисел на простые множители остаётся ещё такой вопрос: а однозначное ли это разложение? То есть понятно, что оно не однозначно с точки зрения порядка сомножителей, однако сами эти эти сомножители будут ли всегда одними и теми же, или могут отличаться? Так сходу это уже не докажешь, потребуется некоторая подготовка.

Займем из далека и начнем с быстрого способа нахождения НОД, который также носит имя Евклида. Пусть нам надо найти величину $(x, y)$. Вначале поделим с остатком $x$ на $y$:
$$x = q_0 y + r_0$$
Теперь поделим с остатком $y$ на $r_0$:
$$y = q_1 r_0 + r_1$$.
И будем продолжать этот нехитрый процесс:
$$r_0 = q_2 r_1 + r_2$$
$$r_1 = q_3 r_2 + r_3$$
$$\dots$$
Последовательность $\{r_i\}$ строго убывающая, поэтому в какой-то момент у нас эти числа поделятся без остатка:
$$r_{n-2} = q_n r_{n-1} + r_n$$
$$r_{n-1} = q_{n+1} r_n$$
Давайте покажем, что $r_n$ является НОД чисел $a$ и $b$. Из последней строчки видно, что $r_n | r_{n-1}$. Перепишем теперь предпоследнюю строчку таким образом:
$$r_{n - 2} - q_n r_{n - 1} = r_n$$
Здесь и $r_n$ и $r_{n-1}$ делятся на $r_n$, стало быть и $r_{n-2}$ обязана делиться на $r_n$. Поднимаясь ещё на строчку выше, получим аналогично, что $r_n|r_{n-3}$. Продолжая процесс нужное количество шагов, мы придём в результате к тому, что $r_n|x$ и $r_n|y$.

Это доказывает, что $r_n$ является общим делителем, но не доказывает, что это делитель наибольший. Пусть $d = (x, y)$. Из первой строчки алгоритма Евклида видно, что $d|r_0$. Из второй строчки теперь можно увидеть, что $d|r_1$. Спускаясь ниже, получаем, что $d|r_n$, но $r_n$ и сам является делителем $x$ и $y$. Это доказывает, что действительно $r_n = (x, y)$.

Сам по себе этот алгоритм хоть и быстр, но нам бесполезен: мы не собираемся искать НОД чисел. Что нам интересно, так это следствия из этого алгоритма.

Перепишем предпоследний шаг алгоритма Евклида как $r_n = r_{n-2} - q_n r_{n-1}$. Поднимемся на строчку выше, и из $r_{n-3} = q_{n-1} r_{n-2} + r_{n-1}$ подставим значение $r_{n-1}$:
$$r_n = r_{n-2} - q_n(r_{n-3} - q_{n-1}r_{n-2}) = (1+q_n q_{n-1})r_{n-2} - q_n r_{n-3}$$
Поднимаясь на строчку выше, мы можем выразить $r_{n-2}$ через $r_{n-3}$ и $r_{n-4}$, после чего мы получим следующее выражение:
$$r_n = \alpha r_{n-3} + \beta r_{n-4}$$
Здесь $\alpha$ и $\beta$~--- это некоторые коэффициенты, выражающиеся через значения $q_i$. Их можно найти строго, но они нас не волнуют. На самом деле возможно тут будет и не сумма, а разность, например, $\alpha r_{n-3} - \beta r_{n-4}$ или наоборот $\beta r_{n-4} - \alpha r_{n-3}$. Мы будем считать, что один из коэффициентов может идти со знаком минус (в следующей главе мы введем отрицательные числа и нам не надо будет делать таких оговорок, но пока будем считать так).

Продолжая процесс построчно вверх по алгоритму Евклида, мы в конечном счете приходим к такому выражению:

\begin{Bezus}
$(x, y) = \lambda x + \mu y$ для любых $x$ и $y$, где $\lambda$ и $\mu$ могут содержать знак минус.
\end{Bezus}
\begin{corollary}
Если числа $x$ и $y$ взаимопросты, то найдутся такие $\lambda$ и $\mu$, что
$$\lambda x + \mu y = 1$$
\end{corollary}
\begin{corollary}
Если числа $x$ и $y$ взаимопросты, то любое число $n$ может быть представлено как
$$n = \lambda x + \mu y$$
\end{corollary}
\begin{proof}
Первое следствие элементарно, поскольку оно фуктически дублирует определение взаимной простоты $(x, y) = 1$. Второе следствие получается из первого, если умножить обе части на $n$:
$$(n\lambda) x + (n\mu) y = n$$
\end{proof}

\begin{EuclidsLemma}
Пусть $a$ и $b$ произвольные числа и $p$~--- простое число такое, что $p|ab$. Тогда либо $p|a$, либо $p|b$.
\end{EuclidsLemma}
\begin{proof}
Предположим, что $p$ не делит $a$. В этом предположении докажем, что оно обязано тогда делить $b$. Если $p$ не делит $b$, то рассуждения будут симметричны, но в отношении $a$.

Поскольку $p$ простое и не делит $a$, это значит, что оно взаимопросто с $a$. По соотношению Безу
$$\lambda p + \mu a = 1$$
Умножим это на $b$:
$$ \lambda bp + \mu ab = b$$
В левой части оба слагаемых делятся на $p$. Значит, правая часть также должна делиться на $p$.
\end{proof}

Вот теперь мы готовы к финальному шагу.

\begin{FTA}
Любое натуральное число однозначно раскладывается на простые множители с точностью до порядка сомножителей.
\end{FTA}
\begin{proof}
Предположим, что это не так и некоторое число одновременно может быть записано как $p_1^{\alpha_1}\ldots p_n^{\alpha_n}$ и как $q_1^{\beta_1}\ldots q_n^{\beta_n}$, где $\{p_i\}$ и $\{q_j\}$ наборы различных простых чисел.

Возьмём некоторое произвольное $p_i$ из первого разложения. Лемма Евклида гарантирует нам, что $p_i$ обязано делить один из множителей $q_j^{\beta_j}$. Аналогично можно сказать, что и каждый из $q_j$ обязан делить один из множителей $p_i^{\alpha_i}$. Если учесть, что множители в одном разложении взаимопросты, то соответсвие получается взаимооднозначным. Таким образом, сами простые множители в обоих разложениях будут одинаковыми, но нам ещё надо доказать, что у них будут совпадать показатели степени.

Итак, пусть теперь у нас есть два разложения $p_1^{\alpha_1}\ldots p_n^{\alpha_n}$ и $p_1^{\beta_1}\ldots p_n^{\beta_n}$ и предположим, что $\alpha_i < \beta_i$ для некоторого $i$ (неравенство может быть и в другую сторону, доказательство в этом случае аналогично). Поделим оба этих разложения на $\alpha_i$. Тогда первое разложение более вообще не будет делиться на $p_i$, а второе разложение будет на него делиться. Из этого противоречия следует, что $\alpha_i=\beta_i$, что завершает доказательство.
\end{proof}

К этому доказательству можно было бы добавить формальных деталей, так как пока оно не слишком строго. Мы покажем как это делается в следующем параграфе.

Основная теорема арифметики имеет ряд немедленных простых следствий, которые составляют следующие упражнения, обязательные для выполнения:

\begin{exercise}
Это для разминки. Пусть $a = 2^{a_1}3^{a_2}5^{a_3}\ldots p_k^{a_k}$ и $b = 2^{b_1}3^{b_2}5^{b_3}\ldots p_k^{b_k}$~--- разложения чисел на простые множители (показатели степеней могут быть нулевыми). Докажите, что
$$ab = 2^{a_1+b_1}3^{a_2+b_2}5^{a_3+b_3}\ldots p_k^{a_k+b_k}$$
$${a\over b} = 2^{a_1-b_1}3^{a_2-b_2}5^{a_3-b_3}\ldots p_k^{a_k-b_k}$$
\end{exercise}

\begin{exercise}
Пусть $a$ и $b$ опять имеют разложение на простые множители так же как в прошлом упражнении. Докажите, что
$$(a, b) = 2^{\min\{a_1, b_1\}} 3^{\min\{a_2, b_2\}}\ldots p_k^{\min\{a_k, b_k\}}$$
$$\lcm(a, b) = 2^{\max\{a_1, b_1\}} 3^{\max\{a_2, b_2\}}\ldots p_k^{\max\{a_k, b_k\}}$$
\end{exercise}

\begin{exercise}
Докажите, что $ab = (a, b) \lcm(a, b)$.
\end{exercise}

\begin{exercise}
Докажите, что число делителей числа $a$ равнаяется величие
$$(a_1 + 1)(a_2 + 1)\ldots(a_k + 1)$$
\end{exercise}

\begin{exercise}
Докажите, что отношение делимости задаёт дистрибутивную решётку на $\mathbb{N}$ (решетки рассматривались в конце~\S2.2). Для тех, кто не разобрался с понятием дистрибутивной решетки, поясню, что это равносильно доказательству двух утверждений:
$$(a, \lcm(b, c)) = (\lcm(a, b), \lcm(a, c))$$
$$\lcm(a, (b, c)) = \lcm((a, b), (a, c))$$
\end{exercise}

\begin{exercise}
Встречаются два давнишних друга и между ними происходит следующий диалог:\\
--- Привет! Как жизнь?\\
--- Да вот, сыновья у меня родились.\\
--- Ого! Давно?\\
--- Да не, сыновья ещё маленькие, в школу даже не ходят.\\
--- А сколько им лет?\\
--- Произведение их возрастов равняется количеству голубей вокруг нашей скамейки.\\
--- Чего-то ничего не понятно.\\
--- Старший похож на мать.\\
--- А, теперь понятно.\\
Вопрос: сколько лет сыновьям?
\end{exercise}

Конечно, эта задача из области головоломок. В следующем абзаце я привожу две подсказки, которые почти сразу дадут решение, поэтому не читайте следующий абзац, если ещё самостоятельно не успели подумать.

Секрет кроется в трёх нюансах. Во-первых, то что дети не ходят в школу даёт нам понять, что им меньше семи лет. Во-вторых, фраза о том, что <<старший похож на мать>> говорит нам о том, что дети не являются близнецами. Третий нюанс самый тонкий: собеседник, который отвечает <<теперь всё понятно>> видит конкретное количество голубей, и нам достоверно известно из диалога, что он сам каким-то образом сумел понять сколько голубей вокруг лавки. То есть это число такого, что двух первых условий оказывается достаточно для именно однозначного ответа. Думайте.

\section{Принцип Дирихле}

Предположим, что у нас есть $n$ клеток и $n+1$ голубь, и мы рассадили голубей по клеткам. Довольно очевидно, что в одной из клеток окажется по крайней мере два голубя. Впервые это тривиальные наблюдение сделал и начал использовать в математике немецкий математик Дирихле в XIX веке. С тех пор это наблюдение называется в России <<принципом Дирихле>>, а за пределами бывшего СССР <<принципом голубиных гнёзд>> (pigeonhole principle). Прежде чем перейти к более сложным примерам применения этого принципа, переформулируем утверждение на языке множеств.

\begin{thm}
Пусть $A$ и $B$ такие множества, что $|A|>|B|$, и дана функция $f:A\to B$. Тогда функция $f$ не является инъективной.
\end{thm}
\begin{proof}
Для инъективной функции $|f(A)| = |A|$, однако в нашем случае это противоречит условию $|A|>|B|$, поскольку $f(A)\subset B$.
\end{proof}

\begin{exercise}
Утверждение принципа Дирихле тривиально в терминах множеств и отображений. А сможете ли вы доказать этот принцип, используя только аксиоматику Пеано?
\end{exercise}

Введём обозначение $\lceil {m\over n} \rceil = \min\{k|kn\ge m\}$. Эта операция называется \term{потолок} и интуитивно представляет собой результат деления $m$ на $n$ с округлением в большую сторону: если число не делится без остатка, то к частному прибавляется единица.

\begin{exercise}
Пусть теперь у нас $m$ голубей и $n$ клеток. Докажите, что в одной из клеток окажется по крайней мере $\lceil{m\over n}\rceil$ голубей. Эта теорема называется \term{обобщённым принципом Дирихле}.
\end{exercise}

Рассмотрим теперь ряд простых применений принципа Дирихле. Частично они позаимствованы из Википедии, частично из замечательной книги <<A Walk Through Combinatorics>>\footnote{<<A Walk Through Combinatorics>>, Miklós Bóna, Singapore: World Scientific, 2011}, частично просто выплыли из глубин памяти.

\begin{example}
Пусть в ящике лежит большое число белых, чёрных и красных носков. Когда мы вытаскиваем носок, мы не видим его цвет до тех пор, пока не вытащим его. Сколько надо достать носков, прежде чем мы гарантированно получим пару одного цвета?
\end{example}

Пусть множество $A$~--- это те носки, которые мы достали из ящика. Пусть далее множество $B$~--- это множество цветов носков. В нашем случае $|B|=3$. Каждый носок имеет цвет, что соответствует отображению $f:A\to B$. В задаче требуется, чтобы у нас минимум два носка обладали одним цветом, то есть чтобы функция $f$ не была инъективной. По принципу Дирихле это достигается когда $|A|>|B|$. Таким образом нам достаточно взять $$|A|=|B|+1 = 4$$ носков для того, чтобы гарантированно найти пару одинаковых цветов.

\begin{exercise}
На праздник пришло $n$ человек. Какие-то из этих людей поздоровались за руку. Докажите, что всегда найдётся такая пара людей, которые совершили одинаковое число рукопожатий.
\end{exercise}

Данное утверждение называется <<леммой о рукопожатиях>> и может быть сформулировано более абстрактно для графов:

\begin{definition}
\term{Степенью} вершины графа называется число рёбер, инцидентных ей.
\end{definition}

\begin{HandshakesLemma}
В любом графе найдутся две вершины с одинаковой степенью.
\end{HandshakesLemma}

\begin{example}
Давайте докажем, что в Москве найдётся по крайней мере два человека с одинаковым числом волос на голове.

Пусть $A$~--- множество жителей Москвы, $B$~--- множество чисел, соответствующих возможному числу волос на голове. Точно мы это множество определить видимо не можем, но по крайней мере можем дать грубую оценку $|B|<1000000$. Однако жителей Москвы явно больше. Опять же мы не знаем этого точно, но по крайней мере $|A|>1000000$, откуда следует то отображение $A\to B$ не может быть инъективным и найдутся два жителя с одинаковым числом волос.
\end{example}

\begin{example}
Рассмотрим последовательность чисел 1, 11, 111, 1111, $\ldots$ и докажем, что в этой последовательности найдётся число, которое делится на 123.

Вначале обозначим элементы этой последовательности как $a_i$ и заметим, что их можно записать как
$a_i = \sum_{k=0}^{i-1}10^k$

Предположим теперь, что числа, которое делится на 123, в этой последовательности нет. Рассмотрим тогда последовательность остатков от деления на 123 элементов этой последовательности:
$$1, 11, 111, 4, 41, 42, 52, 29, \ldots$$

В этой последовательности обязательно найдутся два одинаковых числа, поскольку остаток от деления на 123 может максимум равняться 122, поэтому по принципу Дирихле уже среди первых 123 элементов мы встретим повторение. Пусть это будут остатки от деления чисел $a_i$ и $a_j$ на 123 (для определённости пусть $i>j$), которые мы обозначим за $r$:
$$a_i = 123q_i + r$$
$$a_j = 123q_j + r$$
Вычтя из первого равенства второе имеем
\begin{equation}\label{np:1}
a_i - a_j = 123(q_i - q_j)
\end{equation}
Однако если вспомнить определение $a_i$ то легко увидеть, что
\begin{align*}
a_i - a_j &= \sum_{k=0}^{i-1}10^k - \sum_{k=0}^{j-1}10^k \\&= \sum_{k=j}^{i-1}10^k \\&= 10^j\sum_{k=0}^{i-j}10^k \\&= 10^j a_{i-j}
\end{align*}
Сопоставляя это с \eqref{np:1}, получаем
$$10^j a_{i-j} = 123(q_i - q_j)$$
Здесь правая часть делится на 123, а значит на 123 должна делится и левая часть. Однако $10^j$ и 123 взаимопросты, откуда следует, что $123|a_{i-j}$, что и требовалось.
\end{example}

\begin{example}
Докажем, что за последнюю тысячу лет у читателя был такой предок $A$, который одновременно является предком и отца и матери некоторого другого предка $P$ читателя.

У читателя есть отец и мать (2 человека). У них у каждого так же в свою очередь есть по отцу и матери~--- это дедушки и бабушки читателя, всего их $4=2^2$ человека. У них в свою очередь есть свои мамы и папы (прабабушки и прадедушки), которых всего $8=2^3$ человек. Легко увидеть закономерность: $i$-е поколение предков читателя состоит из $2^i$ человек. Если грубо предположить, что поколения сменяются каждые 25 лет, то за 1000 лет поколения успели смениться 40 раз, и того за 1000 лет читатель имел
$$\sum_{i=1}^{40} 2^i = 2^{41} - 1$$
предков. Последнее равенство легко следует из свойств двоичной системы счисления (проверьте!).

В то же время население нашей планеты в данный момент меньше чем $10^{10}$, а в прошлые года оно было ещё меньше. Как самая грубая верхняя оценка количества людей живущих на земле в последнюю 1000 лет можно взять величину $40\cdot 10^{10}$~--- она намного выше реального количества людей, проживавших на земле, но однако даже эта оценка не превосходит величины $2^{41} - 1$, в чём легко убедиться, взяв в руки калькулятор.

Рассуждение будет понятнее, если посмотреть на генеалогическое дерево на рисунке~3.4.

\begin{figure}[h]
\centering
\begin{tikzpicture}
\def\point{node [circle, draw, fill, inner sep = 0, minimum size = .1cm] }
\draw (0, 0) \point (p0) {};
\draw (-1cm, .5cm) \point (p1) {};
\draw (1cm, .5cm) \point (p2) {};
\draw (-1.5cm, 1cm) \point (p3) {};
\draw (-.5, 1cm) \point (p4) {};
\draw (.5cm, 1cm) \point (p5) {};
\draw (1.5cm, 1cm) \point (p6) {};
\draw (-1.7cm, 1.5cm) \point(p7) {};
\draw (-1.3cm, 1.5cm) \point(p8) {};
\draw (-.7cm, 1.5cm) \point(p9) {};
\draw (-.3cm, 1.5cm) \point(p10) {};
\draw (.3cm, 1.5cm) \point(p11) {};
\draw (.7cm, 1.5cm) \point(p12) {};
\draw (1.3cm, 1.5cm) \point(p13) {};
\draw (1.7cm, 1.5cm) \point(p14) {};

\node [below] at (p0) {читатель ($i=0$)};
\node [left] at (p1) {мама ($i = 1$)};
\node [right] at (p2) {папа};
\node [left] at (p3) {($i=2$)};
\node [left] at (p7) {($i=3$)};

\draw (p0) -- (p1);
\draw (p0) -- (p2);
\draw (p1) -- (p3);
\draw (p1) -- (p4);
\draw (p2) -- (p5);
\draw (p2) -- (p6);
\draw (p3) -- (p7);
\draw (p3) -- (p8);
\draw (p4) -- (p9);
\draw (p4) -- (p10);
\draw (p5) -- (p11);
\draw (p5) -- (p12);
\draw (p6) -- (p13);
\draw (p6) -- (p14);
\end{tikzpicture}
\caption{Генеалогическое дерево}
\end{figure}

Предположение о том, что это дерево будет ветвиться именно в таком виде всю тысячу лет противоречит принципу Дирихле: если бы это было так, то дерево имело бы куда больше узлов, нежели за 1000 лет жило людей на земле. А отсюда значит, что где-то наверху по крайней мере две ветви генеалогического дерева сомкнутся. А это ровно то, что требовалось доказать.
\end{example}

\begin{example}
Мистер и миссис Смит пригласили к себе в гости четыре пары. Некоторые из приглашённых были друзьями мистера Смита, а некоторые друзьями миссис Смит. Когда гости прибыли, те, кто знали друг друга ранее, пожали руки. Когда всё это произошло, мистер Смит говорит: <<Как интересно! Если не считать меня, то здесь никто не поздоровался за руку одинаковое количество раз>>. Вопрос: сколько раз пожала руку миссис Смит?

На первый взгляд задача кажется абсолютно нерешаемой. Тем не менее, проявив стойкость, мы можем её решить опять же с помощью всё того же принципа Дирихле. Начинает задача поддаваться решению, если мы изобразим её в виде графа.

\begin{figure}[h]
\centering
\begin{tikzpicture}
\def\point{node [circle, draw, fill, inner sep = 0, minimum size = .1cm] }
\draw (-1.76cm, 2.42cm) \point (smith) {};
\draw (0, 3cm) \point (p8) {};
\draw (1.76cm, 2.43cm) \point (p7) {};
\draw (2.85cm, 0.93cm) \point (p6) {};
\draw (2.85, -0.93cm) \point (p5) {};
\draw (1.76cm, -2.43cm) \point (p4) {};
\draw (0, -3cm) \point (p3) {};
\draw (-1.76cm, -2.43cm) \point(p2) {};
\draw (-2.85cm, -.93cm) \point(p1) {};
\draw (-2.85cm, .93cm) \point(p0) {};

\node [left] at (smith) {м.Смит};
\node [above] at (p8) {8};
\node [above] at (p7) {7};
\node [above] at (p6) {6};
\node [above] at (p5) {5};
\node [above] at (p4) {4};
\node [above] at (p3) {3};
\node [above] at (p2) {2};
\node [above] at (p1) {1};
\node [above] at (p0) {0};

\end{tikzpicture}
\caption{Граф гостей четы Смит}
\end{figure}

Каждый человек на вечеринке у нас будет представлен отдельной вершиной. Ребра графа будут соответствовать рукопожатиям. Семейные пары будем выделять рамкой. На начальном этапе нам известно лишь, что все присутствующие кроме мистера Смита совершили различное количество рукопожатий. Максимальное количество рукопожатий, которые могло быть совершено~--- 8 (всего десять гостей, причём со своим спутником никто не здоровается, отсюда максимум восемь рукопожатий). Поэтому все вершины мы можем пронумеровать числами от 0 до 8, и одну вершину мы обозначим просто как Смит~--- мы не знаем сколько рукопожатий он совершил. Самих гостей мы будем нумеровать так же, то есть когда мы будем говорить фразу <<пятый гость>>, то мы будем подразумевать гостя, который совершил пять рукопожатий. Получившийся граф представлен на рис.~3.5.

\begin{figure}[h]
\centering
\begin{tikzpicture}
\def\point{node [circle, draw, fill, inner sep = 0, minimum size = .1cm] }
\draw (-1.76cm, 2.42cm) \point (smith) {};
\draw (0, 3cm) \point (p8) {};
\draw (1.76cm, 2.43cm) \point (p7) {};
\draw (2.85cm, 0.93cm) \point (p6) {};
\draw (2.85, -0.93cm) \point (p5) {};
\draw (1.76cm, -2.43cm) \point (p4) {};
\draw (0, -3cm) \point (p3) {};
\draw (-1.76cm, -2.43cm) \point(p2) {};
\draw (-2.85cm, -.93cm) \point(p1) {};
\draw (-2.85cm, .93cm) \point(p0) {};

\node [left] at (smith) {м.Смит};
\node [above] at (p8) {8};
\node [above] at (p7) {7};
\node [above] at (p6) {6};
\node [above] at (p5) {5};
\node [right] at (p4) {4};
\node [below] at (p3) {3};
\node [left] at (p2) {2};
\node [above] at (p1) {1};
\node [above] at (p0) {0};

\draw (p8) -- (smith);
\draw (p8) -- (p1);
\draw (p8) -- (p2);
\draw (p8) -- (p3);
\draw (p8) -- (p4);
\draw (p8) -- (p5);
\draw (p8) -- (p6);
\draw (p8) -- (p7);

\draw[dashed] (0, 3.2cm) -- (.4cm, 2.8cm) -- (-2.9cm, .68cm) -- (-3.3cm, .92cm)  -- cycle;
\end{tikzpicture}
\caption{Граф гостей четы Смит}
\end{figure}

Рассмотрим поближе 8-го гостя. Он не поздоровался лишь с одним человеком с одной стороны, и с другой стороны он очевидно не здоровался со своим спутником. Так же мы знаем точно, что он не здоровался с нулевым гостем, так как нулевой гость не совершил вообще ни одного рукопожатия. Соответственно нулевой гость и есть его спутник. Со всеми остальными гостями он поздоровался. Полученный результат изображён на рисунке~3.6.

\begin{figure}[h]
\centering
\begin{tikzpicture}
\def\point{node [circle, draw, fill, inner sep = 0, minimum size = .1cm] }
\draw (-1.76cm, 2.42cm) \point (smith) {};
\draw (0, 3cm) \point (p8) {};
\draw (1.76cm, 2.43cm) \point (p7) {};
\draw (2.85cm, 0.93cm) \point (p6) {};
\draw (2.85, -0.93cm) \point (p5) {};
\draw (1.76cm, -2.43cm) \point (p4) {};
\draw (0, -3cm) \point (p3) {};
\draw (-1.76cm, -2.43cm) \point(p2) {};
\draw (-2.85cm, -.93cm) \point(p1) {};
\draw (-2.85cm, .93cm) \point(p0) {};

\node [left] at (smith) {м.Смит};
\node [above] at (p8) {8};
\node [above] at (p7) {7};
\node [above] at (p6) {6};
\node [above] at (p5) {5};
\node [right] at (p4) {4};
\node [below] at (p3) {3};
\node [left] at (p2) {2};
\node [above] at (p1) {1};
\node [above] at (p0) {0};

\draw (p8) -- (smith);
\draw (p8) -- (p1);
\draw (p8) -- (p2);
\draw (p8) -- (p3);
\draw (p8) -- (p4);
\draw (p8) -- (p5);
\draw (p8) -- (p6);
\draw (p8) -- (p7);

\draw (p7) -- (smith);
\draw (p7) -- (p2);
\draw (p7) -- (p3);
\draw (p7) -- (p4);
\draw (p7) -- (p5);
\draw (p7) -- (p6);

\draw[dashed] (0, 3.2cm) -- (.4cm, 2.8cm) -- (-2.9cm, .68cm) -- (-3.3cm, .92cm)  -- cycle;
\draw[dashed] (1.76cm, 3cm) -- (2.5cm, 2.5cm) -- (-3cm, -1.3cm) -- (-3.2cm, -.65cm) -- cycle;

\end{tikzpicture}
\caption{Граф гостей четы Смит}
\end{figure}

Теперь рассмотрим 7-го гостя. Он не поздоровался за руку с двумя гостями, один из которых~--- его партнёр, а вторым должен быть нулевой гость (нулевой гость не может быть партнёром 7-го гостя, так как мы уже выяснили, что нулевой и восьмой гости образуют пару). Глядя на граф мы так же видим, что первый гость поздоровался с восьмым гостем, но так как нам известно, что он в принципе поздоровался лишь с одним человеком, то он не мог поздороваться с седьмым гостем. Значит, первый и седьмой гости образуют пару, и седьмой гость поздоровался со всеми кроме нулевого и первого гостя. Это отображено на рисунке~3.7.

\begin{figure}[h]
\centering
\begin{tikzpicture}
\def\point{node [circle, draw, fill, inner sep = 0, minimum size = .1cm] }
\draw (-1.76cm, 2.42cm) \point (smith) {};
\draw (0, 3cm) \point (p8) {};
\draw (1.76cm, 2.43cm) \point (p7) {};
\draw (2.85cm, 0.93cm) \point (p6) {};
\draw (2.85, -0.93cm) \point (p5) {};
\draw (2cm, -2.8cm) \point (p4) {};
\draw (0, -3cm) \point (p3) {};
\draw (-1.76cm, -2.43cm) \point(p2) {};
\draw (-2.85cm, -.93cm) \point(p1) {};
\draw (-2.85cm, .93cm) \point(p0) {};

\node [left] at (smith) {м.Смит};
\node [above] at (p8) {8};
\node [above] at (p7) {7};
\node [above] at (p6) {6};
\node [right] at (p5) {5};
\node [right] at (p4) {4};
\node [below] at (p3) {3};
\node [left] at (p2) {2};
\node [above] at (p1) {1};
\node [above] at (p0) {0};

\draw (p8) -- (smith);
\draw (p8) -- (p1);
\draw (p8) -- (p2);
\draw (p8) -- (p3);
\draw (p8) -- (p4);
\draw (p8) -- (p5);
\draw (p8) -- (p6);
\draw (p8) -- (p7);

\draw (p7) -- (smith);
\draw (p7) -- (p2);
\draw (p7) -- (p3);
\draw (p7) -- (p4);
\draw (p7) -- (p5);
\draw (p7) -- (p6);

\draw (p6) -- (smith);
\draw (p6) -- (p3);
\draw (p6) -- (p4);
\draw (p6) -- (p5);

\draw (p5) -- (smith);
\draw (p5) -- (p4);

\draw[dashed] (0, 3.2cm) -- (.4cm, 2.8cm) -- (-2.9cm, .68cm) -- (-3.3cm, .92cm)  -- cycle;
\draw[dashed] (1.76cm, 3cm) -- (2.5cm, 2.5cm) -- (-3cm, -1.3cm) -- (-3.2cm, -.65cm) -- cycle;
\draw[dashed] (2.85cm, 1.4cm) -- (3.3cm, 0.9cm) -- (-2.2cm, -3cm) -- (-2.26cm, -2cm) -- cycle;
\draw[dashed] (3cm, -0.5cm) -- (3.3cm, -1.1cm) -- (-.02cm, -3.5cm) -- (-.43cm, -2.5cm) -- cycle;

\end{tikzpicture}
\caption{Граф гостей четы Смит}
\end{figure}

Совершенно аналогичным образом мы можем показать, что шестой гость не здоровался с гостями 0, 1 и 2, и что 2-ой гость является его спутником. Отсюда можно аналогично получить, что пятый гость не здоровался с гостями 0, 1, 2 и 3, и что третий гость является его спутником. Результат представлен на рисунке~3.8.

Теперь мы нашли пару для всех людей, кроме четвёртого. Единственная возможная пара для четвёртого человека~--- это мистер Смит, поэтому четвёртый человек и есть миссис Смит и она пожала руку ровно четырём людям.
\end{example}

\begin{exercise}
Будем считать, что костяшка домино равна по размеру двум клеткам на шахматной доске. Удалим из шахматной доски две противоположные угловые клетки (например, А1 и H8). Возможно ли теперь заполнить такую доску целиком костяшками домино? (Костяшки должны покрывать все клетки, не могут вылезать за границы доски, две костяшки не могут занимать одну клетку).
\end{exercise}

Для следующего упражнения этого параграфа опять введём новую нотацию: будем обозначать через $a\Mod b$ остаток от деления $a$ на $b$.

\begin{exercise}
Пусть $n$ и $m$~--- взаимопростые числа и пусть нам известно, что
$$x\Mod n = a$$
$$x\Mod m = b$$
Докажите, используя принцип Дирихле, что это уравнение всегда имеет ровно одно решение $x$, меньшее $mn$.
\end{exercise}

\begin{exercise}
Читатель наверняка сталкивался с программами, называемыми архиваторами, которые сжимают данные на компьютере. Они это делают обыкновенно за счёт обнаружения каких-то закономерностей в данных, что используется для более компактного представления.  Можно привести почти тривиальный пример: пусть в текстовом файле записано какое-то число в десятичной системе счисления. Если программа архивирования способна обнаружить, что в файле записано число, а не текст, то она может заменить текстовые символы (в кодировке ASCII, например, они занимают 1 байт каждый), и записать вместо текста двоичное значение указанного в файле числа, то можно значительно сократить размер файла: например, для 100-значного числа изначально требуется 100 байт, а после записи в двоичной системе счисления всего лишь 42 байта. Это сжатие более чем в два раза, а это значительно. Приведённый пример конечно совершенно тривиальный, реальные данные устроены не так просто и закономерности, которые отыскивают архиваторы, не так тривиальны.

В информатике есть довольно общий теоретический результат: не существует алгоритма, который сжимал бы любые данные. Один из способов доказать это выгладит так: предположим, что такой алгоритм всё же существует и нам дан файл размером $N$ мегабайт. После сжатия этот файл будет иметь размер $N_1 < N$. После повторного сжатия этот файл будет иметь размер $N_2 < N_1$. По нашему предположению этот процесс можно продолжать сколько угодно раз, и в итоге мы придём к файлу нулевого размера. Очевидно, что это невозможно. Мы получили противоречие, следовательно такого алгоритма действительно не существует.

Второй подход к доказательству использует принцип Дирихле и позволяет доказать даже более сильное утверждение: для любого алгоритма и любого размера данных $N$, найдутся такие данные, которые которые не могут быть сжаты. Докажите это более сильное утверждение самостоятельно.
\end{exercise}

\begin{exercise}
В первой главе мы ввели понятие контекстно-свободной грамматики, когда говорили о формальных языках. Для любой контекстно-свободной грамматики справедливо следующее утверждение: в любой достаточно длинной строке найдутся такие две подстроки, что их дублирование приведёт так же к строке данной контекстно-свободной грамматики.

Для примера можно рассмотреть простую грамматику парных скобок:
$$S \to () | \epsilon | (S)$$
Строка $(()())$ принадлежит этой грамматике. Мы можем взять первое и второе вхождение подстроки $()$ и продублировать их: $(()()()())$, либо мы могли бы взять подстроки $(($ и $))$ и опять же продублировать их: $(((()())))$. Это так же будет строка той же грамматики. Вам требуется доказать, что в любой достаточно длинной строке найдутся две подстроки, дублирование которых даёт строку той же контекстно-свободной грамматики.
\end{exercise}

\begin{exercise}
Последнее упражнение помогает понять для отдельных языков, являются ли они контекстно-свободными. Воспользуйтесь им для того, чтобы доказать, что язык, состоящий из строк вида $$aa\ldots abb\ldots b\ldots cc\ldots c$$ где количество символов $a$, $b$ и $c$ одинаково, не определяется никакой контекстно-свободной грамматикой.
\end{exercise}

\section{Принцип индукции}

Пусть $P(n)$~--- некоторое утверждение, в котором как-то фигурирует натуральное число $n$. Пусть нам удалось доказать истинность $P(0)$, а также следствие $P(n)\to P(S(n))$ для любого $n$, где функция $S(n)$ обозначает элемент, следующий за $n$. Используя это следствие, мы можем получить $P(0)\to P(1)$. Отсюда, опять же по тому же следствию $P(1)\to P(2)$. Затем $P(2)\to P(3)$. Продолжая так бесконечно долго, мы получаем, что утверждение истинно вообще для любого натурального $n$.

Приведённое рассуждение довольно неформально, фраза <<продолжая так бесконечно долго>> явно требует уточнения. Строгие рассуждения ниже довольно жёсткие по содержанию, их можно пока пропустить до первого примера ниже, если у вас возникнут проблемы с пониманием. Более обстоятельно мы подобные рассуждения будем рассматривать в главе, посвящённой бесконечным множествам, и тогда у вас уже вопросов по этой теме не должно будет остаться.

Прежде всего попытаемся понять, почему приведённое нами доказательство не может считаться удовлетворительным. Давайте для примера вместо натуральных чисел рассмотрим множество $\mathbb{N}^2$, упорядоченное лексикографически. Напомню, что это значит, что мы рассматриваем пары натуральных чисел $(a, b)$, а для сравнения на больше и меньше используем тот же принцип, по которому упорядочены слова в алфавите: вначале мы сравниваем первые числа, и только если они равны, то сравниваем вторые числа. Например:

$$(1, 2) < (10, 1)$$
$$(4, 4) > (4, 1)$$

Попробуем применить индукцию теперь к этому множеству. Из $P((0, 0))$ следует $P((0, 1))$, отсюда следует $P((0, 2))$, затем $P((0, 3)), P((0, 4))$ и так далее. Очевидно, что в этих рассуждениях мы никогда не достигнем даже значения $P((1, 0))$, поэтому метод индукции на таком множестве не работает.

Мы конечно это множество сконструировали специальным образом, но по большому счёту у нас нет никаких жёстких гарантий, что в случае $\mathbb{N}$ индукция действительно достигнет каждого элемента. Вдруг всё же есть какой-то элемент $k\in \mathbb{N}$, который мы никогда не достигнем? Маловероятно, но нам надо быть уверенными, что подобного случаю $\mathbb{N}^2$ не произойдёт.

Теперь изложим корректное формальное доказательство. Предположим, что индукция не верна для $\mathbb{N}$ и непустое множество чисел, для которых утверждение $P(n)$ не выполняется, обозначим как
$$X = \{n\in\mathbb{N}|\neg P(n)\}$$
Это множество имеет минимальный элемент $m=\min X$, и поскольку это наименьшее число, для которого $P$ ложно, мы знаем, что $P(m-1)$ истинно. Однако отсюда и из импликации $P(n)\to P(S(n))$ следует истинность $m$, а это противоречие. Значит правда: для натуральных чисел индукция работает.

Метод индукции можно обобщить двумя способами. Во-первых, мы могли бы начинать наш отсчёт не с P(0), а с произвольного элемента $P(k)$. На доказательство и общий принцип применения индукции это никак не повлияло бы, поэтому мы не будем рассматривать этот случай.

Второе обобщение индукции заключается в том, что вместо импликации $P(n)\to P(S(n))$ мы могли бы рассматривать импликацию $(\forall m<n, P(m))\to P(n)$. Здесь мы в доказательстве опираемся не только на одно значение $n$, но на все значения, меньшие заданного, для которых уже доказана истинность $P(m)$. Справедливость такого принципа индукции доказывается аналогично: если бы существовало непустое множество чисел $X$, для которых утверждение неверно, мы могли бы взять минимальный элемент этого множества $m$, а отсюда мы сразу же приходим к противоречию как и в первом случае.

Первоначальный подход называется \term{слабой индукцией}, последний подход называется \term{сильной индукцией}. Если рассматривать только натуральные числа, то разница между двумя индукциями невелика. Слабая индукция следует из сильной при замене $\forall m<n P(m)$ на $P(n-1)$, сильная индукция может получиться из слабой, если ввести множества
$$A_i = \{0, 1, 2, \ldots, i\}$$
и доказывать, используя слабую индукцию, утверждение
$$P'(n) = \forall x \in A_n, P(x)$$
вместо первоначального $P(n)$. То есть эти подходы эквивалентны.

Тем не менее, сильная индукция может быть легко обобщена на довольно широкий класс множеств: если внимательно вглядеться в доказательство сильной индукции, то единственное, что мы требуем от множества, на котором формулируем принцип индукции, это чтобы любое его подмножество $X$ имело минимальный элемент. В первом параграфе мы называли такие множества фундированными, и теперь понятно почему они заслуживают отдельного определения: такие множества (и только такие, см. упражнение ниже) допускают применять к себе принцип индукции.

Пока что мы не будем всерьёз заниматься индукцией на произвольных фундированных множествах, вернувшись к ним в главе, посвящённой бесконечным множествам, но в качестве необязательного попробуйте доказать следующее

\begin{exercise}
Пусть на множестве $X$ работает принцип сильной индукции. Докажите, что это множество фундированное.
\end{exercise}

Всё, что мы до сих пор говорили, относилось только к аксиоматике Фреге-Рассела, однако из сформулированных нами аксиом Пеано принцип индукции вывести невозможно. Напомню, что мы определили аксиомы Пеано как некоторое множество с заданной на нём инъективной функцией $S$, для который 0 не имеет обратного элемента. Дополнительно мы ввели определение сложения, умножения и возведения в степень (см.\S3.1). Обозначим этот набор аксиом как $\mathbb{P}^-$.

Если внимательно посмотреть на доказательство слабой индукции, то можно заметить, что мы по элементу $m=\min X$ для построения противоречия искали \term{предыдущий элемент} $m-1$. То есть мы неявно предполагали существование функции
\begin{align*}
S^{-1}: & \mathbb{N}\backslash\{0\} \to \mathbb{N}\\
    & n \mapsto n-1
\end{align*}
Для аксиом Фреге-Рассела определить её легко:
$$S^{-1}(n) = \{k \in n | S(k) \in n\}$$
Эта формула станет понятной, если вы вспомните, что
$$n = \{0, 1, 2, \ldots, (n-1)\}$$
А вот из $\mathbb{P}^-$ определить такую функцию уже невозможно. Почему? Ответ здесь кроется в том, что и $\mathbb{N}$ в терминах Фреге-Рассела, и введённое выше множество $\mathbb{N}^2$ оба являются моделями\footnote{Для формального определения модели требуется как минимум доопределить арифметические операции и потребовать, чтобы операция S продолжалась до строгого линейного порядка, что само по себе ставит перед нами дополнительные вопросы; однако мы не будем вдаваться в формальные подробности глубже, так как на данный момент наша цель сводится лишь к пониманию принципа индукции. В пятой главе отдельные вопросы о моделях арифметики мы рассмотрим подробнее.} для $\mathbb{P}^-$. В последнем случае элемент $(1, 0)$ предыдущего элемента не имеет. Ну и плюс к этому, поскольку в первом случае слабая индукция работает, а во втором не работает, то это значит, что принцип слабой индукции из $\mathbb{P}^-$ вовсе не выводим никак (см.\S1.5).

Итак сформулированные нами до сих пор аксиомы Пеано $\mathbb{P}^-$ не полноценны, пока мы не добавим к ним аксиому индукции:
$$(P(0)\land(\forall n (P(n)\to P(S(n)))) \to \forall m, P(m)$$

Вот теперь определение аксиоматики Пеано нами завершено и с этой дополнительной аксиомой какие-то <<нестандартные модели>> арифметики хоть и возможны, но строятся уже не так просто и мы их рассматривать не будем. Простейшие примеры типа $\mathbb{N}^2$ теперь не проходят, т.к. прицнип слабой индукции накладывает очень серьезные ограничения на модель. Я замечу, что вместо аксиомы индукции можно было бы потребовать выполнения каких-то других аксиом, например потребовать, чтобы функция $S|_{\mathbb{N}\backslash\{0\}}$ была обратима, однако такие варианты аксиом видимо менее интуитивно понятны, поэтому редко формулируются в таком виде.

Перейдём от теории к практике и рассмотрим примеры применения принципа индукции.

\begin{example}
Докажем закон Де Моргана для произвольного конечного набора множеств:
\begin{equation}\label{ni:1}
\left(\bigcup_{i=0}^n A_i\right)^C = \bigcap_{i=0}^n A_i^C
\end{equation}
\end{example}

Формула \eqref{ni:1} будет выступать у нас в роли доказываемого предложения $P(n)$. Мы будем начинать отсчёт индукции не с $P(0)$, а с $P(2)$, поскольку это первый нетривиальный случай:
\begin{equation}\label{ni:2}
(A\cup B)^C = A^C \cap B^C
\end{equation}
Это мы уже доказывали в параграфе \S~2.1. Теперь докажем импликацию $P(n)\to P(n + 1)$. Запись $S(n)$ будем использовать для обозначения исходного выражения.
$$S(n+1) = \left( \bigcup_{i=0}^{n+1} A_i \right)^C = \left( \left( \bigcup_{i=0}^n A_i\right) \bigcup A_{n+1}\right)^C$$
К этому выражению применимо тождество \eqref{ni:2}, из которого получаем
$$S(n+1) = \left( \bigcup_{i=0}^n A_i\right)^C \bigcap A_{n+1}^C$$
Однако мы предполагаем, что $P(n)$ истинно, поэтому к выражению слева мы можем легко применить \eqref{ni:1}:
$$S(n+1) = \left(\bigcup_{i=0}^n A_i^C\right) \bigcap A_{n+1}^C = \bigcup_{i=0}^{n+1} A_i^C$$
А это ровно то, что требовалось доказать. Это же самое доказательство работает и для пересечения множеств и для логических операций.

\begin{example}\label{ex:naturals_assoc}
Докажите, что в выражении
$$a_0 a_1 a_2 \ldots a_n$$
где $a_i$ натуральные числа, не важно как расставлять скобки.
\end{example}

Утверждение кажется очевидным, но на самом деле это не так. Пусть есть четыре числа 100, 234, 135 и 77. Мы тут утверждаем, например, что
$$(100\cdot 234)\cdot (135 \cdot77) = 23400 \cdot 10395 = 3159000 \cdot 77 = ((100\cdot 234)\cdot 135) \cdot77$$
То что это действительно так все знают, это кажется очевидным, так как мы привыкли к такому с детства, хотя в школе этого и не доказывали: к этому приучали. Но вообще-то это надо доказывать, так как если просто смотреть на конкретные числа, данные равенства выглядят уже не слишком убедительно сами по себе.

Прежде чем двигаться дальше, я советую читателю попробовать доказать утверждение примера самостоятельно~--- это совершенно не сложно. Только когда попытаетесь порассуждать сами, читайте дальше.

Простейший нетривиальный случай это $P(3)$~--- его мы доказали в~\S3.1, это обычный закон ассоциативности. Далее воспользуемся сильной индукцией: предположим что верно $\forall m<n, P(m)$ и докажем отсюда $P(n)$. Пусть изначально скобки расставлены следующим образом:
$$(a_0\ldots a_k)(a_{k+1}\ldots a_n)$$
Это означает, что сначала в некотором порядке отдельно будут перемножены группы чисел $a_0\ldots a_k$ и $a_{k+1}\ldots a_n$, а затем будут перемножены результаты. Внутри этих групп числа могут перемножаться в любом порядке в силу предположения индукции. Для того, чтобы доказать возможность произвольной расстановки скобок, нам надо показать, что это выражение эквивалентно
$$(a_0\ldots a_m)(a_{m+1}\ldots a_n)$$
где $m$~--- некий произвольный номер. Будем считать для определённости, что $m>k$, случай $m<k$ доказывается аналогично. Имеем:
$$(a_0\ldots a_k)((a_{k+1}\ldots a_m)(a_{m+1}\ldots a_n))$$
По закону ассоциативности это можно записать как
\begin{align*}
((a_0\ldots a_k)(a_{k+1}\ldots a_m))(a_{m+1}\ldots a_n)\\
= (a_0\ldots  a_m)(a_{m+1}\ldots a_n)
\end{align*}
Теперь мы можем вначале умножить группу чисел $a_0\ldots a_m$, затем группу чисел $a_{m+1}\ldots a_n$, причем обе группы могут умножаться в произвольном порядке, $m$ мы также выбрали произвольно. А это ровно то, что требовалось доказать.

Полностью аналогичное доказательство возможно и для любых прочих операций, обладающих свойством ассоциативности: логических операций, арифметического сложения, операций над множествами и т.д.

В действительности многие доказательства прошлых параграфов были весьма урезанны в силу того, что мы не доказывали индуктивный шаг, полагая это чем-то очевидным. Это, конечно, совершенно не так, и для строгости надо было доказывать это по индукции. Вот несколько примеров:

\begin{exercise}
Объясните чем было плохо доказательство
$$m^n = \underbrace{m\cdot m \cdot \ldots \cdot m}_n$$
и докажите это корректно.
\end{exercise}

\begin{exercise}
Объясните где была дырка в доказательстве основной теоремы арифметики, и заткните эту дырку.
\end{exercise}

\begin{exercise}
Докажите, что декартово произведение конечных множеств~--- конечно.
\end{exercise}

\begin{exercise}
Докажите по индукии равенство $ab=ba$ исходя из аксиом Пеано. Это сложное доказательство и вначале вам придется доказать кучу промежуточных фактов о сложении и умножение, и всё это тоже делается по индукции~--- я не рассчитываю, что вы самостоятельно доведёте работу до конца, но по крайней мере попытка будет не лишней.
\end{exercise}

\begin{exercise}
Последовательности часто задаются как выражение элемента $x_n$ через предыдущие элементы. Например, числа Фибоначчи определяются как
$$x_n = x_{n-1} + x_{n-2}$$
при $x_0 = 0$ и $x_1 = 1$. Подобные определения называются \term{рекурсивными}. Докажите, что такое определение действительно задаёт последовательность $x_n$ для любого $n\in\mathbb{N}$.
\end{exercise}

И так далее. Если доказывать основы арифметики строго и из аксиом Пеано, не опуская скучные детали, то там в каждой теореме индукция будет применяться по три-четыре раза минимум. Этим редко кто занимается, не стали этим заниматься и мы.

Всё сказанное до сих пор относилось больше к теории и по большому счёту мы доказывали то, что уже знали. Перейдём теперь к более приземленным примерам.

\begin{example}Обозначим
$$S(n) = 1 + 2 +3 \ldots + n$$
и
$$S'(n) = {n(n+1)\over 2}$$
Доказываемое утверждение $P$ будет состоять в том, что
$$S(n) = S'(n)$$
то есть что две приведенные две формулы эквивалентны.
\end{example}

Равенство $S(0) = S'(0)$ довольно очевидно, нам теперь надо доказать импликацию $P(n)\to P(n + 1)$, то есть доказать, что если эти формулы совпадают для произвольного $n$, то они будут совпадать и для $n + 1$. В соответствии с определением, для любого $n$
$$S(n + 1) = 1 + 2 + \ldots + n + (n + 1)$$
Из предположения о том, что утверждение верно для $n$, мы первые $n$ слагаемых можем переписать, используя формулу для $S'$:
\begin{align*}
S(n + 1) &= {n(n + 1)\over 2} + (n + 1)\\
	&= {n^2 + n\over 2} + {2n + 2\over 2} \\
	&= {n^ 2 + 3n + 2 \over 2} \\
	&= {(n + 1)(n + 2)\over 2} \\
	&= {S'(n + 1)}
\end{align*}

Всё! Мы доказали, что из равенства $S(n)=S'(n)$ следует также равенство $S(n+1)=S'(n+1)$, а это всё что требуется: теперь принцип индукции гарантирует нам, что действительно для любого $n$ эти формулы совпадают.

Принцип индукции чрезвычайно силён: подаляющее большинство теорем, в которых как-то фигурируют произвольные натуральные числа, могут быть доказаны таким образом, причем во многих случаях доказательство оказывается не сложным. С другой стороны этот метод обладает и явным недостатком: он позволяет нам доказать утверждение, которое нам уже известно, но не даёт никакого способа это утверждение вывести, не зная ответа. Например, если бы задача изначально состояла не в том, чтобы доказать равенство двух формул, а в том, чтобы получить краткое выражение для $S(n)$, метод индукции нам уже почти никак не помог бы.

Часто о решении можно догадаться. Например, если записать первые значения $S(n)$
$$0, 1, 3, 6, 10, 15, 21, 28, 36, \ldots$$
то некоторые смогут увидеть закономерность $S'$. По крайней мере так пишут в учебниках, что догадаться можно, а в жизни я людей, которые легко улавливают такие закономерности, видел очень мало. Но предположим, что догадаться можно.

Чаще формулу можно подобрать. Для $S(n)$ мы могли бы предположить, что формула имеет вид
$S(n) = {an^2 + bn + c\over d}$
где $a, b, c, d$~--- некоторые неизвестные нам числа. Затем, вычислив явно значения $S(0), S(1), S(2), S(3)$ можно убедиться в том, что единственными вариантом, который работает, является набор значений
$$a=1, b=3, c=2, d=2$$
Эти значения приводят к формуле $S'(n)$, но это пока не является доказательством, поскольку мы проверили эту формулу лишь на четырёх значениях $n$, Тем не менее здесь нам уже есть к чему приложить принцип индукции.

Этот подход может показаться также сложным и надуманным, но на самом деле он довольно прост. Если нарисовать значения $S(n)$ на графике, то их расположение будет очень сильно похоже на параболу, которую тут же распознает любой смышленный школьник, а формула с неизвестными, которую я привёл выше~--- это как раз формула параболы в общем виде. Мы пока не рассматривали графики функций, поэтому такое рассуждение может показаться сложным или необычным, но через какое-то время вы научитесь довольно быстро искать такие решения самостоятельно.

Тем не менее даже если мы формулу каким-то образом подобрали, а потом доказали её, нам хочется её еще и понять. Да, мы знаем, что $S(n)=S'(n)$, но с какой стати? Эти две формулы совершенно не похожи друг на друга и их взаимосвязь совершенно не ясна. Это главная проблема принципа индукции: он позволяет доказать очень многое, но он совершенно не проясняет ситуацию.

Попробуем вывести краткую формулу для $S(n)$, зайдя с другого угла. Для этого вначале расположим красные квадраты в ряды один под другим (рис.3.8). В первом ряду у нас будет один квадрат, во втором два, в третьем три и так далее. Наша задача~--- подсчитать сколько всего получается квадратов, если мы располагаем таким образом квадраты в $n$ строчках. Как видно из картинки, эти квадраты образуют некое подобие треугольника, поэтому такие числа называются \term{треугольными}.

\begin{figure}[h]
\centering
\begin{tikzpicture}
  \fill[red] (0, 0) -- (4.2cm, 0) -- (4.2cm, .6cm) -- (3.6cm, .6cm) -- (3.6cm, 1.2cm) -- (3cm, 1.2cm) -- (3cm, 1.8cm) -- (2.4cm, 1.8cm) -- (2.4cm, 2.4cm) -- (1.8cm, 2.4cm) -- (1.8cm, 3cm) -- (1.2cm, 3cm) -- (1.2cm, 3.6cm) -- (.6cm, 3.6cm) -- (.6cm, 4.2cm) -- (0, 4.2cm) -- cycle;
  \draw[step=.6cm,gray,very thin] (0, 0) grid (4.2cm,4.2cm);
\end{tikzpicture}
\caption{Треугольные числа}
\end{figure}

Помимо квадратов я изобразил сетку, одна ячейка которой по размеру равняется величине квадрата, а строк и колонок в ней $n$ штук. Всего клеток, таким образом, имеется $n^2$. Из всех этих клеток красные клетки~--- это те, что лежат на диагонали (их $n$ штук), а также половина от оставшихся клеток. Итого, мы получаем
$$S(n) = {n^2 - n\over 2} + n = {n^2 - n + 2n \over 2} = {n(n+1)\over 2}$$
Без каких-либо хитрых умоизысканий мы получили ту же самую формулу, причем теперь нам вполне понятно откуда она взялась и что означает.

В этом примере метод индукции оказался одновременно и сложнее и менее информативным~--- последний подход показал нам общую идею откуда такая формула берется, а не просто дал доказательство. Часто именно так и происходит, поэтому мы чаще всего будем избегать формальных выкладок по индукции и приводить по возможности методы, которые как-то раскрывают суть теоремы, а не просто доказывать что-то лишь бы доказать.

Тем не менее, последнее доказательство с клетками в строгом смысле доказательством не является. Как мы уже говорили в первой главе, строгое доказательство~--- это последовательность предложений, выводимых из аксиом с использованием четко определенного набора правил. Доказательство с клетками далеко от такого принципа: мы не определили никаких геометрических понятий, мы никак не определили по какому вообще критерию мы отождествляем клетки с числами, мы никак даже не определяем строго положение клеток в пространстве, хотя апеллируем к зрительным образам. Безусловно, доказательство выглядит убедительным (и, пролив много пота и крови, его можно сделать абсолютно строгим) и у большинства людей оно справедливо не будет вызывать сомнений. Однако формально этого не достаточно: многие учителя (не самые умные представители) в университетах такие доказательства не примут, также подобное доказательство не признают многие авторы учебников, эти люди всюду доказывают любую мелочь по индукции. Я считаю это неправильным (правда, я и образования не получил), поэтому мы будем по возможности избирать не всегда самый формально корректный, но по возможности самый содержательный метод доказательства.

\begin{exercise}
Докажите, что
$$1^2 + 2^2 +3^2 \ldots + n^2 = {n(n+1)(2n+1)\over 6}$$
\end{exercise}

\begin{exercise}
Прошлое упражнение задаёт формулу для так называемых \term{пирамидальных квадратичных} чисел. Их можно рассматривать как число блоков из которых состоит кирпичная пирамида с четырёхугольным основанием. Придумайте формулу для числа кирпичей в случае, если бы основанием пирамиды был треугольник. Такие числа называются \term{тэтраэдрическими пирамидальными}.
\end{exercise}

\begin{exercise}
Докажите, что
$$1^3 + 2^3 +3^3 \ldots + n^3 = \left({n(n+1)\over 2}\right)^2$$
\end{exercise}

Скорее всего в этих упражнениях вам придется воспользоваться методом индукции и это не даст вам никакой догадки о том, почему эти формулы работают. Далее мы немного углубимся в природу этих формул и выведем общую закономерность, хотя даже это не сильно поможет нам с интуитивным пониманием~--- хоть они и станут понятнее, красивой интерпретации как с треугольными числами мы получить всё равно не сможем (она в принципе есть в многомерных пространствах, но это не делает вещи проще). Иногда и такое бывает.

\begin{exercise}
В таблице 3.1 числа расположены в квадрате размером $5\times 5$ по порядку, начиная из центра (проследите за последовательностью чисел 1, 2, 3,..., 25 и вы поймёте принцип построения этой таблицы). Диагональные элементы выделены красным цветом. Если их сложить, то мы получим значение 101. Чему будет равно значение суммы диагональных элементов в таблице, построенной по тому же принципу, но произвольного размера $n\times n$ ($n$ нечётное).
\footnote{Это обобщение задачи 28 с сайта projecteuler.net}
\end{exercise}

\begin{table}[h]
\centering
\begin{tabular}{ccccc}
{\color{red} 21} &22& 23& 24& {\color{red} 25}\\
20 & {\color{red} 7} & 8 & {\color{red} 9}& 10\\
19 & 6&  {\color{red} 1}&  2& 11\\
18 & {\color{red} 5} & 4 & {\color{red} 3}& 12\\
{\color{red} 17}& 16& 15& 14& {\color{red} 13}
\end{tabular}
\caption{Расположение чисел по спирали}
\end{table}

\begin{exercise}
Пусть $n$ --- произвольное натуральное число, $p$ --- сумма его цифр, стоящих на нечётной позиции, а $k$ --- сумма его цифр, стоящих на четной позиции. Например, для $n=12345678$:
$$p=8+6+4+2 = 20$$
$$k=7+5+3+1 = 16$$
Докажите, что число $n$ делится на 11 тогда и только тогда, когда разность $p$ и $k$ делится на 11.
\end{exercise}

\begin{exercise}
Напомню, что цепью мы условились называть в \S~2.3 такие пути в графе, которые не проходят через повторяющиеся рёбра; циклом~--- цепь, у которой начальная и конечная вершины совпадают; а степенью вершины~--- количество рёбер, инцидентных ей. Цепь, проходящая через все вершины называется \term{Эйлеровой цепью}. Докажите, что граф имеет эйлерову цепь тогда и только тогда, когда он имеет не более двух вершин с нечётной степенью (Подсказка: вероятно, эту задачу будет проще решить, если вначале доказать, что граф имеет \term{Эйлеров цикл} тогда и только тогда, когда все его вершины имеют чётную степень; под эйлеровым циклом по аналогии понимается цикл, проходящий через все вершины).
\end{exercise}

\begin{figure}[H]
\centering
\includegraphics[width=5cm]{images/konigsberg_bridges.png}
\caption{Мосты Кёнигсберга}
\end{figure}

\begin{exercise}
Прошлая задача ведёт свои корни от старинной \term{Задачи о Кёнигсбергских мостах}, которая была решена Леонардом Эйлером в 1736 году. Схематически мосты Кёнигсберга изображены на рис.~3.10. Возможно ли обойти все эти мосты, пройдя по каждому мосту ровно один раз, и если возможно, то как?
\end{exercise}

\begin{example}
Числами Ферма называются числа
$$F_n = 2^{2^n} + 1$$
Если записать первые числа Ферма
$$3, 5, 17, 257, 65537, 4294967297$$
то можно увидеть закономерность
\begin{equation}\label{ni:3}
F_n = \prod_{i=0}^{n-1}F_i + 2
\end{equation}
\end{example}

Давайте докажем это по индукции. Как обычно предположим, что для формула верна для $F_n$ и докажем из этого её правильность для $F_{n+1}$, переписав на этот раз правда выражение несколько в другой форме:
\begin{align*}
\prod_{i=0}^n F_i &= (\prod_{i=0}^{n-1} F_i) F_n = (F_{n - 1} - 2) F_n \\
	&= (2^{2^n} - 1) (2^{2^n} + 1) = 2^{2\cdot 2^n} - 1 \\
	&= 2^{2^{n+1}} - 1 = F_{n+1} - 2
\end{align*}

Что и требовалось доказать.

Числа Ферма имеют интересную историю. Ферма их придумал в 1640-ом году и выдвинул гипотезу, что все такие числа простые. Действительно: они на вид похожи на простые. Однако в 1732 году Леонард Эйлер сумел разложить четвертое число Ферма на множители:
$$F_5 = 4294967297 = 641 \cdot 6700417$$
На это потребовалось почти сто лет! Интересно, что до сих пор не известно существует ли хоть одно простое число Ферма большее $F_4$~--- до сих пор все числа Ферма оказывались составными, однако исследована лишь малая доля таких чисел. Например, доказано, что число $F_{20}$ не является простым, хотя сами его множители не известны~--- это уже слишком большое число, чтобы для него можно было найти хотя бы один делитель.

Тем не менее числа Ферма с простыми числами явно связаны. Например, легко видеть из~\eqref{ni:3} и теоремы~3.14, что число $F_n$ не имеет общих множителей с числами $F_i$ при $i<n$. Другими словами это значит, что все числа Ферма взаимопросты. Отсюда в частности следует, что простых чисел бесконечно много: если бы их было конечное число, мы не могли бы иметь бесконечную последовательность взаимопростых чисел. Мы таким образом еще раз доказали теорему Евклида, но уже другим способом.

Напоследок рассмотрим два классических примера неправильного применения индукции.

\begin{example}
Докажем, что все лошади одного цвета. Пусть $A_n$~--- множество из $n$ лошадей, индукцию будем проводить по $n$ относительно уствержения <<в любом множестве $A_n$ все лошади имеют одинаковый цвет>>. Для $A_1$ утвержедение очевидно. Предположим, что утверждение верно для $A_m$ при $m<n$, и в этом предположении докажем верность утверждения для $A_n$. Разобьём множество $A_n$ на непустые непересекающиеся подмножества $B\cup C \cup\{x\}$, где $x$~--- одна отдельно взятая лошадь. По предположению индукции во множествах $B\cup\{x\}$ все лошади имеют один цвет, то же самое касается множества $C\cup\{x\}$. Поскольку в оба последних множества входит один и тот же $x$, то цвет лошадей в обоих множествах совпадает. Отсюда по индукции следует, что для любого $A_n$ цвет лошадей в этом множестве будет одинаковым. Что и требовалось доказать.
\end{example}

Рассуждение, очевидно, ошибочно, но обычно в подобных доказательствах не сразу понятно в чем ошибка. Проблема кроется в том, что когда мы предположили верность утверждения для $m<n$ мы забили о том, что $n$ в том числе может быть равно двойке. В то же время никакое множество $A_2$ невозможно разбить на непустые непересекающиеся $B\cup C\cup\{x\}$. Поэтому наше доказательство некорректно, хоть и поучительно: всегда надо держать в голове крайние случаи, так как они часто приводят к проблемам.

\begin{exercise}
Докажем, что мы можем поднять любую гору песка. Будем проводить индукцию по количеству песчинок. Очевидно, что мы можем поднять одну песчинку. Но если предположить, что мы можем поднять $n$ песчинок, то и $n+1$ песчинку тоже поднять сможем, потому что каждая отдельная песчинка ничего почти не весит. А отсюда по индукции мы можем поднять вообще любое количетсво песчинок. Как считаете, почему это доказательство некорректно?
\end{exercise}

%\section{Голубоглазые островитяне}

Последний пример применения индукции многими считается самым контринтуитивным чуть ли не во всей математике.

\begin{example}
   На острове живёт 1000 человек с идеальным логическим складом ума. Из них 100 имеет голубые глаза, и 900~--- карие. Религия запрещает им знать свой цвет глаз и рассказывать другим о цвете глаз. Никаких отражающих поверхностей на острове нет. Если кто-то вдруг узнает свой цвет глаз, то он обязан в ближайшую ночь устроить публичное ритуальное самоубийство.

    В какой-то момент на остров приезжает путешественник, который не знаком с местной религией, но тем не менее довольно успешно вливается в местный коллектив. И однажды он случайно на общем собрании в ходе своей речи невзначай упоминает:

    ---[...] и я был очень удивлён увидеть здесь, в столь отдалённом уголке, голубоглазых людей [...]

    Вопрос: сколько осталось жить голубоглазым и/или кареглазым островитянам?
\end{example}

Казалось бы, ничего произойти не должно. Островитяне не узнали ничего нового: путешественник им сказал лишь то, что на острове есть голубоглазые люди, но ведь все и так уже видели до этого голубоглазых. Однако, не всё так просто.



\section{Перестановки}

Мы часто переставляем физические предметы местами: тасуем карты, перекладываем деньги в кошельке, сортируем данные в компьютере, ставим книги на полке, меняем жизненные приоритеты. Эти действия отражены в математическом понятии перестановок, которым мы и займёмся. Чтобы не было путаницы, мы более не будем придерживаться аксиоматического задания натуральных чисел (если не оговорено обратное), и в этом и последующих параграфах будем использовать обозначение $[n]$ для множества чисел от 1 до $n$.

\begin{definition}
\term{$n$-элементной перестановкой} называется биекция $[n]\to[n]$. Множество всех $n$-элементных перестановок будем обозначать как $S_n$.
\end{definition}

Перестановку удобно записывать в виде строки. Например: $\pi = 35124 \in S_5$. Эта запись означает, что на первой позиции оказался элемент, который раньше стоял на третьей позиции, вслед за ним идёт элемент, которые ранее стоял пятым и т.д.

Как и для любых функций, для перестановок можно рассматривать их композицию (в случае перестановок её часто называют \term{умножением}). Пусть, например, перестановка $\pi$ задана как выше и дана перестановка $\rho = 45231$. В этом случае мы получаем $\pi \circ \rho = 24513$. Для понимания этого примера важно помнить, что $(\pi\circ\rho)(x) = \pi(\rho(x))$, то есть для того, чтобы получить позицию элемента, надо вначале применить перестановку $\rho$, а затем только $\pi$. Такой <<обратный порядок>> применения перестановок может показаться непривычным, но это стандартная форма записи функций, как мы узнали во второй главе. Вот так например можно получить позицию единицы в искомой перестановке:

$$(\pi\circ\rho) (1) = \pi(\rho(1)) = \pi(5) = 2$$

Хотя для композиции функций принято обозначение $\pi\circ\rho$, для умножения перестановок мы будем опускать символ $\circ$ и писать просто $\pi\rho$.

Из того, что перестановка~--- это функция, сразу следует ассоциативность перемножения перестановок (см.~\S2.4): для любых $\rho, \pi, \delta$ мы имеем тождество
$$(\rho\pi)\delta = \rho(\pi\delta) = \rho\pi\delta$$

\begin{exercise}
Покажите, что умножение перестановок не коммутативно, то есть что в общем случае $\rho\pi \not= \pi\rho$
\end{exercise}

\begin{exercise}
Придумайте сами каких-нибудь перестановок и поэкспериментируйте с ними.
\end{exercise}

\begin{thm}\label{thm:perminv}
$(\rho\pi)^{-1} = \pi^{-1}\rho^{-1}$
\end{thm}
\begin{proof}
$$(\rho\pi)(\pi^{-1}\rho^{-1}) = \rho(\pi\pi^{-1})\rho^{-1} = \rho\rho^{-1} = 1_{[n]}$$
\end{proof}

Если вам здесь что-то стало не очень понятно или сложно~--- перечитайте~\S2.4, все обозначения и термины являются непосредственным отражением того, что обсуждалось когда мы говорили об обыкновенных функциях.

Подсчитаем мощность множества $S_n$. На первую позицию мы можем поставить один из $n$ доступных нам элементов. На вторую позицию мы можем выбрать один из оставшихся $n-1$ элементов (итого способов расставить первые два элемента $n(n-1)$). Затем на третью позицию мы можем поставить один из $n-2$ элементов (итого имеем $n(n-1)(n-2)$ расстановок). Продолжая так до конца, на последнюю $n$-ю позицию мы ставим один оставшийся элемент. Для формулировки полученного результата полезно следующее обозначение:

\begin{definition}
Факториалом $n!$ называется величина $$n(n-1)(n-2)\ldots 2\cdot 1$$
Так же считаем, что $0! = 1$.
\end{definition}

Значение для $0!$ оправдано с той точки зрения, что на пустом множестве формально можно задать единственную функцию, которая так же будет биекцией (хоть она ничего и не отображает, формально она есть; см. аналогию с $n^0$ в~\S1).

Рассуждения, приведенные выше, теперь можно сформулировать таким образом:

\begin{thm}\label{thm:symgrpord}
$|S_n| = n!$
\end{thm}

\begin{exercise}
Докажите, что при разложении на простые множители значения $n!$, множитель $p$ встретится ровно
$$\sum_{i=1}^\infty \left\lfloor {n \over p^i} \right\rfloor$$
раз, где как обычно через $\lfloor {a\over b}\rfloor$ обозначается целая часть от деления $a$ на $b$.
\end{exercise}

Помимо представления перестановки в виде строки, их удобно рассматривать в виде \term{циклов}. Рассмотрим опять перестановку $\rho = 45231$. Мы видим, что при этой перестановке 1 переходит на позицию 5, 5 переходит на позицию 2, 2 на позицию 3, 3 на 4, которая переходит опять на позицию 1. Эти переходы записываются строкой чисел в круглых скобках: $\rho = (15234)$. Здесь за каждым числом $x$ следует число $\rho(x)$. Последнее значение переходит на позицию, обозначенную первым числом, что замыкает цикл.

Рассмотрим теперь перестановку $\pi = 35124$. Здесь $\pi(1) = 3$ и $\pi(3) = 1$, что даёт цикл $(13)$. Сюда, однако, вошли не все элементы множества. Рассмотрим оставшиеся элементы: $\pi(2) = 4$, $\pi(4) = 5$, $\pi(5) = 2$, что даёт цикл $(245)$. Итого перестановка представляется в виде произведения двух циклов: $\pi = (13)(245)$.

Количество элементов в цикле мы будем называть его длиной. Если какой-то элемент $x$ отображается сам в себя (то есть $\pi(x) = x$), то мы будем говорить, что перестановка имеет цикл длины 1 $(x)$.

\begin{exercise}
Пусть $\sigma$~--- некоторая перестановка и $(x y \ldots z)$~--- некоторый цикл. Докажите, что
$$\sigma(x y \ldots z)\sigma^{-1} = (\sigma x \:\: \sigma y \ldots \sigma z)$$
(Подсказка: рассмотрите как эта перестановка действует на элемент $\sigma x$).
\end{exercise}

\begin{definition}
\term{Типом перестановки} называется набор $(c_1, c_2, \ldots, c_n)$ , где $c_k$~--- количество циклов длины $k$ заданной перестановки.
\end{definition}

\begin{definition}
\term{Циклической перестановкой} $[n]$ называется перестановка типа $(0, 0, \ldots, 0, 1)$, то есть перестановка, состоящая из единственного цикла, сдвигающего все элементы.
\end{definition}

\begin{example}
Типом перестановки $\pi$, используемой выше, будет $(0, 1, 1, 0, 0)$, типом перестановки $\rho$~--- $(0, 0, 0, 0, 1)$. Тривиальная перестановка $1_{[n]}$, которая оставляет все элементы на месте, имеет тип $(n, 0, 0, 0, 0)$, то есть она состоит из $n$ циклов вида $(a)$.
\end{example}

\begin{exercise}
Покажите, что для любого типа $(c_1, c_2, \ldots, c_n)$ всегда выполнено соотношение
$$\sum_{i=1}^n ic_i = n$$
\end{exercise}

\begin{thm}
Существует
$${n!\over c_1!c_2!\ldots c_n!1^{c_1}2^{c_2}\ldots n^{c_n}}$$
перестановок типа $(c_1, c_2, \ldots, c_n)$
\end{thm}
\begin{proof}
В качестве рабочего примера будем считать, что мы ищем все перестановки типа $(1, 2, 1, 0, 0, 0, 0, 0)$ множества $[8]$.

Вначале выпишем произвольную строку чисел, являющуюся перестановкой $[n]$ (например, 18356724), способов сделать это $n!$. Расставим теперь в этой записи круглые скобки по очереди: вначале $c_1$ скобок с одним элементом внутри, затем $c_2$ скобок с двумя элементами внутри, потом $c_3$ с тремя элементами внутри и так далее. Для нашего примера получится запись
$$(1)(83)(56)(724)$$
Таким образом мы получаем одну из возможных перестановок требуемого типа. Однако, не только первоначально выписанная перестановка даёт нам такое циклическое представление. Во-первых, если мы поменяем местами любые два цикла длины $k$, то мы получим отличное представление той же самой перестановки. Например, мы можем поменять местами две перестановки длины 2:
$$(1)(56)(83)(724)$$
Это ровно та же самая перестановка, но получающаяся из другой строки. Таким образом, поскольку у нас имеется $c_k$ циклов длины $k$, нам необходимо поделить общее количество первоначальных перестановок на $c_k!$.

Во-вторых, каждая из перестановок может быть циклически сдвинута на k элементов. Например, мы так могли бы сдвинуть наш пример, получив тот же результат:
$$(1)(56)(38)(472)$$
Опять же это та же самая перестановка. Для каждого цикла длины $k$ мы имеем $k$ способов произвести такой сдвиг. По этой причине имеющееся у нас количество искомых перестановок надо поделить на $k^{c_k}$ для каждой длины цикла $k$.
\end{proof}
\begin{corollary}
Существует $(n-1)!$ циклических перестановок множества $[n]$.
\end{corollary}
\begin{proof}
В качестве упражнения.
\end{proof}

\begin{exercise}
\term{Порядком перестановки} $|\pi|$ называется такое число $n$, что $\pi^n = 1_{[n]}$ (то есть под $\pi^n(x) = x$ подразумевается $n$-кратное применение перестановки $\pi$). Докажите, что для перестановки $\pi$ типа $(c_1, \ldots, c_n)$  выполняется соотношение
$$|\pi| = \lcm\{c_i\}$$
\end{exercise}

\begin{definition}
\term{Транспозицией} называется перестановка типа $(n-2, 1, 0, \ldots, 0)$, то есть это перестановка, меняющая местами лишь два элемента.
\end{definition}

\begin{exercise}
Докажите, что любая перестановка может быть представлена композицией транспозиций.
\end{exercise}

Ясно, что представление в виде транспозиций неоднозначно, причём может отличаться даже количество транспозиций, используемых для представления перестановки, например:

$$(13)(34) = (23)(14)(24)(23)$$

Интересно, однако, что количество транспозиций для любой заданной перестановки всегда либо чётное, либо нечётное. Это не совсем очевидно и требует доказательства, которое предсталено следующим определением с прилагающимися упражнениями.

\begin{definition}
\term{Инверсией} перестановки называется такая пара $i<j$, что $\pi(i) > \pi(j)$.
\end{definition}

Инверсии~--- это такие пары элементов, которые при применении перестановки меняют свой относительный порядок. Например, для перестановки 35124 имеется 5 инверсий: 3 и 1, 3 и 2, 5 и 1, 5 и 2, 5 и 4. Инверсии имеют важное значение при анализе быстродействия компьютерных алгоритмов, однако нас само количество инверсий интересовать не будет, нас будет интересовать лишь то, чётное это количество или нет.

\begin{definition}
Перестановка называется \term{чётной} (\term{нечётной}), если она имеет \term{чётное} (соответственно \term{нечётное}) число инверсий.
\end{definition}

\begin{exercise}
Докажите, что любая транспозиция~--- это нечётная перестановка.
\end{exercise}

\begin{exercise}
Докажите, что произведение чётных перестановок даёт чётную перестановку, произведение нечётных~--- тоже чётную, а произведениче чётной и нечётной перестановки~--- нечётную перестановку.
\end{exercise}

Тривиальная перестановка, которая оставляет все элементы на месте, чётная. Если умножить её на транспозицию~--- получаем нечётную перестановку. Умножная её на транспозицию ещё раз~--- опять получаем чётную перестановку. Поскольку любая перестановка однозначно либо чётная либо нечётная, то, рассуждая по индукции, приходим к ожидаемому: любая перестановка всегда представлена в виде транспозиций либо чётным их числом, либо нечётным.

\begin{exercise}\label{ex:sgnpermgrp}
Докажите, что для любого $n>1$ количество чётных и нечётных перестановок на $[n]$ одинаково.
\end{exercise}

Пока мы рассмотрели понятие чётности перестановок лишь в качестве упражнения, но довольно скоро вы увидите, что оно играет весьма важную теоретическую роль.

\begin{exercise}
Докажите, что любая чётная перестановка может быть представлена в виде произведения циклов длины 3.
\end{exercise}

\begin{exercise}
Докажите, что с помощью перестановок $(12)$ и $(12\ldots n)$ возможно представить любую перестановку множества $[n]$.
\end{exercise}

\begin{definition}
\term{Беззнаковым числом Стирлинга первого рода} $\fstirling{n}{k}$ называется количество перестановок множества $[n]$, содержащих ровно $k$ циклов.
\end{definition}

Слово <<беззнаковый>> мы будем опускать, поскольку пока оно не имеет для нас никакого смысла (он проявится позже). Прямо из определения следует такое простое соотношение:

\begin{thm}
$$\sum_{k=1}^n \fstirling{n}{k} = n!$$
\end{thm}
\begin{proof}
В качестве упражнения.
\end{proof}

Выше мы уже фактически доказали, что $\fstirling{n}{1} = (n-1)!$. Так же очевидно, что $\fstirling{n}{n} = 1$. Для вычисления остальных значений поможет следующая рекурсивная формула.

\begin{thm}
$$\fstirling{n}{k} = \fstirling{n-1}{k-1} + (n-1)\fstirling{n-1}{k}$$
\end{thm}
\begin{proof}
Рассмотрим какой-нибудь один отдельно взятый элемент $x$ множества $[n]$. Под действием перестановки с $k$ циклами $x$ может либо оставаться на месте, либо куда-то передвигаться. Перестановок, когда $x$ остаётся на месте, ровно $\fstirling{n-1}{k-1}$ штук. Если же $x$ куда-то передвигается, то мы можем рассмотреть перестановку множества $[n-1]$ с $k-1$ циклами: в эту перестановку, записанную в виде циклов, достаточно лишь вставить элемент $x$ после какого-нибудь другого элемента. Всего нам доступно $n-1$ позиций для вставки.
\end{proof}

По этой формуле с помощью компьютера можно сравнительно легко находить числа Стирлинга. Конечно, сами по себе они мало интересны и вряд ли вам придётся когда-либо их вычислять. Интреснее то, как они сочетаются с другими объектами математики, к чему мы вернёмся в последующих главах нашей книги.

\begin{exercise}
Посадил злой вертухай сотню гномиков в тюрьму, каждого пронумеровал и поставил условия:\\
— Завтра каждый из вас по одному будет заходить в специальную камеру, где будет лежать сотня пронумерованных конвертов. В каждом конверте — какой-то номер (просто все числа от одного до ста, разложенные по конвертам случайным образом). Каждый из вас будет иметь право открыть ровно 50 конвертов, чтобы найти свой номер. Если все вы найдете свои номера — освобожу. Если хотя бы один ошибется — казню всех.\\
Гномики после обсуждения общей стратегии больше никак общаться не смогут и вообще друг друга не увидят. Конверты после каждого гномика закрываются обратно, так что входящий в камеру гномик не имеет никакой информации о том, что нашли или не нашли другие гномики и какие конверты они открывали.

Гномикам надо придумать стратегию как искать свой номер, чтобы найти. Если открыть просто наугад 50 из 100 конвертов, то вероятность найти свой номер весьма мала (пополам на пополам). Учитывая, что гномиков сто штук, то их общие шансы при случайной стратегии оказываются вообще ничтожны. Надо поэтому придумать какую-то альтернативную стратегию. (По хорошему в этой задаче еще надо проанализировать в конце полученное решение, но этого пока я от читателя не требую, поскольку я не рассказывал методов, которые позволяют это делать; впрочем, вы можете попытаться дать грубые оценки).
\end{exercise}

\section{Сочетания}

<<$k$-сочетание>>~--- это просто другое название для термина <<$k$-элементное подмножество>>, которое по историческим причинам (хотя многие считают, что так удобнее) принято в комбинаторике. Нас будет интересовать количество $k$-сочетаний взятых из множества $[n]$.

\begin{definition}
\term{Биномиальным коэффициентом} $n \choose k$ называется число сочетаний из $[n]$ по $k$.
\end{definition}

Вместо обозначения ${n \choose k}$ в российской и французской литературе исторически чаще используется обозначение $C^k_n$, однако оно кажется мне менее удобным в случаях если вместо величин $n$ и $k$ используются какие-то длинные выражения. По этой причине мы будем придерживаться общемирового обозначения.

Значения ${n \choose 1} = n$ и ${n \choose n} = 1$ очевидны. Так же удобно полагать, что ${n \choose 0} =1$ (пустое множество всего одно, соответственно есть лишь один способ его выбрать). 

\begin{exercise}
На листочке в клетку начерчен прямоугольник со сторонами $m$ и $n$. Мы двигаемся из левого нижнего угла прямоугольника в правый верхний, сдвигаясь за шаг либо на одну клетку вверх, либо на одну клетку вправо. Сколько всего существует таких путей? Сведите задачу к вычислению количества сочетаний, как их считать я расскажу ниже.
\end{exercise}

\begin{exercise}
Будем считать, что в условиях прошлой задачи $m=n$. Сколько существует путей из левого нижнего угла в правый верхний таких, что они не пересекают диагонали квадрата? (Подсказка: возьмите путь, пересекающий диагональ, и отразите его начало до пересечения относительно этой диагонали).
\end{exercise}

\begin{thm}
$$\sum_{k=0}^n{n \choose k} = 2^n$$
\end{thm}
\begin{proof}
В левой части этого выражения строит общее количество всех подмножеств множества $[n]$. В правой части на самом деле написано то же самое: $2^n$ есть мощность булеана, как вы видели в~\S~3.1.
\end{proof}

\begin{thm}
$${n \choose k} = {n \choose n-k}$$
\end{thm}
\begin{proof}
Выбрать $k$ элементов из множества $[n]$ это всё равно что выбрать $n-k$ элементов, которые мы оставим в множестве.
\end{proof}

\begin{thm}
$${n \choose k} = {n-1 \choose k-1} + {n - 1 \choose k}$$
\end{thm}

\begin{proof}
Рассмотрим все $k$-сочетания из множества $[n]$. Сам элемент $n$ может либо принадлежать выбранному подмножеству, либо не принадлежать. В первом случае количество сочетаний будет равно $n-1\choose k$, во втором случае, поскольку один элемент сочетания уже фиксирован, это будет величина $n-1\choose k - 1$.
\end{proof}

Последняя теорема, как и в случае чисел Стирлинга, позволяет вычислять биномиальные коэффициенты. Однако, для сочетаний мы можем записать и явную формулу.

\begin{definition}
\term{$k$-размещением} мы назовём некоторый упорядоченный набор, состоящий из некоторых $k$ элементов множества $[n]$. Количество размещений из $n$ по $k$ будем обозначать как $n^{\lfloor k\rfloor}$.
\end{definition}

\begin{thm}
$$n^{\lfloor k \rfloor} = \frac{n!}{(n-k)!}$$
\end{thm}
\begin{proof}
Доказательство практически дублирует доказательство для количества перестановок. На первую позицию мы можем поставить один из $n$ элементов. На вторую позицию один из оставшихся $n-1$ элементов. На третью~---один из $n-2$ элементов. Однако в отличии от перестановок, теперь нам надо разместить лишь $k$ элементов, а не все $n$, поэтому мы этот процесс оборвём на $k$-том шаге. В итоге получаем выражение
$$n^{\lfloor k \rfloor} = n (n-1)  (n-2) \ldots (n-k+1)$$
Если это выражение умножить и разделить на $(n-k)!$, получим утверждение теоремы.
\end{proof}

Само обозначение $n^{\lfloor k \rfloor}$ на самом деле почти всегда используется просто для обозначения произведения $k$ подряд убывающих чисел. Эту величину часто называют \term{убывающим факториалом}. В полной аналогии вводится и \term{возрастающий факториал}:
$$n^{\lceil k \rceil} = n(n+1)(n+2)\ldots(n+k-1)$$

\begin{exercise}
Покажите, что $n^{\lfloor k \rfloor} = (n-k+1)^{\lceil k \rceil}$
\end{exercise}

\begin{thm}
$${n \choose k} = \frac{n!}{k!(n-k)!}$$
\end{thm}
\begin{proof}
$k$-расстановку мы можем получить, вначале выбрав $k$-элементное подмножество $[n]$, а затем упорядочив его. Всего существует $n\choose k$ способов выбрать такое подмножество. Способов упорядочить выбранное~--- $k!$. Таким образом получаем соотношение
$$n^{\lfloor k \rfloor} = k!{n\choose k}$$
Поделив обе части на $k!$ и подставив выражение для $n^{\lfloor k \rfloor}$, получаем утверждение теоремы.
\end{proof}

\begin{exercise}
Теоремы 3.22 и 3.23 можно доказать, пользуясь явным представлением биномиального коэффициента, полученным в 3.25. Сделайте это.
\end{exercise}

\begin{exercise}
Чему равно $$\sum_{k=m}^n k^{\lfloor m\rfloor} {n\choose k}$$
\end{exercise}

\begin{exercise}
Докажите, что $${n\choose m}{n-m\choose k} = {n\choose k}{n-k\choose m}$$
\end{exercise}

\begin{exercise}
Докажите, что $${n\choose k-1}{n\choose k+1} \le {n\choose k}{n\choose k}$$
\end{exercise}

\begin{thm}
$$(x+y)^n = \sum_{k=0}^n {n \choose k} x^{n-k} y^k$$
\end{thm}
\begin{proof}
Давайте вначале для наглядности распишем степень подробно:
$$(x+y)(x+y)\ldots(x+y)$$
Раскроем все скобки (если пока не понятно как, то потренируйтесь на каких-то частных случаях типа $n=3$). Раскрытие скобок можно интерпретировать так, что из каждой скобки мы выбираем либо $x$, либо $y$. Все слагаемые в полученной сумме будут иметь вид $x^iy^j$, $i+j=n$ с каким-то коэффициентом, появляющимся за счёт того, что некоторые слагаемые вида $x^iy^j$ появляются несколько раз.

Слагаемое $x^n$ появится лишь один раз в случае, если из всех скобок мы выберем $x$. Слагаемое $x^{n-1}y$ появляется, если мы выбираем $x$ из всех скобок, кроме одной. Эту одну скобку, из которой мы выбираем $y$, мы можем выбрать одним из $n$ способов. По аналогии $x^{n-2}y^2$ появится $n\choose 2$ раз, поскольку мы выбираем уже две скобки, из которых мы возьмём $y$. Продолжая по аналогии приходим к утверждению теоремы.
\end{proof}

\begin{exercise}
Приведённую теорему можно доказать по индукции. Будет полезным проделать это самостоятельно.
\end{exercise}

Приведённая теорема даёт нам ещё один способ подсчитать количество подмножеств множества $[n]$:

$$\sum_{k=0}^n {n \choose k} = \sum_{k=0}^n {n \choose k}1^{n-k}1^k = (1+1)^n = 2^n$$

\begin{exercise}
Докажите равенство
$$3^n = \sum_{k=0}^{n} 2^k {n \choose k}$$
\end{exercise}

Последняя задача в свете теоремы 3.26 решается элементарно, однако я замечу, что такие задачи часто даются на олимпиадах для тех школьников, которые ещё по возрасту не могут знать таких теорем. Предполагается, что школьники могут доказать утверждение по индукции (и некоторые действительно могут). Это одна из причин по которой автор ненавидит олимпиадные задачки и <<задачки на логику>>, которые часто дают на собеседованиях~--- большинство таких задач изначально появляются как тривиальные следствия каких-то более общих результатов, а потом уже задним числом оказывается, что задачу в общем-то можно было бы решить и пользуясь лишь школьной математикой. Однако если сравнивать пользу от решения такой задачи по индукции с пользой от изучения общей теоремы, то последнее явно выигрывает, хотя и не помогает на собеседованиях и олимпиадах.

Сочетания часто используются так же вот в каком ключе. Предположим, мама нам сказала: <<Пойди в магазин и купи $n$ каких-нибудь пирожков>>. Мы приходим в магазин, а там продаётся $k$ наименований пирожков. Сколько всего способов у нас есть удовлетворить мамин запрос? Задача в такой постановке приводит нас к понятию \term{сочетаний с повторениями}, поскольку искомые наборы могут содержать повторяющиеся элементы.

Для решения задачи предположим, что пирожки разного вида мы разложили по разным пакетам (я видел, что в ларьках у метро именно так часто и делают). Схематически мы будем разделять пакеты вертикальной чертой $|$, а пирожки (или, более общо, элементы множества), кружочками $\circ$. Для примера давайте считать, что у нас всего имеется $k=5$ видов пирожков, а купили мы $n=6$ пирожков, причём из них было два пирожка первого вида, три третьего и один четвёртого. Остальных пирожков мы не покупали. На схеме это будет выглядеть как
$$|\circ\circ||\circ\circ\circ|\circ||$$
Теперь мы можем догадаться, как подсчитать общее количество различных сочетаний с повторениями: достаточно просто подсчитать общее количество возможных схем такого вида. В схеме, приведённой выше, если отбросить крайние чёрточки $|$, получится $n+k-1$ различных позиций, на которых могут стоять чёрточки либо кружки. Причём мы точно знаем, что кружков всего будет $n$ штук, а чёрточек $k-1$. Чтобы получить какую-то конкретную схему, нам надо выбрать $n$ позиций под кружки, а остальные позиции мы занимаем чёрточками. Итого для количества сочетаний с повторениями мы имеем выражение
$${n+k-1 \choose n} = {n+k-1\choose k - 1}$$

\begin{exercise}
Докажите, что существует $2^{n-1}$ способов представить число $n$ в виде суммы ненулевых слагаемых. Задача довольно сложная, поэтому дам некоторые наводки. Вначале следует разобрать случай, когда имеется ровно $k$ слагаемых. Подход здесь может быть аналогичным задаче с булочками выше, но надо учитывать то, что чтобы слагаемые были ненулевыми, мы не можем поставить подряд две черты, не поставив между ними кружок. Способов сделать это $n-1\choose k -1$ (докажите!). Отсюда уже довольно легко выводится и результат первоначальной задачи.
\end{exercise}

Предположи

\begin{definition}
Число разбиений множества $[n]$ на множества мощностей $k_1, k_2, \ldots k_m$ называется \term{мультиномиальным коэффициентом} и обозначается как
$$n \choose k_1; k_2;\ldots; k_m$$
\end{definition}

В этом определении, очевидно, $k_1\ldots+ k_m = n$.

\begin{exercise}
Сколько можно получить различных слов путём перестановки букв в слове <<математика>>? Если теперь взять произвольно слово, то сколько слов можно получить различными перестановками?
\end{exercise}

\begin{thm}
$${n \choose k_1; k_2;\ldots; k_m} = \frac{n!}{k_1!k_2!\ldots k_m!}$$
\end{thm}
\begin{proof}
Выберем вначале $k_1$ элемент, способов сделать это $n\choose k_1$. Из оставшихся элементов теперь выберем во второе множество $k_2$ элементов, способов сделать это $n-k_1\choose k_2$. Продолжая рассуждать таким же образом, получаем
\begin{align*}
{n \choose k_1; k_2;\ldots; k_m} & = {n\choose k_1}{n-k_1\choose k_2}{n-k_1-k_2\choose k_3}\ldots{n-k_1-\ldots k_{m-1}\choose k_m} \\
&= \frac{n!}{k_1!(n-k_1)!}\cdot\frac{(n-k_1)!}{k_2!(n-k_1-k_2)!}\cdot\frac{(n-k_1-k_2)!}{k_3!(n-k_1-k_2-k_3)!}\cdot\ldots\\
&=\frac{n!}{k_1!k_2!\ldots k_m!}
\end{align*}
\end{proof}

\begin{thm}
$$(x_1+x_2\ldots x_m)^n = \sum_{k_1+\ldots + k_m = n}{n\choose k_1;k_2;\ldots;k_m}x_1^{k_1}x_2^{k_2}\ldots x_m^{k_m}$$
Здесь суммирование ведётся по всем возможным наборам чисел $\{k_i\}$, дающим в сумме $n$.
\end{thm}
\begin{proof}
Аналогично доказательству теоремы 3.26. Проведите его самостоятельно.
\end{proof}

\begin{exercise}
Покажите, что теорема 3.26 является частным случаем для теоремы 3.29.
\end{exercise}

\begin{exercise}
Покажите, что в теореме 3.29 будет ровно $n + k - 1 \choose n$ слагаемых.
\end{exercise}
\section{Разбиения множеств}

Когда мы выбираем из $n$ элементов $k$ элементов, мы на самом деле разбиваем множество на 2 части: ту, которую мы выбрали (она содержит $k$ элементов), и ту, которую оставили ($n-k$ элементов). Эту идею можно обобщить: можно выбирать элементы множества не в два подмножества, а в произвольное число $m$ подмножеств, где в $i$-е подмножество попадает $k_i$ элементов. При этом должно выполняться соотношение $k_1+k_2+\ldots+k_m = n$. Эта идея чуть более формально отражена в следующих определениях.

\begin{definition}
\term{Упорядоченным разбиением} множества $[n]$ на $m$ подмножеств называется отображение $f:[n]\to[m]$. Мы говорим, что упорядоченное разбиение имеет \term{тип} $(k_1, k_2, \ldots, k_m)$, если $|f^{-1}(i)| = k_i$.
\end{definition}

\begin{definition}
Число упорядоченных разбиений множества $[n]$ типа $(k_1, k_2, \ldots k_m)$ называется \term{мультиномиальным коэффициентом} и обозначается как
$$n \choose k_1; k_2;\ldots; k_m$$
\end{definition}

\begin{exercise}
Сколько можно получить различных слов путём перестановки букв в слове <<математика>>? Если теперь взять произвольно слово, то сколько слов можно получить различными перестановками?
\end{exercise}

\begin{thm}
$${n \choose k_1; k_2;\ldots; k_m} = \frac{n!}{k_1!k_2!\ldots k_m!}$$
\end{thm}
\begin{proof}
Выберем вначале $k_1$ элемент, способов сделать это $n\choose k_1$. Из оставшихся элементов теперь выберем во второе множество $k_2$ элементов, способов сделать это $n-k_1\choose k_2$. Продолжая рассуждать таким же образом, получаем
\begin{align*}
{n \choose k_1; k_2;\ldots; k_m} & = {n\choose k_1}{n-k_1\choose k_2}{n-k_1-k_2\choose k_3}\ldots{n-k_1-\ldots k_{m-1}\choose k_m} \\
&= \frac{n!}{k_1!(n-k_1)!}\cdot\frac{(n-k_1)!}{k_2!(n-k_1-k_2)!}\cdot\frac{(n-k_1-k_2)!}{k_3!(n-k_1-k_2-k_3)!}\cdot\ldots\\
&=\frac{n!}{k_1!k_2!\ldots k_m!}
\end{align*}
\end{proof}

\begin{thm}
$$(x_1+x_2\ldots x_m)^n = \sum_{k_1+\ldots + k_m = n}{n\choose k_1;k_2;\ldots;k_m}x_1^{k_1}x_2^{k_2}\ldots x_m^{k_m}$$
Здесь суммирование ведётся по всем возможным наборам чисел $\{k_i\}$, дающим в сумме $n$.
\end{thm}
\begin{proof}
Аналогично доказательству теоремы 3.26. Проведите его самостоятельно.
\end{proof}

\begin{exercise}
Покажите, что теорема 3.26 является частным случаем для теоремы 3.29.
\end{exercise}

\begin{exercise}
Покажите, что в теореме 3.29 будет ровно $n + k - 1 \choose n$ слагаемых.
\end{exercise}

\begin{definition}
\term{Числами Стирлинга второго рода} $\sstirling{n}{k}$ называется количество способов разбить множество $[n]$ на $k$ подмножеств.
\end{definition}

Я обращу внимание на то, что в определении чисел Стирлинга второго рода речь идёт уже о неупорядоченных разбиениях. Например, если рассматривать множество $\{a, b, c, d\}$, то с точки зрения чисел Стирлинга разбиения $(\{a, b\}, \{c, d\})$ и $(\{c, d\}, \{a, b\})$ будут эквивалентны, а с точки зрения мультиномиального коэффициента~--- нет.

\begin{exercise}
Докажите, что $\sstirling{4}{2} = 8$, но в то же время
$$\sum_{i=0}^4 {4\choose i;(4-i)} = 16$$
\end{exercise}

\begin{exercise}
Выразите $\sstirling{n}{k}$ через мультиномиальные коэффициенты.
\end{exercise}

\begin{definition}
\term{Числами Белла} $B(n)$ называется количество способов разбить множество $[n]$ на подмножества.
\end{definition}

Речь опять же идёт о неупорядочнных разбиениях.

\begin{thm}
$$B(n) = \sum_{k=1}^n\sstirling{n}{k}$$
\end{thm}
\begin{proof}
Очевидно.
\end{proof}

\begin{exercise}
Докажите, что $\sstirling{n}{2} = 2^{n-1} - 1$.
\end{exercise}

Значения $\sstirling{n}{1} = \sstirling{n}{n} = 1$ настолько очевидны, что даже не заслуживают отдельного упражнения. Остальные значения, как и в случае чисел Стирлинга первого рода и количества сочетаний, могут быть вычислены рекурсивно.

\begin{thm}
$$\sstirling{n}{k} = \sstirling{n-1}{k-1} + k\sstirling{n-1}{k}$$
\end{thm}
\begin{proof}
Рассмотрим элемент $n$ множества $[n]$. При разбиении $[n]$ на подмножества $n$ может войти в отдельное подмножество мощности 1, либо же войти в состав более крупного подмножества. В первом случае количество таких разбиений будет $\sstirling{n-1}{k-1}$~--- количество разбиений без учёта элемента $n$. Во втором случае мы опять же строим разбиение множества $[n-1]$, но уже на $k$ подмножеств. Нам остаётся лишь выбрать в какое из этих $k$ подмножеств добавить элемент $n$. Получаем $k\sstirling{n-1}{k}$.
\end{proof}

Числа Белла так же допускают рекурсивное определение, хотя и не особо удобное.

\begin{thm}
$$B(n+1) = \sum_{k=0}^n{n \choose k} B(k)$$
\end{thm}
\begin{proof}
Пусть элемент $n+1$ при некотором разбиении множества $[n+1]$ попал в множество размера $k$. Есть $n\choose k - 1$ способов выбрать это множество, оставшиеся элементы можно разбить на подмножества $B(n+1-k)$ способами. Поскольку $k$ может быть произвольным от 1 до $n+1$, получаем
\begin{align*}
B(n+1) & = \sum_{k=1}^{n+1}{n\choose k-1}B(n+1-k)\\
& = \sum_{k=1}^{n+1}{n\choose n-k+1}B(n-k+1) \\
& = \sum_{k=0}^{n}{n\choose k}B(k)
\end{align*}
\end{proof}

Чтобы эта формула работала, нужно положить $B(0) = 1$ (чтобы увидеть это, примените 3.32 к значению $B(1)$), что соответствует интуиции: существует лишь одно разбиение пустого множества на подмножества, которое само является пустым множеством.

\begin{exercise}
Докажите, что для $n>2$ выполняется оценка
$$n! < \sstirling{2n}{n} < (2n)!$$
Вероятно, вначале будет целесообразно, пользуясь 3.31, доказать, что $B(n) < n!$ при $n>3$.
\end{exercise}

\section{Включения-исключения}

Вот задачка, которая как-то попалась мне в младшем возрасте на олимпиаде по математике, а затем многократно попадалась в разных книжках с занимательными примерами по математике.

Есть класс, в котором известно, что все ученики ходят в какие-то секции. Известно, что 12 учеников ходят на плавание (и, возможно на что-то ещё), 13 на карате, 12 на шахматы, 6 одновременно на плавание и на карате, 6 на карате и шахматы, 2 на шахматы и плавание и 2 ходят во все три секции. Сколько всего учится детей в классе?

Мы не будем решать эту задачу, а сразу перейдём к общему случаю. Пусть у нас есть семейство множеств $\{A_i\}$, нам известны мощности каждого из них, а так же мощности любых их пересечений. Необходимо найти мощность объединения этих множеств.

Напомню, что в~\S~3.1 мы ввели понятие характеристической функции $\chi_A(x)$. Эта функция принимает значение 0, если $x\not\in A$ и 1, если $x\in A$. Нам потребуются дополнительные факты о ней:

\begin{thm}
\begin{enumerate}
\item Пусть $A\subset U$, тогда $|A| = \sum_{x\in U} \chi_A(x)$
\item $\chi_{A^C}(x) = 1 - \chi_A(x)$
\item $\chi_{A\cap B}(x) = \chi_A(x)\chi_B(x)$ 
\end{enumerate}
\end{thm}
\begin{proof}
В качестве упражнения.
\end{proof}

\begin{thm}
$$\left|\bigcup_{i=1}^n A_i \right|= \sum_{i=1}^n|A_i| - \sum_{i<j}|A_i\cap A_j| + \sum_{i<j<k}|A_i\cap A_j\cap A_k| -\ldots$$
Здесь суммирование справа ведётся по всем таким наборам $(i, j)$ таким, что $i<j$, затем по наборам $(i,j,k)$, таким что $i<j<k$ и так далее.
\end{thm}

\begin{proof}
Утверждение на самом деле следует непосредственно из теоремы~3.32. В выкладках ниже я не указываю пределы суммирования, т.к. они и так очевидны:
\begin{align*}
\chi_{\cup A_i}(x) &= \chi_{(\cup A_i)^{CC}}(x) \\
&= 1 -  \chi_{(\cup A_i)^C}(x) \\
&= 1 - \chi_{\cap A_i^C}(x) \\
&= 1 - \prod \chi_{\cap A_i^C}(x) \\
&= 1 - \prod (1 - \chi_{A_i}(x)) \\
&= \sum_{i=1}^n\chi_{A_i}(x) - \sum_{i<j}\chi_{A_i}(x)\chi_{A_j}(x) +\ldots\\
&= \sum_{i=1}^n\chi_{A_i}(x) - \sum_{i<j}\chi_{A_i\cap A_j}(x) +\ldots
\end{align*}
В предпоследней строке мы просто раскрыли скобки. Применяя к полученному пункт 1 теоремы 3.32, получаем утверждение теоремы.
\end{proof}

Теорема легко выводится, но пока, вероятно, не очень понятна по смыслу. Давайте посмотрим на случаи нескольких множеств. Пусть нам надо подсчитать $|A\cup B|$. Мы могли бы просто сложить их мощности $|A| + |B|$, однако в этом случае получится, что часть $A\cap B$ будет учтена в этом выражении дважды~--- один раз от $|A|$ и  один раз от $|B|$, значит из полученного нам надо вычесть $|A\cap B|$, в результате чего получаем

\begin{equation}\label{nii:1}
|A\cup B| = |A| + |B| - |A\cap B|
\end{equation}

Можно было бы увидеть это и по-другому. Множества $X = A\backslash B$, $Y = A\cap B$ и $Z = B\backslash A$ не пересекаются, причём $A = X\cup Y$, $B = Y\cup Z$, $A\cup B = X\cup Y \cup Z$. Т.к. $X$, $Y$ и $Z$ не пересекаются, мы можем записать
$$|A\cup B| = |X\cup Y\cup Z| = |X| + |Y| + |Z|$$
Но поскольку $|A| = |X| + |Y|$, $|B| = |Y| + |Z|$ и $|A\cap B| = |Y|$, мы опять получаем~\eqref{nii:1}.

Перейдём к случаю трёх множеств. Опять же простая сумма $|A|+|B|+|C|$ будет дважды учитывать попарные пересечения множеств и трижды~--- тройное пересечение $A\cap B\cap C$. Чтобы исправить ситуацию, нам надо вычесть все попарные пересечения, однако этого тоже недостаточно~--- вычтя три попарных пересечения мы в том числе три раза вычтем тройное пересечение, а это значит, что оно теперь остаётся вообще не учтено. Его надо прибавить. Получаем

$$|A\cup B \cup C| = |A| + |B| + |C| - |A\cap B| - |A\cap C| - |B\cap C| + |A\cap B \cap C|$$

Собственно утверждение теоремы 3.33 можно было бы получить и по индукции, рассуждая таким образом, однако такой путь явно сложнее.

Для пересечения множеств можно получить результат, аналогичный 3.33.

\begin{thm}
Пусть $A_i \subset U$, тогда
$$\left|\bigcap_{i=1}^nA_i\right| = |U| - \sum_{i=1}^n|A_i^C| + \sum_{i<j}|A_i^C\cap A_j^C| - \ldots$$
\end{thm}
\begin{proof}
Утверждение следует непосредственно из тождества
$$|\cap A_i| = |(\cap A_i)^{CC}| = |U| - |\cup A_i^C|$$
Применяя теперь теорему 3.33 получаем желаемый результат.
\end{proof}

Помимо простых задачек для школьников, конечно же, это всё имеет применения и к чистой математике. Давайте приведём примеры.

\begin{definition}
\term{Функцией Эйлера} $\phi(n)$ называется количество чисел, меньших $n$ и взаимопростых с ним (исключая ноль, включая единицу).
\end{definition}

\begin{thm}
Пусть $n=\prod_{i=1}^m p_i^{\alpha_i}$~--- разложение $n$ на простые множители. Тогда
$$\phi(n) = n - \sum_{i=1}^m\frac{n}{p_i} + \sum_{i<j}\frac{n}{p_ip_j} - \sum_{i<j<k}\frac{n}{p_ip_jp_k} + \ldots$$
\end{thm}
\begin{proof}
Будем рассматривать в качестве универсума (то есть множества, относительно которого мы берём дополнения) множество чисел от 1 до $n-1$. Определим $B_i$ как множество чисел, делящихся на $p_i$ и меньших $n$. Его дополнение $B_i^C$~--- это в точности числа, меньшие $n$, которые не делятся на $p_i$. Соответственно числа, взаимопростые с $n$~--- это пересечение $\cap B_i^C$, то есть это все числа, которые не делятся ни на один из делителей числа $n$. Заметим, что $|B_i| = \frac{n}{p_i}$. Так же в силу того, что все $\{p_i\}$ взаимопросты, $|B_i\cap B_j| = \frac{n}{p_ip_j}$ (в это пересечение попадают только числа, кратные $p_ip_j$). Аналогичные утверждения можно получить для любых пересечений множеств $\{B_i\}$. Отсюда сразу же по теореме 3.34 вытекает требуемое утверждение.
\end{proof}

\begin{exercise}
\term{Беспорядком} называется такая перестановка, которая не оставляет на месте ни один элемент множества, то есть такая, что $\rho(x)\not=x$. Сколько беспорядков возможно задать на множестве $[n]$?
\end{exercise}

\begin{thm}
Существует в точности
$$k^n - {k \choose 1}(k-1)^n + {k\choose 2}(k-2)^n -\ldots$$
сюръекций вида $[n]\to[k]$ (напомню, что это функции $f$ такие, что для каждого $y\in[k]$ существует $x$ такой что $f(x) = y$).
\end{thm}
\begin{proof}
Пусть множество $B_i$~--- это множество таких отображений $f:[n]\to[k]$, что прообраз элемента $i$ не пуст (то есть существует такое $x$, что $f(x) = i$). Множество сюръекций~--- это в точности множество $\cap B_i$. Чтобы применить теорему~3.34 нам необходимо найти мощность всех возможных пересечений множеств семейства $\{B_i^C\}$. Однако, это легко: если $I\subset [n]$, то
$$\left|\bigcap_{i\in I} B_i^C\right| = (k-|I|)^n$$
поскольку это в точности множество отображений $[n]\to([k]\backslash I)$. Причём это выражение зависит только от количества множеств в пересечении $|I|$, а не от самих множеств, которые пересекаются. Отсюда получаем, что для каждого $m = |I|$ мы будем иметь $k\choose m$ слагаемых $(k-m)^n$.
\end{proof}

\begin{exercise}
Предположим, что компьютер запрограммирован выполнить $n_i$ задач типа $i$, всего есть $m$ типов задач. Компьютер выполняет задачи последовательно, но ему надо так распланировать свою работу, чтобы все задачи одного вида не выполнялись последовательно, он обязательно должен задачи чередовать (то есть он не может выполнить подряд $n_i$ задач типа $i$). Сколько всего способов существует для компьютера распланировать свою работу?
\end{exercise}

\section{Двойной счёт}

Напомню, что в графе степенью вершины $\deg(v)$ мы назвали количество рёбер, ей инцидентных.

\begin{thm}
Пусть $V$~--- множество вершин некоторого графа, $e$~--- количество его рёбер. Тогда
$$\sum_{v\in V}\deg(v) = 2e$$
\end{thm}
\begin{proof}
Мы можем пересчитать все рёбра графа двумя способами: собственно, пересчитывая сами рёбра (получим в этом случае $e$), либо же пересчитывая вершины и складывая степени $\deg{v}$. В этом случае, правда, получится, что каждое ребро мы учтём два раза, поскольку каждое ребро инцидентно ровно двум вершинам.
\end{proof}

Это элементарное доказательство является простейшим примером доказательства методом \term{двойного счёта}. Приём этот выглядит всегда одинаково: мы берём некоторый набор объектов и считаем его двумя разными способами, получая в итоге одно и то же значение, но записанное в разном виде. По большому счёту рекурсивные формулы для вычисления числа сочетаний, чисел Белла и чисел Стирлинга, а так же формулы их сумм, доказывались нами именно методом двойного счёта, только мы не произносили этого слова. Остаток этого параграфа мы посвятим разбору ещё нескольких подобных примеров.

\begin{thm}
$$\sum_{k=0}^m {m\choose k}{w\choose n-k} = {m+w\choose n}$$
\end{thm}
\begin{proof}
Пусть у нас имеется $m$ мужчин и $w$ женщин. Нам надо выбрать из них группу $n$ человек. Очевидно, что это можно сделать $m+w\choose n$ способами. С другой стороны, мы можем отдельно рассмотреть все варианты, когда мы выбираем $k$ мужчин и $n-k$ женщин, что даёт нам левую часть.
\end{proof}

\begin{corollary}
$$\sum_{k=0}^n{n\choose k}^2 = {2n\choose n}$$
\end{corollary}
\begin{proof}
Достаточно положить $m=w=n$ и заметить, что ${n\choose k}={n\choose n-k}$.
\end{proof}

\begin{thm}
$$\sum_{k=q}^n{n\choose k}{k\choose q} = 2^{n-q}{n\choose q}$$
\end{thm}
\begin{proof}
Левую часть можно интерпретировать как количество способов выбрать различные подмножества $[n]$ с по крайней мере $q$ элементами, и затем в этих множествах выделить ещё некоторые $q$ элементов. Правая часть даёт ровно ту же самую величину, но в этом случае мы вначале выбираем $q$ помеченных элементов из $[n]$, а затем добавляем к полученному набору один из $2^{n-q}$ подмножеств, составленных из оставшихся элементов.
\end{proof}

\begin{thm}
$$k!\sstirling{n}{k} = k^n - {k \choose 1}(k-1)^n + {k\choose 2}(k-2)^n -\ldots$$
\end{thm}
\begin{proof}
Справа, как мы увидели в конце прошлого параграфа, перечислено количество сюръекций $[n]\to[k]$. Однако, их можно перечислить и по-другому. Пусть $f$~--- некоторая сюръекция, тогда прообразы $f^{-1}(1)$, $f^{-1}(2)$,~... не пусты и задают некоторое разбиение множества $[n]$ на $k$ подмножеств. Таких разбиений существует $\sstirling{n}{k}$ штук. В то же время если $\rho$~--- некоторая перестановка на $[k]$, то $\rho\circ f$ так же задаёт то же самое разбиение множества $[n]$, хотя сюръекция уже будет другой. Поскольку мы имеем $k!$ таких перестановок $[k]$, общее количество сюръекций может быть так же определено как $k!\sstirling{n}{k}$.
\end{proof}

\begin{thm}
$$\sum_{k=1}^n k^2 = {n(n+1)(2n+1)\over 6}$$
\end{thm}
\begin{proof}
Я уже предлагал доказать это утверждение по индукции самостроятельно в упражнении 3.38 по индукции. Сейчас мы докажем эту формулу используя метод двойного счета. Искомую сумму обозначим за $S_n$.

Из того, что $(n+1)^2 = n^2 + 2n + 1$ следует, что каждое следующее слагаемое в сумме $S_n$ равняется предыдущему, увеличенному на $2n+1$. Первый элемент этой суммы 1. Второй элемент суммы по этому разложению мы можем записать как 1+3. Третий элемент как 1+3+5. Продолжая по аналогии, мы можем составить таблицу~3.2:

\begin{table}[h]
\centering
\begin{tabular}{ccccccc}
$1^2$ & $2^2$ & $3^2$ & $4^2$ & $5^2$ & $\ldots$ & $n^2$ \\
\hline
1 & 1 & 1 & 1 & 1 & $\vdots$ & 1 \\
& 3 & 3 & 3 & 3 & $\vdots$ & 3 \\
& & 5 & 5 & 5 & $\vdots$ & 5 \\
& & & 7 & 7 & $\vdots$ & 7 \\
& & & & 9 & $\vdots$ & 9 \\
& & & & & &  $\vdots$  \\
& & & & & &  $2n-1$  \\
\end{tabular}
\caption{Элементы суммы $S_n$}
\end{table}

Первоначальную сумму можно интерпретировать таким образом, что мы вначале в этой таблице складываем стобцы, а затем складываем результаты. Однако, мы можем поступить и по-другому, складывая вначале строки. Если считать строки начиная с верхней, то $i$-ая строка будет состоять из чисел $2i-1$, которых всего будет $n-i+1$ штук. Таким образом получаем:
\begin{align*}
S_n & = \sum_{i=1}^n (n-i+1)(2i-1)\\
& = \sum_{i=1}^n (2in - n - 2i^2 + i + 2i - 1)  \\
& = (2n+3)\sum_{i=1}^n i - 2\sum_{i=1}^n i^2 - n(n+1) \\
& = (2n+3){n(n+1)\over 2} - 2S_n - n(n+1)
\end{align*}
Здесь мы в последней строчке воспользовались формулой для треугольных чисел. Слагаемое $2S_n$ можно <<перенести>> в левую часть уравнения. Получаем:
\begin{align*}
3S_n = {n(n+1)(2n+1)\over 2}
\end{align*}
Разделив теперь обе части на 3 получаем результат.
\end{proof}

Мы до сих пор не сталкивались с переносом слагаемых в уравнениях, это доказательство~--- первый случай. Школьное правило гласит, что любое слагаемое может быть перенесено в другую часть уравнения с противоположным знаком. Это действительно так и я на всякий случай объясню почему. Пусть у нас есть сумма
\begin{equation}\label{ndc:1}
x+y = z
\end{equation}
Знак равенства по сути означает, что слева и справа стоят одинаковые значения, только записанные по-разному. А раз они одинаковые, то применив к ним одинаковые операции, мы так же получим разные значения. Например, мы по этим соображениям можем умножать обе части уравнения на одно и то же число, возводить в одинаковую степень и, в том числе, вычитать из обоих частей одно и то же значение. Вычтя из \eqref{ndc:1} значение $y$ получаем
$$x + y - y = z - y$$
но поскольку $y-y=0$, окончательно получаем
$$x = z - y$$
По сути мы получили то что утрвеждает школьное правило: <<перенесли>> $y$ в другую сторону со сменой знака.

Вернёмся, однако, к двойному счёту. Очень легко можно получить следующую интересную формулу (довольно, правда, бесполезную) для суммы степеней.

\begin{thm}
$$\sum_{i=1}^n i^k = \sum_{j=1}^n {n \choose j}\sstirling{k+1}{j}(j-1)!$$
\end{thm}
\begin{proof}
Рассмотрим множество последовательностей длины $k+1$, таких что все элементы в ней положительны и не превосходят $n$, причём последний элемент последовательности по совместительству максимален.

С одной стороны, если максимальный элемент последовательности $i$ (он же и последний), то оставшиеся $k$ элементов мы можем выбрать произвольным образом от 1 до $i$. Итого получаем $i^k$ таких последовательностей. Учитывая, что $i$ произвольно, сумма, данная в утверждении теоремы слева, перечисляет все интересующие нас последовательности.

С другой стороны мы можем предположить, что всего последовательность состоит из $j$ различных чисел (они могут повторяться). Способов выбрать эти числа $n\choose j$. После этого всю последовательность мы должны разбить на $j$ групп, элементы в каждой из которых будут равны. Итого таких разбиений имеется $\sstirling{k+1}{j}$. Нам осталось присвоить этим группам элементов значения из выбранных $j$ чисел. Учитывая, что последнее число всегда максимально, нам остаётся назначить $j-1$ значение, а способов сделать это столько же, сколько существует перестановок, то есть $(j-1)!$.
\end{proof}

\begin{definition}
Граф называется \term{связным}, если в нём существует путь между любыми двумя вершинами.
\end{definition}

\begin{definition}
\term{Деревом} называется связный граф без циклов.
\end{definition}

Деревья часто применяются в компьютерных системных поиска. Самый распространённый вариант~--- это двоичные деревья поиска, которые представляют собой следующую структуру: каждая вершина дерева обладает некоторым значением и, возможно, тремя гранями, называемых ветвями. Одна ветвь ведёт в направлении корня, другая ветвь, называемая левой, ведёт ко всем вершинам со значениями, меньшими чем текущее, а вторая ветвь, называемая правой, ведёт к большим значениям. Если предположить, что значения~--- это строки, то их порядок может восприниматься как алфавитный. Пример такого бинарного дерева поиска представлен на рисунке 3.11.

\tikzset{
  treenode/.style = {align=center, inner sep=0pt, text centered,
    font=\sffamily},
  arn_n/.style = {treenode, rectangle, black, font=\sffamily\bfseries, draw=white,
    fill=white},% arbre rouge noir, noeud noir
}

\begin{figure}[h]
\centering
\begin{tikzpicture}[->,level/.style={sibling distance = 5cm/#1,
  level distance = 1.5cm}] 
\node [arn_n] {Николай}
    child{ node [arn_n] {Евгения} 
            child{ node [arn_n] {Авдотья}}
            child{ node [arn_n] {Жанна}}                            
    }
    child{ node [arn_n] {Роман}
            child{ node [arn_n] {Ольга}}
            child{ node [arn_n] {Света}}
    }
; 
\end{tikzpicture}
\caption{Бинарное дерево поиска}
\end{figure}

Предположим, что в дереве на рисунке 3.11 так же в каждом узле дерева записан телефон, и что мы захотели найти телефон Ольги. Если бы мы просматривали все телефоны подряд, то в случае их упорядоченности по алфавиту, прежде чем мы наткнулись бы на Олин телефон, нам пришлось бы проверить шесть записей. Однако, вместо этого мы могли бы искать телефон в дереве, двигаясь от корня: вначале мы увидели бы, что имя Ольга должно идти после имени Николай, что значит, что мы должны искать по правой ветви от корня, где мы встречаем имя Роман. Ольга идёт раньше Романа по алфавиту, поэтому мы продолжаем искать её в левой ветви, где и находим её телефон. Итого нам потребовалось три шага поиска: вдвое быстрее чем при последовательном переборе.

\begin{exercise}
Пусть мы ищем телефон Фиофанта. Покажите, что при последовательном поиске нам потребовалось бы 7 шагов, чтобы убедиться, что такого телефона нет, а при поиске  в дереве всего 3 шага.
\end{exercise}

\begin{exercise}
Пусть у нас теперь имеется дерево, в котором записано 2147483647 телефонных номеров. Например, это может база данных ФСБ или ещё какая. Покажите, что используя бинарное дерево поиска, мы можем найти любой телефон (или убедиться, что его нет в базе) максимум за 31 шаг.
\end{exercise}

Последнее упражнение показывает, что использование деревьев может здорово упростить поиск информации (зачастую ускорение получается в миллионы раз). Собственно очень похожим образом устроены почти все базы данных, и без деревьев не было бы ни компании Гугл, ни, наверное, вообще компьютерной техники в современном её виде. Чтобы уметь анализировать скорость работы алгоритмов, нам прежде всего необходимо уметь пересчитывать все деревья.

Бинарные деревья~--- это в общем-то частный случай дерева. Как перечислить все возможные деревья поиска мы поймём позже в нашем курсе (мы изучим общие подходы), а пока что же мы перечислим  просто все возможные деревья с $n$ вершинами. При изучении следующего доказательства важно иметь ввиду следующие уточнения постановки задачи, которые мы подразумеваем:
\begin{enumerate}
\item Дерево не обладает корнем; все вершины равнозначны;
\item Вершины деревьев имеют определённые метки (данные от 1 до $n$), то есть даже если два дерева имеют одинаковую форму внешне, но вершины имеют разные именования, мы рассматриваем эти деревья как различные;
\item Каждая вершина может иметь произвольное количество инцидентных рёбер;
\item Рёбра, в отличие от бинарного дерева поиска, никак не упорядочены, то есть понятия <<левое>> или <<правое>> тут не имеет значения.
\end{enumerate}

Если изменить любое из этих условий, то формула для количества деревьев будет уже совершенно другой, но пока мы не будем рассматривать эти случаи.

\begin{table}[h]
\centering
\begin{tabular}{c|ccccc}
$x$ & 1 & 2 & 3 & 4 & 5 \\
\hline
$f(x)$ & 4 & 1 & 1 & 2 & 4
\end{tabular}
\caption{Пример функции $f$}
\end{table}

\begin{thm}
Существует $n^{n-2}$ деревьев с $n$ вершинами.
\end{thm}
\begin{proof}
Подсчитаем количество функций $f:[n]\to[n]$ двумя способами. С одной стороны количество таких функций $n^n$, это тривиальная теорема, рассмотренная нами в~\S~3.1. Попробуем теперь подсчитать количество функций $f$, сопоставив каждой функции некоторое дерево. Это довольно сложное рассуждение, поэтому будем рассматривать его на примере функции $f$, значения которой заданы в таблице~3.3. Пусть
$$C = \{x \in[n]| \exists k\ f^k(x) = x\}$$
то есть это множество таких элементов $x$ из $[n]$, что применяя к ним $f$ последовательно несколько раз, мы в какой-то момент получим тот же $x$. Для функции из нашего примера $C = \{1,2,4\}$.

Если упорядочить по возрастанию элементы $C$ и рассмотреть ограничение на нём $f|_C$, то эта функция будет действовать как перестановка $C$, которую мы можем записать строкой (в примере будет $f|_C = 412$). Теперь, выбрав в качестве вершин графа элементы $[n]$, мы можем соединить вершины из $C$ в порядке, заданном перестановкой. Элементы $[n]\backslash C$ пока правда остаются изолированными. Чтобы сформировать из них дерево, для каждого значения $x\in [n]\backslash C$, если $f(x)=y$ добавим ребро $xy$. Для функции из примера это будут рёбра 31 и 54. Получившееся дерево изображено на рисунке~3.12.

\begin{figure}[h]
\centering
\begin{tikzpicture}
\def\point{node [circle, draw, fill, inner sep = 0, minimum size = .1cm] }
\draw (0, 0) \point (p1) {};
\draw (1cm, 0cm) \point (p2) {};
\draw (0cm, 1cm) \point (p3) {};
\draw (-1cm, 0cm) \point (p4) {};
\draw (-1cm, -1cm) \point (p5) {};

\node [below] at (p1) {1};
\node [right] at (p2) {2 (конец)};
\node [left] at (p3) {3};
\node [left] at (p4) {(начало) 4};
\node [right] at (p5) {5};

\draw (p1) -- (p2);
\draw (p1) -- (p3);
\draw (p1) -- (p4);
\draw (p4) -- (p5);
\end{tikzpicture}
\caption{Дерево, построенное по функции $f$}
\end{figure}

Надо теперь показать, что и для каждого дерева мы можем задать функцию. Это делается в полной аналогии: вначале выбираем в дереве условные <<начало>> и <<конец>> (внимание!) и находим путь от начала к концу. Вершины этого пути задают множество $C$, а последовательность вершин перестановку $f|_C$. Для вершин $x$, не вошедших в этот путь, в качестве значения $f(x)$ выбираем следующую вершину по пути от $x$ до <<конца>> дерева.

Остаётся лишь заметить, что заданная конструкция даёт нам не просто дерево, а дерево с выбранными <<началом>> и <<концом>>. У дерева с $n$ вершинами есть $n^2$ способов определить начало и конец, поэтому количество произвольных деревьев в $n^2$ раз меньше, чем количество функций $[n]\to[n]$. (По-хорошему так же надо более чётко указать, что соответствие деревьев и функций в данном случае действительно однозначное, но это не сложно и я оставляю это в качестве упражнения читателю).
\end{proof}

Приведу для обшего развития несколько упражений на деревья. Эти задачи используют разные техники, которые мы рассматривали, и решаться могут по-разному. В задачах под деревьями будут пониматься бинарные деревья с корнем. \term{Листьями} мы будем называть вершины без потомков. Рёбра деревьев будут неупорядочены, но листья (и только они) будут иметь метки.

\begin{exercise}
Докажите, что дерево с $n$ листьями, описанное выше, будет иметь $2n-2$ рёбер.
\end{exercise}

\begin{exercise}
Докажите, что деревьев указанного вида с $n$ листьями существует
$$(2n - 1)!!$$
штук, где $n!!$ обозначает \term{двойной факториал}:
$$n!! = n\cdot (n-2) \cdot (n-4) \cdot\ldots$$
\end{exercise}

\begin{exercise}
Рассморим множество $[n]$. Сколько существует способов разбить это множество на пары? (Если $n$ нечётное, то один элемент должен остаться без пары).
\end{exercise}

\begin{exercise}
Докажите, что для чётного $n=2k$
$$n!! = 2^kk!$$
\end{exercise}

\begin{exercise}
Создал боженька $n$ бесполых бессмертных существ и сказал им: <<Плодитесь и размножайтесь!>> Зачать ребёнка (опять же бесполого и бессмертного) может любая пара существ, если они не являются родственниками. <<Родственниками>> существа являются лишь в том случае, если они потомки одного и того же подмножества $n$ первоначальных существ. Сколько всего существ наплодится таким образом? 
\end{exercise}
\section{Прочие комбинаторные величины}

В этом параграфе мы очень кратко рассмотрим прочие комбинаторные величины, которые могут оказаться полезны и которые будут в дальнейшем выступать в качесве примеров.

\subsection{Числа Фибоначчи}

\begin{definition}
\term{Числа Фибоначчи} $F_n$ определяются начальными условиями $F_0 = 0$, $F_1 = 1$ и соотношением
$$F_{n+1} = F_n + F_{n-1}$$
\end{definition}

Начальные значения чисел Фибоначчи выглядят так:
$$0, 1, 1, 2, 3, 5, 8, 13, 21, 34, 55, \ldots$$

Эти числа возникают в целом ряде задач и довольно распространены. Исторически числа Фибоначчи стали широко известны после решения Леонардо Пизанским (<<Фибоначчи>>  было его прозвищем, что переводится как <<сын Боначчи>>) следующей задачи в 1202 году:

\begin{exercise}
Предположим, что каждая взрослая пара кроликов каждый месяц производит на свет ещё одну молодую пару кроликов. Взросление кроликов наступает в течение одного месяца. Изначально у нас есть одна пара молодых клоликов. Через месяц она становится взрослой. Еще через месяц эта пара производит на свет ещё пару молодых кроликов (итого 2 пары, из которых 1 молодая). В следующий месяц эта же пара производит ещё одну молодую пару, а пара, которая была молодой, взрослеет (имеем 3 пары кроликов, 1 молодая). Таким же образом в следующий месяц мы будем иметь 5 пар кроликов, из которых 2 будут молодыми. Сколько кроликов будет в конце года?
\end{exercise}

На самом деле числа, которые мы сегодня называем числами Фибоначчи, были известны ещё древним Индусам. Математик Пиндас в своём трактате <<Чхандас>> (датированным примерно 200 годом до нашей эры) использует их при решении примерно такой задачи:

\begin{exercise}
Пусть нам надо пройти путь длины $n$. При проходе пути мы можем использовать либо шаги длины 1, либо шаги длины 2. Докажите, что существует ровно $F_{n+1}$ способов пройти путь используя такие шаги. (Например для пути длины 3 мы можем сделать три одинарных шага, либо вначале одинарный, а потом двойноё, либо наоборот: итого 3 разных способа пройти путь).
\end{exercise}

Сами индусы, правда, решали хоть и ту же задачу, но из другой предметной области: они исследовали сколько всего существует мелодий, состоящих лишь из одной ноты, которая может иметь либо одинарную, либо двойную длительность. Интерпретация, данная привёденным мной упражнением, используется при решении следующих двух задач:

\begin{exercise}
Докажите следующее тождество, используя двойной счёт:
$$F_{n+1} = \sum_{k=0}^{\lfloor {n\over 2}\rfloor} {n-k\choose k}$$
\end{exercise}

\begin{exercise}
Докажите следующее тождество (идея доказательства очень похожа):
$$F_{n+1} = 1+ \sum_{k=0}^{n-1}F_n$$
\end{exercise}

\begin{thm}
$$F_n F_{n+1} = \sum_{i=1}^n F_i^2$$
\end{thm}
\begin{figure}[h]
\centering
\begin{tikzpicture}
\draw (0, 0) rectangle (6.5cm, -4cm);
\draw (2.5cm, 0) -- (2.5cm, -4cm);
\draw (0, -1.5cm) -- (2.5cm, -1.5cm);
\draw (1cm, 0) -- (1cm, -1.5cm);
\draw (0, -.5cm) -- (1cm, -.5cm);
\draw (.5cm, 0) -- (.5cm, -.5cm);
\end{tikzpicture}
\caption{Прямоугольник Фибоначчи}
\end{figure}
\begin{proof}
Идея докательства продемонстрирована на рисунке~3.13. Прямоугльник Фибоначчи строится следующим образом: вначале рисуем квадрат единичной длиной стороны. Затем справа от него рисуем ещё один такой же квадрат. После под ними рисуем квадрат со стороной равной двум предыдущим квадратам. Затем справа рисуем опять квадрат со стороной, равной двум предыдущим. Совершая последовательно $n$ таких построений, в итоге имеем прямоугольник со сторонами $F_n$ и $F_{n+1}$. Всего его площадь (то есть количество единичных квадратиков, которые в него уместятся), равно $F_nF_{n+1}$. В то же время подсчитав площади квадртав в порядке построения получаем сумму  $\sum_{i=1}^n F_i^2$.
\end{proof}

Следующая наша теорема (называемая теоремой Цекендорфа) утверждает, что любое натуральное число можно представить единственным образом в виде суммы чисел Фибоначчи таким образом, что каждое из чисел будет использоваться максимум единожды, и что в этой сумме не будет присутствовать никакие два последовательные числа Фибоначчи. Числа $F_1$ и $F_0$ в этом представлении так же не учавтсвуют. Таким образом, для любого $a$ мы можем записать:
\begin{equation}\label{non:1}
a = \sum_{i=1}^m F_{\alpha_i}
\end{equation}
где $\alpha_k$~--- это последовательность номеров использованных чисел Фибоначчи.

Помимо интересного практического применения этой теоремы (о чём я напишу несколькими абзацами ниже), такое представление позволяет определить операцию <<Фибоначчиевого умножения>>. Пусть, например, мы представили число $b$ так же в соответствии с теоремой Цекендорфа:
$$b = \sum_{j=1}^n F_{\beta_j}$$

Тогда множение Фибоначчи определяется таким образом:
$$a\circ b = \sum_{i=1}^m\sum_{j=1}^n F_{\alpha_i+\beta_j}$$

\begin{exercise}
Покажите, что $2\circ3 = 13$ и что $4\circ4 = 40$.
\end{exercise}

Легко видеть, что такое умножение коммутативно ($a\circ b = b\circ a$), однако довольно неожиданно, что оно так же является и ассоциативным, то есть выполняется тождество
$$a\circ(b\circ c) = (a\circ b)\circ c$$

Это в общем-то единственное полезное свойство такого умножения, но сам факт довольно интересен и неожиданнен. Доказательство, которое я представляю ниже, сделает это утверждение очевидным.

\begin{thm}
Любое число $n$ допускает единственное представление в виде \eqref{non:1}.
\end{thm}
\begin{proof}
Данная теорема может быть легко доказана по индукции (что и сделал в 1972 году Цекендорф), однако я представлю более остроумное доказательство, придуманное Дональном Кнутом в 1988 году. В той же работе Кнут ввёл понятие Фибоначчиева умножения и показал его ассоциативность, как простое следствие из данное доказательства.

Теорему Цекендорфа можно переформулировать как возможность представить любое число в виде суммы
$$\sum_{i=0}^n d_iF_i$$
где коэффициенты $d_i$ могут принимать значения 0 или 1, причем два коэффициента подряд не могут принимать значение 1. Такую запись называют <<системой счисления Фибоначчи>>, поскольку она очень похожа на позиционные системы счисления, рассмотренные нами в~\S3.2, с той лишь разницей, что теперь вместо степеней некоторого основания мы используем значения $F_i$. По аналогии с позиционными системами счисления мы можем кратко записывать числа в ней, перечисляя последовательно коэффициенты $d_i$. Чтобы отличать Фибоначчиеву систему счисления, мы будем подписывать букву $f$ справа от записи. Например,
$$33 = 101010100_f$$
Два младных разряда (соответствующие $F_0$ и $F_1$) всегда равны нулю.

Чтобы получить такую запись, мы вначале позволим коэфициентам $d_i$ принимать произвольные значения, а затем путём их преобразований будем пошагово идти к требуемому представлению. Заметим, что при отсутствии ограничений на коэффициенты, представить натуральное число $a$ в виде суммы чисел Фибоначчи не представляет труда: достаточно взять значения $d_2 = a$ и $d_i = 0$ в остальных случаях (или, иначе, можно применить тот же подход с остатками от деления, который мы применяли при построении позиционных систем счисления).

Пусть мы получили некоторую запись натурального числа и в ней имеется коэффициент $d_i > 1$. Здесь может быть две ситуации: когда $d_{i-1}> 0$ и когда $d_{i-1} = 0$. В первом случае мы можем увеличить на единицу значение $d_{i+1}$ и уменьшить на единицу значения $d_i$ и $d_{i-1}$ по определнию чисел Фибоначчи. Зачему, что то же преобразование мы можем выполнить в случае
$$d_i = d_{i-1} = 1$$
Если же $d_{i-1} = 0$, то мы можем уменьшить на единицу $d_i$ и увеличить на единицу $d_{i-1}$ и $d_{i-2}$. Мы пришли к ситуации, когда $d_{i-1} > 0$, а как действовать в этом случае мы уже знаем. Применив указанное выше преобразование мы получаем суммарно, что мы увеличили на единицу $d_{i+1}$, уменьшили на двойку $d_i$ и на единицу $d_{i-2}$.

Если в ходе эти преобразований мы на каком-то шаге получаем $d_0 > 0$, то это можно безболезненно заменять на $d_0 = 0$. Если мы получили $d_1 > 0$, то можно смело прибавлять величину $d_1$ к $d_2$ зануляя $d_1$.

Здесь важно заметить два момента: во-первых, если рассматривать набор коэффициентов $\{d_i\}$ как упорядоченный набор, то после каждого такого преобразования новый набор будет больше старого в смысле лексикографического порядка. Применять одно из указанных преобразований мы можем всегда, если только наше число не представлено уже в требуемом виде. В то же время само множество допустимых наборов у нас ограничено, поскольку всегда найдётся $F_n > a$, так что продолжать применять бесконечно долго наши правила преобразования мы не сможем. Таким образом в итоге мы обязательно придём к Фибоначчиевому представлению числа.

Из этой интерпретации легко увидеть и единственность такого представления. Если нам даны два различных таких представления то мы можем вычесть большее в лексикографическом смысле из меньшего. Здесь правда может возникнуть проблема, что нам придётся вычитать единицу из нуля. Будем писать в этом случае $d_i =-1$, что означает, что этого слагаемого нам недостаёт. Поскольку мы вычитаем большее из меньшего, всегда найдётся такое $k$, что $d_{i+k} = 1$. Его можно занулить, увеличив на единицу значения $d_{i+k - 1}$ и $d_{i+k-2}$. Повторяя эту операцию несколько раз мы можем избавиться от всех значений $-1$, однако результат будет ненулевым, т.к. каждая такая операция не может уменьшать количество единиц в записи. Значит, рахзница между двумя представлениями положительно, а следовательно эти представления всё же задают различные числа.
\end{proof}

Очевидно теперь, что фибоначчиево умножение~--- это просто умножение чисел в столбик, с тем только лишь отличием, что мы на этот раз используем фибоначчиеву систему счисления. Отсюда уже легко понять, почему оно ассоциативно (поймите это в качестве упражнения).

Рассмотрим задачу передачи данных по сети. Для просторы будем считать, что мы последовательно передаём числа в диапазоне от 0 до 15 (на практике мы бы вероятнее всего были заинтересоаны в передаче полноценных байтов в диапазоне от 0 до 255). Кодировать числа мы будем битами 1 и 0, что означает наличие и отсутствие напряжения соответственно. Мы легко можем передать значения, используя простое двоичное представления. Пусть, например, мы хотим передать числа 11, 13 и 10. В двоичной записи мы передадим следующие данные:
$$1011\ 1101\ 1010\ \ldots \cong 11,13,10,\ldots$$
Однако, при передаче по сети данные могут теряться из-за помех. Пусть, например, мы потеряли четвёртый бит. Тогда полученные данные будут выглядеть как
$$1011\ 1011\ 0100\ \cong 11, 11, 4, \ldots$$
Очевидно, что все последние данные так же будут повреждены и получатель данных в результате получит совершенно бессмысленый набор данных. Если, например, эти данные~--- это видеотрансляция, то пропажа всего одного бита в потоке (что очень вероятно) по сути приведёт к неутранимой ошибке, что заставит пользователя устанавливать соединение заново. Это явно неприемлемо.

Фибоначчиева система счисления может решить проблему. Её критическим свойством является то, что в ней не могут идти подряд два бита 1, а значит мы можем использовать битовую последовательность 11 для разделения фрагментов данных. Для экономии трафика мы можем немного сэкономить место, избавившись, во-первых, от разрядов $d_0$ и $d_1$ (они всегда нулевые), а так же передавая запись в обратном порядке. Это гарантирует нам, что старшим битом в записи всегда будет 1, поэтому мы можем дописывать всего один бит 1 к нему, а не биты 11. Например, если закодировать таким образом $11=1010000_f$, то в итоге получим значение $001011$, что короче на один бит. Данные, приведённые выше, будет таким образом переданы в виде
$$001011\ 0000011\ 010011\ \ldots \cong 11, 13, 10, \ldots$$
Если теперь у нас произойдёт ошибка связи и какой-то бит потеряется, вставится или изменится, мы конечно получим какие-то данные в ошибочном виде, однако мы всегда знаем, что новое значение начинается после последовательности 11, поэтому любое повреждение будет иметь лишь локальный смысл. Опять же в случае видеопотока такое повреждение будет означать, что мы будем иметь помеху в нескольких каждрах, однако эта помеха сама же и устранится, т.к. последующие данные мы будем получать уже в корректном виде.

На практике, впрочем, разработано множество других более совершенных систем кодирования, но их мы касаться не будем.

\subsection{Числа Каталана}

\begin{definition}
\term{Числа Каталана} $C_n$ определяются начальным значением $C_0 = 1$ и условием
$$C_{n+1} = \sum_{k=0}^n C_kC_{n-k}$$
\end{definition}

Начальные значения чисел Каталана выглядят так:
$$1, 1, 2, 5, 14, 42, 132, 429, 1430, 4862, 16796, \ldots$$

Как и числа Фибоначчи, они возникают во многих задачах. Рассмотрим, например, количество строк длин $n$, состоящих из парных скобок (то есть строк типа "((()()))(())"). Пусть у нас имеется $n+1$ парная скобка. Возьмём первую пару (то есть скобки "(...)..."). Предположим, что внутри этой пары находится $k$ пар скобок. Тогда после этой пары скобок будет находиться ещё $n-k$ пар скобок. По индукции это даёт нам, что всего таких строк есть $C_kC_{n-k}$ штук. Суммируя теперь по всем значениям $k$ получаем, что количество таких строк определяется числами Каталана.

Давайте теперь подсчитаем количество спобов вычислить сумму (или любую другую ассоциативную операцию) $n+1$ слагаемых. В силу ассоциативности количество вычислить эту сумму равно количеству способов расставить $n$ пар скобок в этой записи. Итого опять же получаем, что это количество спобов равно величине $C_n$.

По расстановке скобок в сумме мы можем построить бинарное дерево. Пусть у нас есть $n+1$ слагаемое. Вначале сопоставим каждому слагаемому в соответствие узел дерева (эти узлы будем называть \term{внешними}~--- они не будут имеют потомков). Затем каждой арифметической операции будем ставить в соответствие узел дерева с двумя потомками, соответствующими складываемым значениям (будем называть такие узлы \term{внутренними}). Очевидно, что такие деревья однозначно соответствуют расстановкам скобок, а следовательно таких деревьев будет $C_n$ штук. Здесь $n$~--- это количество операций, оно же количество внутренних узлов (внешних соответствунно $n+1$). Если отбросить интерпретацию со слагаемыми, внешними и внутренними узлами, то получаем, что всего бинарных деревьев с $n$ узлами имеется $C_n$ штук. Рисунок~3.14 демонстрирует идею.

\begin{figure}[h]
\centering
\begin{tikzpicture}
\def\point{node [circle, draw, fill, inner sep = 0, minimum size = .1cm] }
\def\emptynode{node [draw=none,fill=none, below] }
\draw (-2cm, 0) \point (p1) {};
\draw (-1.5cm, .05cm) \emptynode (s12) {};
\draw (-1cm, 0) \point (p2) {};
\draw (-.5cm, .05cm) \emptynode (s23) {};
\draw (0cm, 0) \point (p3) {};
\draw (.5cm, .05cm) \emptynode (s34) {};
\draw (1cm, 0) \point (p4) {};
\draw (1.5cm, .05cm) \emptynode (s45) {};
\draw (2cm, 0) \point (p5) {};

\draw (-.5cm, .5cm) \point (p23) {};
\draw (1.5cm, .5cm) \point (p45) {};
\draw (-1.25cm, 1cm) \point (p123) {};
\draw (.125cm, 1.5cm) \point (p12345) {};

\node [below] at (p1) {(a};
\node [below] at (s12) {+};
\node [below] at (p2) {(b};
\node [below] at (s23) {+};
\node [below] at (p3) {c))};
\node [below] at (s34) {+};
\node [below] at (p4) {(d};
\node [below] at (s45) {+};
\node [below] at (p5) {e)};

\draw (p23) -- (p2);
\draw (p23) -- (p3);
\draw (p45) -- (p4);
\draw (p45) -- (p5);
\draw (p123) -- (p1);
\draw (p123) -- (p23);
\draw (p12345) -- (p123);
\draw (p12345) -- (p45);
\end{tikzpicture}
\caption{Бинарное дерево, построенное по выражению.}
\end{figure}

Стоит отдельно заострить внимание на отличиях от тех деревьев, формулу для числа которых мы получили в теореме~3.42:
\begin{enumerate}
\item Дерево обладает корнем;
\item Вершины не имеют именований, важна только их форма;
\item Каждая вершина может иметь максимум двух потомков;
\item Рёбра упорядочены, то есть важно, нарисовано оно слева или справа от узла.
\end{enumerate}

На самом деле та же величина за небольшой оговоркой показывает и количество произвольных деревьев. Для того, чтобы увидеть это, нам потребуется следующее определение.

\begin{definition}
\term{Лесом} называется граф, состоящий из нескольких несвязанных деревьев.
\end{definition}

Опять же леса рассматриваются в математики разные, но нас будут интересовать лишь те, в которых упорядочены как сами деревья, так и их рёбра. Оказывается, что между бинарными деревьями и лесами можно построить взаимооднозначное соответствие. Я покажу как по лесу построить бинарное дерево, процедура в обратную сторону совершенно аналогична.

В качестве корня бинарного дерева выбирается корень первого дерева в лесе. На рисунке 3.15 это узел $a$. Затем по левой ветви мы строим опять же бинарное дерево, которое строится по аналогии, если рассматривать первое дерево за вычетом корня, как лес (в примере на рисунке это лес, состоящий из деревьев d-h, e-i и f). По правой ветви мы строим бинарное дерево для леса, получающегося после удаления первого дерева (в примере это лес, состоящий из деревьев b-g и c).

\begin{figure}[h]
\begin{tikzpicture}
\def\point{node [circle, draw, fill, inner sep = 0, minimum size = .1cm] }
\draw (-1cm, 0) \point (lc) {};
\draw (-2cm, 0) \point (lb) {};
\draw (-2cm, -1cm) \point (lg) {};
\draw (-4cm, 0) \point (la) {};
\draw (-4cm, -1cm) \point (le) {};
\draw (-5cm, -1cm) \point (ld) {};
\draw (-3cm, -1cm) \point (lf) {};
\draw (-5cm, -2cm) \point (lh) {};
\draw (-4cm, -2cm) \point (li) {};

\node [right] at (lc) {c};
\node [above] at (lb) {b};
\node [right] at (lg) {g};
\node [above] at (la) {a};
\node [left] at (ld) {d};
\node [right] at (le) {e};
\node [right] at (lf) {f};
\node [below] at (lh) {h};
\node [below] at (li) {i};

\draw (lb) -- (lg);
\draw (la) -- (ld);
\draw (ld) -- (lh);
\draw (la) -- (le);
\draw (le) -- (li);
\draw (la) -- (lf);

\draw (-5.5cm, .5cm) rectangle (-.5cm, -4cm);


\draw (3cm, 0) \point (ra) {};
\draw (2cm, -1cm) \point (rd) {};
\draw (4cm, -1cm) \point (rb) {};
\draw (1.5cm, -2cm) \point (rh) {};
\draw (2.5cm, -2cm) \point (re) {};
\draw (3.5cm, -2cm) \point (rg) {};
\draw (4.5cm, -2cm) \point (rc) {};
\draw (2cm, -3cm) \point (ri) {};
\draw (3cm, -3cm) \point (rf) {};

\node [above] at (ra) {a};
\node [right] at (rb) {b};
\node [right] at (rc) {c};
\node [left] at (rd) {d};
\node [right] at (re) {e};
\node [below] at (rf) {f};
\node [right] at (rg) {g};
\node [left] at (rh) {h};
\node [below] at (ri) {i};

\draw (ra) -- (rd);
\draw (rd) -- (rh);
\draw (rd) -- (re);
\draw (re) -- (ri);
\draw (re) -- (rf);
\draw (ra) -- (rb);
\draw (rb) -- (rg);
\draw (rb) -- (rc);

\draw (5.5cm, .5cm) rectangle (.5cm, -4cm);
\end{tikzpicture}
\caption{Соответствие леса и бинарного дерева.}
\end{figure}

Теперь мы готовы к тому, чтобы понять сколько всего произвольных деревьев с $n$ вершинами существует. Если применить соответствие лес-бинарное дерево, то бинарное дерево, полученное из произвольного дерева, не будет иметь правой ветви у корня, левое же поддерево может быть произвольным бинарным. Итого у нас есть свобода выбора как расположить в дереве $n-1$ узел. Итого произвольных деревьев $n$ вершинами существует $C_{n-1}$ штук.

Ну и так далее. Числа Каталана всплывают постоянно и мне даже кажется довольно странным, что они довольно мало известны в народе, в отличие от тех же чисел Фибоначчи, которые применяют даже биржевые трейдеры (последние верят, что числа Фибоначчи имеют какое-то психологически-сакральное значение, что позволяет им предсказывать поведение цен на рынке). Думаю, в пользе чисел Каталана я вас убедил. Остаётся вопрос как можно их более эффективно находить.

Числа Каталана можно интерпретировать ещё вот как. Рассмотрим квадрат размером $n\times n$. Координаны в этом квадрате будем записывать парой $(x, y)$, которую будем называть \term{координатами}, и в которой $x$ и $y$ могут принимать значения от 0 до $n$. Мы начинаем движение из точки с координатами $(0, 0)$ и двигаемся в точку $(n, n)$, увеличивая за один шаг на единицу либо координату $x$, либо координату $y$. Числа Каталана в этом случае показывают количество путей, которые не пересекают диагональ квадрата (то есть никогда не заходят в точку с координатами $(x, x+1)$).

Увидеть это довольно легко по индукции. Пусть утверждение уже доказано для квадратов размером вплоть до $n$ и мы рассматриваем количество путей в квадрате со сторонами $n+1$. Пусть при движении мы в последний касаемся диагонали в позиции $(k, k)$. Количество способов пройти такой путь до точки касания есть $C_k$ (по предположению индукции). После этого дойти до финиша не коснувшись диагонали у нас остаётся $С_{n-k}$ способов (пройти нам надо ещё $n-k+1$ клетку, однако учитывая, что мы не должны касаться диагонали, первый шаг у нас увеличит координату $x$, а последний координату $y$, что по сути сводит задачу к нахождению пути в квадрате со стороной $n-k$). Отсюда следует, что такие пути так же задаются числами Каталана.

С другой стороны те читатели которые читают учебник последовательно, помнят, что вообще-то ровно то же самое количество путей не пересекающих диагональ квадрата предлагалось найти в упражнении~3.60 используя только понятие сочетаний. Я покажу сейчас как решается та задача.

Вначале поймём сколько вообще существует путей, увеличивающих за шаг лишь одну координату, без учёта пересечения диагонали (упражнение~3.59). Если размеры прямоугольинка $m\times n$, то нам надо $m$ раз увеличить координату $x$ и $n$ раз координату $y$. Итого нам надо совершить $m+n$ шагов, из которых надо выбрать $m$ шагов, которые увеличивают $x$. Способов сделать такой выбор имеется ${m + n \choose m}$. В случае, если нам дан квадрат размером $n\times n$, то очевидно количество путей будет равно ${2n\choose n}$.

\begin{figure}[h]
\begin{tikzpicture}
\def\point{node [circle, draw, fill, inner sep = 0, minimum size = .1cm] }
\centering
\draw (-3cm, -3cm) rectangle (3cm, 3cm);
\draw (-3cm, -3cm) \point (z) {};
\node [below] at (z) {(0,0)};

\draw (-3cm, -3cm) -- (3cm, 3cm);

\draw [very thick] (-3cm, -3cm) -- (-1cm, -3cm) -- (-1cm, -2cm) -- (0, -2cm) -- (0, 2cm) -- (1cm, 2cm) -- (1cm, 3cm) -- (3cm, 3cm);
\draw [dashed] (-3cm, -2cm) -- (2cm, 3cm);
\draw [very thick, dashed] (-4cm, -2cm) -- (-4cm, 0) -- (-3cm, 0) -- (-3cm, 1cm) -- (0, 1cm);

\draw (-4cm, -2cm) \point (t) {};
\node [left] at (t) {(-1, 1)};

\end{tikzpicture}
\caption{Отражение начала пути от диагонали.}
\end{figure}

Рассмотрим теперь путь, который диагональ всё же пересекает. Пусть первое пересечение происходит в позиции $(k, k+1)$. Давайте отразим часть пути до этой точки относительно диагонали, сдвинутой вверх на единицу (см. рисунок 3.16). При этом получится некий путь, начинающийся в координатах $(-1, 1)$ и идущий в точку $(n, n)$. Таких путей ровно ${2n\choose n-1}$ штук, поскольку они описывают движение в прямоугольнике $((n-1)\times (n+1)$. Легко так же увидеть, что все пути из точки $(-1, 1)$ после отражения их от начала до точки пересечения диагонали (а они её пересекут обязательно) дают нам путь, который нас не устраивает. Итого нам осталось вычесть количество <<неправильных путей>> из общего числа путей:
$$C_n = {2n\choose n} - {2n\choose n-1} = {{2n\choose n}\over n+1}$$
Последнее равенство легко проверяется, я предлагаю выполнить вам проверку в качестве упражнения.

\subsection{Числа Рамсея}

Эта секция, вероятно, окажется наименее полезной и прикладной из всего, что до сих пор было в учебнике, однако сам результат, изложенный ниже, кого-то может удивить и кому-то может показаться эстетически красивым. Лишним в любом случае не будет.

Напомню, что полным графом $K_n$ мы условились называть граф с $n$ вершинами такой, что любые две его вершины соединены рёбрами. Для нужд этого параграфа мы будем так же считать, что ребра графа разукрашены в разные цвета, которых всего имеется $m$ штук (это предположение эквиваленто заданию отображения $c:E\to[m]$, где $E$~--- множество ребер графа).

\begin{definition}
\term{Числом Рамсея} $r = R(c_1, c_2, \ldots, c_m)$ называется такая величина, что любой граф $K_n, n\ge r$ будет иметь хотя бы один подграф $K_{c_i}$, ребра которого целиком окрашены в цвет $i$.
\end{definition}

Пока это определение наверняка выглядит слишком абстрактно и непонятно. Давайте для примера рассмотрим число Рамсея $R(3, 3)$. Здесь речь идёт о раскраске графа в два цвета (назовём их для определённости синим и красным) и число $R(3, 3)$ показывает минимальное количество вершин, которое должен иметь полный граф, чтобы в нём обязательно нашёлся либо красный подграф $K_3$, либо синий подграф $K_3$. (Обратите внимание, что сам граф $K_3$ имеет вид треугольника).

\begin{exercise}
Раскрасте в два цвета рёбра графа $K_5$ таким образом, чтобы в нём не было одноцветного треугольника.
\end{exercise}

Из приведённого упражнения следует, что $R(3,3)>5$. Давайте теперь рассмотрим граф $K_6$. Возьмём некоторую его вершину $a$. Из этой вершины исходит пять рёбер и среди них обязательно должно быть по крайней мере три ребра одного цвета. Для определённости будем считать, что это три красных ребра, ведущих в вершины $x$, $y$ и $z$. Если так сложилось, что все три ребра, соединяющие $xyz$ синие, то мы получили синий треугольник $K_3$. Если же среди этих ребёр найдётся красный (пусть, для определённости, это ребро $xy$), то мы имеем крачный треугольник $axy$. Таким образом мы в любом случае будем иметь либо синий $K_3$-подграф, либо красный, и, следовательно, $R(3, 3) = 6$.

Приведённые рассуждения часто формулируют в виде следующей задачи: докажите, что среди группы из шести произвольных людей либо найдутся трое попарно знакомых друг с другом, либо трое попарно не знакомых. Эта задача требует по сути доказать, что $R(3, 3) \le 6$, если обозначить людей за вершины графа, а ребра раскрасить в соответствии с тем, знакомы эти люди или нет.

Что совершенно неочевидно про числа Рамсея, так это то что они вообще в принципе существуют. Пусть мы раскрашиваем граф в четыре цвета: красный, желтый и зеленый. Верно ли, что любой достаточно большой граф $K_n$ всегда будет иметь либо красный $K_{12}$-подграф, либо зелёный $K_{351}$-подграф, либо жёлтый $K_{12422}$-подграф? То что любой довольно большой граф удовлетворяет этому свойству само по себе удивительно и до предъявления строгого доказательства должно вызывать сомнения. Однако это так.

Мы разобьём доказательство того, что числа Рамсея всегда определы на два этапа. Вначале мы докажем, что числа Рамсея определены для двухцветных раскрасок, а затем перейдём к общему случаю.

\begin{thm}
Число Рамсея $R(a, b)$ всегда определено, и, более того,
$$R(a, b) \le R(a - 1, b) + R(a, b- 1)$$
\end{thm}
\begin{proof}
Если мы докажем неравенство, то утверждение о существовании чисел Рамсея будет следовать из него само собой. Поэтому сразу возьмёмся за неравенство. Предположим по индукции, что оно выполняется для всех значений $R(x, y)$, где по крайней мере либо $x<a$ либо $y<b$. Первый цвет будем называть красным, второй синим. Рассмотрим полный граф с
$$R(a - 1, b) + R(a, b- 1)$$
вершинами и возьём в нём произвольную вершину $x$. Опять же по обобщённому принципу Дирихле с ним инцидентно либо $R(a-1, b)$ красных рёбер, либо $R(a, b-1)$ синих. Пусть верно первое (второй случай рассматривается аналогично) и рассмотрим все вершины, соединённые с $x$ красными рёбрами. Эти вершины образуют подграф $K_{R(a-1, b)}$. По предположению индукции он либо имеет красный $K_{a-1}$-подграф, либо синий $K_b$-подграф. В последнем случае требуемый подграф найден, в первом же случае мы можем добавить к красному графу $K_{a-1}$ вершину $x$, и, поскольку она соединена с вершинами $K_{R(a-1, b)}$ красными ребрами, мы получаем красный $K_a$-подграф.
\end{proof}

\begin{exercise}
Докажите, что
$$R(a, b) \le {a+b-2\choose a - 1}$$
\end{exercise}

\begin{thm}
Число Рамсея $R(c_1, c_2, \ldots, c_m)$ всегда определено, и, более того,
$$R(c_1, \ldots,c_{m-2}, c_{m-1}, c_m) \le R(c_1, \ldots, c_{m-2}, R(c_{m-1}, c_m)$$
\end{thm}
\begin{proof}
Опять же существование числа Рамсея следует из неравенства. Доказывать мы его будем опять по индукции, но на этот раз будем проводить её по количеству цветов. Предположим, что для количества цветов, меньшего $m$, утверждение верно. Далее, забудем на некоторое время различие между цветами $m-1$ и $m$ и будем считать их эквивалентными. Если взять полный граф с
$$R(c_1, \ldots, c_{m-2}, R(c_{m-1}, c_m))$$
вершинами, то он будет либо иметь $K_{c_i}$-подграф цвета $i$, если $i<m-1$, либо же будет иметь $K_{R(c_{m-1}, c_m)}$-подграф с рёбрами цветов $m-1$ и $m$. В первом случае требуемое доказано. В последнем случае по предположению индукции мы имеем внутри этого подграфа либо $K_{c_{m-1}}$-подграф цвета $m-1$, либо $K_{c_m}$-подграф цвета $m$.
\end{proof}

В отличии от всех предыдущих величин, которые мы приводили, числа Рамсея совершенно не поддаются вычислению (теоретически они могут быть вычислены, но сложность этого просто чудовищная). Например, до сих пор неизвестно значение $R(5, 5)$. Оценка, приведённая нами в упражнении~3.86 говорит нам о том, что $R(5, 5)\le70$. На самом деле в 1992 году было доказано\footnote{<<A new upper bound for the Ramsey number R (5, 5)>>, B.~McKay, S.~Radziszowski, 1992}, что $R(5, 5)\le 49$. Так же в 1987 было доказано, что $R(5, 5) > 42$. То есть всего у нас есть 7 вариантов чему оно может быть равно. Давайте посмотрим как можно было бы точно определить используя эту оценку число Рамсея.

Возьмём граф $K_{46}$~--- число 46 лежит посередине между двумя допустимыми границами, и в случае если мы подтвердим или опровергнем наличие раскрасок, не имеющих $K_5$-одноцветных подграфов, то мы сразу же улучшим существующий научный результат аж в два раза. Очевидный способ, как это можно проверить, это перебрать все возможные графы $K_{46}$. Он имеет $46\times45 = 2070$ ребёр, каждое из которых мы можем раскрасить в один из двух цветов. Итого мы имеем $2^{2070}$ различных раскрасок графов $K_{46}$. Чему равно $2^{2070}$ сказать очень сложно, но это очень много. Калькулятор не имеет столько цифр. Чтобы оценить масштаб этого числа, попробуйте просто подсчитать разные степени числа 2.

Не смотря на то что числа Рамсея практически невычислимы, сам факт их существования может иметь следствия. Пусть у нас есть некоторая последовательность (конечная или бесконечная). Если выкинуть из неё часть членов, то то что останется мы будем называть \term{подпоследовательностью}. Теория Рамсея даёт нам следующий результат:

\begin{exercise}
Докажите используя теорему Рамсея, что для любых чисел $a$ и $b$ найдётся такое $n$, что любая последовательность различных чисел длины $n$ будет иметь либо возрастающую подпоследовательность длины $a$, либо убывающую подпоследовательность длины $b$.
\end{exercise}

\begin{exercise}
На самом деле если не использовать теорию Рамсея, то прошлое утверждение можно уточнить. Докажите, используя принцип Дирихле и двойной счёт, что на самом деле любая последовательность длины $(a-1)(b-1)+1$ будет удовлетворять условиям прошлого упражнения. Для доказательства можно сопоставить каждому элементу последовательности $z_k$ пару $(x_k, y_k)$, где $x_k$~--- наибольная длина возрастающей последовательности, оканчивающийся в позиции $k$, а $y_k$~--- наибольшая длина убывающей последовательности, оканчивающейся в позиции $k$. Данная уточнённая оценка называется \term{теоремой Эрдёша-Секереша}.
\end{exercise}

\begin{thm}
Для любого $r$ найдётся такое $n$, что как бы мы ни разбивали множество $[n]$ на $r$ частей, в одной из них всегда найдутся такие числа $x, y, z$ (не обязательно различные), что $x+y=z$.
\end{thm}
\begin{proof}
Раскрасим элементы множества $[n]$ в $r$ цветов. Построим граф $K_n$, раскрасив каждое ребро $xy, x>y$ тем цветом, в который было раскрашено число $x-y$. Из существования числа $R(3, \ldots, 3)$ следует, что если $n$ достаточно велико, то в $K_n$ найдётся треугольник, раскрашенный одним цветом. Это в свою очередь означает, что найдутся вершины $i> j> k$ такие, что числа $x=i-j$, $y=j-k$, $z=i-k$ раскрашены в один цвет. Очевидно, что $x + y = z$.
\end{proof}

Приведённая теорема называется теоремой Шура. Похожее решение имеют следующие два упражнения.

\begin{exercise}
Докажите, что для любого $r$ найдётся такое $n$, что при любом разбиении $[n]$ найдётся такой элемент разбиения, что в нём будут содержаться числа $a$, $b$, $a+b$, $c$, $b+c$, d, что для них выполняется соотношение
$$a+b+c = d$$
\end{exercise}

\begin{exercise}
Докажите, что теорема Шура останется верна, если наложить условие $x\not= y$.
\end{exercise}
%\section{О честных выборах}

Отвлечёмся немного от теории и обратимся к теме, которая на первый взгляд к серьёзной математике почти не имеет отношения: к политическим выборам. Сделаем мы это с одной стороны ради чистого интереса и веселья, а с другой стороны для того, чтобы на простом интуитивном примере ввести понятие \term{ультрафильтров}, которые впослеедствии постужит нам добрую службу. Материал этого параграфа строго не обязателен для изучения (тем более упражнения), за исключения заключительной части, вводящей определения фильтров и ультрафильтов.

Предположим, что у нас имеется кандидат в президенты, которого готовы отдать голоса 26\% избирателей. Вслед за ним идёт кандидат с 12\%, за ним с 11\% и дальше идёт большое количетво кандидатов, за которые готовы голосовать менее 10\% населения. Предположим теперь, что самый популярный кандидат~--- фашист, людоед и сволочь, и в общем-то кроме 26\% его сторонников никто больше за него голосовать не стал бы и вообще в страшном сне видит его победу. Одна проблема: его противники разбиты на разные идеологические лагери и не имеют единого кандидата. Может быть большинство голосует даже не за своего кандидата, а "лишь бы против людоеда". Вот только против они голосуют неорганизованно. По результатам выборов простым большинством людоед побеждает, хотя это совершенно не отражает волю большинства.

Чтобы таких исторических трагедий не происходило, была придумана двухступенчатая система голосования. В первом этапе голосования определяются два фаворита, а во вотором туре голосовать можно уже только за одного из лидеров. Назовём условно кандидата с 12\% голосов <<либералом>>. Не смотря на то, что в первом туре он значительно уступает людоеду, они оба проходят во второй тур, и уже во втором туре все противники канибаллизма голосуют за либерала. Не потому что он очень им нравится, а потому что они не хотят допустить фашиста во власть. Итог: либерал побеждает с 74\% голосов против всё тех же 26\% у фашиста.

Вроде бы нас двухступенчатая система спасла от трагедии. Но так ли хороша она на самом деле? Фашист ведь не дурак, и при наличии поддержки его значительной долей избирателей, он может ввести на выборы своего искусственного оппонента, а в действительности единомышленника. Если он так поступит, то, скажем, половина избираталей уйдёт к <<аппоненту>> и оба получат примерно по 13\% голосов. Это всё равно больше, чем у либерала, и в итоге во второй тур проходят два практически идентичных людоеда, один из которых становится президентом, а второй премьер-министром.

Такое большое преимущество конечно радко случается, хотя бывает. Предположим, что за фашиста отдают голоса всего 14\% населения. В этом случае ему победить будет уже сложнее, но он всё равно имеет возможность манипулировать выборами. Фашист может создать искуственного конкурента своему наиболее опасному аппоненту, введя на выборы ещё одного либерала. Пусть этот второй либерал не будет популярен, но даже если он отъест хотя бы 2\% от первого либерала, во второй тур либералы уже не пройдут. А кандидат, следующий за либералом с 11\% голосами, вероятно, очень слаб. Например, он сталинист или ЛГБТ-активист, а в России многие предпочтут голосовать скорее за фашиста, нежели за гомосексуалиста (все же нормальные пацаны). Опять же фашист побеждает.

Помимо того, что кандидаты в президенты могут манипулировать выборами, вводя своих фиктивных кандидатов, сами избиратели так же действуют часто тактически. При двуступенчатой системе голосования у действительно симпатичного кандидата редко есть хоть какие-либо шансы на успех и поэтому голосовать за него нет смысла. Куда полезнее определиться с тем, чью победу допускать не хотелось бы вообще никак и бороться против него. В этом случае избирателю разумно определить список хоть как-то приемлемых кандидатов, оценить того, у которого наивысшие шансы на проход во второй тур, и голосовать за него. Многие исследования подтверждают, что в большинстве случае именно так и происходит: люди голосуют не за своего кандидата, а за самого серьезного оппонента тому, кто им неприятен.

Наиболее подвержденны такитческим голосованиям выбора в парламен, который как правило формируется пропорционально: какая доля населения отдала свои голоса за партию, такую долю партия и получит в парламенте. При этом доля может быть не произвольной, а существует некий <<проходной барьер>>. В России, например, этот барьер составляет 5\%, и если партия не получает 5\% голосов, то она вообще не получает представительства в Думе. Необходимость в каком-либо барьере существует чисто техническая: если в парламенте имеется всего 100 мест, а партия набрала 5 голосов из двух миллионов, то очевидно, что места ей не хватит. Другое дело, что правительство часто завышает барьер куда выше, нежели того требуют технические соображения. Цель, которая при этом преследуется, официально состоит в том, чтобы избежать дробления парламента на маленькие фракции и тем самым позволить парламенту более эффективно работать, а так же в том, чтобы не допустить в парламент радикалов, маргиналов и экстремистов.

На практике часто получается так, что проходной барьер оказывается средством манипулирования выборами. На выборах в ГосДуму в 1995 году 45\% голосов набрали те партии, которые не смогли предолеть пятипроцентного барера. Таким образом мнение половины населения при распределении мест в Думе было вообще никак не учтено и парламентские места по сути были отданы людям, против которых голосовала половина населения. На выборах 2011 года проходной барьер был повышен до 7\%, что по сути перекрывало путь в правительство любым оппозиционным партиям.

Процентный барьер могут использовать не только политические партии в своих целях, он так же предполагает широкие возможности по тактическому голосованию для избирателей. Предположим, что на 100 мест парламента претендует четыре партии~--- две партии имеют по 44\% голосов, одна оппозиционная партия 8\% голосов и одна оппозиционная 4\%. При проходном барьере в 5\% последняя партия не проходит в парламент. После голосования без учёта проигравшей партии, две крупных партии получат по 46 мест каждая, и оппозиционная партия, прошедшая барьер получит 8 мест.

Предположим теперь, что часть оппозиционеров (скажем, 2\%) решила отдать свои голоса менее популярной партии. В этом случае обе партии получают по 6\% и обе проходят в парламент. На каждую партию получается меньше мест (6 вместо восьми), однако в сумме оппозиция представлена уже 12-ю парламентариями, а не 8-ю. Крупным партиям так же приходится немного стесниться: вместо 92 мест на двоих они теперь имеют только 88 мест.

Понятно, что когда политики пытаются манипулировать выборами~--- это плохо. Но на самом деле так же плохо и когда избиратели манипулируют выборами, так в этом случае выборы превращаются из процесса определения мнения населения в интеллектуальную стратегическую игру где побеждает не тот, кто дейстивительно предпочтителен обществом, а тот, кто переиграл оппонента. Это явно не соответствует заявленной цели демократических выборов.

Проблема тут кроется не в избирателях или политиках, которые манипулируют выборами, а в самих правилах игры. Как мы видели, двухступенчатые выборы могут защитить нас от ситуации, при которой побеждает наименее желательный кандидат. Можно придумать и другие системы выборов, которые ещё более надёжды.

Простейшая система голосований~--- <<\term{одобрительные выборы}>>, когда избиратель указывает в биллютене не лучшего по его мнению кандидата, а ставит галку напротив всех тех кандидатов, которые в принципе ему приемлемы. Кандидат, который примлем по мнению большинства избирателей, побеждает. При такой системе выборов введение любых новых кандидатов на выборы никак не может повлиять на выбор победителя, поскольку учитывается не доля отданных голосов, а общее количество одобрений.

В то же время такая система выборов оказывается значительно хуже, чем двуступенчатое голосование. Чтобы увидеть это, давайте представим, что у нас есть четыре кандидата: $A$, $B$, $C$ и $D$ и три избирателя. Предпочтения каждого избирателя определяются некоторой перестановкой множества кандидатов.

Пусть первый избиратель имеет предпочтения $A>B>D>C$, второй $B>C>D>A$ и третий $C>A>D>B$. При таких предпочтениях кандидат $D$ оказывается самым худшим: при выборах лишь между двумя кандидатами, он всегда проиграет. В то же время если каждый из избирателей обозначит всех кандидатов как приемлемых, кроме своего наименее желательного кандидата, то кандидат $D$ наберёт наибольшее количество голосов и победит.

Одобрительные выборы имеют ещё и такую проблему: может случиться такое, что кандидат, однозначно предпочитаемый более чем половиной избирателей, в итоге не победит на выборах. Например, пусть 55 избирателей имеют предпочтения $A>B>C$, 35 избирателей $B>C>A$ и 10 избирателей $C>B>A$. 35 избирателей, однозначно предпочитающих кандидата $A$~--- это больше половины. Однако, если все кандидаты считают приемлемыми двух кандидатов из трёх, то в итоге кандидат $A$ окажется приемлемым 45 раз, кандидат $B$ 100 раз и $C$ 45 раз. Побеждает $B$.

К этому явлению можно относиться по-разному. Кто-то скажет, что голосование с таким свойством неприемлемо и при нём побеждают только очень средние кандидаты. Кто-то напротив, скажет, что это вид голосования, максимально учитывающее итересы каждого избирателя, выбирая пусть не лучшего кандидата, но приемлемого.

\begin{exercise}
Можно придумать и систему голосования, при которой каждый избиратель не просто сообщает приемлем ли ему кандидат, но выставляет каждому кандидату оценку как в школе. Все оценки в итоге суммируются и побеждает кандидат, получивший набольшую оценку. Покажите, что при таком голосовании опять же может победить самый слабый кандидат (в том смысле, что при выборах одного из двух он проиграл бы каждому другому кандидату), а так же что кандидат, которого однозначно предпочитает больше половины избирателей, может не победить.
\end{exercise}

Голосование с выставлением оценок можно интерпретировать так, что каждый избиратель оценивает полезность данного кандидата лично для себя. Побеждает кандидат, суммарная полезность которого наиболее высока. Можно провести аналогию, что оценка кандидата~--- это то, сколько условно будет зарабатывать избиратель при победе данного кандидата. После голосования побеждает кандидат, суммурный доход населения при котором будет максимальным. Может показаться, что это оптимальная в экономическом смысле система голосования, однако это не так: никто не гарантирует, что большие доходы будут распределены справедливо. Вполне может быть, что при одном кандидате 100 человек зарабатывает по 50 рублей (условно) и один 200 рублей, а при другом 100 человек зарабатывают по рублю, а один целлый миллион. Победит конечно последний кандидат, но предпочтительнее для большинства, очевидно, первый.

Так же такая методика очень подверждена тактическому голосованию, поскольку она позволяет несправедливо занижать баллы кандидатам.

\begin{exercise}
Если ограничить голосование с оценками так, что каждый избиратель должен упорядочить кандидатов (то есть поставить им всем разные оценки от 1 до $m$, где $m$~--- это число кандидатов), то получится метод голосования, называемый \term{методом Борда}. Докажите, что при этом подходе слабейший кандидат никак не может победить. (Подсказка: используйте двойной счёт). Все остальные недостатки этот подход, однако, сохраняет.
\end{exercise}

1785 году политик, математик и философ Николя де Кондорсе попробовал ввести частичный порядок на множестве кандидатов следующим образом: отношение $A>B$ верно в том случае, если при выборе одного из двух большинство людей проголосовали бы за кандидата $A$. При таком частичном порядке победителя можно было бы определить следующим образом: если для любого кандидата $X$ выполняется отношение $A>X$, то A логично признать победителем. Такой кандидат называется \term{победителем по Кондорсе}. По аналогии вводится понятие \term{проигравший по Кондорсе} (по сути мы их и рассматривали, когда говорили о <<худших кандидатах>>). Хорошая система голосования предполагает, что победитель по Кондорсе всегда побеждает на выборах.

Проблема заключается в том, что такое отношение не является транзитивным, то есть из того, что $A>B$ и $B>C$ вовсе не следует, что $A>C$:

\begin{example}
Пусть один избиратель имеет предпочтения $A>B>C$, один предпочтения $C>A>B$ и один $B>C>A$. В этом случае при голосовании между $A и B$ окажется, что $A>B$, при голосовании между $B и C$ будет $B>C$ и при голосовании между $A$ и $C$ будет $C>A$. В итоге имеем $A>B>C>A$.
\end{example}

Этот факт называется <<парадоксом Кондорсе>> и он показывает, что какого-то общего победителя может и не быть. Общественные предпочтения не обладают свойством транзитивности и это кажется довольно парадоксальным. Тем не менее, когда победитель в смысле Кондорсе всё же имеется, было бы желательно, чтобы на выборах побеждал действительно он. Все рассмотренные нами системы голосования в действительности не обладают этим свойством.

\begin{example}
Пусть 23 избирателя имеют предпочтения $A>C>B$, 19 человек имеют предпочтения $B>C>A$, 16 человек $C>B>A$, 2 человека $C>A>B$. Тогда при голосовании простым большинством побеждает кандидат $A$, при двуступенчатых выборах победит кандидат $B$, однако победителем по Кондорсе является кандидат $C$.
\end{example}

\begin{exercise}
Покажите, что в матоде Борда и при одобрительном голосовании (а так же при голосовании оценками) победитель по Кондорсе вовсе не обязательно окажется победителем.
\end{exercise}

\begin{exercise}
Ещё одна система голосования~---\term{рейтинговое голосование}. Избиратели опять же полностью упорядочивают всех кандиадтов в биллютенях. Затем голоса подсчитаются так, будто бы все проголосовали за первого кандидата. Затем из рассмотрения исключается кандидат, набравший меньшее число голосов и его голоса перераспределяются между другими кандидатами соответственно предпочтениям, голосовавшим за него. Рассмотрим опять пример~3.18. В нём вначале за $A$ будет отдано 23 голоса, за $B$ 19 голосов и за $C$ 18 голосов. $C$ выбывает и его голоса перераспределяются между $A$ и $B$: 16 голосов идёт к $B$ (итого 35) и 2 голоса идёт к $A$ (итого 25), таким образом $B$ побеждает и голосование не выбрало победителя по Кондорсе. Однако, такой метод голосования гарантирует, что никогда не побежит проигравший по Кондорсе и что введение очень похожего кандидата на уже существующего (но не фаварита) не повлияет на итог выборов. Докажите это.
\end{exercise}

\begin{exercise}
Рассмотрим все возможные пары голосований между двумя кандидатами и запишем количество голосов, отданных за каждого из кандидатов. Данные для примера~3.18 будут выглядеть так:
\begin{enumerate}
\item $C>B$: 46
\item $C>A$: 37
\item $B>A$: 35
\item $A>B$: 25
\item $A>C$: 23
\item $B>C$: 19
\end{enumerate}

\begin{figure}[h]
\centering
\begin{tikzpicture}

\def\point{node [circle, draw, fill, inner sep = 0, minimum size = .1cm] }

\draw (0, 1cm) \point (A) {};
\draw (0.866cm, 0) \point (B) {};
\draw (-0.866cm, 0) \point (C) {};

\node [above] at (A) {A};
\node [right] at (B) {B};
\node [left] at (C) {C};

\draw[->, thick] (C) -- (A);
\draw[->, thick] (C) -- (B);
\draw[->, thick] (B) -- (A);

\end{tikzpicture}
\caption{Результат голосования.}
\end{figure}

Я упорядочил этот список по убыванию количества отданных голосов. Если бы в этом списке оказалось два одинаковых значения (например, $X>Y$ и $T>U$), то мы сравнили противоположные пары ($Y>X$ и $U>T$) и поставили бы выше ту, которая имеет меньше голосов за противоположный вариант. Теперь, используя этот упорядоченный список, мы начинаем рисовать ориентированный граф, вершинами которого являются кандидаты. Мы начинаем рисовать рёбра последовательно от начала списка к концу, проская те рёбра, в результате которых образуется цикл. На рисунке~3.17 показан результирующий граф для нешего примера. На таком графе окажутся вершины, из которых исходят рёбра, но в которые рёбра не заходят. Такая вершина показывает победителя выборов (в нашем случае $C$). Докажите, что, такая система голосования всегда выберет победиля по Кондорсе, если таковой имеется, что она никогда не выберет проигравшего по Кондорсе, и что введение фиктивных кандидатов (походих на существующих кандидатов, если они не фавориты) не повлияет на исход голосования. В некоторых случаях эта система не сможет определить победителя, что это за случаи такие?
\end{exercise}

Парадокс Кондорсе показывает, что <<честных выборов>> существовать не может по крайней мене таком смысле: предпочтения социума не транзитивны, а следовательно во многих случаях <<объективного>> победителя может и не быть. Однако, это не значит что все выборы заведомо <<нечестные>>. Во-первых, во многих случаях победитель по Кондорсе всё же существует. Во-вторых, помимо критерия победиля по Кондорсе могут быть и другие критерии: например, суммарная полезность победителя для общества, если избиратели её оценивают в ходе голосования. В конце этого параграфа мы продемонстрируем ещё несколько теорем о <<невозможности справедливых выборов>>, которые исходят уже из других предположений, но прежде покажем довольно простой результат, называемой \term{теоремой Мэя}.

\begin{thm}
Голосование простым большинством является единственной возможной системой голосования, удовлетворящей одновременно следующим критериям:
\begin{enumerate}
\item выбор осуществляется между двумя кандидатами, в результате чего либо один из них побеждает, либо объявляется ничья; каждый избиратель может голосовать либо за одного из кандидатов, либо воздержаться;
\item голования анонимны
\item сама система голосований не зависит от личности кандидатов
\item \term{монотонность}: пусть в результате выборов либо объявляется ничья, либо побеждает кандидат $X$, тогда при тех же самых условиях, если один из избирателей отменит свой голос за кандидата $Y$, либо же воздержавшийся избиратель решит отдать свой голос за $X$, то победит $X$.
\end{enumerate}
\end{thm}
\begin{proof}
Условия 2 и 3 означают, что если все избиратели вдруг резко сменят свой голос на противоположный, то рузультат голосований так же сменится на противоположный. Из этого следует, что если за $X$ и $Y$ проголосовало одинаковое количество человек, то результатом обязательно будет ничья: в противном случае эти условия нарушались бы.

Теперь, пусть за $X$ отдаётся на один голос больше, чем за $Y$. Это всё равно, что из состояния ничьей кто-то однал свой голос за $X$. По условию монотонности $X$ побеждает. Продолжая по индукции мы получаем, что побеждает всегда тот кандидат, за которого отдано большинство голосов.
\end{proof}

Что произойдёт в случае, если кандидатов больше, чем два? Мы уже видели, что в этом случае голосование простым большинством имеет определённые проблемы, хотя и не связанные с условиями, обозначенными в тереме Мэя. Мы докажем, что при опредлённых условиях, накладываемых на <<справедливость>>, адекватной системы голосования может и не быть вовсе.

\begin{definition}
\term{Решающим множеством} избирателей называется такая группа избирателей, что если все избиратели из этой группы проголосуют одинаково, то это результат голосования определится предпочтением этой группы. Если решающее множество избирателей состояит из одного человека, то такой избиратель называется \term{диктатором}.
\end{definition}

\begin{definition}
\term{Фильтром над множеством $X$} называется такое семейство подмножеств $\mathcal{F}\subset 2^X$, для которого выполняются следующие условия:
\begin{enumerate}
\item $\varnothing \not\in \mathcal{F}$, $X\in\mathcal{F}$;
\item если $A\in\mathcal{F}$ и $A\subset B$, то $B\in \mathcal{F}$;
\item если $A\in\mathcal{F}$ и $B\in\mathcal{F}$, то $A\cap B\in\mathcal{F}$.
\end{enumerate}
\end{definition}

Понятно, что решающее множество избирателей определено в общем случае неоднозначно, поэтому мы на самом деле можем говорить не об одном решающем множестве, а сразу о семействе множеств. Легко так же видеть, что семейство решающих множеств так же удовлетворяет всем свойствам фильтра, кроме, возможно последнего. Одной из целей нашего последующего рассуждения будет как раз показать, что последнее свойство при определённых условиях так же работает.

Интуииция про фильтры как раз может быть такой, то фильтр определяет семейство <<больших>> множеств, но не в смысле количества элементов в нём, а в смысле их значительности. Образно представить себе решето, в которое вы бросам множества, и маленькие множества проскакаивают, а большие остаются. Решето очень похоже технически на фильтр, который что-то пропускает, а что-то оставляет, отсюда и терминология.

На будущее замечу, что в определении фильра мы на самом деле никак не использовали свойства множеств, мы лишь использовали символы $\cap$ и $\subset$. Это позволяет нам ввести понятие фильра не только для булеана множеств, но и для произвольной ограниченной решётки, заменив $\subset$ на $<$ и $\cap$ на $\land$. Пока нам это не нужно, но в дальнейшем это окажется полезным.

Как и любое семейство множеств, все фильтры на $X$ можно упорядочить используя отношение включения. В этом смысле следует понимать следующее определение.

\begin{definition}
\term{Ультрафильтром} называется максимальный фильтр.
\end{definition}

\begin{example}
Рассмотрим семейство всех подмножеств $X$, содержащих в качестве элемента некоторый фиксированный $x$. Обозначим это семейство за $<x>$. Легко проверить, что это фильтр, однако $<x>$ так же является и ультрафильтром. Если предположить, что $<x>$ не ультрафильтр, то должен существовать какой-то более <<крупный>> фильтр $\mathcal{F}$, содержащий в себе $<x>$, но он в свою очередь должен содержать в себе множество без элемента $x$. Такое множество будет иметь пустое пересечение с $\{x\}\in<x>$ что нарушает свойство 3 фильтра. Про ультрафильтры $<x>$, что они \term{порождены} элементом $x$
\end{example}

Мы покажем ниже, что, во-первых решающее множество является не просто фильтром, но ультрафильтром, а так же что любой ультрафильтр над конечным множеством порождён единственным элементом. Следовательно, существует диктатор.

\begin{definition}
Мы будем говорить, что семейство множеств обладает свойством \term{FIP}, если любое конечное пересечение элементов этого множества непусто (FIP расшифровывается как Finite-Intersection property, что дословно и переводится как <<свойство конечного пересечения>>).
\end{definition}

\begin{lemma}
Любое FIP-семейство подмножеств $X$ может быть расширено (путём добавления в него новых множеств) до ультрафильра.
\end{lemma}
\begin{proof}
Во-первых, добавим в семейтсво все возможные пересечения его множеств. Затем добавим для каждого множества все его надмножества (то есть для каждого $A\subset X$ добавим все множества $B\subset X$ такие что $A\subset B$). Таким образом мы получили фильтр. Обозначим его как $\mathcal{F}$.

Рассмотрим множество всех фильтров над $X$ и рассмотрим в нём максимальное линейно-упорядоченное подмножество $F_0\subset F_1\subset F_2, \ldots$ (такие подмножества называются \term{цепью}), содержащее в качетсве элемента $\mathcal{F}$. Поскольку все элементы цепи являются подмножествами конечного множества, сама цепь будет конечна, и таким образом в ней найдётся максимальный элемент, который и будет расширением $\mathcal{F}$ до ультрафильтра.

В случае бесконечных множеств доказательство будет очень похожим, но потребует несколько более экстравагантного математического аппарата, который мы введём в пятой граве (доказательство останется тем же, только мы добавим фразу <<по лемме Цорна существует максимальный элемент>>). Пока бесконечный случай нам не потребуется.
\end{proof}

\begin{lemma}
Пусть $\mathcal{F}$~--- ультрафильтр, и $B$ такое множество, что $\forall A\in\mathcal{F}$ пересечение $A\cap B$ непусто. Тогда $B\in\mathcal{F}$.
\end{lemma}
\begin{proof}
Легко увидеть, что семейство множеств $\mathcal{T} = \{B\cap A| A\in\mathcal{F}\}$ обладает свойством FIP, а следовательно может быть расширено до некоторого ультрафильтра $\mathcal{G}$. Однако ультрафильтр $\mathcal{F}$ сам является расширением до ультрафильтра семейства $\mathcal{T}$, и в силу максимальности, отсюда следует, что $\mathcal{G} = \mathcal{F}$. Однако мы знаем, что $B\in\mathcal{T}$, откуда следует утверждение леммы.
\end{proof}

\begin{lemma}
Если $A\cup B\in\mathcal{F}$ и $\mathcal{F}$~--- ультрафильтр, то по крайней мере одно из множеств $A$ или $B$ ему принадлежит.
\end{lemma}
\begin{proof}
Прежположим, что $A,B\not\in\mathcal{F}$. Возьмём такие $C, D\in\mathcal{F}$, что $A\cap C = B\cap D = \varnothing$. Такие множества мы всегда можем выбрать, так как если бы это не удалось, то по прошлой лемме $A$ или $B$ принадлежали бы ультрафильтру. Но при таком выборе $(A\cup B)\cap(C\cap D) = \varnothing$, и, поскольку $C\cap D\in\mathcal{F}$, по третьему свойству фильтров, $A\cup B \not\in \mathcal{F}$. Полученное противоречие завершает доказатальство леммы.
\end{proof}

\begin{thm}
Для того, чтобы фильтр $\mathcal{F}$ над $X$ был ультрафильтром, необходимо и достаточно, чтобы для любого множества $A$ выполнялось либо $A\in\mathcal{F}$, либо $A^C\in\mathcal{F}$.
\end{thm}
\begin{proof}
Пусть вначале $\mathcal{F}$ ультрафильтр. Мы знаем, что $A\cup A^C = X \in\mathcal{F}$. Утверждение теоремы следует из последней леммы.
Если же это свойство выполняется, то это значит, что мы никак не можем расширить фильтр. Рассмотрим, например, мы множество $B$, такое что $A\subset B\not\in\mathcal{F}$. Но тогда $B^C\in\mathcal{F}$, хотя $A\cap B^C=\varnothing$, что противоречит свойству фильтра.
\end{proof}

Теперь мы готовы доказать нашу основную теорему.

\begin{thm}
(Гибберт-Саттервайте) Любая система выборов, в которой может победить любой из кандидатов (при соответствующих голосах за него), и в которой невозможно тактическое голосование, будет иметь диктатора.
\end{thm}

Прежде чем перейти к непосредственному доказательству, уточним немного формулировку и получим парочку лемм. Множество кандидатов будем обозначать как $[m]$, множество избирателей как $[n]$. Избиратель $i\in [n]$ имеет предпочтения, выражающиеся перестановкой, которые мы будем обозначать как $\pi_i$. Тогда место, на которое поставил бы избиратель кандидата $x$ обозначается как $\pi_i(x)$. Сама функция голосования~--- это функция типа $f:S_m^n\to [m]$. Элемент множества $S_m^n$ будем называть \term{профилем голосования}. Для краткости будем обозначать $P = S_m^n$. Соответственно конкретное голосование~--- это элемент $\pi\in P$.

Определим строго что значит <<невозможно тактическое голосование>>. Тактической голосование~--- это когда некий избиратель $i$ записывает в биллютень не реальные свои предпочтения, а некоторые другие. В итоге профиль $\pi$ меняется на $\pi'$. Если при этом $\pi_i(f(\pi')) > \pi_i(f(\pi))$, то это значит, что избиратель $i$ смог как-то сманипулировать выборами в свою пользу. Это и значит, что произошло тактическое голосование. Условия теоремы предполагают, что это должно быть невозможно.

\begin{lemma}

\end{lemma}
\section{Невычислимое, недоказуемое}

В этом параграфе мы поговорим о совсем отвлечённых темах. Можно было бы это всё не рассказывать, но мне изложенное кажется очень уж удивительным, поэтому не могу молчать. Разговор будет совершенно не формальным, а больше ознакомительным. Материал этого параграфа не является необходимым для дальнейшего чтения учебника, так что если что-то станет совсем не понятно, то это можно смело пропускать. Впрочем, если вы поймёте изложенное, то, думаю, получите от этого удовольствие.

Предположим, что у нас есть какое-то логическое утверждение, зависящее от натурального числа $P(n)$. Это утверждение, предположим, можно легко проверить для любого конкретного числа, но не понятно верно ли оно для любого числа.

Например, можно рассмотреть \term{гипотезу Гольдбаха} о том, что любое чётное число может быть представлено в виде суммы двух простых чисел (так, например, $50=19+31$ или $100=11+89$). Для любого конкретного чётного числа можно довольно легко перебором подобрать его разложение на сумму двух простых чисел. По крайней мере, всегда у людей это получалось. А вот возможно ли это сделать в общем случае, то есть если получится представить в виде суммы двух простых вообще любое чётное число~--- вопрос, на который наука пока не нашла ответа.

Или вот можно рассмотреть \term{последовательности Коллатца}. Начальный элемент последовательности~--- это произвольное натуральное число, которое мы обозначим как $K_0$. $K_{n+1}$ получается из $K_n$ по следующему правилу: если $K_n$ чётное, то $K_{n+1}=\frac{K_n}{2}$. Если $K_n$ нечётное, то $K_{n+1} = 3K_n  + 1$. Например, если за $K_0$ принять число 15, то мы получаем следующую последовательность:
$$15, 46, 23, 70, 35, 106, 53, 160, 80, 40, 20, 10, 5, 16, 8, 4, 2, 1$$
Как видим, последовательность пришла в единицу. До сих пор с какого бы числа люди не начинали писать последовательность Коллатца, она всегда приходит в единицу, но приходит ли она в единицу всегда, или всё же есть исключения~--- никто не знает, это открытый вопрос математики.

Интуитивно должно казаться ясным, что чтобы ответить на такие вопросы, не достаточно перебрать просто большое количество вариантов. В обоих упомянутых проблемах математики перебрали уйму чисел (фактически все числа, на которые хватило вычислительных возможностей), но примера, чтобы какое-то число не раскладывалось в сумму двух простых или чтобы последовательность Коллатца для него не приходила в единицу, так и не нашли. Доказывает ли это что-либо? Конечно нет, потому что чисел, которых математики не проверили, куда больше~--- их ещё целая бесконечность, а проверена лишь малая конечная часть.

Так вот. Примерно 50 лет тому назад было доказано, что на самом деле для почти любого утверждения достаточно проверить лишь конечное число случаев. То есть если вплоть до некоторого номера $M$ нам не встретятся чётные числа, не представимые в виде суммы двух простых, или не встретится последовательность Коллатца, не приходящая в единицу, то такие числа нам не встретятся вовсе никогда.

\begin{thm}
Для установления истинности почти любого утверждения $P(n)$ достаточно проверить лишь истинность конечного числа утверждений $P(n)$ для $n$ от 1 до некоторого $M$.
\end{thm}
\begin{proof}
Давайте представим себя на минуточку программистами. Мы хотим найти такое число $f\in\mathbb{N}$, что $P(f)$ окажется неверным, то есть мы ищем контрпример к нашей гипотезе.

Будем считать, что мы можем написать программу для проверки утверждения $P$ для любого конкретного номера (это предположение и составляет значение слова <<почти>> в формулировке теоремы). Напишем программу, которая будем перебирать все числа подряд начиная единицей и которая остановится лишь когда она найдёт такое $f\in\mathbb{N}$, что $P(f)$ не выполняется. Возможно, что мы не найдём его никогда и тогда программа будет работать бесконечно долго, а это значит, что у нас есть время попить чай и поразмышлять.

Код нашей программы имеет конечную длину, обозначим её N. Если рассмотреть все возможные программы длины $N$, то среди них найдутся конечно же те, которые будут работать бесконечно долго (можно сказать, что они <<зависнут>>) и те, которые в итоге остановятся. Обозначим за $B_N$ самое долгое время работы программы длины $N$ (на английском эти числа называются Busy Beaver Numbers по фольклорным причинам, а как это адекватно переводится на русский, я и не знаю). Как вычислить это самое $B_N$ не очень понятно, но такое число вполне определено. Предположим пока, что мы его знаем.

Посмотрим на часы, сколько уже работает наша программа по поиску контрпримера, пока мы пили чай? Что? Уже дольше, чем $B_N$? Но ведь $B_N$ это самое долгое время, которое может работать программа, которая останавливастся. Значит мы знаем, что программа не остановится уже никогда, то есть контрпример не будет найден. Утверждение об истинности $P$ можно считать доказанным, программу можно выключать.

Конечно же, наша программа за конечное время $B_N$ успела проверить лишь конечное число вариантов от 1 до некоторого $M$, но из рассуждений, приведенных выше, следует, что нам этого каким-то образом оказалось достаточно для того, чтобы утверждать об истинности $P$ для совсем любого $n$. В чём тут подвох? Подвоха тут нет. 
\end{proof}

Задумайтесь на секундочку: почти любой сложный вопрос математики, касающийся натуральных чисел, может быть установлен путём перебора конечного числа вариантов. Будь то теорема Ферма, гипотеза Гольдбаха, Коллатца или кого ещё: достаточно проверить конечное число случаев. Не знаю как вам, но на меня в своё время этот факт произвёл неизгладимое впечатление.

Насколько этот результат практичен? На самом деле совершенно не практичен. Во-первых, не понятно как вычислить значение $B_N$. Во-вторых, даже если мы его вычислим для какого-то $N$, то оно наверняка окажется настолько гигантским, что вряд ли поместится в наш калькулятор. Так что выгоды мы из этого всего извлечь судя по всему не сможем, увы. Но сам факт!

Логично было бы, если бы после изложения этого результата я тут же показал бы, как считать числа $B_n$, однако сейчас мы уйдём в сторону и я покажу как эти числа ещё можно было бы использовать, если бы они были в нашем распоряжении.

Попробуем написать такую программу, которая будет анализировать другую программу, и говорить нам о том, зависнет она или же нормально остановится. Можно рассматривать такую программу как функцию $H:P\to\{0,1\}$, где $P$~--- множество всех возможных программ. 0 означает, что программа зависнет, 1 что остановится.

При условии, что мы можем написать программу, вычисляющую $B_n$, мы нашу задачу можем очень легко решить. Пусть нам дана программа $x$ длины $n$, для которой надо установить остановится ли она. Поскольку программа~--- это просто описание каких-то действий, мы можем заставить выполнять эти действия не компьютер, а нашу программу $H$, то есть мы как бы будем симулировать выполнение программы $x$ внутри программы $H$. Это не сложно технически, но я не буду углубляться в детали как это реализуется. Если вы знакомы с компьютерными технологиями, то наверняка слышали про интерпретаторы, виртуализацию, виртуальные машины и подобное. Это именно то, это возможно и это не особо сложно.

По ходу выполнения программы $x$, программа $H$ будет считать, как долго $x$ работает. Если вдруг окажется, что программа $x$ работает дольше, чем время $B_n$, то мы знаем, что она не закончится, и $H$ останавливает $x$, выдавая нам результат 0. Если же $x$ остановится сама раньше, то $H$ даст результат 1. Вроде всё.

А теперь давайте напишем такую программу $A$, которая так же будет принимать на вход какую-то другую программу $x$, затем будет запускать на ней программу $H$, и если $H(x)=1$, то $A$ будет зависать (например, начнёт считать все натуральные числа подряд), а если же $H(x)=0$, то она будет возвращать какой-то результат. Такую программу так же легко написать.

Попробуем понять какой результат мы получим, если на вход программе $A$ мы дадим её же саму, то есть попробуем выполнить $A(A)$. Что будет? Предположим, что $H(A)=0$, тогда программа $A$ должна завершиться, вернув результат. Но это противоречит тому, что $H(A)=0$. Из этого противоречия видно, что $H(A)\not=0$. Пусть теперь $H(A)=1$, но тогда $A$ должна зависнуть, что противоречит тому, что $H(A)=1$, а значит $H(A)\not=1$. Как так? $H$ оказывается не равно ни одному своему возможному значению!

Это противоречие означает, что программа $H$ не может быть написана, не существует такой программы и существовать не может. Но как так, ведь мы же чётко указали, как её можно написать! Если вглядеться, то при описании работы программы $H$ мы опирались на то, что мы можем как-то вычислить числа $B_n$, но мы не показывали этого. И это единственный момент во всём рассуждении, который мы никак не обосновали. Значит, числа $B_n$ вычислить невозможно.

Тот факт, что есть такая функция $B$, которая вполне определена, но которую невозможно вычислить, может показаться невероятным. На самом деле если подумать, то это довольно не сложно. Из определения функции $B$ мы могли бы попробовать вычислить $B(n)$, запустив одновременно все возможные программы длины $n$, и, заведя таймер, пытаясь понять, какое самое большое время работы программы. Однако с учетом того, что есть программы, которые зависают, мы никогда не узнаем, когда таймер пора выключать, хотя конечно же такой момент во времени существует. Это и есть пример того, что функцию невозможно вычислить.

Мы получили два результата.

\begin{thm}
Невозможно написать программу, которая бы могла по коду другой программы определить остановится ли она или зависнет.
\end{thm}
\begin{thm}
Числа $B_n$ невычислимы. Другими словами невозможно определить максимальное время, которое может работать независающая программа длины $n$.
\end{thm}

Приведённые рассуждения можно вывернуть ещё и вот под каким углом. Напишем программу $E$, которая будет проверять истинность произвольного утверждения $x$ путём последовательного перебора всех возможных доказательств (программа может начинать с коротких доказательств и постепенно переходить к более длинным, так что такая программа определённо может быть написана). Если предположить, что всё можно доказать или опровергнуть, то через какое-то время наша программа либо найдёт доказательство для $x$, либо доказательство для $\neg x$.

Используя эту программу, однако, мы опять же легко можем написать пресловутую программу $H$ проверяющую, остановится ли программа $y$ или нет. Просто вместо запуска программы $y$ и ожидания времени $B_n$ мы на этот раз будем искать доказательство остановки или зависания $y$. Поскольку мы уже знаем, что программа $H$ не может существовать, наше предположение, что программа $E$ всегда либо найдёт доказательство $x$ либо $\neg x$ так же не верно. Это значит, что существуют такие утверждения $x$, которые невозможно ни доказать ни опровергнуть. Как вы вероятно помните, это есть ни что иное как теорема Гёделя о неполноте:

\begin{thm}
Существуют утверждения, которые невозможно ни доказать ни опровергнуть.
\end{thm}

Я упомянул, что числа $B_n$ окажутся скорее всего довольно большими. Оказывается, мы можем примерно прикинуть насколько именно большими они будут.

\begin{thm}
Числа $B_n$ с ростом $n$ растут быстрее, чем любая последовательность, которую мы могли бы вычислить.
\end{thm}
\begin{proof}
Предположим, что это не так, и что существует некоторая последовательность $x_n$ такая, что мы можем её вычислить и что $B_n\le x_n$. Но тогда мы можем запустить все программы длины $n$ и останавливать их как только они отработают время $x_n$. Среди программ, остановившихся раньше, выберем ту, что работала дольше. Время, которое она работала~--- это и есть $B_n$. Но это значит, что мы смогли их вычислить, а это противоречит теореме 3.51. Значит, такой последовательности $x_n$ всё же не существует.
\end{proof}

\begin{exercise}
Если предположить, что мы имеем в нашем распоряжения числа $B_n$ и неограниченный вычислительный ресурс, проверка истинности произвольных утверждений может всё равно вызывать трудности. Гипотеза Коллатца является как раз таким примером. Если наша программа поиска контрпримера не остановилась за время $B_n$, то она не остановится действительно никогда, но это может иметь разное значение: либо она нашла такое $m$, что последовательность Коллатца для него будет бесконечной (и соответственно программа будет бесконечно долго вычислять её значения), либо же что она такого $m$ никогда не найдёт. Таким образом мы не получили из работы программы никакой полезной информации. Однако простая модификация алгоритма поиска контрпримера может всё таки дать нам способ понять, найден ли контрпример, либо же он не будет найден никогда. Придумайте эту модификацию.
\end{exercise}

\begin{exercise}
\term{Великая теорема Ферма} (на английском её называют последней теоремой) утверждает, что уравнение
$$x^n + y^n = z^n$$
не имеет решений для $n>2$. Покажите, как можно было бы её доказать при неограниченном вычислимом ресурсе и известных $B_n$.
\end{exercise}

У Великой теоремы Ферма интересная история. Ферма сформулировал её в 1637 году в виде пометки на полях книги, которую он читал. Там же он указал, что он нашёл элементарное доказательство этой теоремы, но поскольку места недостаточно, он его приводить не будет. В итоге первое доказательство этой теоремы появилось только в 1994 году (спустя 356 лет!) и занимало оно 130 страниц. В этом доказательстве спустя год была найдена ошибка, которую ещё год исправляли. В итоге финальное доказательство без ошибки было представлено лишь в 1996 году. Такая вот история.

В заключение этого параграфа я замечу, что, конечно, всё написанное здесь очень неформально. Из написанного мной совершенно, например, не понятно каким образом компьютерные программы увязываются с аксиоматикой натуральных чисел и теорией множеств. Так же совершенно не понятно что именно из себя представляют сами программы. Использовать в качестве модели реальные компьютеры и реальные языки программирования не получится, так как в реальных системах есть много факторов, которые мешают идеализированному математическому описанию вычислительного процесса: механизмы мультипоточности, пайплайнинг, всякие мьютексы, графический интерфейс, ввод-вывод и вот всё подобное не дают оставляют возможности рассуждать о вычислениях абстрактно.

Чтобы избавиться от всех мешающих факторов реальных компьютеров используется модель \term{машины Тьюринга}, которая представляет собой видимо самую простую модель компьютера. Я не буду её описывать, но если вы хотите понять как примерно устроен формализм того, что я здесь изложил, вы можете поискать статьи о ней и о том, как она соответствует реальным вычислительным машинам, рекурсивным функциям и понятию множеств. Так же рекомендую почитать трагичную и важную биографию самого Алана Тьюринга: будучи блестящим учёным, работающим в области вычислимости и криптографии (значительным образом, благодаря именно его работе союзники могли перехватывать шифрованные сообщения Германии во Вторую Мировую войну; теорема~3.50 так же была сформулирована и доказана именно им), он был обвинён в гомосексуальных связях и принуждён к химической кастрации и употреблению гормональных таблеток <<для лечения гомосексуализма>>. Закончилась история лишением его всех военных наград и званий, а так же полным запретом на занятия наукой. Итогом таких мер по борьбе с гомосексуализмом стал его суицид.

Сегодня ему ставят памятники, его именем названа самая престижная награда в области компьютерных наук. На этой трагичной ноте мы и закончим эту главу.


\end{document}
