\section{Определение}

Натуральные числа~--- это 0, 1, 2, 3, и~т.д. Все это вроде знают. Но как определить понятие натурального числа строго? Чтобы оценить задачу, прежде чем читать дальше, попробуйте дать такое определение самостоятельно, заодно с определение арифметических операций и не опираясь на физическую интуицию, а лишь на логику. Замечу, что этот параграф носит малоприкладной характер~--- его цель лишь в определении натуральных чисел. Я же не могу написать: <<а с этого момента давайте использовать натуральные числа>>. Определить я их обязан как-то, но в то же время подробные определения, которые я здесь привожу, вряд ли будут кому-то действительно полезными. Поэтому данный параграф можно читать наискосок либо не читать вообще, если вы помните свойства натуральных чисел.

Дать строгое определение пытались многими простыми способами, но все более-менее интуитивные подходы неизменно заводят в тупик. На данный момент мейнстримом в определении натуральных чисел являются два подхода: определение непосредственно на основе теории множеств (определение Фреге-Рассела) и чуть более сложный, но также важный в силу полезных обобщений, подход на основании аксиом Пеано. Мы не будем рассматривать подробно и формально эти подходы со всеми выкладками, а лишь рассмотрим суть этих определений. Недостающие пробелы вы можете закрыть сами~--- это довольно большая работа, но без каких-либо принципиальных сложностей. Более подробно и обстоятельно мы вернёмся к аксиоматизации натуральных чисел позже в шестой главе, когда будем говорить о бесконечных множествах.

Определение Фреге-Рассела отражает идею о том, что натуральные числа определяют количество элементов во множествах. Это довольно понятно на пальцах, но надо дать строгое определение. Первая идея, которая обычно приходит на ум~--- это взять вообще все множества в принципе и разбить их на классы эквивалентности так, чтобы в одном классе оказались множества одинакового размера.

Ввести такую эквивалентность для множеств несложно: в \S~2.4 мы уже вводили понятие равномощности. В соответствии с тем определением, два множества $A$ и $B$ называются равномощными, если существует некоторая биективная (то есть взаимооднозначная) функция $f:A\to B$. Пусть например у нас есть множество трёх букв алфавита $A=\{a, b, c\}$ и странное множество $B=\{\heartsuit, \clubsuit, \spadesuit\}$. Эти два множества равномощны, так как существует биекция $$f=\{(a, \clubsuit), (b, \heartsuit), (c, \spadesuit)\}$$ --- то есть существует способ назначить каждому элементу множества $A$ некий элемент множества $B$ и обратно. Если мы рассмотрим теперь множество с одним дополнительным элементом $C=\{\heartsuit, \clubsuit, \spadesuit, \Diamond\}$, то увидим, что никаких способов задать тут биекцию не существует, и стало было множество $C$ не равномощно $A$ и $B$.

\begin{exercise}
Докажите, что на любом множестве, состоящим из множеств, отношение биекции~--- это отношение эквивалентности.
\end{exercise}

Теперь, когда у нас есть некое отношение эквивалентности, мы хотим задать классы эквивалентности~--- и эти классы эквивалентности мы могли бы объявить натуральными числами, тогда каждое натуральное число представляло собой класс всех множеств одинаковой мощности. К сожалению, поступить таким образом мы не можем, так как классы эквивалентности могут быть построены лишь на некотором множестве, и в нашем случае было бы необходимым рассматривать множество всех множеств. Последнего, однако, как мы показали в \S~2.5, не существует.

Тем не менее, эта идея всё же может быть доведена до конца, если вместо задания сразу всего класса эквивалентности, мы зададим лишь по одному представителю из этого класса, а затем, чтобы определить принадлежность некоторого множества к классу, мы будем сравнивать это множество с представителями классов. Это похоже на то, что делают в физике при измерениях: вначале определяют некий один эталонный объект (скажем, метр), а затем все остальные сравнения производятся уже относительно него.

Самый маленький класс множеств~--- это пустое множество. Представителем этого класса логично выбрать $\emptyset$. Обозначим его как $$0 = \emptyset$$

Следующим за ним класс должен состоять из только одного элемента. Этим элементом можно взять как раз число 0, и определить представителя нашего нового класса как $$1 = \{0\} = \{\emptyset\}$$

Теперь определим представителя для ещё более крупного множества, в котором на один элемент больше. Для этого естественно использовать уже имеющиеся у нас элементы 0 и 1: $$2 = \{0, 1\} = \{\emptyset, \{\emptyset\}\}$$

Ну и до кучи: $$3 = \{0, 1, 2\} = \{\emptyset, \{\emptyset\}, \{\emptyset, \{\emptyset\}\}\}$$

Процедура, в общем-то, проста. Если у нас есть число $n$, то мы можем определить следующий за ним элемент: $$S(n) = n \cup \{n\} = \{0, 1, \ldots, n\}$$

Применяя эту процедуру бесконечное число раз, мы можем получить все натуральные числа. Это на самом деле не столь очевидно, но Infinity Axiom гарантирует нам, что подобным образом действительно возможно определить некое множество, которое мы будем обозначать как $\mathbb{N}$ и называть его множеством натуральных чисел. Теперь мы готовы для наших первых определений.

\begin{definition}
Множеством \term{натуральных чисел} $\mathbb{N}$ называется множество чисел, полученных из $0=\emptyset$ применением функции $S$.
\end{definition}

\begin{definition}
\term{Последовательностью} элементов множества $X$ называется функция $x:\mathbb{N}\to X$. Обозначается последовательность как $\{x_n\}$.
\end{definition}

\begin{definition}
\term{Конечной последовательностью} элементов множества $X$ называется функция $n \to X$.
\end{definition}

\begin{definition}
Множество $X$ называется \term{конечным}, если существует равномощное ему множество $n\in\mathbb{N}$. В противном случае $X$ называется \term{бесконечным}.
\end{definition}

\begin{definition}
\term{Мощностью} конечного множества $X$ называется такое число $n\in \mathbb{N}$, что $X$ и $n$ равномощны. Обозначается это как $|X| = n$.
\end{definition}

Здесь, вероятно, требуются примеры. Возьмём опять наше множество $C=\{\heartsuit, \clubsuit, \spadesuit, \Diamond\}$. Если без формализма, а на уровне интуиции, то его мощность~--- это количество элементов в нём. Это очевидным образом связано с определением, данным нами выше, если вспомнить, что $4 = \{0, 1, 2, 3\}$. Тогда биекцией $4\to C$, устанавливающей равномощность, может быть функция $f = \{(0, \heartsuit), (1, \clubsuit), (2, \spadesuit), (3, \Diamond)\}$. Аналогичным образом получается связь и других определений с нашей интуицией~--- здесь может быть непривычным лишь то, что мы рассматриваем натуральные числа как множества, но именно это, если вдуматься, позволяет нам устанавливать между ними и множествами соответствия, используя привычный механизм функций.

\begin{definition}
Мы говорим, что число $m$ \term{меньше или равно} $n$ и обозначаем это как $m\le n$, если $m\subset n$.
\end{definition}

\begin{definition}
Мы говорим, что число $m$ \term{меньше} $n$ и обозначаем это как $m<n$, если $m\subset n$ и $m \not= n$.
\end{definition}

\begin{thm}
Отношение $\le$ задаёт линейный порядок на $\mathbb{N}$.
\end{thm}
\begin{proof}В качестве упражнения.\end{proof}

\begin{thm}
$S(n) > n$ для любого $n$.
\end{thm}
\begin{proof}
Элементарно.
\end{proof}

\begin{example}
Сравним числа 3 и 4. Мы знаем, что $3 = \{0, 1, 2\}$ и $4 = \{0, 1, 2, 3\}$. Очевидно, что $3\subset 4$, и, следовательно, $3\le4$. Поскольку $3\not= 4$, то $3<4$.
\end{example}

В полной аналогии с приведёнными определениями можно определить также сравнения \term{больше} ($>$) и \term{больше или равно} ($\ge$).

\begin{thm}
Множество $\mathbb{N}$~--- бесконечное.
\end{thm}
\begin{proof}
Пусть $\mathbb{N}$ конечно и $n = \max\{\mathbb{N}\}$. Тогда $m = |\mathbb{N}| = S(n) > n$, но поскольку $n$~--- максимальный элемент, $m\not\in\mathbb{N}$, что противоречит определению $\mathbb{N}$. Что и требовалось доказать.
\end{proof}

На самом деле для порядка натуральных чисел справедливо даже более сильное утверждение, к которому мы будем обращаться время от времени, а потом подробнее рассмотрим его в шестой главе:

\begin{definition}
Множество называется \term{фундированным}, если любое его подмножество имеет минимальный элемент.
\end{definition}

\begin{definition}
Множество называется \term{вполне упорядоченным}, если оно одновременно фундированное и линейно упорядочено.
\end{definition}

\begin{thm}
Множество $\mathbb{N}$ вполне упорядочено относительно порядка $\le$.
\end{thm}
\begin{proof}
В качестве упражнения.
\end{proof}

Теперь немного отвлечёмся от подхода Фреге-Рассела и посмотрим на аксиомы Пеано. Эти аксиомы не отвечают на вопрос что же вообще такое натуральные числа, но постулируют существование некоего множества $\mathbb{N}$ такого, что в нем существует выделенный элемент 0, и на котором задана некая инъективная функция $S:\mathbb{N}\to\mathbb{N}$ такая, что элемент 0 не имеет обратного. Это не совсем полная аксиоматика~--- мы будем её дополнять по мере надобности, но уже сейчас очевидно, что это определение практически дублирует подход Фреге-Рассела за исключением того, что мы не определяем конкретный вид элементов множества $\mathbb{N}$, но этого, на самом деле вполне достаточно~--- этим определением можно пользоваться, хотя жизнь при этом получается намного сложнее, как мы увидим ниже.

Перейдём теперь к определению арифметических операций. Вначале я буду давать интуитивное определение, затем доводить его до строгого вида в аксиоматике Фреге-Рассела, и затем определение для аксиом Пеано.

\begin{definition}
Пусть у нас есть непересекающиеся множества $M$ и $N$, такие что $|M|=m$ и $|N| = n$. Будем писать, что $|M\cup N| = m+n$, а число $m+n$ называть \term{суммой} $m$ и $n$.
\end{definition}

Это имеет очень простой комбинаторный смысл: если у нас есть некоторый набор, состоящий из $n$ объектов и мы его объединяем с набором, состоящим из $m$ объектов, то в результате мы получим набор, состоящий из $n+m$ объектов. Это то что объясняют в первом классе школы, но только более абстрактно.

Тем не менее, это не слишком хорошее определение. Во-первых, оно даёт нам понятие суммы не в терминах самих чисел, а в терминах неких множеств, причём непересекающихся. Это плохо, поскольку сами числа, будучи множествами, всегда пересекаются друг с другом (кроме числа 0). Поэтому если у нас есть два числа, не привязанных к конкретным множествам, это определение не даёт нам понять как определить их сумму.

Во-вторых, встаёт такой неприятный вопрос: пусть у нас есть непересекающиеся множества $A$ и $B$, такие что $|A|=|N|$ и $|B|=|M|$, можем ли мы в этом случае гарантировать, что $|A\cup B| = |N\cup M|$? Интуитивно это очевидно, но как это доказать~---вопрос нетривиальный.

И в третьих, всё ещё хуже: даже если $A$ и $B$ конечны, то где гарантия, что их объединение будет также конечным? Опять же, это очевидно, но поди докажи (а когда мы будем говорить о бесконечных множествах, выяснится, что подобные очевидные рассуждения часто банально неверны).

Первая проблема устраняется с помощью следующих двух упражнений:

\begin{exercise}
Докажите, что $|m\times 1| = m$ для $m\in\mathbb{N}$.
\end{exercise}

\begin{exercise}
Докажите, что $(m\times 1) \cap m = \emptyset$.
\end{exercise}

Пользуясь этими двумя утверждениями, можно ввести такое определение:

\begin{definition}
$m + n = |m\times1 \cup n|$.
\end{definition}

Это решает первую и вторую (после некоторых несложных раздумий) обозначенные нами проблемы, но не решает третью. Её можно решить либо с помощью метода матиндукции, который мы отложим на последующие параграфы, либо используя изначально более абстрактные конструкции и обобщая натуральные числа на случай бесконечных множеств, что мы оставим до шестой главы нашего учебника для начинающих.

А теперь то же самое определение, но уже в аксиоматике Пеано:

\begin{definition}
Сложение определяется следующим образом:\\*
$n + 0 = n$\\*
$n + S(m) = S(m + n)$
\end{definition}

\begin{example}
$$n + 3 = n + S(2) = S(n+2) 
=S(n+S(1)) = S(S(n+1)) = S(S(S(n)))$$
\end{example}

Как видно из примера, это определение фактически говорит, что выражение $m + n$ означает, что к числу $m$ применяется операция $S$ (фактически, увеличение на единицу), $n$ раз. Интуитивно это должно быть понятно, строгое же доказательство того, что такое определение правомочно, будет дано позже.

\begin{thm}
Справедливы следующие свойства сложения:
\begin{enumerate}
\item нейтральность нуля: $a + 0 = a$
\item коммутативность: $a + b = b + a$,
\item ассоциативность: $a + (b + c) = (a + b) + c = a + b + c$
\item если $a < b$, то $a + c < b + c$
\end{enumerate}
\end{thm}
\begin{proof}
Нейтральность нуля очевидна. Для коммутативности и ассоциативности используя определение Фреге-Рассела довольно легко построить биекцию между левыми и правыми частями равенства. Последнее равенство следует из того, что если $a < b$, то $S(a) < S(b)$ (доказывается элементарно), а прибавление любого натурального числа равносильно многократному применению операции $S$. Используя только аксиоматику Пеано это можно доказать, опять же, по индукции, о чем будет отдельный параграф.
\end{proof}

Пользуясь сложением легко определить линейный порядок на $\mathbb{N}$ в случае аксиом Пеано (для Фреге-Рассела это будет элементарная теорема):

\begin{definition}
$a < b$, если существует такое $n$, что $a + n = b$.
\end{definition}

\begin{definition}
Операция вычитания: мы пишем $a = b - c$, если $a + c = b$. В этом случае $a$ называется \term{разностью} $b$ и $c$.
\end{definition}

\begin{thm}
Операция $a - b$ определена только в том случае, если $a \ge b$.
\end{thm}
\begin{proof}В качестве упражнения\end{proof}

\begin{definition}
Операция умножения для Фреге-Рассела: $ab = |a\times b|$
\end{definition}

Здесь опять же надо внимательно отнестись к тем комментариям, которые я приводил для сложения, я на этом уже не буду подробно останавливаться.

Если смотреть на умножение комбинаторно, то получается простая интерпретация: для произвольных множеств $A$ и $B$ с мощностями $a$ и $b$ соответственно, имеем $|A\times B| = ab$. То есть произведение чисел~--- это количество элементов в декартовом произведении множеств соответствующих размеров. Это очень часто используется в самой базовой комбинаторике, например, так:

\begin{exercise}
Пусть у нас есть три бабы и два мужика. Сколько гетеросексуальных пар из них можно составить?
\end{exercise}

\begin{definition}
Операция умножения для аксиом Пеано:\newline
\item $m\cdot 0 = 0$\newline
\item $m\cdot S(n) = mn + m$
\end{definition}

\begin{thm}
Справедливы следующие свойства:
\begin{enumerate}
\item $0\cdot a = 0$
\item нейтральность единицы: $1\cdot a = a$
\item коммутативность: $ab = ba$
\item ассоциативность: $a(bc) = (ab)c = abc$
\item дистрибутивность: $a(b+c) = ab + ac$
\item если $a < b$ и $c \not= 0$, то $ac < bc$
\end{enumerate}
\end{thm}
\begin{proof}
Здесь всё аналогично доказательству подобных свойств для сложения~--- необходимо просто построить биекцию для левой и правой части (причём это не так просто, если ударяться прямо в формализм ZFC, хотя и возможно по индукции). Рекомендую самостоятельно попытаться строго проработать случай коммутативности, поскольку он несложен, но далеко не все понимают его.

Если отойти от формализма и посмотреть на вопрос геометрически, то $mn$ можно рассматривать как количество ячеек в таблице с $m$ строками и $n$ столбцами, а $nm$~--- количество ячеек в той же таблице, поставленной на бок~--- в этом случае строки и столбцы меняются местами, но количество ячеек при этом не меняется.

Из этой табличной интерпретации можно уже построить конкретную биекцию. Как увязываются таблицы и декартовы произведения рассматривалось в \S~2.2. Остальные приведённые здесь свойства могут быть интуитивно мотивированы подобным же образом.
\end{proof}

Используя обозначение $S(n) = n+1$ и свойство дистрибутивности можно легко понять смысл определения умножения в аксиомах Пеано: $mS(n) = m(n+1) = mn + m$.

\begin{thm}
Для любого $a$ и $b > 0$ найдутся такие числа $r<b$ и $q$, что $a = qb + r$.
\end{thm}
\begin{proof}
Будем строить конечную последовательность $\{r_n\}$ следующим образом: $r_n = a - nb$ для тех значений $n$, для которых вычитание будет определено. Множество элементов этой последовательности имеет минимальный элемент, который мы и обозначим как $r = a - qb$. Это и есть утверждение теоремы.
\end{proof}

\begin{exercise}
Где в доказательстве предыдущей теоремы неявно используется условие $b>0$?
\end{exercise}

\begin{definition}
Пусть $a = qb + r$. $q$ называется \term{частным от деления} $a$ на $b$, а $r$ \term{остатком от деления}.
\end{definition}

\begin{definition}
Если остаток от деления $a$ на $b$ равен нулю, то говорят, что $a$ \term{делится} на $b$, или что $b$ \term{делит} $a$. Обозначается это как $a\vdots b$ и  $b|a$ соответственно, а частное в этом случае обозначается как $a\over b$.
\end{definition}

\begin{exercise}
Докажите, что отношение делимости задаёт частичный порядок на $\mathbb{N}$.
\end{exercise}

\begin{definition}
Возведение в степень для Фреге-Рассела: пусть $m = |M|$ и $n = |N|$, тогда $m^n = |M^N|$
\end{definition}

Напомню, что $M^N = \{f|f:N\to M\}$~--- то есть это множество функций из $N$ в $M$. Это имеет простой комбинаторный смысл. Пусть, например, $M$~--- множество цветов рубашек и $N$~--- множество мужчин. Каждый мужчина выбирает себе цвет рубашки. Если все мужчины сделали свой выбор, то этот выбор представляется функцией $f: N \to M$. Сколько всего есть вариантов выбора цветов для всех мужчин сразу? Ровно столько, сколько есть таких функций, то есть ровно столько, какова мощность множества $M^N$. Остаётся только вопрос в том, как посчитать эту величину.

\begin{exercise}
Пусть для кодирования мы используем символы $\{a, b, c, d\}$. Сколько существует различных кодовых слов длины 5?
\end{exercise}

\begin{thm}
$m^n = \underbrace{m\cdot m \cdot \ldots \cdot m}_n$
\end{thm}
\begin{proof}
Пусть $|M| = m$ и $|N| = n$. Я приведу не самые строгие рассуждения, строго это опять же надо доказывать по индукции. Возьмём некоторый элемент $N$. Он может быть отображён в один из элементов $M$, которых $n$ штук. Возьмём другой элемент $N$, он также может быть отражён на один из $n$ элементов $N$. Итого для первых двух элементов $N$ существует $m\cdot m$ вариантов отображения. Если теперь рассмотреть ещё один элемент $N$, то он также может быть отображён в $m$ элементов, итого вариантов для отображения первых трёх элементов оказывается равно $m\cdot m\cdot m$. Продолжая рассуждения мы получим утверждение теоремы, поскольку в множестве $N$ всего $n$ элементов.
\end{proof}

Приведённое рассуждение подсказывает нам как можно определить степень для аксиоматики Пеано:

\begin{definition}
Степень в аксиоматике Пеано:\newline
$a^0 = 1$\newline
$a^{S(n)} = a^na$
\end{definition}

\begin{thm}
Для степеней справедливы следующие свойства:
\begin{enumerate}
\item $a^0 = 1$
\item $a^b a^c = a^{b+c}$
\item $(a^b)^c = a^{bc}$
\item если $a > b$, то для любого $c > 0$, $a^c > b^c$
\item если $a > b$, то для любого $c > 1$, $c^a > c^b$
\end{enumerate}
\end{thm}
\begin{proof}
Первое свойство дублирует определение Пеано, но его можно увидеть и из определения Фреге-Рассела: существует всего лишь одна функция из пустого множества в некоторое другое, и эта функция сама является пустым множеством. Функция вида $f:\emptyset\to X$ совершенно легальна и единственна, хоть она ничего и не отображает.

Второе свойство: $a^ba^c = \underbrace{\underbrace{a\cdot\ldots\cdot a}_b \cdot \underbrace{a \cdot\ldots \cdot a}_c}_{b+c} = a^{b+c}$

Третье свойство: $(a^b)^c = \underbrace{a^ba^b\ldots a^b}_c = a^{bc}$

Оставшиеся свойства предлагаю доказать самостоятельно в качестве упражнения.
\end{proof}

Приведённое определение Фрегге-Рассела может почти сразу дать нам ответ на вопрос о том сколько всего подмножеств имеет некоторое множество, а заодно объяснить обозначение $2^X$ для булеана. Прежде, однако, нам понадобится одно вспомогательное понятие.

\begin{definition}
Пусть $X\subset U$. Функция $\chi_X: U\to \{0,1\} = 2$ называется \term{индикаторной}, или \term{характеристической} функцией множества $X$, если $\chi_X(t) = 1$ при $t\in X$ и $\chi_X(t) = 0$ в противном случае.
\end{definition}

Каждому подмножеству $U$ соответствует своя характеристическая функция, ровно как и характеристической функции соответствует подмножество. Это соответствие позволяет нам легко получить желаемую теорему:

\begin{thm}
$|2^X| = 2^{|X|}$
\end{thm}
\begin{proof}
Для того, чтобы определить количество элементов в булеан, нам достаточно определить количество различных характеристических функций на множестве $X$. Поскольку характеристическая функция имеет вид $\chi:X\to 2$, множеством всех таких функций является $2^X$. По теореме 3.9 мощность этого множества $2^{|X|}$
\end{proof}
