\section{Голубоглазые островитяне}

Последний пример применения индукции многими считается самым контринтуитивным чуть ли не во всей математике.

\begin{example}
   На острове живёт 1000 человек с идеальным логическим складом ума. Из них 100 имеет голубые глаза, и 900~--- карие. Религия запрещает им знать свой цвет глаз и рассказывать другим о цвете глаз. Никаких отражающих поверхностей на острове нет. Если кто-то вдруг узнает свой цвет глаз, то он обязан в ближайшую ночь устроить публичное ритуальное самоубийство.

    В какой-то момент на остров приезжает путешественник, который не знаком с местной религией, но тем не менее довольно успешно вливается в местный коллектив. И однажды он случайно на общем собрании в ходе своей речи невзначай упоминает:

    ---[...] и я был очень удивлён увидеть здесь, в столь отдалённом уголке, голубоглазых людей [...]

    Вопрос: сколько осталось жить голубоглазым и/или кареглазым островитянам?
\end{example}

Казалось бы, ничего произойти не должно. Островитяне не узнали ничего нового: путешественник им сказал лишь то, что на острове есть голубоглазые люди, но ведь все и так уже видели до этого голубоглазых. Однако, не всё так просто.


