\section{Зачем это всё?}

В популярных книгах мотивационные вступления обычно принято писать в самом начале, чтобы человек понимал, что он и зачем изучает. В книгах по чистой математике это не принято, поскольку никаких практических применений материал почти никогда не находит, и читают математики книги просто для извлечения их них знаний. Само содержание книги~--- уже мотивация.

Тем не менее, какие-то общие слова о том, зачем всё изложенное нужно, я всё же могу сказать.

Я как автор учебника рассматривал для себя аж сразу пять причин, по которым я включил эту главу в курс:

\begin{enumerate}
\item Классическая логика является формальной основой для 99\% современной математики. На практике правила вывода, модели и логические операции математики почти никогда не применяются, но тем не менее, чтобы сформулировать аксиомы теории множеств, логика необходима. Мне хотелось показать читателю как все же строится современная математика, и чтобы иметь эту возможность, я сделал небольшой экскурс в логику.
\item Есть буквально пара мест даже в классических областях институтской математики, где формальная логика встречается (в основном это касается кванторов и законов де Моргана). В институте правда все обычно ограничивается словами «этот значок означает „для любого“». Знать более подробно что же это такое все же полезно.
\item Несмотря на то, что логика вроде бы как-то стоит особняком от остальной математики, есть все же несколько теорем, которые интересны широким слоям населения. Доказательство их в курсе так же надеюсь будет приятным занятием.
\item Математическая логика в современном мире является довольно базовым знанием, которое в скором времени скорее всего будут преподавать во всех школах. Простой пример: любая база данных является множеством, а пользователь, задавая вопрос базе данных, на самом деле на специализированном языке формулирует предикат. Эту операцию выполняет часто и продвинутый бухгалтер и юрист, и финансист, только они чаще всего не знают слов типа «подмножество» и «предикат». Понимание логики и теории множеств может здорово упростить им работу, если они захотят более подробно изучать работу с базами данных (что не сложно). Можно привести более сложные примеры с языками программирования, но это уже в общем-то дебри относительно нашего курса, поэтому не будем углубляться. Просто скажем, что все же эти знания пусть и не в чистом виде, но могут оказаться полезны.
\item Из всего, что я могу написать в этой книге, с одной стороны логика в значительной степени абстрактна, с другой же стороны она сравнительно проста и позволит расширять в дальнейшем набор примеров, которые мы будем использовать. Данная глава на мой взгляд является довольно не плохой тренировкой для мозгов.
\end{enumerate}

Если уйти от контекста книги с нашими узкими целями и говорить о более академических направлениях, то основное практическое применение логики (за вычетом философии и оснований математики, о чем я упоминал) --- это попытки разработки систем автоматического доказательства теорем. Ситуация здесь двоякая --- с одной стороны пока не существует никакой адекватной автоматизированной системы автоматических доказательств и работа в этом направлении вроде бы идёт не особо успешно. С другой стороны в качестве частных случаев были примеры компьютерного доказательств сложных теорем именно методами логики, которые человек до этого доказать оказывался не способен. Правда, эти доказательства оказываются небольшой гигантской последовательностью символов, которые формально с точки зрения логики верны, но интуитивно совершенно непонятны. Поэтому такого рода доказательства значительная часть математиков не признает.

Так же применения изложенный материал имеет в электронных схемах. Происходящее там уже правда мало походит на логику --- $1$ означает наличие заряда, а $0$ его отсутствие, а логические операции в этом случае превращаются уже в чистую арифметику. Об этом в этом курсе будет параграф в третьей главе.

Какие-то ещё применения логики встречаются в разработках искусственного интеллекта, где окружающий мир описывается набором высказываний, и компьютер знает какие действия как на эти высказывания влияют. При наличии информации об окружающем мире и конечной цели (или недостающей информации), компьютер может самостоятельно построить последовательность действий, необходимую для достижения результата. В ограниченных количествах подобные технологии вроде бы даже применяются где-то в робототехнике, но это скорее единичные случаи, нежели постоянная практика.
