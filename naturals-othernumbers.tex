\section{Прочие комбинаторные величины}

В этом параграфе мы очень кратко рассмотрим прочие комбинаторные величины, которые могут оказаться полезны и которые будут в дальнейшем выступать в качесве примеров.

\subsection{Числа Фибоначчи}

\begin{definition}
\term{Числами Фибоначчи} $F_n$ называются числа, определяемые начальными условиями $F_0 = 0$, $F_1 = 1$ и соотношением
$$F_{n+1} = F_n + F_{n-1}$$
\end{definition}

Начальные значения чисел Фибоначчи выглядят так:
$$0, 1, 1, 2, 3, 5, 8, 13, 21, 34, 55, 89, 144, \ldots$$

Эти числа возникают в целом ряде задач и довольно распространены. Исторически числа Фибоначчи стали широко известны после решения Леонардо Пизанским (<<Фибоначчи>>  было его прозвищем, что переводится как <<сын Боначчи>>) следующей задачи в 1202 году:

\begin{exercise}
Предположим, что каждая взрослая пара кроликов каждый месяц производит на свет ещё одну молодую пару кроликов. Взросление кроликов наступает в течение одного месяца. Изначально у нас есть одна пара молодых клоликов. Через месяц она становится взрослой. Еще через месяц эта пара производит на свет ещё пару молодых кроликов (итого 2 пары, из которых 1 молодая). В следующий месяц эта же пара производит ещё одну молодую пару, а пара, которая была молодой, взрослеет (имеем 3 пары кроликов, 1 молодая). Таким же образом в следующий месяц мы будем иметь 5 пар кроликов, из которых 2 будут молодыми. Сколько кроликов будет в конце года?
\end{exercise}

На самом деле числа, которые мы сегодня называем числами Фибоначчи, были известны ещё древним Индусам. Математик Пиндас в своём трактате <<Чхандас>> (датированным примерно 200 годом до нашей эры) использует их при решении примерно такой задачи:

\begin{exercise}
Пусть нам надо пройти путь длины $n$. При проходе пути мы можем использовать либо шаги длины 1, либо шаги длины 2. Докажите, что существует ровно $F_{n+1}$ способов пройти путь используя такие шаги. (Например для пути длины 3 мы можем сделать три одинарных шага, либо вначале одинарный, а потом двойноё, либо наоборот: итого 3 разных способа пройти путь).
\end{exercise}

Сами индусы, правда, решали хоть и ту же задачу, но из другой предметной области: они исследовали сколько всего существует мелодий, состоящих лишь из одной ноты, которая может иметь либо одинарную, либо двойную длительность. Интерпретация, данная привёденным мной упражнением, используется при решении следующих двух задач:

\begin{exercise}
Докажите следующее тождество, используя двойной счёт:
$$F_{n+1} = \sum_{k=0}^{\lfloor {n\over 2}\rfloor} {n-k\choose k}$$
\end{exercise}

\begin{exercise}
Докажите следующее тождество (идея доказательства очень похожа):
$$F_{n+1} = 1+ \sum_{k=0}^{n-1}F_n$$
\end{exercise}

\begin{thm}
$$F_n F_{n+1} = \sum_{i=1}^n F_i^2$$
\end{thm}
\begin{proof}

\end{proof}